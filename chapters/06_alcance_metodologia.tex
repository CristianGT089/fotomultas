\section{Alcance}

\subsection{Enfoque y delimitación geográfica}
Este trabajo se circunscribe al proceso de generación, gestión y verificación de \textbf{multas de tránsito automatizadas (fotomultas)} emitidas por la Secretaría Distrital de Movilidad de Bogotá.  Se excluyen deliberadamente:
\begin{itemize}
  \item Multas impuestas de forma presencial por agentes de tránsito.
  \item Procesos sancionatorios de otras ciudades o entidades territoriales.
  \item Funcionalidades de recaudo y pasarelas de pago (solo se registra el estado del pago, no se procesa el pago en sí).
\end{itemize}

\subsection{Componentes del prototipo}
El prototipo aborda los siguientes módulos funcionales:

\begin{enumerate}
  \item \textbf{Registro inmutable de la infracción}  
        Captura de metadatos (placa, fecha, hora, ubicación y tipo de infracción) y publicación del identificador de la evidencia en la \emph{blockchain} (Hyperledger Fabric).
  \item \textbf{Almacenamiento descentralizado de evidencias}  
        Carga de la imagen o video de la fotomulta en IPFS y obtención de su \emph{hash}.
  \item \textbf{Verificación pública}  
        Servicio de consulta que permite contrastar el hash guardado en la cadena con el archivo almacenado en IPFS.
  \item \textbf{Gestión del ciclo de vida de la multa}  
        Estados: \textsf{Generada} $\rightarrow$ \textsf{Notificada} $\rightarrow$ \textsf{En apelación} $\rightarrow$ \textsf{Pagada} $\rightarrow$ \textsf{Cerrada}.  
        Cada transición queda registrada mediante eventos de contrato inteligente.
  \item \textbf{Interfaz mínima}  
        Panel Web para: (i) agentes que registran la infracción y (ii) ciudadanos que consultan la autenticidad y el estado de su fotomulta.
\end{enumerate}

\subsection{Fuera del alcance}
\begin{itemize}
  \item Integración completa con sistemas legados del RUNT o SIMIT; se simula mediante datos de prueba.
  \item Implementación de un modelo económico (tarifas de gas, costos operativos reales).
  \item Implementación de algoritmos de detección automática de infracciones (visión por computador).  
        Se parte de que la cámara ya detectó la infracción y generó la evidencia.
\end{itemize}

\subsection{Entregables}
\begin{itemize}
  \item Contrato inteligente en Solidity (o «chaincode» en Go, según la red seleccionada) con pruebas unitarias.
  \item Script de despliegue de red Hyperledger Fabric e instalación de IPFS local.
  \item Aplicación Web de demostración (\emph{frontend} ligero) conectada a los servicios anteriores.
  \item Manual técnico que documenta la arquitectura y el flujo de datos.
  \item Informe de resultados de las pruebas funcionales y de rendimiento básico.
\end{itemize}

\subsection{Criterios de éxito}
\begin{enumerate}
  \item Tiempo medio de publicación de una infracción $\leq$ 3 s en entorno de laboratorio.
  \item Coincidencia 100 \% entre el hash almacenado en la cadena y la evidencia recuperada desde IPFS.
  \item Trazabilidad completa del historial de estados para al menos 50 multas de prueba.
  \item Ausencia de fallos críticos en pruebas de carga con 10 transacciones concurrentes.
\end{enumerate}

\section{Metodología }
La metodología de este proyecto se divide en dos componentes principales: la metodología de investigación y la metodología de desarrollo de software.  

\subsection{Metodología de investigación }
La investigación se clasifica de la siguiente manera:
\begin{itemize}
  \item En función de su aplicación práctica, corresponde a una investigación aplicada, pues se orienta a resolver la falta de seguridad que existe en la gestión de 25 infracciones de tránsito dentro de la Municipalidad Provincial del Cusco. De acuerdo con Sánchez (2011), la investigación aplicada o técnica se centra en ofrecer soluciones concretas o generar innovaciones y mejoras en procesos o productos.
  \item Por su propósito, el estudio es de tipo descriptivo, ya que pretende identificar y detallar las características más relevantes de la implementación de un servicio web basado en Blockchain, con el objetivo de reforzar la seguridad en la gestión de las infracciones de tránsito en dicha municipalidad. Como señala Gómez (2006), los estudios descriptivos buscan especificar las propiedades, características y aspectos significativos del fenómeno que se analiza.
\end{itemize}
\subsection{Metodología de desarrollo de software: Enfoque por Prototipos }
Para el desarrollo de este proyecto, se adoptará la Metodología de Desarrollo por Prototipos. Esta elección se fundamenta en la naturaleza innovadora del proyecto, que combina tecnologías emergentes como Blockchain e IPFS en un dominio específico (gestión de fotocomparendos), donde los requisitos exactos y los desafíos técnicos pueden no ser completamente evidentes desde el inicio. La metodología por prototipos es inherentemente iterativa y se centra en la construcción rápida de versiones funcionales (prototipos) del sistema, permitiendo la validación temprana de conceptos, la recopilación de retroalimentación continua y la adaptación flexible a los descubrimientos realizados durante el desarrollo. 