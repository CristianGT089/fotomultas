\section{Alcance}
 El ámbito de aplicación de este estudio y desarrollo de prototipo se delimita estrictamente al entorno operativo y regulatorio de la Secretaría Distrital de Movilidad de Bogotá. Esta focalización geográfica y contextual se justifica por la heterogeneidad de los sistemas de gestión de infracciones a nivel regional en Colombia, permitiendo así un análisis pertinente y detallado. Consecuentemente, el diseño y la evaluación del prototipo adoptarán la perspectiva de la entidad administradora, con énfasis en asegurar la integridad registral, la cadena de custodia de la evidencia y la auditabilidad del sistema. La arquitectura tecnológica propuesta se basa en la utilización de Hyperledger Fabric para el registro distribuido y el sistema IPFS para el almacenamiento de evidencia, cuya funcionalidad y características de rendimiento básicas serán objeto de validación mediante pruebas en un ambiente controlado, verificando la operatividad del flujo propuesto y la inmutabilidad de la información registrada. 

  \section{Metodología }
  La metodología de este proyecto se divide en dos componentes principales: la metodología de investigación y la metodología de desarrollo de software.  

    \subsection{Metodología de investigación }
 La metodología de investigación se clasifica de la siguiente manera: a) Por la forma en que la investigación es usada: El siguiente proyecto desea dar solución a los problemas que presenta la falta de seguridad en la gestión de 25 infracciones de tránsito en la Municipalidad Provincial del Cusco. Por lo tanto, es considerado como una investigación APLICADA. Siendo la definición: "La investigación Aplicada o Técnica tiende a la resolución de problemas o al desarrollo de ideas, dirigidas a conseguir innovaciones, mejoras de procesos o productos, etc." (Sanchez, 2011). b) Por el propósito del estudio: El presente proyecto realizará la identificación de las características más sobresalientes de la implementación de un Servicio Web con Blockchain, destacando los aspectos más sobresalientes para la mejora en la seguridad de gestión de infracciones de tránsito en la Municipalidad Provincial del Cusco. Por lo tanto, es considerada como DESCRIPTIVA. Siendo la definición: "Los estudios descriptivos buscan especificar las propiedades, las características y los aspectos importantes del fenómeno que se somete a análisis" (Gomez, 2006) 

   \subsection{Metodología de desarrollo de software: Enfoque por Prototipos }
Para el desarrollo de este proyecto, se adoptará la Metodología de Desarrollo por Prototipos. Esta elección se fundamenta en la naturaleza innovadora del proyecto, que combina tecnologías emergentes como Blockchain e IPFS en un dominio específico (gestión de fotocomparendos), donde los requisitos exactos y los desafíos técnicos pueden no ser completamente evidentes desde el inicio. La metodología por prototipos es inherentemente iterativa y se centra en la construcción rápida de versiones funcionales (prototipos) del sistema, permitiendo la validación temprana de conceptos, la recopilación de retroalimentación continua y la adaptación flexible a los descubrimientos realizados durante el desarrollo. 