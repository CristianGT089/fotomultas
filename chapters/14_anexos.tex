\section{Anexos}



\subsection{Anexo A: repositorios del proyecto}

\subsubsection{Enlaces a los Repositorios}

El proyecto de fotomultas blockchain está distribuido en los siguientes repositorios:

\begin{itemize}
    \item \textbf{Frontend (React + TypeScript):} \url{https://github.com/k-delta/fotomultas-front}
    \item \textbf{Backend (Smart Contracts + API):} \url{https://github.com/CristianGT089/backend-multas}
    \item \textbf{Documento LaTeX (Tesis):} \url{https://github.com/CristianGT089/fotomultas}
\end{itemize}

\paragraph{Descripción de cada repositorio:}

\begin{itemize}
    \item \textbf{fotomultas-front:} Contiene la aplicación web desarrollada en React con TypeScript, incluyendo la interfaz de usuario para agentes de tránsito y ciudadanos, así como la integración con la API del backend.
    
    \item \textbf{backend-multas:} Incluye los smart contracts de Solidity para Ethereum, la API REST desarrollada en Node.js/TypeScript, y la configuración de Hardhat para desarrollo y despliegue de contratos.
    
    \item \textbf{fotomultas:} Contiene el documento LaTeX completo de la tesis de grado, incluyendo todos los capítulos, anexos, figuras y tablas que documentan el proyecto.
\end{itemize}

\subsection{Anexo B: manual de usuario}

\subsubsection{Manual para Agentes de Tránsito}

\paragraph{1. Iniciar Sesión.}
\begin{itemize}
    \item Acceder a la URL del sistema
    \item Ingresar credenciales proporcionadas por el administrador
    \item Seleccionar rol "Agente de Tránsito"
\end{itemize}

\paragraph{2. Registrar una Multa.}
\begin{itemize}
    \item En el menú principal, seleccionar "Registrar Multa"
    \item Completar el formulario con:
    \begin{itemize}
        \item Número de placa del vehículo
        \item Tipo de infracción (seleccionar de lista desplegable)
        \item Ubicación (GPS automático o manual)
        \item Costo de la multa (calculado automáticamente según tipo)
    \end{itemize}
    \item Cargar evidencia fotográfica (máximo 5MB, formato JPG/PNG)
    \item Hacer clic en "Registrar Multa"
    \item Esperar confirmación de blockchain (aprox. 2-5 segundos)
    \item Anotar el ID de multa generado para referencia
\end{itemize}

% Espacio reservado para captura de pantalla del proceso
\begin{figure}[htbp]
    \centering
    % \includegraphics[width=0.7\textwidth]{Images/UI_Registro_Multa.png}
    \caption{Pantalla de Registro de Multa - Panel del Agente}
\end{figure}

\paragraph{3. Actualizar Estado de multa.}
\begin{itemize}
    \item Buscar multa por ID o número de placa
    \item Seleccionar "Actualizar Estado"
    \item Elegir nuevo estado (Pagada, En Apelación, etc.)
    \item Ingresar razón del cambio
    \item Confirmar actualización
\end{itemize}

\subsubsection{Manual para ciudadanos}

\paragraph{1. Consultar multas.}
\begin{itemize}
    \item Acceder a la sección pública (sin autenticación requerida)
    \item Ingresar número de placa del vehículo
    \item Hacer clic en "Buscar"
    \item Revisar lista de multas asociadas
\end{itemize}

% Espacio reservado para captura de pantalla
\begin{figure}[htbp]
    \centering
    % \includegraphics[width=0.7\textwidth]{Images/UI_Consulta_Ciudadano.png}
    \caption{Pantalla de Consulta Pública - Panel Ciudadano}
\end{figure}

\paragraph{2. Verificar Integridad de evidencia.}
\begin{itemize}
    \item Seleccionar una multa de la lista
    \item Hacer clic en "Verificar Integridad"
    \item El sistema compara el hash de la evidencia en blockchain 
          con el archivo en IPFS
    \item Se muestra resultado: "Evidencia Verificada" o 
          "Evidencia Alterada"
\end{itemize}

\paragraph{3. Presentar apelación.}
\begin{itemize}
    \item Crear cuenta en el sistema (requiere verificación de identidad)
    \item Seleccionar multa a apelar
    \item Completar formulario de apelación con argumentos
    \item Cargar evidencias de respaldo (opcional)
    \item Enviar apelación
    \item Esperar notificación de resolución (máximo 30 días hábiles)
\end{itemize}

\subsection{Anexo C: glosario de términos}

\begin{longtable}{|p{3.5cm}|p{10.5cm}|}
\caption{Glosario de Términos Técnicos} \\
\hline
\textbf{Término} & \textbf{Definición} \\ \hline
\endfirsthead

\multicolumn{2}{c}%
{{\bfseries \tablename\ \thetable{} -- continuación de la página anterior}} \\
\hline
\textbf{Término} & \textbf{Definición} \\ \hline
\endhead

\hline \multicolumn{2}{r}{{}} \\ \hline
\endfoot

\hline
\endlastfoot
ABI (Application Binary Interface) & Interfaz que define cómo llamar funciones de un Smart Contract desde aplicaciones externas. Contiene nombres de funciones, parámetros y tipos de retorno. \\ \hline
Blockchain & Tecnología de registro distribuido que almacena datos en bloques encadenados mediante hashes criptográficos, garantizando inmutabilidad. \\ \hline
CA (Certificate Authority) & Entidad que emite y gestiona certificados digitales en una red Hyperledger Fabric, controlando identidades y permisos. \\ \hline
Chaincode & Smart Contract en el contexto de Hyperledger Fabric, generalmente escrito en Go, que define la lógica de negocio. \\ \hline
CID (Content Identifier) & Hash único que identifica un archivo en IPFS. Se genera mediante criptografía del contenido del archivo. \\ \hline
Consenso & Mecanismo mediante el cual los nodos de una blockchain acuerdan la validez de las transacciones. Ejemplos: PBFT, PoS, PoW. \\ \hline
DLT (Distributed Ledger Technology) & Tecnología de libro mayor distribuido que mantiene registros sincronizados entre múltiples nodos sin autoridad central. \\ \hline
Ethers.js & Biblioteca JavaScript para interactuar con la blockchain de Ethereum, permitiendo leer datos y enviar transacciones. \\ \hline
Gas & Unidad de medida del costo computacional en Ethereum. Cada operación consume gas que se paga en Ether. \\ \hline
Hardhat & Framework de desarrollo para Ethereum que facilita compilación, testing y despliegue de Smart Contracts. \\ \hline
Hash Criptográfico & Función matemática que convierte datos de cualquier tamaño en una cadena de longitud fija. Ejemplos: SHA-256, Keccak-256. \\ \hline
Hyperledger Fabric & Plataforma de blockchain permisionada empresarial, parte del proyecto Hyperledger de Linux Foundation. \\ \hline
Inmutabilidad & Propiedad de blockchain que garantiza que datos una vez escritos no pueden ser alterados sin dejar evidencia. \\ \hline
IPFS (InterPlanetary File System) & Sistema de archivos peer-to-peer distribuido que usa direccionamiento por contenido mediante CIDs. \\ \hline
Ledger & Libro mayor que registra todas las transacciones en una blockchain. Es distribuido y sincronizado entre nodos. \\ \hline
Nodo (Node) & Computadora que participa en una red blockchain, manteniendo una copia del ledger y validando transacciones. \\ \hline
OpenZeppelin & Librería de Smart Contracts auditados y seguros para Ethereum, proporciona implementaciones estándar de tokens, control de acceso, etc. \\ \hline
Orderer & Nodo en Hyperledger Fabric que ordena transacciones y las agrupa en bloques para distribuir a los peers. \\ \hline
PBFT (Practical Byzantine Fault Tolerance) & Algoritmo de consenso tolerante a fallas bizantinas usado en Hyperledger Fabric, eficiente para redes permisionadas. \\ \hline
Peer & Nodo en Hyperledger Fabric que mantiene una copia del ledger y ejecuta chaincode. \\ \hline
Pinning & En IPFS, mantener un archivo almacenado permanentemente en un nodo para garantizar su disponibilidad. \\ \hline
PoS (Proof of Stake) & Mecanismo de consenso donde validadores son seleccionados según la cantidad de criptomoneda que poseen. \\ \hline
PoW (Proof of Work) & Mecanismo de consenso que requiere resolver acertijos criptográficos complejos para validar bloques. \\ \hline
Private Data Collections & Funcionalidad de Hyperledger Fabric para almacenar datos privados que solo ciertos nodos pueden acceder. \\ \hline
Smart Contract & Programa autoejecutante almacenado en blockchain que ejecuta lógica de negocio cuando se cumplen condiciones. \\ \hline
Solidity & Lenguaje de programación orientado a objetos para escribir Smart Contracts en Ethereum. \\ \hline
Testnet & Red de prueba de blockchain que imita el funcionamiento de la red principal pero sin valor real. Ejemplo: Sepolia. \\ \hline
Transaction Hash & Identificador único de una transacción en blockchain, generado mediante hash criptográfico de su contenido. \\ \hline
TypeScript & Superset de JavaScript con tipado estático, usado para desarrollo backend del proyecto. \\ \hline
Wallet & Software que almacena claves privadas y permite firmar transacciones en blockchain. \\ \hline
\end{longtable}
