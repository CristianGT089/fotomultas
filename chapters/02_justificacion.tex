\section{\large Justificación}
La gestión de registros públicos, como los fotocomparendos, en arquitecturas centralizadas presenta debilidades en materia de seguridad, transparencia y auditabilidad. En Bogotá, el sistema FÉNIX ilustra estos desafíos, según auditorías de la Contraloría \parencite{Informe 170100-0054-24} \parencite{Informe de Cumplimiento No. 90, 2023}, que destacan limitaciones en la integridad de los datos y una fricción operativa reflejada en más de 153.000 PQRSD en un semestre. Esta situación subraya la necesidad de modelos arquitectónicos alternativos que fortalezcan la confianza pública, independizándola de la dependencia exclusiva en procedimientos y administradores internos. Se transita así de un sistema donde la integridad se presume y se audita retrospectivamente, a uno donde es intrínseca y verificable criptográficamente desde el origen.

El propósito de este proyecto no es modificar el sistema actual, sino diseñar y evaluar un prototipo autocontenido que demuestre un modelo de confianza diferente. Para ilustrar las diferencias estructurales entre el modelo convencional y el propuesto, la Tabla \ref{tab:comparacion_modelos} compara sus características clave:

\footnotesize
\renewcommand{\arraystretch}{1.15}
\setlength{\LTpre}{10pt}
\setlength{\LTpost}{10pt}
\begin{longtable}{p{1.8cm} p{2.8cm} p{2.8cm} p{3.5cm}}

\caption{Comparación entre un modelo centralizado y un modelo descentralizado}
\label{tab:comparacion_modelos} \\
\toprule
\textbf{Característica} & \textbf{Modelo Centralizado} & \textbf{Modelo Descentralizado} & \textbf{Relevancia Contextual} \\
\midrule
\endfirsthead

\caption[]{(Continuación)} \\
\toprule
\textbf{Característica} & \textbf{Modelo Centralizado} & \textbf{Modelo Descentralizado} & \textbf{Relevancia Contextual} \\
\midrule
\endhead

\midrule
\multicolumn{4}{r}{\textit{Continúa en la siguiente página}} \\
\endfoot

\bottomrule
\multicolumn{4}{p{13cm}}{\textbf{Nota.} Elaboración propia, con hallazgos basados en la Auditoría de Cumplimiento No. 90 de la Contraloría de Bogotá D.C. (octubre de 2023) y la Auditoría de Cumplimiento 170100-0054-24.} \\
\endlastfoot

Modelo de Confianza & Basado en la confianza en los administradores del sistema y en la robustez de los controles internos definidos. & Basado en un consenso criptográfico distribuido, donde la confianza reside en el protocolo y no en un intermediario. & La correcta asignación de roles es fundamental. La auditoría observó ``ausencia de un profesional responsable de Seguridad de la Información'' (págs. 20--25), subrayando la criticidad de los factores de gobernanza. \\
\midrule
Integridad de Datos & La integridad se asegura mediante controles de acceso y logs de auditoría internos gestionados por la entidad. & La integridad es una propiedad intrínseca de la estructura de datos; los registros son inmutables por diseño. & La efectividad de los controles internos es fundamental. La auditoría documentó ``Falta de control sobre la integridad y calidad de los datos migrados'' (págs. 38--40) como punto de atención. \\
\midrule
Gestión de Seguridad & Dependiente de políticas y procedimientos de seguridad definidos y ejecutados por la institución. & La seguridad es una propiedad inherente a la capa de protocolo, auditada de forma continua y global por la comunidad. & La formalización de procedimientos es clave. La auditoría identificó ``falta de gestión formal de riesgos y controles'' y ``ausencia de un plan de seguridad para la infraestructura en la nube'' (págs. 25--30). \\
\midrule
Auditabilidad y Trazabilidad & La auditoría se realiza a través de logs internos, con acceso gestionado por la entidad y sujeto a sus políticas de retención y seguridad. & La traza de auditoría es transparente, inalterable por diseño y públicamente verificable por cualquier actor autorizado. & La consistencia de los registros internos es un factor de éxito. La auditoría observó ``retrasos y baja velocidad de desarrollo'' (págs. 15--20), subrayando la importancia de una gobernanza rigurosa. \\
\bottomrule
\end{longtable}



\subsection{Pertinencia social, tecnológica y legal}
La pertinencia de este proyecto se enmarca en tres dimensiones:
\begin{itemize}
\item \textbf{Social y ciudadana:} Ofrece un modelo alternativo que responde a la necesidad de transparencia, permitiendo la verificación independiente y empoderando al ciudadano con herramientas de auditoría directa.
\item \textbf{Tecnológica:} Demuestra cómo la integración de Blockchain (para registros inmutables) e IPFS (para evidencias con contenido direccionable) puede abordar los desafíos de seguridad y trazabilidad documentados en sistemas centralizados.
\item \textbf{Legal e institucional:} El prototipo se alinea con los principios de eficiencia y transparencia exigidos por los organismos de control, sirviendo como un caso de estudio sobre cómo la tecnología puede fortalecer la rendición de cuentas.
\end{itemize}

\subsection{Originalidad e innovación}
La innovación de esta monografía radica en la concepción del prototipo como un laboratorio para un nuevo modelo de confianza. Mientras los sistemas tradicionales se centran en controles administrativos, esta propuesta explora un modelo distribuido y resistente a la manipulación por diseño. La DApp funciona como una prueba de concepto que integra inmutabilidad, gobernanza automatizada y almacenamiento descentralizado para demostrar una solución a una clase de problemas que las bases de datos centralizadas, por su naturaleza, no pueden resolver de manera nativa.

\subsection{Impacto esperado}
El impacto del proyecto se manifiesta en varias dimensiones:
\begin{itemize}
\item \textbf{Confianza por Diseño:} Muestra cómo la verificación independiente puede fortalecer la legitimidad de los procesos públicos.
\item \textbf{Gobernanza Automatizada:} Ilustra cómo los contratos inteligentes pueden ejecutar reglas de negocio de forma predecible, reduciendo la dependencia de la supervisión humana.
\item \textbf{Escalabilidad en GovTech:} Constituye un caso de uso transferible a otros procesos que demandan alta integridad, sirviendo como un precedente para futuras innovaciones en la administración pública.
\end{itemize}

\subsection{Relación con los objetivos del proyecto} 
Este prototipo responde a una problemática documentada en Bogotá y se inserta en la tendencia global de GovTech. Por ello, la adopción de blockchain en esta propuesta no es una preferencia, sino una respuesta técnica deliberada a los desafíos de integridad y confianza inherentes a los modelos centralizados, proponiendo una arquitectura donde la veracidad es una propiedad intrínseca y verificable del sistema. 