\section{Justificación}
La gestión de registros públicos, como los fotocomparendos, en arquitecturas centralizadas presenta debilidades en materia de seguridad, transparencia y auditabilidad. En Bogotá, el sistema FÉNIX ilustra estos desafíos, según auditorías de la Contraloría \parencite{Informe170100005424} \parencite{InformeCumplimiento90}, que destacan limitaciones en la integridad de los datos y una fricción operativa reflejada en más de 153.000 PQRSD en un semestre. Esta situación subraya la necesidad de modelos arquitectónicos alternativos que fortalezcan la confianza pública, independizándola de la dependencia exclusiva en procedimientos y administradores internos. Se transita así de un sistema donde la integridad se presume y se audita retrospectivamente, a uno donde es intrínseca y verificable criptográficamente desde el origen.

El propósito de este proyecto no es modificar el sistema actual, sino diseñar y evaluar un prototipo autocontenido que demuestre un modelo de confianza diferente. Para ilustrar las diferencias estructurales entre el modelo convencional y el propuesto, la Tabla \ref{tab:comparacion_modelos} compara sus características clave:

\footnotesize
\renewcommand{\arraystretch}{1.15}
\setlength{\LTpre}{10pt}
\setlength{\LTpost}{10pt}
\begin{longtable}{p{1.8cm} p{2.8cm} p{2.8cm} p{3.5cm}}

\caption{Comparación entre un modelo centralizado y un modelo descentralizado}
\label{tab:comparacion_modelos} \\
\toprule
\textbf{Característica} & \textbf{Modelo Centralizado} & \textbf{Modelo Descentralizado} & \textbf{Relevancia Contextual} \\
\midrule
\endfirsthead

\caption[]{(Continuación)} \\
\toprule
\textbf{Característica} & \textbf{Modelo Centralizado} & \textbf{Modelo Descentralizado} & \textbf{Relevancia Contextual} \\
\midrule
\endhead

\midrule
\multicolumn{4}{r}{\textit{Continúa en la siguiente página}} \\
\endfoot

\bottomrule
\multicolumn{4}{p{13cm}}{\textbf{Nota.} Elaboración propia, con hallazgos basados en la Auditoría de Cumplimiento No. 90 de la Contraloría de Bogotá D.C. (octubre de 2023) y la Auditoría de Cumplimiento 170100-0054-24.} \\
\endlastfoot

Modelo de Confianza & Basado en la confianza en los administradores del sistema y en la robustez de los controles internos definidos. & Basado en un consenso criptográfico distribuido, donde la confianza reside en el protocolo y no en un intermediario. & La correcta asignación de roles es fundamental. La auditoría observó ``ausencia de un profesional responsable de Seguridad de la Información'' (págs. 20--25), subrayando la criticidad de los factores de gobernanza. \\
\midrule
Integridad de Datos & La integridad se asegura mediante controles de acceso y logs de auditoría internos gestionados por la entidad. & La integridad es una propiedad intrínseca de la estructura de datos; los registros son inmutables por diseño. & La efectividad de los controles internos es fundamental. La auditoría documentó ``Falta de control sobre la integridad y calidad de los datos migrados'' (págs. 38--40) como punto de atención. \\
\midrule
Gestión de Seguridad & Dependiente de políticas y procedimientos de seguridad definidos y ejecutados por la institución. & La seguridad es una propiedad inherente a la capa de protocolo, auditada de forma continua y global por la comunidad. & La formalización de procedimientos es clave. La auditoría identificó ``falta de gestión formal de riesgos y controles'' y ``ausencia de un plan de seguridad para la infraestructura en la nube'' (págs. 25--30). \\
\midrule
Auditabilidad y Trazabilidad & La auditoría se realiza a través de logs internos, con acceso gestionado por la entidad y sujeto a sus políticas de retención y seguridad. & La traza de auditoría es transparente, inalterable por diseño y públicamente verificable por cualquier actor autorizado. & La consistencia de los registros internos es un factor de éxito. La auditoría observó ``retrasos y baja velocidad de desarrollo'' (págs. 15--20), subrayando la importancia de una gobernanza rigurosa. \\
\bottomrule
\end{longtable}



\subsection{Pertinencia del proyecto}

La pertinencia de este proyecto se enmarca en tres dimensiones complementarias que justifican la necesidad de explorar arquitecturas descentralizadas para la gestión de registros públicos críticos:

\begin{itemize}
	\item \textbf{Social y ciudadana:} En un contexto donde la desconfianza en los procesos administrativos genera una tasa de impugnación del 34.1\% (más de 155.000 PQRSD semestrales según se detalla en el Estado del Arte, subsección sobre el contexto de Bogotá), este proyecto ofrece un modelo alternativo que responde a la necesidad de transparencia, permitiendo la verificación independiente y empoderando al ciudadano con herramientas de auditoría directa sobre la autenticidad de las sanciones.

	\item \textbf{Tecnológica:} Demuestra cómo la integración de \textit{blockchain} (para registros inmutables) e \textit{IPFS} (para evidencias con contenido direccionable) puede abordar los desafíos de seguridad y trazabilidad documentados en sistemas centralizados, respondiendo específicamente a las limitaciones estructurales del sistema FÉNIX identificadas por la Contraloría de Bogotá (ver análisis detallado en el Estado del Arte).

	\item \textbf{Legal e institucional:} El prototipo se alinea con los principios de eficiencia, transparencia y rendición de cuentas exigidos por los organismos de control. Frente a los incumplimientos normativos y brechas de protección de datos identificados en el sistema actual, esta propuesta sirve como caso de estudio sobre cómo garantías técnicas intrínsecas pueden fortalecer el cumplimiento normativo y reducir los riesgos de detrimento patrimonial asociados a modelos centralizados.
\end{itemize}

Para un análisis detallado del contexto operativo y legal del sistema actual de fotocomparendos en Bogotá, así como de las limitaciones críticas identificadas en FÉNIX que motivan esta propuesta, consultar la subsección correspondiente en el Estado del Arte.

\subsection{Originalidad e innovación}
La innovación de esta monografía radica en la concepción del prototipo como un laboratorio experimental para un nuevo modelo de confianza aplicado a la gestión de fotocomparendos en Bogotá. A diferencia del sistema FÉNIX actual, que depende de controles administrativos centralizados y ha generado más de 155.000 PQRSD semestrales debido a disputas por falta de transparencia, esta propuesta implementa un modelo distribuido resistente a la manipulación mediante:

\begin{itemize}
	\item \textbf{Inmutabilidad criptográfica:} Cada fotocomparendo queda registrado en blockchain con un hash único que impide alteraciones posteriores, resolviendo el problema de integridad que ha generado una tasa de impugnación del 34.1\% en el sistema actual.

	\item \textbf{Gobernanza automatizada:} Contratos inteligentes (\textit{smart contracts}) ejecutan las reglas de tránsito de manera predecible, eliminando la discrecionalidad administrativa que ha derivado en un detrimento patrimonial estimado en más de \$8.000 millones.

	\item \textbf{Almacenamiento descentralizado:} \textit{IPFS} garantiza que las evidencias fotográficas sean inalterables y accesibles sin intermediarios, abordando las vulnerabilidades de confidencialidad identificadas en auditorías de la Contraloría.
\end{itemize}

La DApp funciona como una prueba de concepto que integra estas tecnologías para demostrar una solución a problemas específicos de gestión pública que las bases de datos centralizadas no pueden resolver de manera nativa: la verificación independiente de más de 1.9 millones de comparendos emitidos entre 2018 y 2024, sin depender de la confianza en instituciones centralizadas.

\subsection{Impacto y objetivos}

Este prototipo responde directamente a la problemática documentada en el sistema de fotocomparendos de Bogotá y se posiciona como una contribución pionera al GovTech colombiano. El impacto esperado se materializa en dimensiones cuantificables que abordan los desafíos específicos identificados por la Contraloría de Bogotá:

\begin{itemize}
	\item \textbf{Confianza por Diseño:} La verificación independiente mediante \textit{blockchain} fortalece la legitimidad de los procesos públicos, permitiendo que ciudadanos y autoridades verifiquen la autenticidad de cualquier comparendo sin intermediarios. Esto cumple con el objetivo específico de garantizar integridad y autenticidad, potencialmente reduciendo la tasa de impugnación del 34.1\% actual mediante evidencia técnica irrefutable.

	\item \textbf{Gobernanza Automatizada:} Los contratos inteligentes (\textit{smart contracts}) ejecutan automáticamente las reglas del Código Nacional de Tránsito, desde la detección hasta la resolución de apelaciones, reduciendo la dependencia de supervisión humana que ha generado sobrecargas administrativas superiores a 155.000 PQRSD semestrales. Esta automatización se alinea con el objetivo de desarrollar un sistema transparente y confiable que minimice riesgos de corrupción inherentes a procesos manuales.

	\item \textbf{Escalabilidad en GovTech:} Este caso de uso de fotocomparendos es directamente transferible a otros procesos críticos como permisos de construcción, licencias ambientales o registros civiles. El prototipo establece un precedente técnico replicable que puede extenderse a más de 20 procesos administrativos identificados en el Plan Distrital de Gobierno Digital, cumpliendo con el objetivo de crear un modelo escalable para la administración pública colombiana.
\end{itemize}

La adopción de \textit{blockchain} e \textit{IPFS} en esta propuesta no representa una preferencia tecnológica, sino una respuesta técnica deliberada y fundamentada a los desafíos específicos de integridad identificados en el sistema FÉNIX: proponiendo una arquitectura donde la veracidad no se presume ni se audita retrospectivamente, sino que es una propiedad intrínseca y verificable criptográficamente desde el momento de registro de cada comparendo.