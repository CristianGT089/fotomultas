\section{\large Justificación}
La gestión actual de los fotocomparendos en Bogotá, centralizada en el sistema Fénix, presenta graves limitaciones de seguridad y transparencia que han erosionado la confianza pública en el proceso sancionatorio. Esta se manifiesta en un riesgo sistémico de corrupción, un fenómeno documentado tanto a nivel internacional, como el escándalo de «ticket-fixing» en Nueva York, como a nivel nacional, donde existen redes ilícitas que manipulan y eliminan comparendos en el sistema \parencite{barbaro2011ticketfixing, blogAletta, procuraduriaBucaramanga}.

La raíz de este problema reside en la arquitectura centralizada de las bases de datos tradicionales, que permite a personal interno modificar registros sin dejar un rastro auditable, comprometiendo la integridad de todo el proceso. Esta debilidad se ve agravada por las deficiencias en ciberseguridad identificadas en el \textit{Informe Final de Auditoría AC-SDM-090 de 2023}, que señala controles de acceso insuficientes y falta de monitoreo en los sistemas de la Secretaría de Movilidad \parencite{auditoriaSDM}. Estas falencias no solo incrementan el riesgo de ataques, sino que debilitan la capacidad institucional para defender la legitimidad de las sanciones impuestas.
Estas diferencias estructurales entre el modelo centralizado actual y la alternativa descentralizada se resumen en la Tabla~\ref{tab:comparacion_bd_blockchain}, evidenciando las ventajas de Blockchain en términos de seguridad, integridad y gobernanza de datos.

Con el fin de evidenciar las diferencias estructurales entre el modelo convencional y el prototipo propuesto, en la Tabla~\ref{tab:comparacion_modelos} se presenta una comparación detallada de sus características:

\begin{table}[htbp]
    \begin{flushleft}
        \textbf{Tabla 2}\\[2em]
        \textit{Comparación entre un modelo centralizado y un modelo descentralizado}
    \end{flushleft}
    \vspace{1em}
    \addcontentsline{lot}{table}{Tabla 2. Comparación entre un modelo centralizado y un modelo descentralizado}
    \centering
    \begin{tabular}{p{3.5cm} p{5.5cm} p{5.5cm} p{3.5cm}}
        \toprule
        \textbf{Característica} & \textbf{Modelo Centralizado} & \textbf{Modelo Descentralizado} & \textbf{Relevancia Contextual (Basado en el Caso de Estudio de la Auditoría No. 90)} \\
        \midrule
        \textbf{Modelo de Confianza} & Basado en la confianza en los administradores del sistema y en la robustez de los controles internos definidos. & Basado en un consenso criptográfico distribuido, donde la confianza reside en el protocolo y no en un intermediario. & Este modelo de confianza depende de la correcta asignación de roles. La auditoría observó que el proceso de implementación se llevó a cabo con la ``ausencia de un profesional responsable de Seguridad de la Información'' (págs. 20--25), lo que subraya la criticidad de los factores de gobernanza en este paradigma. \\
        \midrule
        \textbf{Integridad de Datos} & La integridad se asegura mediante controles de acceso y logs de auditoría internos gestionados por la entidad. & La integridad es una propiedad intrínseca de la estructura de datos; los registros son inmutables por diseño. & La efectividad de los controles internos es fundamental. La auditoría documentó un desafío en esta área, señalando la ``Falta de control sobre la integridad y calidad de los datos migrados'' (págs. 38--40) como un punto de atención. \\
        \midrule
        \textbf{Gestión de Seguridad} & Dependiente de políticas y procedimientos de seguridad definidos y ejecutados por la institución. & La seguridad es una propiedad inherente a la capa de protocolo, la cual es auditada de forma continua y global por la comunidad. & La formalización de estos procedimientos es clave. La auditoría recomendó fortalecer esta área al identificar una ``falta de gestión formal de riesgos y controles'' y la ``ausencia de un plan de seguridad para la infraestructura en la nube'' (págs. 25--30). \\
        \midrule
        \textbf{Auditabilidad y Trazabilidad} & La auditoría se realiza a través de logs internos, con acceso gestionado por la entidad y sujeto a sus políticas de retención y seguridad. & La traza de auditoría es transparente, inalterable por diseño y públicamente verificable por cualquier actor autorizado. & La consistencia de los registros internos es un factor de éxito. La auditoría señaló que la supervisión del proyecto observó ``retrasos y baja velocidad de desarrollo'' (págs. 15--20), lo que subraya la importancia de una gobernanza rigurosa para asegurar la fiabilidad de los controles. \\
        \bottomrule
    \end{tabular}
    \vspace{2em}
    \begin{flushleft}
        \textit{Fuente:} Elaboración propia, con hallazgos basados en la Auditoría de Cumplimiento No. 90 de la Contraloría de Bogotá D.C. (octubre de 2023) y la Auditoría de Cumplimiento 170100-0054-24.
    \end{flushleft}
    \refstepcounter{table}\label{tab:comparacion_modelos}
\end{table}



Frente a este escenario, la tecnología Blockchain, en conjunto con IPFS, ofrece un cambio de paradigma hacia un modelo más seguro y transparente. Como se observa en la comparación, a diferencia de un sistema centralizado donde la confianza recae en una única entidad falible, una solución Blockchain distribuye los datos en una red criptográficamente enlazada. Esto garantiza que cada registro, una vez validado, sea inmutable y verificable por todas las partes autorizadas, haciendo que cualquier intento de alteración sea computacionalmente inviable y fácilmente detectable. Se elimina así la dependencia de intermediarios y se crea una fuente única y confiable de verdad.

En síntesis, la adopción de este prototipo se justifica por su capacidad para:
\begin{itemize}
\item \textbf{Mitigar la corrupción}, al garantizar la integridad de los datos y eliminar la posibilidad de manipulación unilateral.
\item \textbf{Fortalecer la seguridad de la información}, mediante una arquitectura distribuida y tolerante a fallos.
\item \textbf{Aumentar la confianza ciudadana}, al ofrecer mecanismos transparentes y auditables para la validación de infracciones.
\item \textbf{Optimizar los procesos administrativos}, automatizando registros, auditorías y la verificación de evidencias.
\end{itemize}

Esta propuesta no solo responde a desafíos técnicos y éticos urgentes en Bogotá, sino que también se alinea con las tendencias globales en gobernanza digital (\textit{GovTech}), sentando un precedente innovador para la gestión de sanciones públicas con mayor fiabilidad y transparencia.