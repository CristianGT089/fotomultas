\section{\large Justificación}

La gestión actual de los fotocomparendos en Bogotá, articulada principalmente a través del sistema FENIX —desarrollado por la Oficina de Tecnologías de la Información y las Comunicaciones con una inversión de \$3.800 millones de pesos colombianos— presenta limitaciones críticas que afectan la transparencia, seguridad y confianza pública en el proceso sancionatorio vial \parencite{resolucionFenix}.

Uno de los principales desafíos es el riesgo constante de corrupción. Existen evidencias documentadas de prácticas ilícitas para borrar o manipular comparendos en distintas ciudades del país, como Bucaramanga, lo cual pone en entredicho la integridad de los sistemas actuales \parencite{blogAletta,procuraduriaBucaramanga}. Estas situaciones reflejan una debilidad estructural asociada al uso de bases de datos centralizadas que pueden ser alteradas por personal interno con permisos elevados.

A esto se suman las falencias en ciberseguridad señaladas en el \textit{Informe Final de Auditoría AC-SDM-090 de 2023}, en el cual se identifican graves vulnerabilidades en los sistemas informáticos de la Secretaría Distrital de Movilidad, tales como insuficientes controles de acceso, falta de monitoreo efectivo y exposición innecesaria de datos sensibles\parencite{auditoriaSDM}. Estas deficiencias no solo incrementan el riesgo de ataques externos, sino que también debilitan la confianza ciudadana y la capacidad de las autoridades para defender la legitimidad de los comparendos emitidos.

En este contexto, resulta pertinente analizar las diferencias estructurales y funcionales entre los sistemas tradicionales de gestión de información —basados en bases de datos centralizadas— y los modelos descentralizados que emplean tecnologías como Blockchain e IPFS. Las bases de datos convencionales, aunque ampliamente utilizadas, presentan limitaciones significativas en contextos donde la inmutabilidad, la trazabilidad y la resistencia a la manipulación son requisitos esenciales. Por el contrario, los sistemas distribuidos y criptográficamente seguros ofrecen un marco más robusto para garantizar la integridad y transparencia de la información.

A continuación, se presenta una comparación entre ambos enfoques tecnológicos que permite esclarecer sus ventajas y desventajas en términos de seguridad, confiabilidad y gobernanza de los datos:

\begin{table}[htbp]
    \centering
    \caption{Comparación entre bases de datos tradicionales y blockchain para gestión de registros gubernamentales}
    \begin{tabular}{p{4.5cm} p{5.2cm} p{5.2cm}}
        \toprule
        \textbf{Característica} & \textbf{Base de Datos Convencional} & \textbf{Blockchain} \\
        \midrule
        Modelo de confianza & Se basa en un administrador central (entidad de TI) & Confianza distribuida entre múltiples nodos \\
        Inmutabilidad & Registros pueden ser modificados o eliminados por administradores & Los registros son inmutables por diseño \\
        Trazabilidad / Auditoría & Depende de la implementación y control interno & Historial completo e inalterable disponible \\
        Riesgo de corrupción interna & Alto, si hay privilegios indebidos o colusión & Bajo, no se puede alterar sin consenso de la red \\
        Seguridad criptográfica & Opcional, no siempre integrada nativamente & Integrada (firmas digitales, hashes, cifrado) \\
        Disponibilidad / tolerancia a fallos & Riesgo de puntos únicos de falla & Alta disponibilidad por replicación descentralizada \\
        Velocidad de operación & Alta velocidad en lectura/escritura & Menor velocidad, prioriza integridad y consenso \\
        \bottomrule
    \end{tabular}
    \vspace{1em}
    \begin{flushleft}
        \textit{Nota.} Elaboración propia.
    \end{flushleft}
    \refstepcounter{table}\label{tab:comparacion_bd_blockchain}
\end{table}

Como se observa, la tecnología blockchain resulta especialmente útil en escenarios donde la \textbf{integridad de los datos, la resistencia a la manipulación y la auditabilidad} son esenciales, como ocurre en la administración pública y particularmente en la gestión de evidencias sancionatorias. A diferencia de una base de datos central, donde un administrador podría alterar registros sin dejar rastro, en blockchain cualquier modificación es prácticamente inviable sin el consenso de toda la red.

Esto es particularmente relevante para combatir la corrupción administrativa, ya que reduce la posibilidad de que funcionarios alteren o eliminen evidencias. A su vez, permite a ciudadanos, entes de control y entidades judiciales verificar el historial completo de cada comparendo sin necesidad de confiar ciegamente en la autoridad emisora.

En resumen, la adopción de este prototipo permitirá:

\begin{itemize}
    \item \textbf{Prevenir la corrupción}, al eliminar la intervención humana en la manipulación de datos sancionatorios.
    \item \textbf{Fortalecer la seguridad de la información}, al distribuirla en una red tolerante a fallos y ataques.
    \item \textbf{Mejorar la confianza ciudadana}, al ofrecer mecanismos transparentes y verificables para validar las infracciones.
    \item \textbf{Reducir los costos administrativos y legales}, mediante la automatización de registros, auditorías y procesos de verificación.
\end{itemize}

Además de responder a problemas técnicos y éticos actuales, esta solución se alinea con tendencias globales en gobernanza digital (\textit{GovTech}), sentando un precedente innovador para otras ciudades que enfrentan desafíos similares en la gestión de sanciones de tránsito. 