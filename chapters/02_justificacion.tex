\section{\large Justificación}
La gestión actual de los fotocomparendos en Bogotá, centralizada en el sistema FENIX, presenta graves limitaciones de seguridad y transparencia que han erosionado la confianza pública en el proceso sancionatorio. Esta vulnerabilidad no es teórica, sino que se manifiesta en un riesgo sistémico de corrupción, un fenómeno documentado tanto a nivel internacional, como el escándalo de "ticket-fixing" en Nueva York, como a nivel nacional, donde existen redes ilícitas que manipulan y eliminan comparendos en el sistema \parencite{barbaro2011ticketfixing, blogAletta, procuraduriaBucaramanga}.

La raíz de este problema reside en la arquitectura centralizada de las bases de datos tradicionales, que permite a personal interno modificar registros sin dejar un rastro auditable, comprometiendo la integridad de todo el proceso. Esta debilidad se ve agravada por las deficiencias en ciberseguridad identificadas en el \textit{Informe Final de Auditoría AC-SDM-090 de 2023}, que señala controles de acceso insuficientes y falta de monitoreo en los sistemas de la Secretaría de Movilidad \parencite{auditoriaSDM}. Estas falencias no solo incrementan el riesgo de ataques, sino que debilitan la capacidad institucional para defender la legitimidad de las sanciones impuestas.
Estas diferencias estructurales entre el modelo centralizado actual y la alternativa descentralizada se resumen en la siguiente tabla, evidenciando las ventajas de Blockchain en términos de seguridad, integridad y gobernanza de datos:

\begin{table}[htbp]
    \begin{flushleft}
        \textbf{Tabla 1}\\[2em]
        \textit{Comparación entre bases de datos tradicionales y blockchain para gestión de registros gubernamentales}
    \end{flushleft}
    \vspace{1em}
    \addcontentsline{lot}{table}{Tabla 1. Comparación entre bases de datos tradicionales y blockchain para gestión de registros gubernamentales}
    \centering
    \begin{tabular}{p{4.5cm} p{5.2cm} p{5.2cm}}
        \toprule
        \textbf{Característica} & \textbf{Base de Datos Convencional} & \textbf{Blockchain} \\
        \midrule
        Modelo de confianza & Se basa en un administrador central (entidad de TI) & Confianza distribuida entre múltiples nodos \\
        Inmutabilidad & Registros pueden ser modificados o eliminados por administradores & Los registros son inmutables por diseño \\
        Trazabilidad / Auditoría & Depende de la implementación y control interno & Historial completo e inalterable disponible \\
        Riesgo de corrupción interna & Alto, si hay privilegios indebidos o colusión & Bajo, no se puede alterar sin consenso de la red \\
        Seguridad criptográfica & Opcional, no siempre integrada nativamente & Integrada (firmas digitales, hashes, cifrado) \\
        Disponibilidad / tolerancia a fallos & Riesgo de puntos únicos de falla & Alta disponibilidad por replicación descentralizada \\
        Velocidad de operación & Alta velocidad en lectura/escritura & Menor velocidad, prioriza integridad y consenso \\
        \bottomrule
    \end{tabular}
    \vspace{2em}
    \begin{flushleft}
        \textit{Nota.} Elaboración propia.
    \end{flushleft}
    \refstepcounter{table}\label{tab:comparacion_bd_blockchain}
\end{table}

Frente a este escenario, la tecnología Blockchain, en conjunto con IPFS, ofrece un cambio de paradigma hacia un modelo más seguro y transparente. Como se observa en la comparación, a diferencia de un sistema centralizado donde la confianza recae en una única entidad falible, una solución Blockchain distribuye los datos en una red criptográficamente enlazada. Esto garantiza que cada registro, una vez validado, sea inmutable y verificable por todas las partes autorizadas, haciendo que cualquier intento de alteración sea computacionalmente inviable y fácilmente detectable. Se elimina así la dependencia de intermediarios y se crea una fuente única y confiable de verdad.

En síntesis, la adopción de este prototipo se justifica por su capacidad para:
\begin{itemize}
\item \textbf{Mitigar la corrupción}, al garantizar la integridad de los datos y eliminar la posibilidad de manipulación unilateral.
\item \textbf{Fortalecer la seguridad de la información}, mediante una arquitectura distribuida y tolerante a fallos.
\item \textbf{Aumentar la confianza ciudadana}, al ofrecer mecanismos transparentes y auditables para la validación de infracciones.
\item \textbf{Optimizar los procesos administrativos}, automatizando registros, auditorías y la verificación de evidencias.
\end{itemize}

Esta propuesta no solo responde a desafíos técnicos y éticos urgentes en Bogotá, sino que también se alinea con las tendencias globales en gobernanza digital (\textit{GovTech}), sentando un precedente innovador para la gestión de sanciones públicas con mayor fiabilidad y transparencia.