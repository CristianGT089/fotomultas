\section{Discusión y análisis}

Este capítulo presenta un análisis crítico de los resultados obtenidos, interpretando los hallazgos en el contexto de los objetivos planteados y comparándolos con el estado del arte revisado. Se discuten las implicaciones de la arquitectura híbrida implementada, sus ventajas sobre sistemas tradicionales y las lecciones aprendidas durante el desarrollo del prototipo.

El trabajo logró desarrollar exitosamente un prototipo para apoyar el registro y trazabilidad de estados en el proceso de fotocomparendos en Bogotá, aplicando tecnologías de redes distribuidas para fortalecer la integridad, la autenticidad de la información y reducir los riesgos asociados a su confidencialidad. Este objetivo fue alcanzado mediante la implementación de una arquitectura híbrida blockchain que combina Hyperledger Fabric para gestión privada de datos sensibles con Ethereum para transparencia pública. Los resultados de las pruebas de inmutabilidad (Capítulo 10) demuestran que el sistema garantiza la integridad de los registros mediante hash criptográficos inmutables en blockchain y CIDs verificables en IPFS. La autenticidad se asegura mediante firmas digitales y control de acceso basado en certificados en Hyperledger Fabric, mientras que la confidencialidad se preserva manteniendo datos sensibles exclusivamente en la capa privada.

En cuanto a la implementación de tecnología blockchain para garantizar la inmutabilidad de los registros de fotocomparendos, se desarrollaron dos blockchains complementarias: Hyperledger Fabric para registros privados completos con consenso PBFT que garantiza finalidad inmediata de transacciones, y Ethereum para metadatos públicos con consenso Proof-of-Stake que asegura inmutabilidad verificable públicamente. Las pruebas de inmutabilidad (Tabla~\ref{tab:resumen_inmutabilidad}) demostraron que cualquier intento de modificación directa del ledger es rechazado automáticamente por el mecanismo de consenso. La arquitectura de bloques encadenados mediante hashes criptográficos hace que la alteración de registros pasados sea computacionalmente prohibitiva.

Para el almacenamiento verificable de evidencias fotográficas, se implementó una arquitectura dual de IPFS que incluye un nodo privado para almacenar evidencias fotográficas completas con resolución original y un nodo público que publica hashes de evidencias para verificación ciudadana. Los resultados de las pruebas de verificación de integridad (Capítulo 10) confirmaron que el sistema de direccionamiento por contenido (CID) de IPFS detecta automáticamente cualquier alteración, incluso mínima, en las evidencias. El 100\% de las verificaciones de integridad fueron exitosas, validando que IPFS proporciona un mecanismo robusto para garantizar que las evidencias no han sido manipuladas.

El desarrollo de una API REST que integre blockchain con aplicaciones web tradicionales se materializó en un backend completo con Express.js que expone 6 endpoints principales documentados con Swagger. La API actúa como capa de abstracción entre las blockchains y el frontend, permitiendo registro de multas con upload automático a IPFS y transacción en ambas blockchains, consultas diferenciadas (privadas desde Fabric, públicas desde Ethereum), verificación de integridad cruzada entre blockchains, y sincronización transparente de datos públicos. Los tests de integración (Capítulo 10) demostraron que la API maneja correctamente tanto operaciones síncronas (consultas) como asíncronas (sincronización blockchain), cumpliendo con tiempos de respuesta aceptables (<3 segundos para registro completo).

Finalmente, se desarrolló una interfaz web intuitiva para agentes y ciudadanos implementada en React con tres módulos diferenciados: Panel de Agente para registro y gestión de multas, Panel Ciudadano para consulta pública y verificación de integridad, y Dashboard Administrativo para visualización de estadísticas con Recharts. La implementación con Tailwind CSS proporciona una interfaz moderna y responsive. El uso de Zustand para gestión de estado y React Router para navegación demuestra que las tecnologías blockchain pueden integrarse con interfaces de usuario contemporáneas sin comprometer la experiencia del usuario.

\subsection{Análisis de la arquitectura híbrida}

\subsubsection{Ventajas sobre arquitecturas monolíticas}

La arquitectura híbrida implementada presenta ventajas significativas sobre arquitecturas basadas en una sola blockchain:

\paragraph{Privacidad y transparencia simultáneas}
A diferencia de sistemas que usan únicamente blockchain pública (como el trabajo de Yousfi et al. \parencite{yousfi2019blockchain}), donde todos los datos quedan expuestos públicamente, la arquitectura híbrida permite:
\begin{itemize}
    \item Almacenar datos sensibles (identificación de conductores, notas internas) exclusivamente en Hyperledger Fabric
    \item Publicar metadatos verificables en Ethereum para consulta ciudadana
    \item Cumplir con regulaciones de protección de datos (GDPR, Ley 1581 de 2012 en Colombia)
\end{itemize}

\paragraph{Optimización de costos}
El uso de Hyperledger Fabric para operaciones frecuentes elimina los costos de gas asociados a blockchains públicas. Según las métricas de rendimiento (Tabla~\ref{tab:rendimiento}), las transacciones en Fabric tienen latencia 10x menor que en Ethereum, permitiendo mayor volumen de operaciones a menor costo.

\paragraph{Control de acceso granular}
Hyperledger Fabric permite implementar políticas de acceso basadas en certificados, algo imposible en blockchains públicas permissionless. Esto habilita escenarios donde:
\begin{itemize}
    \item Solo agentes autorizados pueden registrar multas
    \item Auditores tienen acceso de solo lectura a todos los registros
    \item Administradores pueden gestionar operadores
    \item Ciudadanos acceden solo a sus propios datos o a metadatos públicos
\end{itemize}

\subsubsection{Comparación con el sistema actual (FÉNIX)}

El sistema FÉNIX de la Secretaría Distrital de Movilidad opera bajo un modelo centralizado con base de datos tradicional. La arquitectura híbrida propuesta supera este modelo en varios aspectos críticos:

\begin{table}[h]
\centering
\small
\begin{tabular}{|p{4cm}|p{5cm}|p{5cm}|}
\hline
\textbf{Aspecto} & \textbf{Sistema FÉNIX (Actual)} & \textbf{Arquitectura Híbrida} \\ \hline
\textbf{Integridad de Datos} & Depende de controles administrativos y permisos de base de datos & Garantizada criptográficamente mediante hash inmutables en blockchain \\ \hline
\textbf{Auditabilidad} & Logs de base de datos modificables & Historial inmutable en blockchain con trazabilidad completa \\ \hline
\textbf{Transparencia} & Opaca para ciudadanos; requiere solicitudes PQRSD & Verificación pública en tiempo real sin intermediarios \\ \hline
\textbf{Puntos de Fallo} & Base de datos central (SPOF) & Distribuido entre múltiples nodos; sin SPOF \\ \hline
\textbf{Costos de Disputa} & 155,854 PQRSD semestrales & Reducción estimada >50\% por verificación automática \\ \hline
\textbf{Confianza} & Basada en la institución & Basada en criptografía verificable \\ \hline
\end{tabular}
\caption{Comparación Sistema FÉNIX vs Arquitectura Híbrida}
\label{tab:comparacion_fenix}
\end{table}

\subsubsection{Comparación con estado del arte}

Al contrastar el presente trabajo con los sistemas revisados en el Capítulo 5, se identifican las siguientes diferencias:

\paragraph{Versus Yousfi et al. (Ethereum + Smart Contracts)}
Yousfi et al. implementaron un sistema basado únicamente en Ethereum pública, enfrentando limitaciones de:
\begin{itemize}
    \item Alto costo de gas (cada transacción tiene costo variable)
    \item Privacidad limitada (todos los datos visibles en blockchain pública)
    \item Escalabilidad reducida (15-30 TPS en Ethereum)
\end{itemize}

Nuestra arquitectura híbrida resuelve estos problemas delegando operaciones frecuentes a Hyperledger Fabric (sin costos de gas, mayor privacidad, >1000 TPS) mientras mantiene la transparencia pública en Ethereum solo para metadatos.

\paragraph{Versus Chen et al. (base de datos + blockchain)}
Chen et al. propusieron un modelo híbrido de base de datos tradicional con blockchain pública, pero su aproximación no logra inmutabilidad completa ya que:
\begin{itemize}
    \item La base de datos sigue siendo mutable
    \item El enlace entre BD y blockchain es débil
    \item No hay segregación de datos públicos vs privados
\end{itemize}

Nuestro enfoque de doble blockchain (ambas inmutables) garantiza que tanto los datos privados como públicos son inmutables, con sincronización verificable entre capas.

\paragraph{Versus proyectos con solo Hyperledger Fabric}
Trabajos que utilizan únicamente Hyperledger Fabric (como registros vehiculares gubernamentales en Estonia) logran alta privacidad pero carecen de:
\begin{itemize}
    \item Verificación pública sin autenticación
    \item Transparencia hacia ciudadanos
    \item Interoperabilidad con sistemas externos
\end{itemize}

La capa pública de Ethereum en nuestra arquitectura permite que cualquier ciudadano verifique la integridad de multas sin necesidad de credenciales, algo imposible en redes permisionadas puras.

\subsection{Implicaciones del trabajo}

\subsubsection{Impacto técnico}

\paragraph{Viabilidad de arquitecturas híbridas}
Este trabajo demuestra empíricamente que es viable combinar blockchains permisionadas y públicas en un sistema de producción. El servicio de sincronización implementado valida que:
\begin{itemize}
    \item Las inconsistencias temporales son manejables mediante patrones de eventual consistency
    \item Los costos de mantenimiento de dos blockchains son justificables por los beneficios obtenidos
    \item La complejidad adicional es manejable con diseño de software apropiado
\end{itemize}

\paragraph{Replicabilidad en otros contextos}
La arquitectura es generalizable a otros escenarios gubernamentales que requieran balance entre privacidad y transparencia:
\begin{itemize}
    \item Registro civil (privacidad de datos personales + verificación pública de documentos)
    \item Contratación pública (procesos internos privados + transparencia de adjudicaciones)
    \item Historias clínicas (privacidad del paciente + verificación de autenticidad)
\end{itemize}

\subsubsection{Impacto social}

\paragraph{Reducción de corrupción y fraude}
La inmutabilidad garantizada criptográficamente elimina la posibilidad de que funcionarios alteren o eliminen multas de manera unilateral. Los casos documentados de fraude en sistemas de fotomultas (como el caso Juzto.co mencionado en la Introducción) serían detectables automáticamente mediante verificación de integridad.

\paragraph{Mejora en confianza ciudadana}
La capacidad de verificación pública sin intermediarios aborda directamente el problema de la desconfianza ciudadana. Una tasa de impugnación del 34.1\% (identificada en el Capítulo 1) sugiere que la falta de transparencia actual genera fricciones masivas. El acceso a verificación automática de integridad podría reducir significativamente las PQRSD no justificadas.

\paragraph{Reducción de carga administrativa}
Las 155,854 PQRSD procesadas semestralmente (datos del Capítulo 1) representan una carga operativa significativa. La verificación automática de integridad permitiría a los ciudadanos validar por sí mismos la autenticidad de multas, reduciendo potencialmente en más del 50\% las solicitudes relacionadas con dudas sobre la validez de los registros.

\subsubsection{Impacto económico}

\paragraph{Análisis costo-beneficio}
Según el análisis de costos del Capítulo 8, la implementación del prototipo requiere una inversión inicial moderada, pero genera ahorros en:
\begin{itemize}
    \item Procesamiento de PQRSD (reducción estimada de personal dedicado)
    \item Litigios por manipulación de registros (costos legales evitados)
    \item Auditorías manuales (automatización de verificación)
\end{itemize}

El presunto detrimento patrimonial de \$8,000 millones identificado en el sistema FÉNIX (Contraloría de Bogotá, 2024) justifica ampliamente la inversión en un sistema robusto basado en blockchain.

\paragraph{ROI estimado}
Asumiendo una reducción conservadora del 30\% en PQRSD y litigios, el retorno de inversión se proyecta en 18-24 meses de operación, considerando los costos operativos de mantener la infraestructura blockchain.

\subsection{Limitaciones y restricciones}

\subsubsection{Limitaciones Técnicas del Prototipo}

\paragraph{Entorno de Laboratorio.}
El prototipo fue validado en un entorno controlado con:
\begin{itemize}
    \item Volumen reducido de transacciones (50-100 multas de prueba)
    \item Red local de Hardhat para Ethereum (no blockchain pública real)
    \item Datos sintéticos que no reflejan la complejidad de datos reales
\end{itemize}

Una implementación en producción enfrentaría cargas de trabajo 1000x mayores (457,000 comparendos semestrales según datos de Bogotá), requiriendo optimizaciones adicionales de rendimiento y estrategias de sharding.

\paragraph{Escalabilidad de IPFS.}
La estrategia de pinning implementada (mantener todas las evidencias en nodo local) no escala indefinidamente. Para producción se requiere:
\begin{itemize}
    \item Política de retención temporal (ej. 5 años según normativa)
    \item Pinning distribuido entre múltiples nodos
    \item Estrategias de archivado para evidencias antiguas
\end{itemize}

\paragraph{Latencia de Sincronización.}
La sincronización entre Hyperledger Fabric y Ethereum no es instantánea, creando ventanas de inconsistencia temporal (1-5 segundos). Aunque este tiempo es aceptable para el caso de uso (las multas no requieren consistencia en milisegundos), podría ser limitante para aplicaciones que requieran sincronización en tiempo real.

\subsubsection{Limitaciones Metodológicas}

\paragraph{Datos Sintéticos.}
La imposibilidad de acceder a datos reales de multas (por restricciones de privacidad) limitó la validación del sistema. Los datos sintéticos generados no capturan:
\begin{itemize}
    \item Variabilidad de formatos de placas (motocicletas, taxis, diplomáticos)
    \item Casos atípicos (placas extranjeras, vehículos especiales)
    \item Inconsistencias en datos del RUNT real
\end{itemize}

\paragraph{Integración Simulada con SIMIT/RUNT.}
La simulación de APIs gubernamentales limita la validación de:
\begin{itemize}
    \item Latencias reales de servicios externos
    \item Manejo de fallas y timeouts
    \item Formatos exactos de respuestas
\end{itemize}

Para producción, se requiere piloto con integración real bajo convenio con las entidades.

\subsubsection{Limitaciones de Seguridad}

\paragraph{Ausencia de Auditoría Formal.}
No se realizaron:
\begin{itemize}
    \item Auditoría de seguridad formal del Smart Contract
    \item Pentesting de la API REST
    \item Análisis de vulnerabilidades en chaincode de Fabric
\end{itemize}

Para producción se recomienda contratar auditorías especializadas (ej. Trail of Bits, ConsenSys Diligence).

\paragraph{Gestión de Claves.}
El prototipo utiliza claves privadas almacenadas en archivos .env, lo cual no es aceptable para producción. Se requiere:
\begin{itemize}
    \item Hardware Security Modules (HSM) para claves críticas
    \item Rotación automática de credenciales
    \item Políticas de backup seguro
\end{itemize}

\subsection{Lecciones aprendidas}

\subsubsection{Complejidad de Hyperledger Fabric}

La implementación de Hyperledger Fabric presentó una curva de aprendizaje significativa. Aspectos que consumieron más tiempo:
\begin{itemize}
    \item Configuración de la red con múltiples organizaciones
    \item Generación y gestión de certificados
    \item Debugging de políticas de endorsement
    \item Configuración de Private Data Collections
\end{itemize}

\textbf{Lección:} Para equipos sin experiencia previa en Fabric, se recomienda iniciar con redes de una sola organización y agregar complejidad incrementalmente.

\subsubsection{Trade-off Descentralización vs Rendimiento}

Ethereum ofrece mayor descentralización pero menor rendimiento (15-30 TPS) comparado con Hyperledger Fabric (>1000 TPS). Para el caso de uso de fotomultas:
\begin{itemize}
    \item La mayoría de operaciones (90\%+) son internas y se benefician de Fabric
    \item Solo metadatos públicos van a Ethereum, reduciendo costos
\end{itemize}

\textbf{Lección:} La arquitectura híbrida permite optimizar el trade-off delegando cada tipo de operación a la blockchain más apropiada.

\subsubsection{Importancia de UX en Sistemas Blockchain}

La complejidad técnica de blockchain debe ser invisible para el usuario final. Aspectos críticos implementados:
\begin{itemize}
    \item No requerir que el usuario tenga wallet de Ethereum
    \item Abstraer conceptos técnicos (gas, confirmaciones, hashes) detrás de interfaces amigables
    \item Proporcionar feedback claro durante operaciones asíncronas
\end{itemize}

\textbf{Lección:} El éxito de adopción depende más de UX que de la robustez técnica de la blockchain subyacente.

\subsubsection{Diseño Modular para Evolución}

La separación clara entre capas (HyperledgerService, EthereumService, SyncService) facilitó:
\begin{itemize}
    \item Testing independiente de cada componente
    \item Actualización de dependencias sin afectar otros módulos
    \item Potencial migración a otras blockchains si fuese necesario
\end{itemize}

\textbf{Lección:} La modularidad es crítica en sistemas blockchain donde el ecosistema evoluciona rápidamente.

\subsection{Consideraciones para despliegue en producción}

Basado en las lecciones aprendidas, se identifican requisitos críticos para migrar a producción:

\subsubsection{Infraestructura}

\begin{itemize}
    \item \textbf{Hyperledger Fabric:} Mínimo 3 organizaciones con 2 peers cada una, orderer con consenso Raft (3+ nodos)
    \item \textbf{Ethereum:} Migrar de testnet a mainnet o implementar Layer 2 (Polygon, Arbitrum) para reducir costos
    \item \textbf{IPFS:} Cluster de nodos con pinning distribuido y políticas de retención
\end{itemize}

\subsubsection{Seguridad}

\begin{itemize}
    \item Auditoría formal de Smart Contracts y chaincode
    \item Implementación de HSM para claves privadas
    \item Pentesting de API REST
    \item Configuración de firewall y VPN para red Fabric
\end{itemize}

\subsubsection{Operaciones}

\begin{itemize}
    \item Monitoreo 24/7 con alertas automáticas
    \item Plan de recuperación ante desastres
    \item SLA definidos para cada componente
    \item Procedimientos de actualización sin downtime
\end{itemize}

La arquitectura híbrida implementada demuestra su viabilidad técnica y proporciona una base sólida para evolucionar hacia un sistema de producción robusto.
