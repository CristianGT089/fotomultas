\section{\large Marco Conceptual}
Este marco conceptual define los elementos tecnológicos, componentes y términos clave que constituyen el "Prototipo para la Gestión de Fotocomparendos mediante Tecnología Blockchain". Su objetivo es proporcionar una comprensión clara y precisa del "qué" de cada componente utilizado en el sistema propuesto. 

\paragraph{Blockchain / Tecnología de Ledger Distribuido (DLT)}
Una blockchain es un tipo específico de Tecnología de Ledger Distribuido (Distributed Ledger Technology, DLT), un sistema de registro digital caracterizado por ser distribuido, sincronizado y asegurado criptográficamente entre múltiples participantes \parencite{narayanan2016bitcoin}. La estructura fundamental se compone de bloques de datos que contienen transacciones validadas y un hash criptográfico que lo vincula al bloque anterior, formando una cadena cronológica. La red está mantenida por nodos que almacenan copias del ledger y ejecutan un protocolo de consenso (ej., Proof-of-Work descrito por \parencite{nakamoto2008bitcoin}, o Proof-of-Stake por \parencite{king2012ppcoin}) para validar y agregar nuevos bloques. Las características clave resultantes son la descentralización, la inmutabilidad (resistencia a la alteración de datos pasados), la transparencia configurable y la seguridad criptográfica \parencite{swan2015blockchain}. DLT es el término general, mientras que Blockchain se refiere específicamente a la estructura de bloques encadenados \parencite{ukgov2016dlt}.

\paragraph{Transacción (en Blockchain)} 

En el contexto de Blockchain, una transacción es una operación firmada digitalmente que se propaga a la red para su validación e inclusión en un bloque \parencite{antonopoulos2023mastering}. Una vez confirmada, modifica el estado del ledger de forma permanente. Para este proyecto, cada transacción crucial encapsula el hash CID de la imagen del comparendo obtenida de IPFS y los metadatos asociados (ej., fecha, lugar, placa, infracción). Funciona como el registro inmutable que vincula la prueba visual con los datos descriptivos del evento. 

\paragraph{Redes P2P (Peer-to-Peer)} 

 Una red P2P (del inglés Peer-to-Peer, o red entre pares) es un modelo de arquitectura de red en el que los participantes, denominados "pares" o "nodos", se conectan y comparten recursos directamente entre sí, sin necesidad de un servidor central que actúe como intermediario. A diferencia del modelo cliente-servidor tradicional, en una red P2P cada nodo puede actuar simultáneamente como cliente y como servidor. 

En el contexto de este proyecto, el paradigma P2P es fundamental, ya que es la base sobre la que se construyen tanto la tecnología Blockchain como el sistema IPFS. Este modelo permite la descentralización inherente al sistema, eliminando los puntos únicos de fallo y aumentando la resiliencia y la resistencia a la censura. La comunicación directa entre nodos es lo que posibilita que el ledger se mantenga sincronizado y que los archivos en IPFS puedan ser recuperados desde múltiples fuentes, garantizando la disponibilidad y la integridad de la información sin depender de una autoridad central. 
\paragraph{IPFS (InterPlanetary File System)} 

IPFS es un protocolo y red P2P diseñado para el almacenamiento y la compartición de archivos distribuida y direccionable por contenido \parencite{benet2014ipfs}. Su funcionamiento básico implica dividir archivos en bloques, calcular el hash de cada bloque y construir una estructura de datos Merkle DAG cuyo hash raíz es el CID del archivo. Para almacenar, un nodo anuncia los hashes de los bloques que posee. Para recuperar, un nodo solicita el archivo por su CID, y la red IPFS utiliza mecanismos como DHT \parencite{maymounkov2002kademlia} para localizar y obtener los bloques de los nodos pares que los poseen \parencite{benet2014ipfs}. 

\paragraph{Hash Criptográfico (y Hash Único/CID)} 

Un hash criptográfico es una salida de longitud fija generada por una función hash a partir de una entrada de datos, actuando como su huella digital única \parencite{menezes1996handbook}. Debe ser resistente a colisiones y unidireccional. Su aplicación en el proyecto es doble: En IPFS, como Content Identifier (CID), identifica unívocamente la imagen del comparendo, asegura su integridad y permite su recuperación \parencite{benet2014ipfs}. En Blockchain, se utiliza para enlazar bloques (asegurando la integridad de la cadena) y para identificar transacciones \parencite{nakamoto2008bitcoin}. 

\paragraph{Metadatos}  

Los metadatos son «datos acerca de datos» \parencite{gilliland2008setting}. En este proyecto, se refieren a la información estructurada que describe el contexto del fotocomparendo (fecha, hora, lugar, tipo de infracción, placa, etc.). Estos metadatos se registran directamente en la transacción Blockchain, junto al CID de la imagen, proporcionando un contexto inmutable y fácilmente verificable para la evidencia visual. 

\paragraph{Smart Contract (Contrato Inteligente)} 

Un Smart Contract es un programa autoejecutable almacenado en una Blockchain, cuyo código define e impone automáticamente los términos de un acuerdo o proceso cuando se cumplen condiciones predefinidas \parencite{szabo1997smart, wood2014ethereum}. Su aplicación potencial en este proyecto podría extenderse a la gestión automatizada del ciclo de vida del comparendo (actualización de estado de pago, aplicación de plazos o sanciones) o a la implementación de reglas de acceso y auditoría más complejas \parencite{buterin2014next}. 

\paragraph{Proceso de Verificación Digital (en este contexto)}  

Este proceso se refiere a los pasos técnicos para confirmar la autenticidad e integridad de un fotocomparendo usando las tecnologías del prototipo. El mecanismo implica: consultar la transacción en la Blockchain mediante un identificador, extraer el CID y los metadatos registrados, usar el CID para recuperar la imagen original de IPFS, y permitir al usuario comparar la información. La fiabilidad del proceso se basa en la inmutabilidad de la Blockchain \parencite{nakamoto2008bitcoin} y el direccionamiento por contenido de IPFS \parencite{benet2014ipfs}, que conjuntamente aseguran que tanto el registro como la evidencia visual son auténticos y no han sido alterados. 