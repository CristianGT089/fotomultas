\section{Metodología}
La metodología de este proyecto se divide en dos componentes principales: la metodología de investigación y la metodología de desarrollo de software.  

\subsection{Metodología de investigación}
La investigación se clasifica de la siguiente manera:
\begin{itemize}
  \item En función de su aplicación práctica, corresponde a una investigación aplicada, pues se orienta a resolver la falta de seguridad que existe en la gestión de infracciones de tránsito en la ciudad de Bogotá. De acuerdo con \parencite{coulouris2011}, la investigación aplicada o técnica se centra en ofrecer soluciones concretas o generar innovaciones y mejoras en procesos o productos.
  \item Por su propósito, el estudio es de tipo descriptivo, ya que pretende identificar y detallar las características más relevantes de la implementación de un servicio web basado en Blockchain, con el objetivo de reforzar la seguridad en la gestión de las infracciones de tránsito en dicha municipalidad. Como señala \parencite{vanSteen2017}, los estudios descriptivos buscan especificar las propiedades, características y aspectos significativos del fenómeno que se analiza.
\end{itemize}
\subsection{Metodología de desarrollo de software: Enfoque por Prototipos}
Para el desarrollo de este proyecto, se adoptará la Metodología de Desarrollo por Prototipos. Esta elección se fundamenta en la naturaleza innovadora del proyecto, que combina tecnologías emergentes como Blockchain e IPFS en un dominio específico (gestión de fotocomparendos), donde los requisitos exactos y los desafíos técnicos pueden no ser completamente evidentes desde el inicio. La metodología por prototipos es inherentemente iterativa y se centra en la construcción rápida de versiones funcionales (prototipos) del sistema, permitiendo la validación temprana de conceptos, la recopilación de retroalimentación continua y la adaptación flexible a los descubrimientos realizados durante el desarrollo. 
\subsection{Introducción a los artefactos técnicos del diseño}
Con el fin de estructurar de manera clara el desarrollo de la solución propuesta, en esta sección se presentan los principales artefactos utilizados durante la etapa de diseño. Estos elementos permiten representar gráficamente tanto la lógica de funcionamiento como la arquitectura del sistema, sirviendo como guía para la implementación y posterior validación del prototipo.

El conjunto de diagramas que se incluye responde a la necesidad de modelar distintos aspectos del sistema. Por un lado, se usan diagramas de casos de uso para identificar las funcionalidades clave desde la perspectiva del usuario. Por otro, los diagramas de clases permiten definir la estructura del software, mientras que los diagramas de despliegue muestran cómo se distribuyen los componentes en el entorno tecnológico. Además, se incluyen diagramas de flujo que describen el comportamiento del sistema ante eventos específicos, facilitando la comprensión de su dinámica interna.

Cada uno de estos artefactos está alineado con los objetivos del proyecto y fue elaborado considerando tanto las necesidades funcionales como las características propias de las tecnologías involucradas, en particular el uso de Blockchain e IPFS. De esta forma, se busca garantizar coherencia técnica en el diseño y establecer una base sólida para el desarrollo e implementación de la solución.
