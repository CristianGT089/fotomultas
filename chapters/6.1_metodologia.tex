\section{Metodología}
La metodología de este proyecto se divide en dos componentes principales: la metodología de investigación y la metodología de desarrollo de software.  

\subsection{Metodología de investigación}
La investigación se clasifica de la siguiente manera:
\begin{itemize}
  \item En función de su aplicación práctica, corresponde a una investigación aplicada, pues se orienta a resolver la falta de seguridad que existe en la gestión de infracciones de tránsito en la ciudad de Bogotá. De acuerdo con Sánchez (2011), la investigación aplicada o técnica se centra en ofrecer soluciones concretas o generar innovaciones y mejoras en procesos o productos.
  \item Por su propósito, el estudio es de tipo descriptivo, ya que pretende identificar y detallar las características más relevantes de la implementación de un servicio web basado en Blockchain, con el objetivo de reforzar la seguridad en la gestión de las infracciones de tránsito en dicha municipalidad. Como señala Gómez (2006), los estudios descriptivos buscan especificar las propiedades, características y aspectos significativos del fenómeno que se analiza.
\end{itemize}
\subsection{Metodología de desarrollo de software: Enfoque por Prototipos}
Para el desarrollo de este proyecto, se adoptará la Metodología de Desarrollo por Prototipos. Esta elección se fundamenta en la naturaleza innovadora del proyecto, que combina tecnologías emergentes como Blockchain e IPFS en un dominio específico (gestión de fotocomparendos), donde los requisitos exactos y los desafíos técnicos pueden no ser completamente evidentes desde el inicio. La metodología por prototipos es inherentemente iterativa y se centra en la construcción rápida de versiones funcionales (prototipos) del sistema, permitiendo la validación temprana de conceptos, la recopilación de retroalimentación continua y la adaptación flexible a los descubrimientos realizados durante el desarrollo. 