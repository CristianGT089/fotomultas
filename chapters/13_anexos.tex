\section{Anexos}



\subsection{Anexo A: repositorios del proyecto}

\subsubsection{Enlaces a los repositorios}

El proyecto de fotomultas \textit{blockchain} está distribuido en los siguientes repositorios de código fuente:

\begin{itemize}
    \item \textbf{Frontend (\textit{React} + \textit{TypeScript}):} \url{https://github.com/k-delta/fotomultas-front}
    \item \textbf{Backend (\textit{Smart Contracts} + \textit{API}):} \url{https://github.com/CristianGT089/backend-multas}
\end{itemize}

\subsubsection{Descripción técnica de cada repositorio}

\paragraph{Repositorio Frontend: fotomultas-front}
\textbf{URL:} \url{https://github.com/k-delta/fotomultas-front}

\textbf{Tecnologías principales:}
\begin{itemize}
    \item \textbf{Lenguaje:} \textit{TypeScript} (99.2\%) con configuración \textit{ESM}
    \item \textbf{Framework:} \textit{React} 18+ con \textit{Vite} como bundler
    \item \textbf{Estilos:} \textit{Tailwind CSS} para diseño responsive
    \item \textbf{Estado:} \textit{Zustand} para gestión de estado global
    \item \textbf{Testing:} \textit{Jest} con \textit{React Testing Library}
    \item \textbf{Licencia:} \textit{MIT License}
\end{itemize}

\textbf{Contenido:}
\begin{itemize}
    \item Interfaz de usuario para agentes de tránsito (registro de multas)
    \item Panel ciudadano para consulta y verificación de multas
    \item Dashboard administrativo con estadísticas y métricas
    \item Integración con \textit{API REST} del backend
    \item Componentes reutilizables y diseño modular
\end{itemize}

\paragraph{Repositorio Backend: backend-multas}
\textbf{URL:} \url{https://github.com/CristianGT089/backend-multas}

\textbf{Tecnologías principales:}
\begin{itemize}
    \item \textbf{Lenguaje:} \textit{TypeScript} (93.1\%), \textit{JavaScript} (4.1\%), \textit{Solidity} (2.8\%)
    \item \textbf{Blockchain:} \textit{Smart Contracts} en \textit{Solidity} para \textit{Ethereum}
    \item \textbf{Framework:} \textit{Express.js} con \textit{TypeScript} para \textit{API REST}
    \item \textbf{Desarrollo:} \textit{Hardhat} para compilación y despliegue de contratos
    \item \textbf{Testing:} \textit{Vitest} para pruebas unitarias e integración
    \item \textbf{Almacenamiento:} Integración con \textit{IPFS} para evidencias
\end{itemize}

\textbf{Contenido:}
\begin{itemize}
    \item \textit{Smart Contract} \texttt{FineManagement.sol} para gestión de multas
    \item \textit{API REST} con endpoints para registro, consulta y actualización
    \item Servicios de integración con \textit{IPFS} y blockchain
    \item Configuración de \textit{Hardhat} para desarrollo local y testnet
    \item Scripts de despliegue y testing automatizado
    \item Documentación \textit{Swagger} para la \textit{API}
\end{itemize}

\subsubsection{Instrucciones de acceso}

Para acceder al código fuente del proyecto:

\begin{enumerate}
    \item \textbf{Clonar repositorios:}
    \begin{verbatim}
    git clone https://github.com/CristianGT089/backend-multas
    git clone https://github.com/k-delta/fotomultas-front
    \end{verbatim}
    
    \item \textbf{Revisar documentación:} Cada repositorio incluye archivos README con instrucciones de instalación y configuración
    
    \item \textbf{Explorar código:} El código está organizado en carpetas lógicas (src/, contracts/, test/, etc.)
\end{enumerate}

\subsection{Anexo B: manual de usuario}

\subsubsection{Manual para agentes de tránsito}

\paragraph{1. Iniciar Sesión.}
\begin{itemize}
    \item Acceder a la URL del sistema
    \item Ingresar credenciales proporcionadas por el administrador
    \item Seleccionar rol "Agente de Tránsito"
\end{itemize}

\paragraph{2. Registrar una Multa.}
\begin{itemize}
    \item En el menú principal, seleccionar "Registrar Multa"
    \item Completar el formulario con:
    \begin{itemize}
        \item Número de placa del vehículo
        \item Tipo de infracción (seleccionar de lista desplegable)
        \item Ubicación (GPS automático o manual)
        \item Costo de la multa (calculado automáticamente según tipo)
    \end{itemize}
    \item Cargar evidencia fotográfica (máximo 5MB, formato JPG/PNG)
    \item Hacer clic en "Registrar Multa"
    \item Esperar confirmación de blockchain (aprox. 2-5 segundos)
    \item Anotar el ID de multa generado para referencia
\end{itemize}

% Figura de registro de multa en anexos: (eliminada del PDF)
% Se comenta la figura para que no aparezca en la lista de figuras ni en el documento final.
% \begin{figure}[htbp]
%     \centering
%     % \includegraphics[width=0.7\textwidth]{Images/UI_Registro_Multa.png}
%     \caption{Pantalla de Registro de Multa - Panel del Agente}
% \end{figure}

\paragraph{3. Actualizar Estado de multa.}
\begin{itemize}
    \item Buscar multa por ID o número de placa
    \item Seleccionar "Actualizar Estado"
    \item Elegir nuevo estado (Pagada, En Apelación, etc.)
    \item Ingresar razón del cambio
    \item Confirmar actualización
\end{itemize}

\subsubsection{Manual para ciudadanos}

\paragraph{1. Consultar multas.}
\begin{itemize}
    \item Acceder a la sección pública (sin autenticación requerida)
    \item Ingresar número de placa del vehículo
    \item Hacer clic en "Buscar"
    \item Revisar lista de multas asociadas
\end{itemize}

% Figura de consulta pública en anexos: (eliminada del PDF)
% Se comenta la figura para que no aparezca en la lista de figuras ni en el documento final.
% \begin{figure}[htbp]
%     \centering
%     % \includegraphics[width=0.7\textwidth]{Images/UI_Consulta_Ciudadano.png}
%     \caption{Pantalla de Consulta Pública - Panel Ciudadano}
% \end{figure}

\paragraph{2. Verificar Integridad de evidencia.}
\begin{itemize}
    \item Seleccionar una multa de la lista
    \item Hacer clic en "Verificar Integridad"
    \item El sistema compara el hash de la evidencia en blockchain 
          con el archivo en IPFS
    \item Se muestra resultado: "Evidencia Verificada" o 
          "Evidencia Alterada"
\end{itemize}

\paragraph{3. Presentar apelación.}
\begin{itemize}
    \item Crear cuenta en el sistema (requiere verificación de identidad)
    \item Seleccionar multa a apelar
    \item Completar formulario de apelación con argumentos
    \item Cargar evidencias de respaldo (opcional)
    \item Enviar apelación
    \item Esperar notificación de resolución (máximo 30 días hábiles)
\end{itemize}

\subsection{Anexo C: glosario de términos}

La Tabla~\ref{tab:glosario_terminos} presenta las definiciones de los principales términos técnicos utilizados en este documento.

\footnotesize
\renewcommand{\arraystretch}{1.3}
\setlength{\LTpre}{10pt}
\setlength{\LTpost}{10pt}
\begin{longtable}{p{4cm}p{9.5cm}}

\caption{Glosario de Términos Técnicos}
\label{tab:glosario_terminos} \\
\toprule
\textbf{Término} & \textbf{Definición} \\
\midrule
\endfirsthead

\caption[]{(Continuación)} \\
\toprule
\textbf{Término} & \textbf{Definición} \\
\midrule
\endhead

\midrule
\multicolumn{2}{r}{\textit{Continúa en la siguiente página}} \\
\endfoot

\bottomrule
\multicolumn{2}{l}{\textbf{Nota.} Elaboración propia.} \\
\endlastfoot

ABI (Application Binary Interface) & Interfaz que define cómo llamar funciones de un Smart Contract desde aplicaciones externas. Contiene nombres de funciones, parámetros y tipos de retorno. \\
\midrule

Blockchain & Tecnología de registro distribuido que almacena datos en bloques encadenados mediante hashes criptográficos, garantizando inmutabilidad. \\
\midrule

CA (Certificate Authority) & Entidad que emite y gestiona certificados digitales en una red Hyperledger Fabric, controlando identidades y permisos. \\
\midrule

Chaincode & Smart Contract en el contexto de Hyperledger Fabric, generalmente escrito en Go, que define la lógica de negocio. \\
\midrule

CID (Content Identifier) & Hash único que identifica un archivo en IPFS. Se genera mediante criptografía del contenido del archivo. \\
\midrule

Consenso & Mecanismo mediante el cual los nodos de una blockchain acuerdan la validez de las transacciones. Ejemplos: PBFT, PoS, PoW. \\
\midrule

DLT (Distributed Ledger Technology) & Tecnología de libro mayor distribuido que mantiene registros sincronizados entre múltiples nodos sin autoridad central. \\
\midrule

Ethers.js & Biblioteca JavaScript para interactuar con la blockchain de Ethereum, permitiendo leer datos y enviar transacciones. \\
\midrule

Gas & Unidad de medida del costo computacional en Ethereum. Cada operación consume gas que se paga en Ether. \\
\midrule

Hardhat & Framework de desarrollo para Ethereum que facilita compilación, testing y despliegue de Smart Contracts. \\
\midrule

Hash Criptográfico & Función matemática que convierte datos de cualquier tamaño en una cadena de longitud fija. Ejemplos: SHA-256, Keccak-256. \\
\midrule

Hyperledger Fabric & Plataforma de blockchain permisionada empresarial, parte del proyecto Hyperledger de Linux Foundation. \\
\midrule

Inmutabilidad & Propiedad de blockchain que garantiza que datos una vez escritos no pueden ser alterados sin dejar evidencia. \\
\midrule

IPFS (InterPlanetary File System) & Sistema de archivos peer-to-peer distribuido que usa direccionamiento por contenido mediante CIDs. \\
\midrule

Ledger & Libro mayor que registra todas las transacciones en una blockchain. Es distribuido y sincronizado entre nodos. \\
\midrule

Nodo (Node) & Computadora que participa en una red blockchain, manteniendo una copia del ledger y validando transacciones. \\
\midrule

OpenZeppelin & Librería de Smart Contracts auditados y seguros para Ethereum, proporciona implementaciones estándar de tokens, control de acceso, etc. \\
\midrule

Orderer & Nodo en Hyperledger Fabric que ordena transacciones y las agrupa en bloques para distribuir a los peers. \\
\midrule

PBFT (Practical Byzantine Fault Tolerance) & Algoritmo de consenso tolerante a fallas bizantinas usado en Hyperledger Fabric, eficiente para redes permisionadas. \\
\midrule

Peer & Nodo en Hyperledger Fabric que mantiene una copia del ledger y ejecuta chaincode. \\
\midrule

Pinning & En IPFS, mantener un archivo almacenado permanentemente en un nodo para garantizar su disponibilidad. \\
\midrule

PoS (Proof of Stake) & Mecanismo de consenso donde validadores son seleccionados según la cantidad de criptomoneda que poseen. \\
\midrule

PoW (Proof of Work) & Mecanismo de consenso que requiere resolver acertijos criptográficos complejos para validar bloques. \\
\midrule

Private Data Collections & Funcionalidad de Hyperledger Fabric para almacenar datos privados que solo ciertos nodos pueden acceder. \\
\midrule

Smart Contract & Programa autoejecutante almacenado en blockchain que ejecuta lógica de negocio cuando se cumplen condiciones. \\
\midrule

Solidity & Lenguaje de programación orientado a objetos para escribir Smart Contracts en Ethereum. \\
\midrule

Testnet & Red de prueba de blockchain que imita el funcionamiento de la red principal pero sin valor real. Ejemplo: Sepolia. \\
\midrule

Transaction Hash & Identificador único de una transacción en blockchain, generado mediante hash criptográfico de su contenido. \\
\midrule

TypeScript & Superset de JavaScript con tipado estático, usado para desarrollo backend del proyecto. \\
\midrule

Wallet & Software que almacena claves privadas y permite firmar transacciones en blockchain. \\
\bottomrule
\end{longtable}

