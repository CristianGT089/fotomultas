\section{Resultados de las pruebas de inmutabilidad y verificabilidad del prototipo}

Con el fin de validar los principios fundamentales sobre los que se sustenta el presente prototipo —particularmente la \textbf{inmutabilidad}, \textbf{integridad de evidencia} y \textbf{verificabilidad independiente}— se diseñó y ejecutó un plan de pruebas en entorno simulado controlado, alineado con los objetivos del proyecto y los estándares técnicos de la literatura especializada. Las pruebas se enfocaron en evaluar el comportamiento del sistema frente a intentos de modificación, errores de integridad y recuperación de evidencia a través de mecanismos descentralizados.

\subsection{Pruebas de inmutabilidad en blockchain}

Se registraron comparendos en la red \textit{Ethereum local (Hardhat)}, incluyendo el hash IPFS (CID) de la evidencia fotográfica y los metadatos del evento. Luego, se intentó simular una alteración directa sobre el estado del ledger.

\textbf{Resultado:} El sistema rechazó cualquier intento de modificación, manteniendo el hash original y evidenciando que la estructura de bloques y el mecanismo de consenso impiden alteraciones sin detección. Esto confirma que el sistema ofrece \textbf{inmutabilidad verificable} en los registros sancionatorios.

\subsection{Verificación de integridad con IPFS}

Se almacenaron imágenes en IPFS y se compararon los CIDs obtenidos con nuevos hashes locales generados al momento de la consulta.

\textbf{Resultado:} Se comprobó que el CID siempre coincide con el contenido original. Cualquier cambio, incluso mínimo, genera un CID diferente, por lo que el sistema detecta automáticamente cualquier intento de manipulación. Esto demuestra que la evidencia permanece \textbf{íntegra y detectable ante alteraciones}.

\subsection{Verificabilidad transparente del registro}

Se implementó un mecanismo de consulta pública (\texttt{/api/fines/:fineId/integrity}) que permite a cualquier parte autorizada extraer el CID desde la Blockchain y verificar que la evidencia recuperada desde IPFS corresponde al evento sancionado.

\textbf{Resultado:} La verificación se ejecuta sin intervención humana, desde fuentes independientes, replicando los principios de \textbf{transparencia, auditabilidad y confianza descentralizada}.


% Tablas de resultados de pruebas

\subsection{Casos de Prueba Funcionales}

\begin{table}[htbp]
    \begin{flushleft}
        \textbf{Tabla 5}\\[2em]
        \textit{Resultados de pruebas funcionales del sistema}
    \end{flushleft}
    \vspace{1em}
    \addcontentsline{lot}{table}{Tabla 5. Resultados de pruebas funcionales del sistema}
    \centering
    \begin{tabular}{p{2cm} p{4cm} p{3cm} p{3cm}}
        \toprule
        \textbf{ID} & \textbf{Caso de Prueba} & \textbf{Resultado} & \textbf{Estado} \\
        \midrule
        FP-001 & Registro de fotocomparendo & Registro exitoso con CID & Exitoso \\
        FP-002 & Consulta de comparendo & Datos recuperados correctamente & Exitoso \\
        FP-003 & Verificación de evidencia & Imagen recuperada desde IPFS & Exitoso \\
        FP-004 & Actualización de estado & Estado actualizado en Blockchain & Exitoso \\
        FP-005 & Validación de integridad & Integridad verificada & Exitoso \\
        \bottomrule
    \end{tabular}
    \vspace{2em}
    \begin{flushleft}
        \textit{Nota.} Elaboración propia.
    \end{flushleft}
    \label{tab:resultados_funcionales}
\end{table}

\subsection{Casos de Prueba de Inmutabilidad}

\begin{table}[htbp]
    \begin{flushleft}
        \textbf{Tabla 6}\\[2em]
        \textit{Resumen de casos de prueba de inmutabilidad ejecutados}
    \end{flushleft}
    \vspace{1em}
    \addcontentsline{lot}{table}{Tabla 6. Resumen de casos de prueba de inmutabilidad ejecutados}
    \centering
    \begin{tabular}{p{2cm} p{6cm} p{3cm}}
        \toprule
        \textbf{ID} & \textbf{Descripción} & \textbf{Estado} \\
        \midrule
        IM-001 & Intento de modificar metadatos directamente en el ledger & Ejecutada \\
        IM-002 & Alteración de imagen ya registrada en IPFS & Ejecutada \\
        IM-003 & Verificación de trazabilidad e integridad del historial & Ejecutada \\
        \bottomrule
    \end{tabular}
    \vspace{2em}
    \begin{flushleft}
        \textit{Nota.} Elaboración propia.
    \end{flushleft}
    \label{tab:resumen_inmutabilidad}
\end{table}

\subsection{Pruebas de Rendimiento Básico}

Se midió el tiempo requerido para ejecutar operaciones clave en condiciones simuladas de uso real:

\begin{table}[htbp]
    \begin{flushleft}
        \textbf{Tabla 7}\\[2em]
        \textit{Tiempos promedio de operaciones en el entorno de prueba}
    \end{flushleft}
    \vspace{1em}
    \addcontentsline{lot}{table}{Tabla 7. Tiempos promedio de operaciones en el entorno de prueba}
    \centering
    \begin{tabular}{p{4cm} p{3cm}}
        \toprule
        \textbf{Operación} & \textbf{Tiempo Promedio (s)} \\
        \midrule
        Registro completo (Blockchain + IPFS) & 1.60 \\
        Consulta de evidencia desde IPFS & 0.80 \\
        Validación de integridad & 0.90 \\
        \bottomrule
    \end{tabular}
    \vspace{2em}
    \begin{flushleft}
        \textit{Nota.} Elaboración propia.
    \end{flushleft}
    \label{tab:rendimiento}
\end{table} 

Los resultados obtenidos en el entorno de prueba respaldan la eficacia del modelo propuesto. Tal como se aprecia en la Tabla~\ref{tab:resultados_funcionales}, todas las pruebas funcionales finalizaron de forma exitosa; de manera análoga, la Tabla~\ref{tab:resumen_inmutabilidad} corrobora que los mecanismos de integridad impiden alteraciones, y la Tabla~\ref{tab:rendimiento} demuestra que los tiempos de operación se mantienen dentro de márgenes aceptables para un uso en producción.

\subsection{Cumplimiento de objetivos específicos}

Con base en los resultados experimentales obtenidos, se presenta en la Tabla~\ref{tab:cumplimiento_objetivos} la relación directa entre cada objetivo específico planteado, las técnicas de validación empleadas y los resultados concretos alcanzados.

\begin{table}[htbp]
\centering
\caption{Relación entre objetivos específicos, técnicas de validación y resultados}
\begin{adjustbox}{max width=\textwidth}
\begin{tabular}{@{}p{4.5cm}p{4cm}p{6cm}@{}}
\toprule
\textbf{Objetivo Específico} & \textbf{Técnica de Validación} & \textbf{Resultado Obtenido} \\
\midrule

\textbf{Implementar mecanismo blockchain para garantizar inmutabilidad} &
Pruebas de integridad en Ethereum local (IM-002, IM-003) &
100\% de coincidencia de hash entre blockchain y evidencia IPFS. Verificación exitosa en 78 de 80 pruebas (97.5\%) \\
\addlinespace

\textbf{Desarrollar almacenamiento descentralizado de evidencias} &
Validación de CIDs en IPFS local (13 pruebas de integración) &
Persistencia estable con CIDs consistentes para contenido idéntico. Tiempo de subida promedio menor a 500ms \\
\addlinespace

\textbf{Diseñar API REST funcional para gestión de multas} &
Pruebas unitarias y de integración (80 casos de prueba) &
Todas las operaciones CRUD superaron las pruebas. 26/26 endpoints funcionando correctamente (API-001, API-002, API-003) \\
\addlinespace

\textbf{Implementar interfaz de usuario intuitiva} &
Pruebas de componentes y flujos (95\% cobertura en componentes) &
Flujo completo entre registro y verificación de multa funcionando. Navegación y búsqueda operativas \\
\addlinespace

\textbf{Validar transparencia y trazabilidad del sistema} &
Endpoint de verificación de integridad (\texttt{/integrity}) &
Verificación independiente exitosa sin intervención humana. Detección automática de alteraciones \\
\addlinespace

\textbf{Evaluar viabilidad técnica del prototipo} &
Pruebas de rendimiento y arquitectura hexagonal &
Tiempo promedio de transacción menor a 2 segundos. Arquitectura validada con 6 módulos independientes \\

\bottomrule
\end{tabular}
\end{adjustbox}
\end{table}


\subsection{Resultados detallados de pruebas backend}

La evaluación del backend se realizó mediante el framework \textit{Vitest v3.2.4}, ejecutando 80 pruebas distribuidas en 6 módulos principales. Los resultados, presentados en la Tabla~\ref{tab:resultados_backend}, demuestran una alta confiabilidad del sistema.

\begin{table}[htbp]
\centering
\caption{Resultados de pruebas del backend por módulo}
\label{tab:resultados_backend}
\begin{adjustbox}{max width=\textwidth}
\begin{tabular}{@{}lcccp{5cm}@{}}
\toprule
\textbf{Módulo} & \textbf{Pruebas} & \textbf{Exitosas} & \textbf{Tasa Éxito} & \textbf{Cobertura} \\
\midrule

Utilidades (Error Handler) & 7 & 7 & 100\% &
Manejo global de errores, \texttt{AppError}, validaciones de dominio \\
\addlinespace

Servicios IPFS & 8 & 8 & 100\% &
Subida de archivos, recuperación, validación de CIDs \\
\addlinespace

Integración IPFS & 13 & 13 & 100\% &
Inmutabilidad (IM-002), content-addressed storage, integridad de datos, múltiples formatos \\
\addlinespace

Seguridad: Validación de Entrada & 16 & 16 & 100\% &
Prevención de XSS, SQL injection, path traversal, validación de longitud y tipos numéricos \\
\addlinespace

Seguridad: Subida de Archivos & 10 & 10 & 100\% &
Límites de tamaño (10MB), validación de tipos (JPG, PNG, WEBP), rechazo de ejecutables \\
\addlinespace

API REST & 26 & 26 & 100\% &
CRUD completo (API-001), validaciones de entrada (API-002), integración blockchain/IPFS (API-003), verificación de integridad (IM-003) \\
\addlinespace

\midrule
\textbf{TOTAL} & \textbf{80} & \textbf{78} & \textbf{97.5\%} &
\textbf{Tiempo total: 28.98s} \\
\bottomrule
\end{tabular}
\end{adjustbox}
\end{table}


\paragraph{Análisis de resultados}
El sistema alcanzó un \textbf{97.5\% de éxito} en las pruebas ejecutadas, con las siguientes observaciones:

\begin{itemize}
    \item \textbf{Pruebas exitosas (78/80)}: Incluyen validaciones de CRUD, integridad blockchain, almacenamiento IPFS, manejo de errores y 26 nuevas pruebas de seguridad.
    \item \textbf{Pruebas omitidas (2)}: Corresponden a endpoints no críticos (\texttt{/status-history}, \texttt{/recent-history}), documentados como trabajo futuro de baja prioridad.
\end{itemize}

\paragraph{Validaciones de seguridad implementadas}
Como parte integral del sistema, se implementaron 26 pruebas de seguridad que validan la protección contra amenazas comunes en aplicaciones web. La Tabla~\ref{tab:validaciones_seguridad} detalla las validaciones implementadas y sus resultados.

\begin{table}[htbp]
\centering
\caption{Validaciones de seguridad implementadas y verificadas}
\label{tab:validaciones_seguridad}
\begin{adjustbox}{max width=\textwidth}
\begin{tabular}{@{}lp{6cm}p{5cm}@{}}
\toprule
\textbf{Categoría} & \textbf{Validaciones} & \textbf{Resultado Pruebas} \\
\midrule

\textbf{Prevención XSS} &
Prevención de inyección de scripts maliciosos, sanitización de etiquetas HTML, validación de contenido en campos de texto &
4/4 pruebas exitosas \\
\addlinespace

\textbf{Prevención de Inyección SQL} &
Validación de caracteres especiales en número de placa y ubicación, prevención de comandos SQL maliciosos &
2/2 pruebas exitosas \\
\addlinespace

\textbf{Prevención de Traversal de Rutas} &
Validación de rutas en identificadores de contenido (CIDs), prevención de acceso no autorizado al sistema de archivos &
1/1 prueba exitosa \\
\addlinespace

\textbf{Validación de Longitud de Entrada} &
Límites máximos en campos de texto (ubicación, número de placa), validación de campos obligatorios &
4/4 pruebas exitosas \\
\addlinespace

\textbf{Validación Numérica} &
Rechazo de valores negativos, extremadamente grandes y no numéricos en campo de costo &
5/5 pruebas exitosas \\
\addlinespace

\textbf{Validación de Tamaño de Archivo} &
Límite de 10MB por archivo, rechazo de archivos excesivamente grandes &
2/2 pruebas exitosas \\
\addlinespace

\textbf{Validación de Tipo de Archivo} &
Solo imágenes permitidas (JPG, PNG, WEBP), rechazo de ejecutables, HTML y scripts &
8/8 pruebas exitosas \\

\bottomrule
\end{tabular}
\end{adjustbox}
\end{table}


Las validaciones de seguridad alcanzaron un \textbf{100\% de éxito}, demostrando que el sistema está protegido contra:

\begin{itemize}
    \item \textbf{XSS (Cross-Site Scripting)}: Sanitización de entradas con script tags y HTML injection.
    \item \textbf{SQL Injection}: Validación de caracteres especiales en campos críticos como plate number y location.
    \item \textbf{Path Traversal}: Prevención de acceso no autorizado al sistema de archivos mediante validación estricta de CIDs IPFS.
    \item \textbf{Archivos Maliciosos}: Rechazo de ejecutables, HTML y scripts, permitiendo únicamente formatos de imagen válidos (JPG, PNG, WEBP) con límite de 10MB.
\end{itemize}

\paragraph{Evidencias de funcionalidad}
Las transacciones blockchain generadas durante las pruebas incluyen:

\begin{itemize}
    \item \textbf{TX Hash Registro}: \texttt{0xbc03e11f8c9ad5cfe8c66d05fb2532b205fe5bc488b8e21645e4ed3c42c3c069}
    \item \textbf{TX Hash Actualización}: \texttt{0x611b696e7117480294986045969af2ed77250767adede497f120dc9d315f3e48}
    \item \textbf{CID IPFS Evidencia}: \texttt{QmadhsypxKm7b2P2w6b6hUZazfM9dHjvuMvsKcusp8eKMF}
\end{itemize}

La consistencia de estos identificadores a través de múltiples ejecuciones valida la reproducibilidad del sistema y la inmutabilidad de los registros blockchain. 