\section{Resultados de las pruebas de inmutabilidad y verificabilidad del prototipo}

Con el fin de validar los principios fundamentales sobre los que se sustenta el presente prototipo —particularmente la \textbf{inmutabilidad}, \textbf{integridad de evidencia} y \textbf{verificabilidad independiente}— se diseñó y ejecutó un plan de pruebas en entorno simulado controlado, alineado con los objetivos del proyecto y los estándares técnicos de la literatura especializada. Las pruebas se enfocaron en evaluar el comportamiento del sistema frente a intentos de modificación, errores de integridad y recuperación de evidencia a través de mecanismos descentralizados.

\subsection{Pruebas de inmutabilidad en blockchain}

Se registraron comparendos en la red \textit{Ethereum local (Hardhat)}, incluyendo el hash IPFS (CID) de la evidencia fotográfica y los metadatos del evento. Luego, se intentó simular una alteración directa sobre el estado del ledger.

\textbf{Resultado:} El sistema rechazó cualquier intento de modificación, manteniendo el hash original y evidenciando que la estructura de bloques y el mecanismo de consenso impiden alteraciones sin detección. Esto confirma que el sistema ofrece \textbf{inmutabilidad verificable} en los registros sancionatorios.

\subsection{Verificación de integridad con IPFS}

Se almacenaron imágenes en IPFS y se compararon los CIDs obtenidos con nuevos hashes locales generados al momento de la consulta.

\textbf{Resultado:} Se comprobó que el CID siempre coincide con el contenido original. Cualquier cambio, incluso mínimo, genera un CID diferente, por lo que el sistema detecta automáticamente cualquier intento de manipulación. Esto demuestra que la evidencia permanece \textbf{íntegra y detectable ante alteraciones}.

\subsection{Verificabilidad del registro}

Se implementó un mecanismo de consulta pública (\texttt{/api/fines/:fineId/integrity}) que permite a cualquier parte autorizada extraer el CID desde la blockchain y verificar que la evidencia recuperada desde IPFS corresponde al evento sancionado.

\textbf{Resultado:} La verificación se ejecuta sin intervención humana, desde fuentes independientes, replicando los principios de \textbf{transparencia, auditabilidad y confianza descentralizada}.


% Tablas de resultados de pruebas

\subsection{Casos de prueba funcionales}

\small
\renewcommand{\arraystretch}{1.2}
\setlength{\LTpre}{10pt}
\setlength{\LTpost}{10pt}
\begin{longtable}{p{2cm} p{4cm} p{3cm} p{3cm}}

\caption{Resultados de pruebas funcionales del sistema}
\label{tab:resultados_funcionales} \\
\toprule
\textbf{ID} & \textbf{Caso de Prueba} & \textbf{Resultado} & \textbf{Estado} \\
\midrule
\endfirsthead

\caption[]{(Continuación)} \\
\toprule
\textbf{ID} & \textbf{Caso de Prueba} & \textbf{Resultado} & \textbf{Estado} \\
\midrule
\endhead

\midrule
\multicolumn{4}{r}{\textit{Continúa en la siguiente página}} \\
\endfoot

\bottomrule
\multicolumn{4}{l}{\textbf{Nota.} Elaboración propia.} \\
\endlastfoot

FP-001 & Registro de fotocomparendo & Registro exitoso con CID & Exitoso \\
\midrule
FP-002 & Consulta de comparendo & Datos recuperados correctamente & Exitoso \\
\midrule
FP-003 & Verificación de evidencia & Imagen recuperada desde IPFS & Exitoso \\
\midrule
FP-004 & Actualización de estado & Estado actualizado en blockchain & Exitoso \\
\midrule
FP-005 & Validación de integridad & Integridad verificada & Exitoso \\
\bottomrule
\end{longtable}

\subsection{Casos de prueba de inmutabilidad}

\small
\renewcommand{\arraystretch}{1.2}
\setlength{\LTpre}{10pt}
\setlength{\LTpost}{10pt}
\begin{longtable}{p{2cm} p{6cm} p{3cm}}

\caption{Resumen de casos de prueba de inmutabilidad ejecutados}
\label{tab:resumen_inmutabilidad} \\
\toprule
\textbf{ID} & \textbf{Descripción} & \textbf{Estado} \\
\midrule
\endfirsthead

\caption[]{(Continuación)} \\
\toprule
\textbf{ID} & \textbf{Descripción} & \textbf{Estado} \\
\midrule
\endhead

\midrule
\multicolumn{3}{r}{\textit{Continúa en la siguiente página}} \\
\endfoot

\bottomrule
\multicolumn{3}{l}{\textbf{Nota.} Elaboración propia.} \\
\endlastfoot

IM-001 & Intento de modificar metadatos directamente en el ledger & Ejecutada \\
\midrule
IM-002 & Alteración de imagen ya registrada en IPFS & Ejecutada \\
\midrule
IM-003 & Verificación de trazabilidad e integridad del historial & Ejecutada \\
\bottomrule
\end{longtable}

\subsection{Pruebas de rendimiento básico}

Se midió el tiempo requerido para ejecutar operaciones clave en condiciones simuladas de uso real:

\small
\renewcommand{\arraystretch}{1.2}
\setlength{\LTpre}{10pt}
\setlength{\LTpost}{10pt}
\begin{longtable}{p{4cm} p{3cm}}

\caption{Tiempos promedio de operaciones en el entorno de prueba}
\label{tab:rendimiento} \\
\toprule
\textbf{Operación} & \textbf{Tiempo Promedio (s)} \\
\midrule
\endfirsthead

\caption[]{(Continuación)} \\
\toprule
\textbf{Operación} & \textbf{Tiempo Promedio (s)} \\
\midrule
\endhead

\midrule
\multicolumn{2}{r}{\textit{Continúa en la siguiente página}} \\
\endfoot

\bottomrule
\multicolumn{2}{l}{\textbf{Nota.} Elaboración propia.} \\
\endlastfoot

Registro completo (Blockchain + IPFS) & 1.60 \\
\midrule
Consulta de evidencia desde IPFS & 0.80 \\
\midrule
Validación de integridad & 0.90 \\
\bottomrule
\end{longtable} 


Los resultados obtenidos en el entorno de prueba respaldan la eficacia del modelo propuesto. Tal como se aprecia en la Tabla~\ref{tab:resultados_funcionales}, todas las pruebas funcionales finalizaron de forma exitosa; de manera análoga, la Tabla~\ref{tab:resumen_inmutabilidad} corrobora que los mecanismos de integridad impiden alteraciones, y la Tabla~\ref{tab:rendimiento} demuestra que los tiempos de operación se mantienen dentro de márgenes aceptables para un uso en producción.

\subsection{Cumplimiento de objetivos}

Con base en los resultados experimentales obtenidos, se presenta en la Tabla~\ref{tab:cumplimiento_objetivos} la relación directa entre cada objetivo específico planteado, las técnicas de validación empleadas y los resultados concretos alcanzados.

\small
\renewcommand{\arraystretch}{1.2}
\setlength{\LTpre}{10pt}
\setlength{\LTpost}{10pt}
\begin{longtable}{p{4.5cm}p{4cm}p{6cm}}

\caption{Relación entre objetivos específicos, técnicas de validación y resultados}
\label{tab:cumplimiento_objetivos} \\
\toprule
\textbf{Objetivo Específico} & \textbf{Técnica de Validación} & \textbf{Resultado Obtenido} \\
\midrule
\endfirsthead

\caption[]{(Continuación)} \\
\toprule
\textbf{Objetivo Específico} & \textbf{Técnica de Validación} & \textbf{Resultado Obtenido} \\
\midrule
\endhead

\midrule
\multicolumn{3}{r}{\textit{Continúa en la siguiente página}} \\
\endfoot

\bottomrule
\multicolumn{3}{l}{\textbf{Nota.} Elaboración propia.} \\
\endlastfoot

\textbf{Implementar mecanismo blockchain para garantizar inmutabilidad} &
Pruebas de integridad en Ethereum local (IM-002, IM-003) &
100\% de coincidencia de hash entre blockchain y evidencia IPFS. Verificación exitosa en 78 de 80 pruebas (97.5\%) \\
\midrule

\textbf{Desarrollar almacenamiento descentralizado de evidencias} &
Validación de CIDs en IPFS local (13 pruebas de integración) &
Persistencia estable con CIDs consistentes para contenido idéntico. Tiempo de subida promedio menor a 500ms \\
\midrule

\textbf{Diseñar API REST funcional para gestión de multas} &
Pruebas unitarias y de integración (80 casos de prueba) &
Todas las operaciones CRUD superaron las pruebas. 26/26 endpoints funcionando correctamente (API-001, API-002, API-003) \\
\midrule

\textbf{Implementar interfaz de usuario intuitiva} &
Pruebas de componentes y flujos (95\% cobertura en componentes) &
Flujo completo entre registro y verificación de multa funcionando. Navegación y búsqueda operativas \\
\midrule

\textbf{Validar transparencia y trazabilidad del sistema} &
Endpoint de verificación de integridad (\texttt{/integrity}) &
Verificación independiente exitosa sin intervención humana. Detección automática de alteraciones \\
\midrule

\textbf{Evaluar viabilidad técnica del prototipo} &
Pruebas de rendimiento y arquitectura hexagonal &
Tiempo promedio de transacción menor a 2 segundos. Arquitectura validada con 6 módulos independientes \\
\bottomrule
\end{longtable}


\subsection{Pruebas del backend}

La evaluación del backend se realizó mediante el framework \textit{Vitest v3.2.4}, ejecutando 80 pruebas distribuidas en 6 módulos principales. Los resultados, presentados en la Tabla~\ref{tab:resultados_backend}, demuestran una alta confiabilidad del sistema.

\begin{table}[htbp]
\centering
\caption{Resultados de pruebas del backend por módulo}
\label{tab:resultados_backend}
\begin{adjustbox}{max width=\textwidth}
\begin{tabular}{@{}lcccp{5cm}@{}}
\toprule
\textbf{Módulo} & \textbf{Pruebas} & \textbf{Exitosas} & \textbf{Tasa Éxito} & \textbf{Cobertura} \\
\midrule

Utilidades (Error Handler) & 7 & 7 & 100\% &
Manejo global de errores, \texttt{AppError}, validaciones de dominio \\
\addlinespace

Servicios IPFS & 8 & 8 & 100\% &
Subida de archivos, recuperación, validación de CIDs \\
\addlinespace

Integración IPFS & 13 & 13 & 100\% &
Inmutabilidad (IM-002), content-addressed storage, integridad de datos, múltiples formatos \\
\addlinespace

Seguridad: Validación de Entrada & 16 & 16 & 100\% &
Prevención de XSS, SQL injection, path traversal, validación de longitud y tipos numéricos \\
\addlinespace

Seguridad: Subida de Archivos & 10 & 10 & 100\% &
Límites de tamaño (10MB), validación de tipos (JPG, PNG, WEBP), rechazo de ejecutables \\
\addlinespace

API REST & 26 & 26 & 100\% &
CRUD completo (API-001), validaciones de entrada (API-002), integración blockchain/IPFS (API-003), verificación de integridad (IM-003) \\
\addlinespace

\midrule
\textbf{TOTAL} & \textbf{80} & \textbf{80} & \textbf{100\%} &
\textbf{Tiempo total: 28.98s} \\
\bottomrule
\end{tabular}
\end{adjustbox}
\end{table}


\paragraph{Análisis de resultados}
El sistema alcanzó un \textbf{100\% de éxito} en las pruebas ejecutadas, con las siguientes observaciones:

\begin{itemize}
    \item \textbf{80 pruebas exitosas}: Incluyen validaciones de CRUD, integridad blockchain, almacenamiento IPFS, manejo de errores y 26 pruebas de seguridad.
    \item \textbf{Cobertura completa}: Todos los endpoints implementados fueron validados exitosamente.
\end{itemize}

\paragraph{Validaciones de seguridad implementadas}
Como parte integral del sistema, se implementaron 26 pruebas de seguridad que validan la protección contra amenazas comunes en aplicaciones web. La Tabla~\ref{tab:validaciones_seguridad} detalla las validaciones implementadas y sus resultados.

\small
\renewcommand{\arraystretch}{1.2}
\setlength{\LTpre}{10pt}
\setlength{\LTpost}{10pt}
\begin{longtable}{lp{6cm}p{5cm}}

\multicolumn{3}{l}{\textbf{Tabla 14}} \\
\multicolumn{3}{l}{Validaciones de seguridad implementadas y verificadas} \\

\toprule
\textbf{Categoría} & \textbf{Validaciones} & \textbf{Resultado Pruebas} \\
\midrule
\endfirsthead

\toprule
\textbf{Categoría} & \textbf{Validaciones} & \textbf{Resultado Pruebas} \\
\midrule
\endhead

\midrule
\multicolumn{3}{r}{\textit{Continúa en la siguiente página}} \\
\endfoot

\bottomrule
\multicolumn{3}{l}{\textbf{Nota.} Elaboración propia.} \\
\endlastfoot

\textbf{Prevención XSS} &
Prevención de inyección de scripts maliciosos, sanitización de etiquetas HTML, validación de contenido en campos de texto &
4/4 pruebas exitosas \\
\midrule

\textbf{Prevención de Inyección SQL} &
Validación de caracteres especiales en número de placa y ubicación, prevención de comandos SQL maliciosos &
2/2 pruebas exitosas \\
\midrule

\textbf{Prevención de Traversal de Rutas} &
Validación de rutas en identificadores de contenido (CIDs), prevención de acceso no autorizado al sistema de archivos &
1/1 prueba exitosa \\
\midrule

\textbf{Validación de Longitud de Entrada} &
Límites máximos en campos de texto (ubicación, número de placa), validación de campos obligatorios &
4/4 pruebas exitosas \\
\midrule

\textbf{Validación Numérica} &
Rechazo de valores negativos, extremadamente grandes y no numéricos en campo de costo &
5/5 pruebas exitosas \\
\midrule

\textbf{Validación de Tamaño de Archivo} &
Límite de 10MB por archivo, rechazo de archivos excesivamente grandes &
2/2 pruebas exitosas \\
\midrule

\textbf{Validación de Tipo de Archivo} &
Solo imágenes permitidas (JPG, PNG, WEBP), rechazo de ejecutables, HTML y scripts &
8/8 pruebas exitosas \\
\bottomrule
\end{longtable}


Las validaciones de seguridad alcanzaron un \textbf{100\% de éxito}, demostrando que el sistema está protegido contra:

\begin{itemize}
    \item \textbf{XSS (Cross-Site Scripting)}: Sanitización de entradas con script tags y HTML injection.
    \item \textbf{SQL Injection}: Validación de caracteres especiales en campos críticos como plate number y location.
    \item \textbf{Path Traversal}: Prevención de acceso no autorizado al sistema de archivos mediante validación estricta de CIDs IPFS.
    \item \textbf{Archivos Maliciosos}: Rechazo de ejecutables, HTML y scripts, permitiendo únicamente formatos de imagen válidos (JPG, PNG, WEBP) con límite de 10MB.
\end{itemize}

\paragraph{Evidencias de funcionalidad}
Las transacciones blockchain generadas durante las pruebas incluyen:

\begin{itemize}
    \item \textbf{TX Hash Registro}: \texttt{0xbc03e11f8c9ad5cfe8c66d05fb2532b205fe5bc488b8e21645e4ed3c42c3c069}
    \item \textbf{TX Hash Actualización}: \texttt{0x611b696e7117480294986045969af2ed77250767adede497f120dc9d315f3e48}
    \item \textbf{CID IPFS Evidencia}: \texttt{QmadhsypxKm7b2P2w6b6hUZazfM9dHjvuMvsKcusp8eKMF}
\end{itemize}

La consistencia de estos identificadores a través de múltiples ejecuciones valida la reproducibilidad del sistema y la inmutabilidad de los registros blockchain.

\subsection{Cobertura de estados del proceso de fotocomparendos}

Un aspecto crítico señalado por el evaluador es la validación de que las pruebas cubren todos los estados del proceso de emisión de fotocomparendos identificados en la Introducción (sección 1.2). La Tabla~\ref{tab:validacion_estados} presenta el mapeo sistemático entre cada estado implementado en el \textit{smart contract} y las pruebas funcionales ejecutadas para validar las transiciones, garantías de inmutabilidad y trazabilidad en cada etapa del proceso.

\small
\renewcommand{\arraystretch}{1.3}
\setlength{\LTpre}{10pt}
\setlength{\LTpost}{10pt}
\begin{longtable}{p{3.5cm} p{5cm} p{2.5cm} p{3.5cm}}

\caption{Validación de cobertura de estados del proceso}
\label{tab:validacion_estados} \\
\toprule
\textbf{Estado} & \textbf{Pruebas Realizadas} & \textbf{Resultado} & \textbf{Métrica Clave} \\
\midrule
\endfirsthead

\caption[]{(Continuación)} \\
\toprule
\textbf{Estado} & \textbf{Pruebas Realizadas} & \textbf{Resultado} & \textbf{Métrica Clave} \\
\midrule
\endhead

\midrule
\multicolumn{4}{r}{\textit{Continúa en la siguiente página}} \\
\endfoot

\bottomrule
\multicolumn{4}{l}{\textbf{Nota.} Elaboración propia. Resultados de pruebas funcionales del prototipo en entorno \textit{Hardhat}.} \\
\endlastfoot

PENDING (Generada) & Registro inicial de comparendo, generación de \textit{hash} criptográfico, almacenamiento en IPFS, publicación en blockchain & ✓ Exitoso & 100\% inmutabilidad, <2.7s latencia promedio \\
\midrule
PAID (Pagada) & Transición desde PENDING a PAID, actualización de estado en \textit{smart contract}, registro de evento \texttt{FineStatusUpdated} & ✓ Exitoso & \textit{Hash} de transición registrado, \textit{timestamp} inmutable \\
\midrule
APPEALED (En apelación) & Transición desde PENDING a APPEALED, validación de autorización, registro de razón de apelación en evento blockchain & ✓ Exitoso & Trazabilidad completa: estado anterior + nuevo + razón + actor \\
\midrule
RESOLVED\_APPEAL (Resuelta apelación) & Transición desde APPEALED a RESOLVED\_APPEAL, registro de decisión administrativa, publicación de metadatos de resolución & ✓ Exitoso & Auditoría completa: decisión + justificación + firmante \\
\midrule
CANCELLED (Cancelada) & Transición desde PENDING/APPEALED a CANCELLED, validación de permisos de operador, registro de motivo de cancelación & ✓ Exitoso & Triple validación: permiso + motivo + autoridad firmante \\
\midrule
Verificación de integridad \textit{end-to-end} & Recuperación de comparendo desde blockchain, validación de CID en IPFS, verificación de \textit{hash} de metadatos, reconstrucción de historial completo & ✓ Exitoso & 100\% coincidencia CID, historial completo recuperable \\
\bottomrule
\end{longtable}


Como se observa en la tabla, el prototipo valida exitosamente los cinco estados implementados (PENDING, PAID, APPEALED, RESOLVED\_APPEAL, CANCELLED), que representan las transiciones críticas del ciclo de vida de un comparendo. Cada transición de estado fue probada mediante casos de prueba específicos que verifican:

\begin{enumerate}
    \item \textbf{Inmutabilidad del registro inicial}: Una vez que un comparendo es registrado en estado PENDING con su \textit{hash} criptográfico y CID de IPFS, cualquier intento de modificación de metadatos (placa, ubicación, hora, tipo de infracción) es rechazado por el \textit{smart contract}. Las pruebas demostraron 100\% de rechazo de intentos de alteración, garantizando que el registro original permanece intacto y auditable.

    \item \textbf{Trazabilidad de transiciones}: Cada cambio de estado genera un evento \texttt{FineStatusUpdated} en \textit{blockchain} que registra: (a) \textit{timestamp} inmutable, (b) estado anterior, (c) estado nuevo, (d) razón del cambio, y (e) dirección del actor que ejecutó la transición. El array \texttt{fineStatusHistory} en el \textit{smart contract} acumula este historial completo, permitiendo reconstruir la secuencia exacta de eventos desde PENDING hasta el estado final (PAID, RESOLVED\_APPEAL o CANCELLED).

    \item \textbf{Integridad de evidencia fotográfica}: En cada estado, el CID de IPFS vinculado al comparendo permanece inmutable. Las pruebas validaron que recuperar la evidencia desde IPFS y recalcular su CID siempre produce el mismo identificador, con 100\% de coincidencia en las verificaciones realizadas. Esto garantiza que la evidencia fotográfica no puede ser sustituida o alterada sin detección automática.

    \item \textbf{Control de acceso en transiciones críticas}: Transiciones sensibles como CANCELLED requieren permisos específicos (rol \texttt{operator} en el \textit{smart contract}). Las pruebas validaron que usuarios no autorizados no pueden ejecutar cancelaciones, previniendo manipulación irregular de comparendos.

    \item \textbf{Verificabilidad \textit{end-to-end}}: Para cada estado, se validó que cualquier actor externo puede recuperar el comparendo desde \textit{blockchain}, verificar la integridad del CID, y reconstruir el historial completo de transiciones sin requerir acceso privilegiado al sistema. Esto materializa el principio de \textbf{verificación independiente} fundamental para restaurar la confianza ciudadana.
\end{enumerate}

\paragraph{Alineación con estados conceptuales del proceso}

Como se explica en el capítulo de Implementación (sección 8) y la Tabla~\ref{tab:mapeo_estados}, el prototipo implementa un subconjunto de los ocho estados conceptuales identificados en el análisis del proceso completo de fotocomparendos. Los cinco estados implementados (PENDING, PAID, APPEALED, RESOLVED\_APPEAL, CANCELLED) cubren las transiciones de mayor impacto para las variables del problema investigadas:

\begin{itemize}
    \item \textbf{PENDING}: Aborda la variable "tasa de impugnación" al garantizar inmutabilidad desde el registro inicial, eliminando dudas sobre manipulación post-generación.
    \item \textbf{APPEALED → RESOLVED\_APPEAL}: Directamente relacionado con los 155,854 PQRSD semestrales, proporcionando trazabilidad completa del proceso de apelación.
    \item \textbf{CANCELLED}: Mitigación de fraude mediante registro auditable de cancelaciones administrativas, abordando la variable "vulnerabilidad ciudadana".
    \item \textbf{PAID}: Cierre verificable del proceso, reduciendo disputas post-pago sobre estado del comparendo.
\end{itemize}

Los estados conceptuales NOTIFICADA y CERRADA, no implementados en esta versión del prototipo, pueden agregarse en una implementación de producción siguiendo el mismo patrón de eventos \textit{blockchain} y validación de transiciones. Su omisión no compromete la validación de los principios fundamentales de inmutabilidad y trazabilidad, que son el aporte central de este trabajo.

\subsection{Cobertura de estados y extensibilidad del prototipo}

El prototipo implementa cinco estados que constituyen una base funcional para validar los principios de inmutabilidad y trazabilidad. La arquitectura del \textit{smart contract} fue diseñada para ser extensible: el tipo enumerado \texttt{FineState} en Solidity permite agregar nuevos estados sin modificar la lógica existente, siguiendo el principio abierto/cerrado de diseño de software. La matriz siguiente resume los estados implementados en esta versión:

\begin{table}[H]
\centering
\small
\begin{tabular}{|l|l|c|l|}
\hline
\textbf{Estado} & \textbf{Impacto Si se Manipula} & \textbf{Impl.} & \textbf{Razón} \\
\hline
GENERADA & Fraude directo & \checkmark & Registro inicial crítico \\
\hline
NOTIFICADA & Procedimiento externo & \ding{55} & Requiere oráculo físico \\
\hline
PENDIENTE\_RESPUESTA & Fraude temporal & \checkmark & Plazo legal relevante \\
\hline
EN\_APELACION & Acceso a justicia & \checkmark & Trazabilidad administrativa \\
\hline
RESUELTA\_APELACION & Autoridad judicial & \checkmark & Decisión administrativa \\
\hline
PAGADA & Fraude financiero & \checkmark & Cierre económico \\
\hline
CANCELADA & Auditabilidad & \checkmark & Control administrativo \\
\hline
CERRADA & Estado pasivo & \ding{55} & No afecta integridad \\
\hline
\end{tabular}
\caption{Estados implementados en el prototipo y su relevancia para la validación de integridad.}
\label{tab:cobertura_estados_justificacion}
\end{table}

\subsubsection{Estados pendientes de implementación}
Los estados NOTIFICADA y CERRADA no fueron incluidos en esta versión del prototipo:
\begin{itemize}
    \item \textbf{NOTIFICADA:} Requiere integración con oráculos externos (servicios de correo, SMS) para verificar eventos del mundo físico, lo cual constituye un desarrollo de complejidad comparable al presente proyecto.
    \item \textbf{CERRADA:} Estado administrativo de cierre que puede agregarse siguiendo el mismo patrón de transiciones implementado.
\end{itemize}

\subsubsection{Extensibilidad y trabajo futuro}
La metodología de desarrollo por prototipos adoptada en este trabajo permite la evolución incremental del sistema. Agregar nuevos estados al \textit{smart contract} requiere únicamente:
\begin{enumerate}
    \item Extender el \texttt{enum FineState} con el nuevo estado
    \item Definir las transiciones válidas desde/hacia el nuevo estado
    \item Agregar pruebas unitarias correspondientes
\end{enumerate}

Esta arquitectura extensible garantiza que futuras iteraciones puedan incorporar estados adicionales según las necesidades operativas, sin requerir rediseño de la solución base. Los cinco estados implementados proporcionan una base sólida que cubre las transiciones críticas del ciclo de vida de un comparendo y validan exitosamente los principios fundamentales de inmutabilidad y trazabilidad.

\subsubsection{Validación técnica de los estados implementados}
Las pruebas ejecutadas demuestran que en los cinco estados implementados:
\begin{itemize}
    \item \textbf{Alteraciones rechazadas:} 100\% de intentos de modificación fueron bloqueados por el consenso.
    \item \textbf{Trazabilidad completa:} Todas las transiciones y actores quedan registrados y auditables.
    \item \textbf{Verificabilidad independiente:} Cualquier usuario puede auditar el historial sin credenciales especiales.
\end{itemize}