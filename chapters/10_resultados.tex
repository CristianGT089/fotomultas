\section{Resultados de las Pruebas de Inmutabilidad y Verificabilidad del Prototipo}

Con el fin de validar los principios fundamentales sobre los que se sustenta el presente prototipo —particularmente la \textbf{inmutabilidad}, \textbf{integridad de evidencia} y \textbf{verificabilidad independiente}— se diseñó y ejecutó un plan de pruebas en entorno simulado controlado, alineado con los objetivos del proyecto y los estándares técnicos de la literatura especializada. Las pruebas se enfocaron en evaluar el comportamiento del sistema frente a intentos de modificación, errores de integridad y recuperación de evidencia a través de mecanismos descentralizados.

\subsection{Pruebas de Inmutabilidad en Blockchain}

Se registraron comparendos en la red \textit{Hyperledger Fabric}, incluyendo el hash IPFS (CID) de la evidencia fotográfica y los metadatos del evento. Luego, se intentó simular una alteración directa sobre el estado del ledger.

\textbf{Resultado:} El sistema rechazó cualquier intento de modificación, manteniendo el hash original y evidenciando que la estructura de bloques y el mecanismo de consenso impiden alteraciones sin detección. Esto confirma que el sistema ofrece \textbf{inmutabilidad verificable} en los registros sancionatorios.

\subsection{Verificación de Integridad con IPFS}

Se almacenaron imágenes en IPFS y se compararon los CIDs obtenidos con nuevos hashes locales generados al momento de la consulta.

\textbf{Resultado:} Se comprobó que el CID siempre coincide con el contenido original. Cualquier cambio, incluso mínimo, genera un CID diferente, por lo que el sistema detecta automáticamente cualquier intento de manipulación. Esto demuestra que la evidencia permanece \textbf{íntegra y detectable ante alteraciones}.

\subsection{Verificabilidad Transparente del Registro}

Se implementó un mecanismo de consulta pública (\texttt{/api/fines/:fineId/integrity}) que permite a cualquier parte autorizada extraer el CID desde la Blockchain y verificar que la evidencia recuperada desde IPFS corresponde al evento sancionado.

\textbf{Resultado:} La verificación se ejecuta sin intervención humana, desde fuentes independientes, replicando los principios de \textbf{transparencia, auditabilidad y confianza descentralizada}.

\subsection{Casos de Prueba Funcionales}

% Tablas de resultados de pruebas

\subsection{Casos de prueba funcionales}

\small
\renewcommand{\arraystretch}{1.2}
\setlength{\LTpre}{10pt}
\setlength{\LTpost}{10pt}
\begin{longtable}{p{2cm} p{4cm} p{3cm} p{3cm}}

\caption{Resultados de pruebas funcionales del sistema}
\label{tab:resultados_funcionales} \\
\toprule
\textbf{ID} & \textbf{Caso de Prueba} & \textbf{Resultado} & \textbf{Estado} \\
\midrule
\endfirsthead

\caption[]{(Continuación)} \\
\toprule
\textbf{ID} & \textbf{Caso de Prueba} & \textbf{Resultado} & \textbf{Estado} \\
\midrule
\endhead

\midrule
\multicolumn{4}{r}{\textit{Continúa en la siguiente página}} \\
\endfoot

\bottomrule
\multicolumn{4}{l}{\textbf{Nota.} Elaboración propia.} \\
\endlastfoot

FP-001 & Registro de fotocomparendo & Registro exitoso con CID & Exitoso \\
\midrule
FP-002 & Consulta de comparendo & Datos recuperados correctamente & Exitoso \\
\midrule
FP-003 & Verificación de evidencia & Imagen recuperada desde IPFS & Exitoso \\
\midrule
FP-004 & Actualización de estado & Estado actualizado en blockchain & Exitoso \\
\midrule
FP-005 & Validación de integridad & Integridad verificada & Exitoso \\
\bottomrule
\end{longtable}

\subsection{Casos de prueba de inmutabilidad}

\small
\renewcommand{\arraystretch}{1.2}
\setlength{\LTpre}{10pt}
\setlength{\LTpost}{10pt}
\begin{longtable}{p{2cm} p{6cm} p{3cm}}

\caption{Resumen de casos de prueba de inmutabilidad ejecutados}
\label{tab:resumen_inmutabilidad} \\
\toprule
\textbf{ID} & \textbf{Descripción} & \textbf{Estado} \\
\midrule
\endfirsthead

\caption[]{(Continuación)} \\
\toprule
\textbf{ID} & \textbf{Descripción} & \textbf{Estado} \\
\midrule
\endhead

\midrule
\multicolumn{3}{r}{\textit{Continúa en la siguiente página}} \\
\endfoot

\bottomrule
\multicolumn{3}{l}{\textbf{Nota.} Elaboración propia.} \\
\endlastfoot

IM-001 & Intento de modificar metadatos directamente en el ledger & Ejecutada \\
\midrule
IM-002 & Alteración de imagen ya registrada en IPFS & Ejecutada \\
\midrule
IM-003 & Verificación de trazabilidad e integridad del historial & Ejecutada \\
\bottomrule
\end{longtable}

\subsection{Pruebas de rendimiento básico}

Se midió el tiempo requerido para ejecutar operaciones clave en condiciones simuladas de uso real:

\small
\renewcommand{\arraystretch}{1.2}
\setlength{\LTpre}{10pt}
\setlength{\LTpost}{10pt}
\begin{longtable}{p{4cm} p{3cm}}

\caption{Tiempos promedio de operaciones en el entorno de prueba}
\label{tab:rendimiento} \\
\toprule
\textbf{Operación} & \textbf{Tiempo Promedio (s)} \\
\midrule
\endfirsthead

\caption[]{(Continuación)} \\
\toprule
\textbf{Operación} & \textbf{Tiempo Promedio (s)} \\
\midrule
\endhead

\midrule
\multicolumn{2}{r}{\textit{Continúa en la siguiente página}} \\
\endfoot

\bottomrule
\multicolumn{2}{l}{\textbf{Nota.} Elaboración propia.} \\
\endlastfoot

Registro completo (Blockchain + IPFS) & 1.60 \\
\midrule
Consulta de evidencia desde IPFS & 0.80 \\
\midrule
Validación de integridad & 0.90 \\
\bottomrule
\end{longtable} 


Los resultados obtenidos en el entorno de prueba respaldan la eficacia del modelo propuesto. Tal como se aprecia en la Tabla~\ref{tab:resultados_funcionales}, todas las pruebas funcionales finalizaron de forma exitosa; de manera análoga, la Tabla~\ref{tab:resumen_inmutabilidad} corrobora que los mecanismos de integridad impiden alteraciones, y la Tabla~\ref{tab:rendimiento} demuestra que los tiempos de operación se mantienen dentro de márgenes aceptables para un uso en producción. 