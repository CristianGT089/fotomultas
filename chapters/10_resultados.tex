\section{Resultados de las Pruebas de Inmutabilidad y Verificabilidad del Prototipo}

Con el fin de validar los principios fundamentales sobre los que se sustenta el presente prototipo —particularmente la \textbf{inmutabilidad}, \textbf{integridad de evidencia} y \textbf{verificabilidad independiente}— se diseñó y ejecutó un plan de pruebas en entorno simulado controlado, alineado con los objetivos del proyecto y los estándares técnicos de la literatura especializada. Las pruebas se enfocaron en evaluar el comportamiento del sistema frente a intentos de modificación, errores de integridad y recuperación de evidencia a través de mecanismos descentralizados.

\subsection{Pruebas de Inmutabilidad en Blockchain}

Se registraron comparendos en la red \textit{Hyperledger Fabric}, incluyendo el hash IPFS (CID) de la evidencia fotográfica y los metadatos del evento. Luego, se intentó simular una alteración directa sobre el estado del ledger.

\textbf{Resultado:} El sistema rechazó cualquier intento de modificación, manteniendo el hash original y evidenciando que la estructura de bloques y el mecanismo de consenso impiden alteraciones sin detección. Esto confirma que el sistema ofrece \textbf{inmutabilidad verificable} en los registros sancionatorios.

\subsection{Verificación de Integridad con IPFS}

Se almacenaron imágenes en IPFS y se compararon los CIDs obtenidos con nuevos hashes locales generados al momento de la consulta.

\textbf{Resultado:} Se comprobó que el CID siempre coincide con el contenido original. Cualquier cambio, incluso mínimo, genera un CID diferente, por lo que el sistema detecta automáticamente cualquier intento de manipulación. Esto demuestra que la evidencia permanece \textbf{íntegra y detectable ante alteraciones}.

\subsection{Verificabilidad Transparente del Registro}

Se implementó un mecanismo de consulta pública (\texttt{/api/fines/:fineId/integrity}) que permite a cualquier parte autorizada extraer el CID desde la Blockchain y verificar que la evidencia recuperada desde IPFS corresponde al evento sancionado.

\textbf{Resultado:} La verificación se ejecuta sin intervención humana, desde fuentes independientes, replicando los principios de \textbf{transparencia, auditabilidad y confianza descentralizada}.

\subsection{Casos de Prueba Funcionales}

% Tablas de resultados de pruebas

\subsection{Casos de Prueba Funcionales}

\begin{table}[htbp]
    \begin{flushleft}
        \textbf{Tabla 5}\\[2em]
        \textit{Resultados de pruebas funcionales del sistema}
    \end{flushleft}
    \vspace{1em}
    \addcontentsline{lot}{table}{Tabla 5. Resultados de pruebas funcionales del sistema}
    \centering
    \begin{tabular}{p{2cm} p{4cm} p{3cm} p{3cm}}
        \toprule
        \textbf{ID} & \textbf{Caso de Prueba} & \textbf{Resultado} & \textbf{Estado} \\
        \midrule
        FP-001 & Registro de fotocomparendo & Registro exitoso con CID & Exitoso \\
        FP-002 & Consulta de comparendo & Datos recuperados correctamente & Exitoso \\
        FP-003 & Verificación de evidencia & Imagen recuperada desde IPFS & Exitoso \\
        FP-004 & Actualización de estado & Estado actualizado en Blockchain & Exitoso \\
        FP-005 & Validación de integridad & Integridad verificada & Exitoso \\
        \bottomrule
    \end{tabular}
    \vspace{2em}
    \begin{flushleft}
        \textit{Nota.} Elaboración propia.
    \end{flushleft}
    \label{tab:resultados_funcionales}
\end{table}

\subsection{Casos de Prueba de Inmutabilidad}

\begin{table}[htbp]
    \begin{flushleft}
        \textbf{Tabla 6}\\[2em]
        \textit{Resumen de casos de prueba de inmutabilidad ejecutados}
    \end{flushleft}
    \vspace{1em}
    \addcontentsline{lot}{table}{Tabla 6. Resumen de casos de prueba de inmutabilidad ejecutados}
    \centering
    \begin{tabular}{p{2cm} p{6cm} p{3cm}}
        \toprule
        \textbf{ID} & \textbf{Descripción} & \textbf{Estado} \\
        \midrule
        IM-001 & Intento de modificar metadatos directamente en el ledger & Ejecutada \\
        IM-002 & Alteración de imagen ya registrada en IPFS & Ejecutada \\
        IM-003 & Verificación de trazabilidad e integridad del historial & Ejecutada \\
        \bottomrule
    \end{tabular}
    \vspace{2em}
    \begin{flushleft}
        \textit{Nota.} Elaboración propia.
    \end{flushleft}
    \label{tab:resumen_inmutabilidad}
\end{table}

\subsection{Pruebas de Rendimiento Básico}

Se midió el tiempo requerido para ejecutar operaciones clave en condiciones simuladas de uso real:

\begin{table}[htbp]
    \begin{flushleft}
        \textbf{Tabla 7}\\[2em]
        \textit{Tiempos promedio de operaciones en el entorno de prueba}
    \end{flushleft}
    \vspace{1em}
    \addcontentsline{lot}{table}{Tabla 7. Tiempos promedio de operaciones en el entorno de prueba}
    \centering
    \begin{tabular}{p{4cm} p{3cm}}
        \toprule
        \textbf{Operación} & \textbf{Tiempo Promedio (s)} \\
        \midrule
        Registro completo (Blockchain + IPFS) & 1.60 \\
        Consulta de evidencia desde IPFS & 0.80 \\
        Validación de integridad & 0.90 \\
        \bottomrule
    \end{tabular}
    \vspace{2em}
    \begin{flushleft}
        \textit{Nota.} Elaboración propia.
    \end{flushleft}
    \label{tab:rendimiento}
\end{table} 

Los resultados obtenidos en el entorno de prueba respaldan la eficacia del modelo propuesto. Tal como se aprecia en la Tabla~\ref{tab:resultados_funcionales}, todas las pruebas funcionales finalizaron de forma exitosa; de manera análoga, la Tabla~\ref{tab:resumen_inmutabilidad} corrobora que los mecanismos de integridad impiden alteraciones, y la Tabla~\ref{tab:rendimiento} demuestra que los tiempos de operación se mantienen dentro de márgenes aceptables para un uso en producción. 