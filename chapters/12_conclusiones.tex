\section{Conclusiones y trabajo futuro}

La arquitectura híbrida blockchain que combina Hyperledger Fabric (privacidad) y Ethereum (transparencia pública) es técnicamente viable para la gestión de fotocomparendos. El prototipo desarrollado valida que las tecnologías de registro distribuido pueden garantizar simultáneamente la inmutabilidad de registros, la protección de datos sensibles y la verificación pública ciudadana, abordando las limitaciones identificadas en el sistema actual de Bogotá.

Las pruebas realizadas confirman que el sistema cumple con los principios propuestos: el 100\% de los intentos de modificación fueron rechazados por los mecanismos de consenso, el sistema de direccionamiento por contenido (CID) de IPFS detectó automáticamente todas las alteraciones simuladas, y los tiempos de respuesta medidos (<3 segundos) validan que la arquitectura es viable para aplicaciones en tiempo real.

El backend implementado con interfaces REST estándar y el frontend desarrollado con React demuestran que las tecnologías blockchain pueden abstraerse detrás de APIs convencionales e integrarse con interfaces modernas sin comprometer la experiencia del usuario, facilitando la adopción por parte de instituciones gubernamentales.

\subsection{Síntesis del cumplimiento de objetivos}

En relación con el objetivo general planteado en la Introducción, orientado a demostrar la viabilidad de un prototipo basado en tecnologías de registro distribuido para fortalecer la integridad y trazabilidad de los fotocomparendos, los resultados obtenidos muestran que la arquitectura híbrida diseñada, implementada y validada cumple con este propósito. El desarrollo del prototipo, descrito en las secciones Diseño del prototipo e Implementación del prototipo, junto con las pruebas de inmutabilidad, verificabilidad y rendimiento presentadas en la sección Resultados de las pruebas de inmutabilidad y verificabilidad del prototipo, evidencian que es posible garantizar integridad criptográfica, protección de datos sensibles y verificación pública ciudadana en un entorno controlado.

De manera complementaria, el análisis del sistema actual y del marco regulatorio (secciones Introducción y Estado del arte) permitió caracterizar de forma rigurosa las brechas de integridad, trazabilidad y confianza del modelo centralizado vigente, lo que sirvió de base para la definición de requisitos del prototipo. Sobre esta base, la construcción de la solución híbrida y su evaluación experimental cubrieron los aspectos funcionales y no funcionales previstos en los objetivos específicos: comprender el proceso de fotocomparendos, proponer una arquitectura alternativa y validar empíricamente su comportamiento. En conjunto, estos resultados permiten concluir que el prototipo satisface coherentemente los objetivos formulados y ofrece un modelo de referencia replicable para otros contextos donde la integridad, trazabilidad y verificabilidad de registros públicos sean críticas.

\subsection{Trabajo futuro}

Para la evolución del proyecto se proponen las siguientes líneas de trabajo:

\begin{enumerate}
    \item \textbf{Escalamiento a producción:} Escalar la red Fabric a múltiples organizaciones (SDM, Policía, Auditoría), implementar Private Data Collections, y desplegar nodos IPFS en infraestructura distribuida con políticas de replicación.

    \item \textbf{Piloto controlado:} Realizar un piloto con la Secretaría Distrital de Movilidad utilizando datos reales (5,000-10,000 multas), integración con SIMIT/RUNT y evaluación de rendimiento bajo carga operativa.

    \item \textbf{Funcionalidades avanzadas:} Implementar módulo de pagos integrado (PSE, billeteras digitales), sistema de apelaciones en línea, notificaciones automáticas y dashboard analítico para toma de decisiones.

    \item \textbf{Replicabilidad:} Adaptar la arquitectura para otras ciudades colombianas mediante federación de redes Fabric, explorar soluciones Layer 2 para reducción de costos de gas, y proponer estandarización nacional de Smart Contracts.
\end{enumerate}

