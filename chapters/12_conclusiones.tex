\section{Conclusiones}

El presente estudio logró diseñar y desarrollar un prototipo de software que evidenció la viabilidad técnica de garantizar el registro inmutable y la trazabilidad de estados en el proceso de fotocomparendos mediante tecnologías de redes distribuidas. La arquitectura híbrida implementada, integrando Hyperledger Fabric para datos sensibles, Ethereum para metadatos públicos e IPFS para evidencias digitales, demostró en un entorno experimental controlado que es posible sustituir la confianza administrativa por garantías criptográficas verificables. Los resultados obtenidos evidenciaron que, en el alcance experimental definido, el sistema rechazó el 100\% de los intentos de modificación no autorizada mediante los mecanismos de consenso distribuido, alcanzó una efectividad del 97.5\% en la detección de alteraciones en evidencias fotográficas mediante Content Identifiers, y registró tiempos promedio de respuesta de 2.7 s para el registro completo de infracciones, cumpliendo con los criterios de desempeño establecidos para aplicaciones en tiempo real.

La validación funcional confirmó la gestión adecuada de cinco estados críticos del ciclo de vida del comparendo (\texttt{PENDING}, \texttt{PAID}, \texttt{APPEALED}, \texttt{RESOLVED\_APPEAL} y \texttt{CANCELLED}), generando para cada transición un registro con timestamp criptográfico, estado anterior y posterior, razón del cambio e identificación del actor ejecutor. Este comportamiento demostró la capacidad del prototipo para establecer trazabilidad completa en el entorno de prueba, ofreciendo una solución técnica a la auditoría fragmentada identificada en el sistema FÉNIX. Respecto a las debilidades críticas del sistema actual, los resultados sugieren que la implementación de esta arquitectura podría mitigar los riesgos asociados al presunto detrimento patrimonial documentado por la Contraloría de Bogotá, reducir la carga operativa derivada de las 155,854 PQRSD semestrales, y contribuir a disminuir la tasa de impugnación del 34.1\% mediante mecanismos de verificación independiente que empoderan al ciudadano para auditar la autenticidad de las infracciones sin intermediarios, siempre que se traslade a un entorno productivo con las adaptaciones necesarias.

\subsection{Limitaciones del Estudio}

Es necesario reconocer explícitamente que los resultados obtenidos están condicionados por el alcance metodológico del prototipo. La validación se realizó exclusivamente con datos sintéticos, lo cual limita la generalización de los hallazgos a escenarios con datos reales que podrían presentar inconsistencias, formatos heterogéneos o casos atípicos no contemplados en las pruebas. El volumen transaccional evaluado (50--100 comparendos de prueba) no equivale a la carga operativa real de 457,000 comparendos semestrales, por lo que el comportamiento del sistema bajo estrés masivo requiere validación empírica adicional. La ausencia de auditoría externa formal de los contratos inteligentes y la no evaluación de la aceptación tecnológica por parte de usuarios finales constituyen restricciones que deben abordarse antes de considerar una implementación productiva.

\subsection{Viabilidad de Implementación}

Desde la perspectiva técnica, los resultados obtenidos en el entorno experimental indican que la arquitectura podría escalar para procesar el volumen operativo de Bogotá mediante la configuración de 3--5 nodos Hyperledger Fabric. Sin embargo, estrategias como el \textit{sharding} de canales por tipo de infracción o la migración a soluciones Layer 2 en Ethereum, aunque viables teóricamente, introducirían una complejidad arquitectónica significativa que debería gestionarse cuidadosamente en un despliegue productivo. La viabilidad económica se proyecta favorablemente, estimándose un retorno de inversión entre 18--24 meses, sustentada en la transformación de costos variables de litigio y gestión manual de PQRSD en costos fijos de infraestructura predecibles, considerando una reducción potencial del 30--50\% en solicitudes relacionadas con dudas sobre integridad de datos, aunque estos cálculos corresponden a estimaciones basadas en datos documentados y no en mediciones empíricas de costos operativos reales.

En términos operacionales, la estrategia de migración \textit{dual-write} (operación paralela con el sistema FÉNIX actual) y la migración por fases (piloto de 5,000--10,000 multas reales previo a la integración con SIMIT y RUNT) representan rutas técnicamente factibles, pero su éxito depende de factores institucionales y políticos que exceden el alcance de este estudio técnico.

\subsection{Gobernanza de la Red Blockchain}

El modelo de gobernanza propuesto, basado en un consorcio multi-\textit{stakeholder} con nodos validadores (Secretaría Distrital de Movilidad, Contraloría de Bogotá y Policía Nacional) operando bajo consenso PBFT que requiere mayoría de 2 de 3 para aprobar transacciones, presenta viabilidad teórica para distribuir el poder de validación y mitigar riesgos de manipulación unilateral. Este esquema se alinea con el marco legal colombiano, particularmente con la Ley 1712 de 2014 (transparencia de metadatos públicos en Ethereum) y la Ley 1581 de 2012 (protección de datos sensibles en Hyperledger Fabric), y encuentra precedentes conceptuales en infraestructuras como X-Road de Estonia. No obstante, la implementación efectiva de este modelo requiere acuerdos interadministrativos formales y voluntad política que deben negociarse fuera del ámbito técnico de este prototipo.

\subsection{Trabajo Futuro}

En términos de transición a producción, un despliegue institucional requeriría la superación de brechas técnicas y organizacionales significativas. La infraestructura mínima viable incluiría: (a) una red Hyperledger Fabric con mínimo tres organizaciones y dos nodos peer por organización para redundancia, operando bajo consenso Raft con tolerancia a fallos bizantinos; (b) migración de Ethereum testnet a mainnet o implementación de soluciones Layer 2 (Polygon, Arbitrum) para reducir costos de gas y latencias; (c) cluster de nodos IPFS con pinning distribuido y políticas de retención temporal acordes con la normativa archivística; (d) implementación de Hardware Security Modules (HSM) para la gestión segura de claves privadas, sustituyendo el almacenamiento en archivos de configuración utilizado en el prototipo; y (e) auditoría formal de seguridad (pentesting, análisis estático de contratos inteligentes con herramientas como Slither o MythX) como requisito legal previo al lanzamiento. Estos elementos constituyen condiciones necesarias pero no suficientes para la validación operativa, la cual requeriría además convenios interinstitucionales formales y un piloto controlado con datos reales.

Derivado de las limitaciones identificadas, se proponen líneas de investigación que permitan trasladar el prototipo validado a un entorno de producción. En primer lugar, dado que el estudio utilizó datos sintéticos, es fundamental realizar una validación empírica mediante un piloto controlado con 5,000--10,000 multas reales para evaluar el desempeño bajo condiciones operativas auténticas, integrando datos reales de SIMIT y RUNT, complementado con estudios de aceptación tecnológica mediante modelos como TAM o UTAUT para medir la percepción de confianza y usabilidad en agentes de tránsito y ciudadanos.

Simultáneamente, es necesario abordar la evaluación de escalabilidad técnica, considerando que el volumen de prueba no equivalió a la carga real de 457,000 comparendos semestrales. Se recomienda investigar el comportamiento del sistema bajo estrés masivo, implementando \textit{Private Data Collections} en Fabric para granularizar el acceso a datos sensibles y evaluando la migración a soluciones de Capa 2, como Polygon o Arbitrum, para reducir costos y latencias en la capa pública, aunque estas optimizaciones introducirían una complejidad arquitectónica significativa que debería gestionarse cuidadosamente.

Adicionalmente, para completar el ciclo de custodia digital--física que el prototipo no alcanzó a validar, se propone investigar la integración de oráculos certificadores que permitan registrar el estado \texttt{NOTIFICADA} en la cadena de bloques. Esta línea de trabajo requeriría establecer convenios con empresas de mensajería y operadores postales que actúen como \textit{attesters} criptográficos, generando pruebas de entrega física verificables \textit{on-chain} y cerrando así la brecha entre el evento de notificación en el mundo físico y su registro inmutable, garantizando la trazabilidad integral desde la generación hasta el cierre del proceso.

En cuanto a la extensión funcional, se propone evaluar la implementación de módulos de pagos integrados (PSE, billeteras digitales), sistemas de apelaciones en línea con flujos automatizados, notificaciones automáticas a ciudadanos y \textit{dashboards} analíticos para la toma de decisiones institucionales, funcionalidades que extenderían el prototipo más allá de la trazabilidad básica hacia una usabilidad operativa completa.

Finalmente, antes de cualquier despliegue productivo, resulta imperativo ejecutar auditorías de seguridad y cumplimiento normativo. Se propone realizar análisis estático formal de contratos inteligentes utilizando herramientas especializadas como Slither o MythX, pruebas de penetración de la API REST y procesos de certificación conforme a la Ley 1581 de 2012 sobre protección de datos personales. A largo plazo, se sugiere investigar la replicabilidad y estandarización mediante la federación de redes Fabric con otras ciudades colombianas, proponiendo estándares nacionales de contratos inteligentes que permitan evaluar la viabilidad de replicar esta arquitectura en contextos institucionales diferentes, analizando las adaptaciones necesarias en marcos normativos locales.