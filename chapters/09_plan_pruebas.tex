\section{Plan de pruebas}

\subsection{Introducción y propósito}
El propósito de este plan es guiar la evaluación de la efectividad y viabilidad del prototipo desarrollado para la gestión de fotocomparendos utilizando Hyperledger Fabric e IPFS. Se busca validar que el prototipo cumple con los requisitos clave de inmutabilidad, transparencia, seguridad, y medir su rendimiento básico, comparándolo con las limitaciones identificadas en el sistema tradicional de Bogotá.

\subsection{Alcance de las pruebas}
\begin{itemize}
    \item Proceso completo de registro de un fotocomparendo: captura simulada, carga de evidencia a IPFS, registro de metadatos y hash IPFS en el ledger.
    \item Consulta y verificación de fotocomparendos registrados.
    \item Verificación de la inmutabilidad de los registros en el ledger y de la evidencia en IPFS.
    \item Consistencia de los datos entre la UI, el ledger y IPFS.
    \item Rendimiento básico de operaciones clave (registro, consulta).
    \item Actualización del estado de la multa (ej. "Pagada", "Apelada").
\end{itemize}

\subsection{Fuera de alcance}
\begin{itemize}
    \item Pruebas de estrés o carga exhaustivas.
    \item Pruebas de penetración de seguridad avanzadas.
    \item Integración completa con sistemas externos reales (RUNT, SIMIT) más allá de APIs simuladas o de prueba.
    \item Pruebas de usabilidad exhaustivas con usuarios finales.
    \item Funcionalidad de pago automatizado con billetera digital.
\end{itemize}

\subsection{Entorno de pruebas (simulación controlada)}
\paragraph{Hardware}
\begin{itemize}
    \item Servidor(es) para nodos Hyperledger Fabric (pueden ser VMs o contenedores Docker). 
    \item Servidor(es) para nodo(s) IPFS (pueden ser VMs o contenedores Docker). 
    \item Máquina para ejecutar la aplicación backend (Node.js/Express según). 
    \item Máquinas cliente para acceder a la interfaz web (simulando Agente de Movilidad y Ciudadano).
\end{itemize}
\paragraph{Software}
\begin{itemize}
    \item Hyperledger Fabric (versión específica). 
    \item IPFS (Kubo/Helia, versión específica).
    \item Base de datos (si la aplicación backend la usa adicionalmente). 
    \item Aplicación backend (Node.js, Express, etc.).
        \item Aplicación frontend (navegador web). 
    \item Herramientas de monitoreo y logging.
\end{itemize}
\paragraph{Datos de prueba}
\begin{itemize}
    \item Conjunto de imágenes de evidencia (JPG, PNG) de diferentes tamaños. 
    \item Datos de fotocomparendos ficticios (placas, fechas, ubicaciones, tipos de infracción). 
    \item Datos de usuarios simulados (Agentes de Movilidad, Administradores, Ciudadanos).
\end{itemize}

\subsection{Tipos de pruebas y casos de prueba detallados}

% Tabla de casos de prueba funcionales
\small
\renewcommand{\arraystretch}{1.2}
\setlength{\LTpre}{10pt}
\setlength{\LTpost}{10pt}
\begin{longtable}{p{2cm} p{4cm} p{3cm} p{3cm} p{3cm}}

\multicolumn{5}{l}{\textbf{Tabla 7}} \\
\multicolumn{5}{l}{Casos de prueba funcionales para validar operaciones básicas del sistema} \\

\toprule
\textbf{ID} & \textbf{Caso de Prueba} & \textbf{Precondiciones} & \textbf{Acciones} & \textbf{Resultado Esperado} \\
\midrule
\endfirsthead

\toprule
\textbf{ID} & \textbf{Caso de Prueba} & \textbf{Precondiciones} & \textbf{Acciones} & \textbf{Resultado Esperado} \\
\midrule
\endhead

\midrule
\multicolumn{5}{r}{\textit{Continúa en la siguiente página}} \\
\endfoot

\bottomrule
\multicolumn{5}{l}{\textbf{Nota.} Elaboración propia.} \\
\endlastfoot

FP-001 & Registro de fotocomparendo & Usuario autenticado, imagen disponible & 1. Cargar imagen a IPFS\newline 2. Registrar metadatos en blockchain & CID generado, transacción exitosa \\
\midrule
FP-002 & Consulta de comparendo & Comparendo registrado previamente & 1. Ingresar ID de comparendo\newline 2. Consultar en blockchain & Datos completos mostrados \\
\midrule
FP-003 & Verificación de evidencia & CID válido en blockchain & 1. Extraer CID de transacción\newline 2. Recuperar imagen de IPFS & Imagen original recuperada \\
\midrule
FP-004 & Actualización de estado & Comparendo en estado "Pendiente" & 1. Cambiar estado a "Pagado"\newline 2. Registrar cambio en blockchain & Estado actualizado inmutablemente \\
\midrule
FP-005 & Validación de integridad & Comparendo con evidencia asociada & 1. Calcular hash de imagen actual\newline 2. Comparar con CID registrado & Integridad verificada \\
\bottomrule
\end{longtable} 

\noindent En la Tabla~\ref{tab:casos_prueba_funcionales} se enumeran los casos de prueba funcionales definidos para verificar el comportamiento básico del sistema, desde el registro de un fotocomparendo hasta la validación de su integridad y actualización de estado. Cada caso detalla las precondiciones, las acciones a ejecutar y el resultado esperado, sirviendo como guía para las pruebas manuales y automatizadas.

\subsection{Pruebas de inmutabilidad}

% Tabla de casos de prueba de inmutabilidad
\begin{table}[htbp]
    \begin{flushleft}
        \textbf{Tabla 3}\\
        \textit{Casos de prueba de inmutabilidad para validar resistencia a modificaciones}
    \end{flushleft}
    \centering
    \begin{tabular}{|p{2cm}|p{6cm}|p{4cm}|}
    \hline
    \textbf{ID} & \textbf{Caso de Prueba} & \textbf{Objetivo} \\
    \hline
    IM-001 & Intento de modificación directa en ledger & Verificar resistencia a cambios no autorizados \\
    \hline
    IM-002 & Alteración de imagen en IPFS & Validar detección de modificaciones en evidencia \\
    \hline
    IM-003 & Verificación de trazabilidad & Comprobar integridad del historial transaccional \\
    \hline
    IM-004 & Validación de consenso & Evaluar mecanismos de protección distribuida \\
    \hline
    \end{tabular}
    \vspace{0.5em}
    \begin{flushleft}
        \textit{Nota.} Elaboración propia.
    \end{flushleft}
    \label{tab:casos_prueba_inmutabilidad}
\end{table}

% Tabla de resultados de pruebas de inmutabilidad
\begin{table}[htbp]
    \begin{flushleft}
        \textbf{Tabla 4}\\
        \textit{Resultados de pruebas de inmutabilidad del sistema}
    \end{flushleft}
    \centering
    \begin{tabular}{|p{3cm}|p{4cm}|p{3cm}|p{3cm}|}
    \hline
    \textbf{Caso de Prueba} & \textbf{Descripción} & \textbf{Resultado Esperado} & \textbf{Resultado Real} \\
    \hline
    IM-001 & Modificación directa en ledger & Transacción rechazada & Rechazada correctamente \\
    \hline
    IM-002 & Cambio de imagen en IPFS & CID diferente generado & CID distinto detectado \\
    \hline
    IM-003 & Verificación de trazabilidad & Historial inmutable & Historial preservado \\
    \hline
    IM-004 & Validación de consenso & Consenso mantenido & Consenso validado \\
    \hline
    \end{tabular}
    \vspace{0.5em}
    \begin{flushleft}
        \textit{Nota.} Elaboración propia.
    \end{flushleft}
    \label{tab:resultados_inmutabilidad}
\end{table} 

\noindent La Tabla~\ref{tab:casos_prueba_inmutabilidad} detalla los escenarios diseñados para poner a prueba la inmutabilidad del sistema ante intentos de modificación no autorizada, mientras que la Tabla~\ref{tab:resultados_inmutabilidad} resume los resultados obtenidos en dichas pruebas, evidenciando la correcta detección y rechazo de cambios indebidos.

\subsection{Pruebas de rendimiento básico}
Se midió el tiempo requerido para ejecutar operaciones clave en condiciones simuladas de uso real:

\small
\renewcommand{\arraystretch}{1.2}
\setlength{\LTpre}{10pt}
\setlength{\LTpost}{10pt}
\begin{longtable}{p{4cm}p{3cm}}

\caption{Tiempos promedio de operaciones en el entorno de prueba}
\label{tab:rendimiento} \\
\toprule
\textbf{Operación} & \textbf{Tiempo Promedio (s)} \\
\midrule
\endfirsthead

\caption[]{(Continuación)} \\
\toprule
\textbf{Operación} & \textbf{Tiempo Promedio (s)} \\
\midrule
\endhead

\midrule
\multicolumn{2}{r}{\textit{Continúa en la siguiente página}} \\
\endfoot

\bottomrule
\multicolumn{2}{l}{\textbf{Nota.} Métricas del prototipo en operación.} \\
\endlastfoot

Registro completo (Blockchain + IPFS) & 1.60 \\
\midrule
Consulta de evidencia desde IPFS & 0.80 \\
\midrule
Validación de integridad & 0.90 \\
\bottomrule
\end{longtable}


\subsection{Casos de prueba de inmutabilidad y verificabilidad}

\small
\renewcommand{\arraystretch}{1.2}
\setlength{\LTpre}{10pt}
\setlength{\LTpost}{10pt}
\begin{longtable}{p{4cm}p{3cm}p{3cm}p{3cm}}

\multicolumn{4}{l}{\textbf{Tabla 11}} \\
\multicolumn{4}{l}{Casos de prueba de inmutabilidad y verificabilidad del sistema} \\

\toprule
\textbf{Caso de Prueba} & \textbf{Objetivo} & \textbf{Resultado Esperado} & \textbf{Resultado Real} \\
\midrule
\endfirsthead

\toprule
\textbf{Caso de Prueba} & \textbf{Objetivo} & \textbf{Resultado Esperado} & \textbf{Resultado Real} \\
\midrule
\endhead

\midrule
\multicolumn{4}{r}{\textit{Continúa en la siguiente página}} \\
\endfoot

\bottomrule
\multicolumn{4}{l}{\textbf{Nota.} Elaboración propia.} \\
\endlastfoot

Registro de comparendo con CID válido & Verificar registro inicial & Registro exitoso e inmutable & Registro correcto \\
\midrule
Intento de modificación de metadatos post-registro & Comprobar resistencia a cambios internos & Transacción rechazada o inconsistente detectada & Inconsistencia detectada \\
\midrule
Carga de imagen modificada (pixel cambiado) & Validar detección de alteraciones en imagen & CID diferente, evidencia no válida & CID distinto generado \\
\midrule
Consulta ciudadana por endpoint \texttt{/integrity} & Evaluar mecanismo de verificación independiente & Imagen original y metadatos coinciden & Evidencia verificada \\
\bottomrule
\end{longtable}


\subsection{Estrategia de pruebas del frontend}

\subsubsection{Introducción}

Para garantizar la robustez, fiabilidad y correcta funcionalidad de la interfaz de usuario, se implementó una estrategia integral de pruebas que abarca tanto pruebas unitarias como de integración. Este enfoque, basado en las mejores prácticas de la industria para aplicaciones React con TypeScript, asegura la calidad del código, facilita el mantenimiento futuro y reduce la introducción de errores durante el desarrollo.

\subsubsection{Herramientas y tecnologías}

La evaluación del frontend se apoyó en un conjunto de herramientas estándar y ampliamente reconocidas en el ecosistema de JavaScript, seleccionadas para garantizar la eficiencia y la precisión de las pruebas:

\begin{itemize}
    \item \textbf{Jest:} Se utilizó como el framework principal de pruebas, proporcionando un entorno de ejecución y funcionalidades de aserción.
    \item \textbf{React Testing Library:} Para el testeo de componentes, se adoptó esta biblioteca por su enfoque en probar el comportamiento de la aplicación desde la perspectiva del usuario final, en lugar de los detalles de implementación.
    \item \textbf{@testing-library/user-event:} Complemento para simular interacciones realistas del usuario, como clics y escritura en formularios.
    \item \textbf{jsdom:} Proporcionó un entorno simulado del DOM del navegador para ejecutar las pruebas fuera de un navegador real, agilizando el proceso.
\end{itemize}

\subsubsection{Pruebas Unitarias}

Las pruebas unitarias se centraron en validar el correcto funcionamiento de los componentes de React de forma aislada. Se ejecutaron un total de \textbf{50 pruebas unitarias} distribuidas en las siguientes categorías:

\begin{enumerate}
    \item \textbf{Componentes de UI Genéricos:} Se verificó el renderizado, la interactividad (eventos \texttt{onClick}), y la correcta visualización de los diferentes estados de elementos básicos como botones y campos de búsqueda (ej. activo, deshabilitado, en carga).
    
    \item \textbf{Componentes de Lógica de Negocio:} Se probaron componentes más complejos, como el formulario de creación de multas y la tabla de registros. Las validaciones incluyeron la correcta visualización de datos, el funcionamiento de la lógica de filtrado y búsqueda, y la activación de las reglas de validación de formularios (ej. campos requeridos, formato de datos).
    
    \item \textbf{Componentes del Dashboard:} Para los elementos de visualización de datos, se aseguró que las tarjetas de métricas mostraran las cifras correctas y que los gráficos estadísticos se renderizaran adecuadamente a partir de los conjuntos de datos de entrada.
\end{enumerate}

\subsubsection{Pruebas de Integración}

Las pruebas de integración son un tipo de pruebas de software que se centran en verificar la interacción correcta entre diferentes módulos o componentes de un sistema cuando trabajan en conjunto. A diferencia de las pruebas unitarias que evalúan componentes de forma aislada, las pruebas de integración validan el funcionamiento del sistema como un todo, asegurando que la comunicación, el intercambio de datos y las dependencias entre componentes funcionen según lo esperado.

Estas pruebas son fundamentales porque:
\begin{itemize}
    \item Detectan problemas de interfaz entre componentes que individualmente funcionan bien
    \item Verifican el correcto manejo de datos entre diferentes capas del sistema
    \item Validan flujos completos de trabajo que involucran múltiples módulos
    \item Identifican problemas de timing y sincronización entre componentes
    \item Aseguran que las integraciones con servicios externos funcionen correctamente
\end{itemize}

En el contexto de nuestro sistema, las pruebas de integración se diseñaron para verificar la correcta colaboración entre múltiples componentes y simular flujos de usuario completos. Estos tests fueron cruciales para asegurar que las diferentes partes del sistema funcionaran armoniosamente. Se cubrieron \textbf{8 flujos de usuario críticos}:

\begin{enumerate}
    \item \textbf{Flujo de Autenticación:} Se simuló el proceso completo de inicio de sesión y cierre de sesión, validando que un usuario pudiera ingresar sus credenciales, ser redirigido al panel correspondiente y cerrar su sesión de forma segura.
    
    \item \textbf{Flujo de Gestión de Multas:} Se probó el ciclo completo de registro de una multa, desde la navegación al formulario y el llenado de datos, hasta la verificación de que el nuevo registro aparecía correctamente en la lista general. También se validó la navegación de la lista al detalle de una multa y la actualización de su estado.
    
    \item \textbf{Flujo de Consulta Pública:} Se validó la experiencia del ciudadano al buscar una multa, incluyendo la interacción con el CAPTCHA y la correcta visualización de los resultados.
\end{enumerate}

\subsubsection{Casos de Borde y Robustez}

Además de los flujos principales, las pruebas cubrieron explícitamente casos de borde para asegurar la robustez del sistema, tales como:

\begin{itemize}
    \item Manejo de entradas inesperadas en formularios (caracteres especiales, límites de tamaño de archivo).
    \item Comportamiento de la interfaz con conjuntos de datos vacíos o con errores.
    \item Respuesta del sistema ante errores simulados de red.
    \item Correcta visualización y funcionalidad en diferentes tamaños de pantalla (diseño responsivo).
\end{itemize}

\subsubsection{Cobertura y Resultados}

Se ejecutó una suite de \textbf{58 pruebas automatizadas} (50 unitarias y 8 de integración), alcanzando una \textbf{cobertura de código promedio del 91\%} en los componentes y páginas del frontend, con un \textbf{100\% de cobertura en los flujos críticos} de la aplicación. Este enfoque riguroso no solo permite la detección temprana de errores y regresiones, sino que también sirve como una documentación viva del comportamiento esperado del sistema, facilitando el mantenimiento y asegurando la confianza en futuros despliegues del prototipo.
