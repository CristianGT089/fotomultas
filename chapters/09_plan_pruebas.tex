\section{Plan de Pruebas}

\subsection{Introducción y Propósito}
El propósito de este plan es guiar la evaluación de la efectividad y viabilidad del prototipo desarrollado para la gestión de fotocomparendos utilizando Hyperledger Fabric e IPFS. Se busca validar que el prototipo cumple con los requisitos clave de inmutabilidad, transparencia, seguridad, y medir su rendimiento básico, comparándolo con las limitaciones identificadas en el sistema tradicional de Bogotá.

\subsection{Alcance de las Pruebas}
\begin{itemize}
    \item Proceso completo de registro de un fotocomparendo: captura simulada, carga de evidencia a IPFS, registro de metadatos y hash IPFS en el ledger.
    \item Consulta y verificación de fotocomparendos registrados.
    \item Verificación de la inmutabilidad de los registros en el ledger y de la evidencia en IPFS.
    \item Consistencia de los datos entre la UI, el ledger y IPFS.
    \item Rendimiento básico de operaciones clave (registro, consulta).
    \item Actualización del estado de la multa (ej. "Pagada", "Apelada").
\end{itemize}

\subsection{Fuera de Alcance}
\begin{itemize}
    \item Pruebas de estrés o carga exhaustivas.
    \item Pruebas de penetración de seguridad avanzadas.
    \item Integración completa con sistemas externos reales (RUNT, SIMIT) más allá de APIs simuladas o de prueba.
    \item Pruebas de usabilidad exhaustivas con usuarios finales.
    \item Funcionalidad de pago automatizado con billetera digital.
\end{itemize}

\subsection{Entorno de Pruebas (Simulación Controlada)}
\paragraph{Hardware:}
\begin{itemize}
    \item Servidor(es) para nodos Hyperledger Fabric (pueden ser VMs o contenedores Docker). 
    \item Servidor(es) para nodo(s) IPFS (pueden ser VMs o contenedores Docker). 
    \item Máquina para ejecutar la aplicación backend (Node.js/Express según). 
    \item Máquinas cliente para acceder a la interfaz web (simulando Agente de Movilidad y Ciudadano).
\end{itemize}
\paragraph{Software:}
\begin{itemize}
    \item Hyperledger Fabric (versión específica). 
    \item IPFS (Kubo/Helia, versión específica).
    \item Base de datos (si la aplicación backend la usa adicionalmente). 
    \item Aplicación backend (Node.js, Express, etc.).
        \item Aplicación frontend (navegador web). 
    \item Herramientas de monitoreo y logging.
\end{itemize}
\paragraph{Datos de Prueba:}
\begin{itemize}
    \item Conjunto de imágenes de evidencia (JPG, PNG) de diferentes tamaños. 
    \item Datos de fotocomparendos ficticios (placas, fechas, ubicaciones, tipos de infracción). 
    \item Datos de usuarios simulados (Agentes de Movilidad, Administradores, Ciudadanos).
\end{itemize}

\subsection{Tipos de Pruebas y Casos de Prueba Detallados}

% Tabla de casos de prueba funcionales
\paragraph{Pruebas Funcionales}
\begin{table}[htbp]
    \centering
    \footnotesize
    \caption{Casos de Prueba Funcionales}
    \label{tab:casos_funcionales}

    \begin{tabular}{|
        >{\raggedright\arraybackslash}p{0.07\textwidth}|
        >{\raggedright\arraybackslash}p{0.20\textwidth}|
        >{\raggedright\arraybackslash}p{0.40\textwidth}|
        >{\raggedright\arraybackslash}p{0.20\textwidth}|}
        \hline
        \textbf{ID} & \textbf{Descripción} & \textbf{Pasos de Ejecución} & \textbf{Datos de Entrada} \\
        \hline
        % Fila 1
        \textbf{FT-001} & 
        Registro exitoso de fotocomparendo & 
        1. Login en SisFotocomp. \newline 
        2. Ir a "Registrar nueva multa". \newline 
        3. Ingresar datos (placa, fecha, tipo). \newline 
        4. Adjuntar imagen. \newline 
        5. Enviar. & 
        Placa: XYZ789, Fecha: [Hoy], Tipo: Exceso Velocidad, Imagen: evidencia01.jpg \\
        \hline
        % Fila 2
        \textbf{FT-002} & 
        Consulta y verificación (Agente/Admin) & 
        1. Login como Agente/Admin. \newline 
        2. Ir a "Gestión de multas". \newline 
        3. Buscar multa FT-001 por ID o placa. \newline 
        4. Ver detalles. \newline 
        5. Verificar información e imagen IPFS. & 
        ID/Placa de la multa FT-001. \\
        \hline
        % Fila 3
        \textbf{FT-003} & 
        Consulta ciudadana & 
        1. Acceder a "Consulta de Multas". \newline 
        2. Ingresar documento, número y placa. \newline 
        3. Ingresar CAPTCHA. \newline 
        4. Consultar. & 
        Datos del propietario/vehículo de FT-001. \\
        \hline
        % Fila 4
        \textbf{FT-004} & 
        Registro con datos incompletos & 
        1. Intentar registrar multa sin placa o sin imagen. & 
        Placa: Vacía, Imagen: No adjuntada. \\
        \hline
        % Fila 5
        \textbf{FT-005} & 
        Actualización de estado & 
        1. Seleccionar multa FT-001. \newline 
        2. Cambiar estado (ej. "Apelada", "Pagada"). \newline 
        3. Guardar. & 
        Multa FT-001, Nuevo estado: "Apelada". \\
        \hline
        % Fila 6
        \textbf{FT-006} & 
        Consistencia Ledger-IPFS & 
        1. Registrar multa (similar a FT-001). \newline 
        2. Anotar CID de IPFS y metadatos. \newline 
        3. Recuperar transacción del ledger. \newline 
        4. Recuperar imagen de IPFS. & 
        Nueva multa, nueva imagen. \\
        \hline
    \end{tabular}
\end{table} 

\noindent En la Tabla~\ref{tab:casos_prueba_funcionales} se enumeran los casos de prueba funcionales definidos para verificar el comportamiento básico del sistema, desde el registro de un fotocomparendo hasta la validación de su integridad y actualización de estado. Cada caso detalla las precondiciones, las acciones a ejecutar y el resultado esperado, sirviendo como guía para las pruebas manuales y automatizadas.

\subsection{Pruebas de Inmutabilidad}

% Tabla de casos de prueba de inmutabilidad
\begin{table}[htbp]
    \begin{flushleft}
        \textbf{Tabla 3}\\[1em]
        \textit{Casos de prueba de inmutabilidad para validar resistencia a modificaciones}
    \end{flushleft}
    \vspace{1em}
    \addcontentsline{lot}{table}{Tabla 3. Casos de prueba de inmutabilidad para validar resistencia a modificaciones}
    \centering
    \begin{tabular}{p{2cm} p{6cm} p{4cm}}
        \toprule
        \textbf{ID} & \textbf{Caso de Prueba} & \textbf{Objetivo} \\
        \midrule
        IM-001 & Intento de modificación directa en ledger & Verificar resistencia a cambios no autorizados \\
        IM-002 & Alteración de imagen en IPFS & Validar detección de modificaciones en evidencia \\
        IM-003 & Verificación de trazabilidad & Comprobar integridad del historial transaccional \\
        IM-004 & Validación de consenso & Evaluar mecanismos de protección distribuida \\
        \bottomrule
    \end{tabular}
    \vspace{1em}
    \begin{flushleft}
        \textit{Nota.} Elaboración propia.
    \end{flushleft}
    \refstepcounter{table}\label{tab:casos_prueba_inmutabilidad}
\end{table}

% Tabla de resultados de pruebas de inmutabilidad
\begin{table}[htbp]
    \begin{flushleft}
        \textbf{Tabla 4}\\[1em]
        \textit{Resultados de pruebas de inmutabilidad del sistema}
    \end{flushleft}
    \vspace{1em}
    \addcontentsline{lot}{table}{Tabla 4. Resultados de pruebas de inmutabilidad del sistema}
    \centering
    \begin{tabular}{p{3cm} p{4cm} p{3cm} p{3cm}}
        \toprule
        \textbf{Caso de Prueba} & \textbf{Descripción} & \textbf{Resultado Esperado} & \textbf{Resultado Real} \\
        \midrule
        IM-001 & Modificación directa en ledger & Transacción rechazada & Rechazada correctamente \\
        IM-002 & Cambio de imagen en IPFS & CID diferente generado & CID distinto detectado \\
        IM-003 & Verificación de trazabilidad & Historial inmutable & Historial preservado \\
        IM-004 & Validación de consenso & Consenso mantenido & Consenso validado \\
        \bottomrule
    \end{tabular}
    \vspace{1em}
    \begin{flushleft}
        \textit{Nota.} Elaboración propia.
    \end{flushleft}
    \refstepcounter{table}\label{tab:resultados_inmutabilidad}
\end{table} 

\noindent La Tabla~\ref{tab:casos_prueba_inmutabilidad} detalla los escenarios diseñados para poner a prueba la inmutabilidad del sistema ante intentos de modificación no autorizada, mientras que la Tabla~\ref{tab:resultados_inmutabilidad} resume los resultados obtenidos en dichas pruebas, evidenciando la correcta detección y rechazo de cambios indebidos.

\subsection{Pruebas de Rendimiento Básico}

Se midió el tiempo requerido para ejecutar operaciones clave en condiciones simuladas de uso real:

\begin{table}[htbp]
    \centering
    \caption{Tiempos promedio de operaciones en el entorno de prueba}
    \begin{tabular}{|p{4cm}|p{3cm}|}
        \hline
        \textbf{Operación} & \textbf{Tiempo Promedio (s)} \\
        \hline
        Registro completo (Blockchain + IPFS) & 1.60 \\
        \hline
        Consulta de evidencia desde IPFS & 0.80 \\
        \hline
        Validación de integridad & 0.90 \\
        \hline
    \end{tabular}
    \vspace{1em}
    \begin{flushleft}
        \textit{Nota.} Elaboración propia.
    \end{flushleft}
\end{table}


\subsection{Casos de Prueba de Inmutabilidad y Verificabilidad}

\begin{center}
\begin{tabular}{|p{4cm}|p{3cm}|p{3cm}|p{3cm}|}
    \hline
    \textbf{Caso de Prueba} & \textbf{Objetivo} & \textbf{Resultado Esperado} & \textbf{Resultado Real} \\
    \hline
    Registro de comparendo con CID válido & Verificar registro inicial & Registro exitoso e inmutable & Registro correcto \\
    \hline
    Intento de modificación de metadatos post-registro & Comprobar resistencia a cambios internos & Transacción rechazada o inconsistente detectada & Inconsistencia detectada \\
    \hline
    Carga de imagen modificada (pixel cambiado) & Validar detección de alteraciones en imagen & CID diferente, evidencia no válida & CID distinto generado \\
    \hline
    Consulta ciudadana por endpoint \texttt{/integrity} & Evaluar mecanismo de verificación independiente & Imagen original y metadatos coinciden & Evidencia verificada \\
    \hline
\end{tabular}

\vspace{1em}
\noindent\textbf{Cuadro 3}\\[2em]
\textit{Casos de prueba de inmutabilidad y verificabilidad del sistema}
\end{center} 

\subsection{Estrategia de pruebas del frontend}

\paragraph{Introducción}
El frontend de la aplicación de gestión de multas implementa una estrategia integral de pruebas que abarca tanto pruebas unitarias como de integración, utilizando las mejores prácticas de testing en React con TypeScript. Esta estrategia garantiza la calidad del código, facilita el mantenimiento y reduce la introducción de errores durante el desarrollo.

\subsubsection{Herramientas y Tecnologías}
\begin{itemize}
    \item \textbf{Jest}: Framework principal de testing con soporte para TypeScript.
    \item \textbf{React Testing Library}: Biblioteca para testing de componentes React con enfoque en comportamiento del usuario.
    \item \textbf{@testing-library/jest-dom}: Matchers adicionales para Jest.
    \item \textbf{@testing-library/user-event}: Simulación de eventos de usuario.
    \item \textbf{jsdom}: Entorno DOM para pruebas en Node.js.
\end{itemize}

\subsubsection{Pruebas Unitarias}

\subsubsection{Pruebas de Integración}
