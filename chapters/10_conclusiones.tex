\section{Conclusiones}
\begin{enumerate}
    \item El uso combinado de Blockchain permisionada e IPFS garantiza la inmutabilidad y verificabilidad de los registros sancionatorios, cumpliendo con el objetivo general del proyecto. La implementación del prototipo demostró que es posible registrar comparendos de forma segura y auditable, asegurando que tanto los metadatos como las evidencias fotográficas permanezcan protegidas ante manipulaciones, incluso frente a ataques internos o errores administrativos.
    \item La evaluación funcional del sistema evidenció que los flujos principales de registro, consulta, verificación y actualización de multas operan correctamente, permitiendo una interacción fluida entre los actores del sistema: agentes de tránsito, ciudadanos y administradores. Esto confirma que los requisitos funcionales identificados en la etapa de análisis fueron cubiertos adecuadamente, y que la arquitectura distribuida no impide la usabilidad del sistema.
    \item El modelo desarrollado representa un avance significativo hacia una gestión más transparente y confiable de los fotocomparendos en Bogotá, y sienta las bases para su adopción en contextos reales. Si bien el prototipo fue probado en un entorno simulado, sus resultados técnicos, el cumplimiento de los objetivos específicos y su alineación con las necesidades ciudadanas sugieren que su implementación a gran escala podría fortalecer la confianza institucional y reducir los casos de corrupción y disputa legal asociados a los sistemas actuales.
\end{enumerate} 