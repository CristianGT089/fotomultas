\section{\large Introducción}

La tecnología \textit{blockchain}, introducida por Nakamoto en 2008 con Bitcoin \parencite{nakamoto2008bitcoin}, ha evolucionado desde su aplicación original en criptomonedas hacia casos de uso empresariales y gubernamentales. Esta tecnología de registro distribuido (\textit{DLT}) se caracteriza por proporcionar inmutabilidad mediante encadenamiento criptográfico de bloques, consenso descentralizado entre múltiples nodos, y transparencia verificable sin requerir intermediarios de confianza \parencite{swan2015blockchain}. A diferencia de las bases de datos tradicionales donde un administrador central controla y puede modificar registros, \textit{blockchain} distribuye el poder de validación entre múltiples participantes, haciendo computacionalmente prohibitiva la alteración retroactiva de información \parencite{tapscott2016blockchain}.

En la última década, gobiernos alrededor del mundo han explorado \textit{blockchain} como infraestructura para servicios públicos digitales. Estonia implementó \textit{Keyless Signature Infrastructure} (\textit{KSI}) basada en \textit{blockchain} para proteger registros de salud, tributarios y judiciales, procesando más de 1 billón de verificaciones anuales \parencite{sullivan2021estonia}. Suecia realizó pilotos de registro de tierras en \textit{blockchain} que redujeron tiempos de transacción de meses a días mientras eliminaban fraudes documentales \parencite{karamitsos2018design}. Dubai lanzó la Estrategia \textit{Blockchain} 2020 con el objetivo de migrar todos los documentos gubernamentales a esta tecnología, proyectando ahorros de 25 millones de horas laborales anuales y 1,500 millones de dólares en costos de procesamiento \parencite{dubai2016blockchain}.

En América Latina, Chile ha utilizado \textit{blockchain} para certificar datos de la red eléctrica nacional, garantizando transparencia en tarifas energéticas \parencite{galvez2018blockchain}. Brasil explora registros médicos en \textit{blockchain} para interoperabilidad entre instituciones de salud \parencite{azaria2016medrec}. En Colombia, el marco regulatorio ha avanzado con iniciativas del Ministerio de Tecnologías de la Información y las Comunicaciones (MinTIC) para promover la transformación digital del Estado, incluyendo lineamientos para adopción de tecnologías emergentes en el sector público \parencite{mintic2020estrategia}.

Paralelamente, el almacenamiento descentralizado mediante \textit{InterPlanetary File System} (\textit{IPFS}) ha emergido como complemento natural de \textit{blockchain} para gestión de documentos gubernamentales. \textit{IPFS} utiliza direccionamiento por contenido (\textit{Content Identifier}, \textit{CID}) en lugar de direccionamiento por ubicación, garantizando que cualquier alteración de un archivo genere un identificador completamente diferente \parencite{benet2014ipfs}. Esta propiedad lo hace ideal para evidencia digital que requiere verificación de integridad, como lo demuestran implementaciones en sistemas judiciales de Noruega y archivos históricos de Portugal \parencite{truong2019gdpr}.

A pesar de este potencial, la gestión de infracciones de tránsito ha recibido atención limitada en la literatura de \textit{blockchain} gubernamental. Trabajos como el de \textcite{yousfi2019blockchain} proponen sistemas de multas en \textit{blockchain} pública (Ethereum), pero enfrentan limitaciones de privacidad al exponer datos personales sensibles. \textcite{chen2024blockchain} sugieren arquitecturas híbridas combinando bases de datos con \textit{blockchain}, pero su modelo mantiene mutabilidad en la capa de datos primaria, debilitando garantías de integridad. Ninguna de estas propuestas aborda el desafío específico de balancear transparencia pública con protección de datos personales en cumplimiento de regulaciones como GDPR o la Ley 1581 de 2012 colombiana.

El contexto colombiano presenta un caso de estudio particularmente relevante. La gestión de fotocomparendos en Bogotá ha sido objeto de controversia debido a posibles fallas en el sistema, documentadas por la Contraloría de Bogotá \parencite{contraloria2024detrimento}. La falta de un sistema confiable ha generado desconfianza entre los ciudadanos, manifestada en una tasa de impugnación del 34.1\% que refleja déficit de confianza en la integridad de los datos \parencite{sdm2024estadisticas}. Casos de fraude como Juzto.co, donde miles de ciudadanos fueron estafados con promesas de impugnaciones garantizadas \parencite{semana2023juzto}, evidencian que el sistema actual no cumple una necesidad del ciudadano de una solución que garantice integridad, inmutabilidad y verificabilidad de la información.

Las tecnologías de registro distribuido, y en particular \textit{blockchain}, ofrecen alternativas arquitectónicas para abordar estas deficiencias. A través de \textit{smart contracts} (contratos inteligentes), es posible automatizar la validación y el procesamiento de fotocomparendos con reglas transparentes y auditables, reduciendo la intervención humana discrecional y minimizando el riesgo de corrupción o errores administrativos \parencite{buterin2014next}. La combinación de \textit{blockchain} permisionado para datos sensibles con \textit{blockchain} público para metadatos verificables, junto con almacenamiento IPFS para evidencia fotográfica, constituye una arquitectura híbrida que puede satisfacer simultáneamente requisitos de privacidad, transparencia y verificabilidad.

Este trabajo propone el diseño e implementación de un prototipo basado en una arquitectura híbrida \textit{blockchain} para la gestión de fotocomparendos en Bogotá, con el objetivo de garantizar la transparencia del proceso mediante garantías criptográficas en lugar de confianza administrativa. El prototipo integra \textit{Hyperledger Fabric} (red permisionada para datos privados), \textit{Ethereum} (red pública para metadatos verificables) e \textit{IPFS} (almacenamiento descentralizado de evidencia), validando mediante pruebas empíricas que esta arquitectura puede proporcionar inmutabilidad del 100\%, trazabilidad completa de transiciones de estado, y tiempos de respuesta compatibles con operación en tiempo real (≤3 segundos). La investigación demuestra cómo las tecnologías de redes distribuidas pueden fortalecer la confianza en los procesos de control de tránsito, reducir la carga operativa de PQRSD, y mitigar el detrimento patrimonial asociado a la gestión actual de comparendos en la capital colombiana.

\subsection{Formulación del problema}
La gestión de comparendos en Bogotá es un proceso de gran escala. Según datos del Observatorio de Movilidad, entre enero de 2018 y agosto de 2024 se emitieron más de 1.9 millones de comparendos a través de cámaras salvavidas, evidenciando la importancia sistémica de este proceso para la regulación del tránsito en la ciudad, como se presenta en la Figura~\ref{fig:estadisticas_comparendos} se observa los diferentes métodos utilizados para crear los comparendos. Esta operación se apoya en el sistema FÉNIX, una aplicación con infraestructura en la nube, cuya arquitectura de datos y control de acceso opera bajo un paradigma centralizado.

\begin{figure}[h!]
    \begin{flushleft}
        \textbf{Figura 1}\\[0.5em]
        \textit{Estadísticas de comparendos emitidos en Bogotá entre enero de 2018 y agosto de 2024}
    \end{flushleft}
    \vspace{0.3em}
    \addcontentsline{lof}{figure}{Figura 1. Estadísticas de comparendos emitidos en Bogotá entre enero de 2018 y agosto de 2024}
    \centering
    \includegraphics[width=0.75\textwidth]{Images/numComparendos.png}
    \vspace{0.5em}
    \begin{flushleft}
        \textit{Nota.} Estadísticas de comparendos emitidos en Bogotá entre 2018 y 2024, mostrando los diferentes métodos de creación.
    \end{flushleft}
    \refstepcounter{figure}\label{fig:estadisticas_comparendos}
\end{figure}

En el sistema actual, la validez e inmutabilidad de los registros de infracciones se fundamenta en los procedimientos administrativos y en la gestión de los funcionarios responsables del sistema \parencite{C112_2018}. Los cambios en la información solo pueden ser detectados por las entidades autorizadas, lo que implica que el control sobre los registros depende directamente de la correcta aplicación de las políticas internas y del seguimiento realizado por dichas entidades \parencite{Sentencia123_2019}.

La evidencia generada se conserva bajo un modelo centralizado, en el cual la confianza en la integridad de los datos se sostiene en mecanismos administrativos y controles internos, más que en garantías técnicas accesibles públicamente \parencite{DAFP_Lineamientos_2021}. La potestad sancionatoria y el debido procedimiento administrativo aseguran la validez de los actos administrativos y la correcta motivación en la imposición de sanciones (Corte Constitucional, 2022; Gamero Casado y Fernández Ramos, s.f.).

De acuerdo con la Auditoría de Cumplimiento de la Contraloría de Bogotá (2024), en el proceso de desarrollo del sistema FÉNIX se identificaron dificultades relacionadas con la supervisión contractual, lo que derivó en retrasos, duplicidad de sistemas y un presunto detrimento patrimonial estimado en más de \$8.000 millones de pesos. Estos hallazgos reflejan que, desde su implementación, la plataforma ha enfrentado retos significativos en materia de gobernanza y gestión, los cuales han tenido impacto en la eficiencia administrativa y en la sostenibilidad financiera del proyecto.

Estas debilidades se manifiestan en la operación técnica actual. A nivel operativo, el riesgo de integridad se materializa en una fricción a gran escala con la ciudadanía. Un análisis correlacional de fuentes oficiales para el primer semestre de 2025 revela la magnitud de esta fricción: frente a 457.000 comparendos impuestos [Observatorio de Movilidad, 2025], se gestionaron 155.854 PQRSD [Informe de Gestión PQRSD, 2025].

De estos datos se deduce una Tasa de Impugnación general del 34.1\%, un indicador cuantitativo que sugiere que al menos uno de cada tres actos administrativos del sistema genera una disputa formal, reflejando una carga administrativa insostenible y un déficit de confianza.

La desconfianza generada por estas opacidades y dificultades procesales crea un vacío que es explotado por terceros, afectando directamente al ciudadano. Reportajes de prensa documentan cómo la ausencia de canales oficiales percibidos como confiables ha fomentado la aparición de redes de fraude, como el caso de Juzto.co, donde miles de ciudadanos fueron estafados con promesas de impugnaciones garantizadas, resultando en trámites inconclusos y mayores deudas \parencite{Semana_Juzto_2023}.

La identificación de estas limitaciones permite estructurar el problema en torno a variables que reflejan tanto el modelo de confianza actual como sus impactos técnicos, operativos y financieros. La Tabla~\ref{tab:variables_problema} sintetiza las variables del problema de investigación y los indicadores asociados, mostrando cómo el paradigma centralizado de gestión condiciona la integridad de los datos, la eficiencia administrativa, la confianza ciudadana y la sostenibilidad del sistema.

\small
\renewcommand{\arraystretch}{1.2}
\setlength{\LTpre}{10pt}
\setlength{\LTpost}{10pt}
\begin{longtable}{p{4.5cm} p{4.5cm} p{3.5cm} p{2.5cm}}

\caption{Variables del problema de investigación}
\label{tab:variables_problema} \\
\toprule
\textbf{Variable} & \textbf{Definición} & \textbf{Medición Actual} & \textbf{Meta con Prototipo} \\
\midrule
\endfirsthead

\caption[]{(Continuación)} \\
\toprule
\textbf{Variable} & \textbf{Definición} & \textbf{Medición Actual} & \textbf{Meta con Prototipo} \\
\midrule
\endhead

\midrule
\multicolumn{4}{r}{\textit{Continúa en la siguiente página}} \\
\endfoot

\bottomrule
\multicolumn{4}{l}{\textbf{Nota.} Elaboración propia basada en datos de \textcite{sdm2024estadisticas} y \textcite{contraloria2024detrimento}.} \\
\endlastfoot

Tasa de Impugnación & Porcentaje de comparendos que generan PQRSD por parte de ciudadanos que cuestionan su validez o evidencia & 34.1\% (155,854 PQRSD de 457,000 comparendos semestrales) & Reducción esperada por mayor confianza en integridad de evidencia \\
\midrule
Detrimento Patrimonial & Pérdida económica estimada para el Distrito por comparendos impugnados exitosamente o declarados nulos & \$8,000+ millones de pesos semestrales & Cuantificación de reducción mediante trazabilidad verificable \\
\midrule
Carga Operativa PQRSD & Cantidad de solicitudes de petición, queja, reclamo y denuncia que deben procesarse administrativamente & 155,854 solicitudes por semestre (2024-I) & Reducción por transparencia y verificabilidad autónoma \\
\midrule
Vulnerabilidad Ciudadana & Exposición del ciudadano a fraudes o manipulación de registros de comparendos (ej. casos Juzto.co) & Casos documentados de suplantación y modificación irregular & Mitigación por inmutabilidad criptográfica y registro distribuido \\
\bottomrule
\end{longtable}


Para comprender las implicaciones técnicas de estas variables, la Tabla~\ref{tab:comparacion_bd_blockchain} contrasta las características fundamentales entre el modelo de base de datos convencional actualmente utilizado y una arquitectura basada en blockchain, evidenciando las diferencias en los mecanismos de confianza, inmutabilidad y trazabilidad que motivan la propuesta de este trabajo.

\begin{table}[htbp]
    \centering
    \caption{Comparación entre bases de datos tradicionales y blockchain para gestión de registros gubernamentales}
    \begin{tabular}{p{4.5cm} p{5.2cm} p{5.2cm}}
        \toprule
        \textbf{Característica} & \textbf{Base de Datos Convencional} & \textbf{Blockchain} \\
        \midrule
        Modelo de confianza & Se basa en un administrador central (entidad de TI) & Confianza distribuida entre múltiples nodos \\
        Inmutabilidad & Registros pueden ser modificados o eliminados por administradores & Los registros son inmutables por diseño \\
        Trazabilidad / Auditoría & Depende de la implementación y control interno & Historial completo e inalterable disponible \\
        Riesgo de corrupción interna & Alto, si hay privilegios indebidos o colusión & Bajo, no se puede alterar sin consenso de la red \\
        Seguridad criptográfica & Opcional, no siempre integrada nativamente & Integrada (firmas digitales, hashes, cifrado) \\
        Disponibilidad / tolerancia a fallos & Riesgo de puntos únicos de falla & Alta disponibilidad por replicación descentralizada \\
        Velocidad de operación & Alta velocidad en lectura/escritura & Menor velocidad, prioriza integridad y consenso \\
        \bottomrule
    \end{tabular}
    \vspace{1em}
    \begin{flushleft}
        \textit{Nota.} Elaboración propia.
    \end{flushleft}
    \refstepcounter{table}\label{tab:comparacion_bd_blockchain}
\end{table}

En síntesis, el problema se formula como un Riesgo de Integridad de Datos inherente al paradigma de confianza centralizada del sistema de fotocomparendos. Este riesgo se encuentra documentado por debilidades fundacionales en la gobernanza del proyecto y se manifiesta en consecuencias medibles: (i) una Tasa de Impugnación del 34.1\%; (ii) una carga operativa superior a 155 mil PQRSD semestrales; (iii) un presunto detrimento patrimonial por más de \$8.000 millones; y (iv) la vulnerabilidad de la ciudadanía a esquemas fraudulentos derivados de la falta de transparencia institucional.

Ante este panorama, surge la necesidad de explorar arquitecturas que permitan sustituir la confianza administrativa por garantías criptográficas. La pregunta central que guía este trabajo es:

\textbf{¿Cómo mitigar el riesgo de pérdida o alteración de la integridad de los datos asociados a todos los estados en el proceso de fotocomparendos en Bogotá mediante el uso de tecnologías de redes distribuidas que garanticen el registro, la trazabilidad, la autenticidad y la confidencialidad de la información?}

\subsection{Estados del proceso de fotocomparendos}

Para comprender el alcance del problema de integridad de datos, es fundamental identificar los estados por los que transita un comparendo desde su emisión hasta su resolución final. El proceso de fotocomparendos en Bogotá contempla los siguientes estados principales:

\begin{enumerate}
    \item \textbf{GENERADA:} Estado inicial del comparendo, creado por un agente de tránsito o sistema automático de fotodetección al registrar una infracción. En este estado se capturan los metadatos esenciales: placa del vehículo, tipo de infracción, fecha, hora, ubicación y evidencia fotográfica.

    \item \textbf{NOTIFICADA:} El comparendo ha sido oficialmente notificado al ciudadano propietario o conductor del vehículo, según lo establecido en el Código Nacional de Tránsito. La notificación puede realizarse de forma electrónica, por correo certificado o personalmente. Este estado habilita los términos legales para respuesta ciudadana.

    \item \textbf{PENDIENTE\_RESPUESTA:} El ciudadano ha sido notificado y se encuentra dentro del plazo legal para tomar una acción: pagar voluntariamente con descuento, solicitar acuerdos de pago, o presentar una apelación mediante PQRSD. Este es un estado crítico donde la integridad de la evidencia es fundamental para la toma de decisiones.

    \item \textbf{EN\_APELACION:} El ciudadano ha presentado formalmente una Petición, Queja, Reclamo, Sugerencia o Denuncia (PQRSD) cuestionando la validez del comparendo. Durante este estado, la autoridad de tránsito debe revisar la evidencia, verificar el cumplimiento del debido proceso, y emitir una resolución motivada. La trazabilidad completa del comparendo es esencial en esta etapa.

    \item \textbf{RESUELTA\_APELACION:} La autoridad competente ha tomado una decisión sobre la apelación presentada, que puede ser: (a) confirmar el comparendo, (b) revocarlo total o parcialmente, o (c) declararlo nulo por defectos procedimentales. La decisión debe estar debidamente fundamentada y ser notificada al ciudadano.

    \item \textbf{PAGADA:} El comparendo ha sido pagado, ya sea voluntariamente por el ciudadano, mediante acuerdo de pago, o tras agotar el proceso de apelación con resultado confirmatorio. El pago liquida la obligación económica pero el registro del comparendo permanece en el sistema con fines estadísticos y de antecedentes.

    \item \textbf{CANCELADA:} El comparendo ha sido cancelado administrativamente por razones como: anulación judicial, revocación por defectos procedimentales, o corrección de errores en la emisión (ej. placa incorrecta). Este estado requiere la mayor trazabilidad posible para prevenir cancelaciones irregulares.

    \item \textbf{CERRADA:} Estado final y definitivo del comparendo. Todas las acciones administrativas, judiciales o de pago han concluido. El registro se mantiene de forma permanente en el sistema con fines de auditoría, estadística y consulta histórica.
\end{enumerate}

Cada transición entre estados representa un punto crítico donde la alteración, pérdida o manipulación de datos puede comprometer la validez del acto administrativo y generar las consecuencias descritas anteriormente: impugnaciones, detrimento patrimonial y vulnerabilidad ciudadana. En el desarrollo del proyecto se identifican explícitamente como puntos de intervención técnica: (i) el registro inicial de metadatos y evidencia en los estados \textbf{GENERADA} y \textbf{NOTIFICADA}; (ii) la trazabilidad de cambios y decisiones durante \textbf{PENDIENTE\_RESPUESTA}, \textbf{EN\_APELACION} y \textbf{RESUELTA\_APELACION}, con especial énfasis en \textbf{CANCELADA}; y (iii) la verificación independiente y permanente del cierre del caso en \textbf{CERRADA}. El prototipo propuesto busca garantizar la inmutabilidad y trazabilidad completa en todas estas transiciones mediante el uso de tecnologías de registro distribuido.

\subsection{Objetivos}
\paragraph{Objetivo General}
Diseñar un prototipo basado en una metodología de software que garantice el registro inmutable y la trazabilidad completa de estados en el proceso de fotocomparendos en Bogotá mediante tecnologías de redes distribuidas, con el fin de reducir los riesgos asociados a la integridad, confidencialidad y auditabilidad de la información.

\paragraph{Objetivos específicos}
\begin{itemize}
    \item Analizar el proceso actual de registro de fotocomparendos en Bogotá, a partir del marco jurídico y regulatorio que lo rige y de los informes de auditoría emitidos por la secretaria distrital de movilidad sobre la gestión de comparendos, para identificar vulnerabilidades, requisitos funcionales y no funcionales.
    \item Desarrollar un prototipo con arquitectura híbrida basado en \textit{blockchain} permisionado (\textit{Hyperledger Fabric}) y \textit{blockchain} público (\textit{Ethereum}), integrando almacenamiento distribuido mediante \textit{IPFS}, asegurando que cada transacción incorpore los metadatos del comparendo y disponiendo de una interfaz de programación de aplicaciones (API \textit{REST}) que permita operaciones de registro, consulta y verificación de estados.
    \item Evaluar la viabilidad técnica y funcional del prototipo a través de un plan de pruebas que incluya: pruebas de inmutabilidad ante intentos de modificación de registros, verificación de trazabilidad mediante validación de \textit{hashes} criptográficos en cada transición de estado, análisis de rendimiento para validar tiempos de respuesta ≤3 segundos, y evaluación de integridad de documentos mediante \textit{Content Identifiers} (\textit{CID}) de \textit{IPFS}.
\end{itemize}

\subsection{Impacto esperado}

El desarrollo de este prototipo tiene como propósito demostrar la viabilidad técnica de una arquitectura descentralizada para la gestión de fotocomparendos, con el potencial de:

\begin{itemize}
    \item \textbf{Fortalecer la confianza ciudadana:} Mediante la verificación independiente de infracciones y el acceso transparente a la información, sin intermediarios.
    \item \textbf{Reducir la fricción operativa:} Disminuir los recursos destinados a la gestión de PQRSD y disputas administrativas, actualmente estimados en más de 155.000 casos semestrales.
    \item \textbf{Prevenir fraudes:} Mitigar la vulnerabilidad de los ciudadanos ante esquemas de estafa derivados de la falta de canales oficiales confiables.
    \item \textbf{Establecer un precedente técnico:} Servir como referencia para la implementación de soluciones similares en otros procesos gubernamentales que requieran alta integridad de datos.
\end{itemize}

\textbf{Nota:} Para la especificación detallada del alcance del proyecto, los componentes del prototipo, criterios de éxito y limitaciones metodológicas, consultar la sección Alcance. 