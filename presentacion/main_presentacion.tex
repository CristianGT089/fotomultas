\documentclass[10pt, aspectratio=169]{beamer}

% ==================== TEMA Y CONFIGURACIÓN ====================
\usetheme{metropolis}
\metroset{block=fill}

% ==================== PACKAGES ====================
\usepackage[utf8]{inputenc}
\usepackage[spanish]{babel}
\usepackage{graphicx}
\usepackage{booktabs}
\usepackage{longtable}
\usepackage{adjustbox}
\usepackage{tikz}
\usepackage{xcolor}
\usepackage{hyperref}
\usepackage{tabularx}
\usepackage{multirow}
\usepackage{colortbl}

% ==================== CONFIGURACIÓN DE COLORES ====================
% Colores institucionales Universidad Distrital
\definecolor{udblue}{RGB}{0,51,102}
\definecolor{udlightblue}{RGB}{51,102,153}
\definecolor{udgray}{RGB}{89,89,89}
\definecolor{udorange}{RGB}{230,126,34}  % Naranja para títulos

\setbeamercolor{palette primary}{bg=udlightblue,fg=white}
\setbeamercolor{palette secondary}{bg=udblue,fg=white}
\setbeamercolor{frametitle}{bg=udlightblue,fg=white}  % Título de frames en azul claro
\setbeamercolor{title}{fg=udblue}  % Título principal en azul oscuro (texto)
\setbeamercolor{title separator}{fg=udlightblue}
\setbeamercolor{block title}{bg=udblue,fg=white}
\setbeamercolor{alerted text}{fg=udlightblue}

% ==================== CONFIGURACIÓN DE HYPERREF ====================
\hypersetup{
    colorlinks=true,
    linkcolor=udblue,
    urlcolor=udlightblue,
    citecolor=udblue
}

% ==================== INFORMACIÓN DE LA PRESENTACIÓN ====================
\title{Prototipo para apoyar el registro y trazabilidad de estados en el proceso de fotocomparendos aplicando tecnologías de redes distribuidas}

\author{%
    Laura Catalina Preciado Ballén \\
    Cristian Stiven Guzmán Tovar
}

\institute{%
    Universidad Distrital Francisco José de Caldas \\
    Facultad de Ingeniería \\
    Programa de Ingeniería de Sistemas \\[1em]
    \textbf{Director:} Julio Barón Velandia
}

\date{}

% ==================== CONFIGURACIONES ADICIONALES ====================
% Mostrar número de slide en el footer
\setbeamertemplate{frame numbering}[fraction]

% Ajustar espaciado de items
\setlength{\leftmargini}{1em}
\setlength{\leftmarginii}{1.5em}

% ==================== COMANDOS PERSONALIZADOS ====================
% Comando para crear secciones con página de transición
\newcommand{\sectionframe}[1]{
    \section{#1}
    \begin{frame}[plain,noframenumbering]
        \centering
        \vfill
        {\Huge \textbf{#1}}
        \vfill
    \end{frame}
}

% Comando para destacar texto
\newcommand{\highlight}[1]{\textcolor{udlightblue}{\textbf{#1}}}

% ==================== INICIO DEL DOCUMENTO ====================
\begin{document}

% ==================== PORTADA ====================
\begin{frame}[plain,noframenumbering]
    \vfill
    \begin{center}
        \includegraphics[width=0.15\textwidth]{images/Escudo_UD.png}

        \vspace{0.8em}

        \normalsize
        \inserttitle

        \vspace{1.5em}

        \insertauthor

        \vspace{1em}

        \small
        Director: Julio Barón Velandia

        \vspace{1.5em}

        Universidad Distrital Francisco José de Caldas \\
        Facultad de Ingeniería \\
        Programa de Ingeniería de Sistemas
    \end{center}
    \vfill
\end{frame}

% ==================== TABLA DE CONTENIDOS ====================
\begin{frame}{Agenda}
    \tableofcontents[hideallsubsections]
\end{frame}

% ==================== CONTENIDO ====================
% Sección 1: Introducción y Contexto
\section{\large Introducción}
En Colombia, la gestión de fotocomparendos ha sido objeto de controversia debido a fallas en la transparencia y posibles manipulaciones en el proceso de registro y validación de infracciones. La falta de un sistema confiable ha generado desconfianza entre los ciudadanos, lo que evidencia la necesidad de una solución que garantice la integridad, inmutabilidad y verificabilidad de la información.

Las tecnologías de registro distribuido (DLT), y en particular blockchain, han demostrado ser alternativas eficaces para el almacenamiento seguro y descentralizado de datos, asegurando que una vez registrados, estos no puedan ser alterados sin dejar rastro (ver Capítulo 3 para fundamentos teóricos detallados). A través de contratos inteligentes, es posible automatizar la validación y el procesamiento de fotocomparendos, reduciendo la intervención humana y minimizando el riesgo de corrupción o errores administrativos.

Este trabajo propone el diseño e implementación de un prototipo basado en una arquitectura híbrida blockchain para la gestión de fotocomparendos en Bogotá, con el objetivo de garantizar la transparencia del proceso. Se utilizarán contratos inteligentes para registrar cada infracción, permitiendo que cualquier actor autorizado pueda verificar su autenticidad sin necesidad de intermediarios. Mediante pruebas y simulaciones, se evaluará la viabilidad del sistema, demostrando cómo esta tecnología puede fortalecer la confianza en los procesos de control de tránsito y mejorar la eficiencia en la gestión de sanciones.

\subsection{Formulación del problema}
La gestión de comparendos en Bogotá es un proceso de gran escala. Según datos del Observatorio de Movilidad, entre enero de 2018 y agosto de 2024 se emitieron más de 1.9 millones de comparendos a través de cámaras salvavidas, evidenciando la importancia sistémica de este proceso para la regulación del tránsito en la ciudad, como se presenta en la Figura~\ref{fig:estadisticas_comparendos} se observa los diferentes métodos utilizados para crear los comparendos. Esta operación se apoya en el sistema FÉNIX, una aplicación con infraestructura en la nube, cuya arquitectura de datos y control de acceso opera bajo un paradigma centralizado.

\begin{figure}[htbp]
    \begin{flushleft}
        \textbf{Figura 1}\\[2em]
        \textit{Estadísticas de comparendos emitidos en Bogotá entre enero de 2018 y agosto de 2024}
    \end{flushleft}
    \vspace{1em}
    \addcontentsline{lof}{figure}{Figura 1. Estadísticas de comparendos emitidos en Bogotá entre enero de 2018 y agosto de 2024}
    \centering
    \includegraphics[width=0.8\textwidth]{Images/numComparendos.png}
    \vspace{2em}
    \begin{flushleft}
        \textit{Nota.} Elaboración propia basado en datos del Observatorio de Movilidad.
    \end{flushleft}
    \refstepcounter{figure}\label{fig:estadisticas_comparendos}
\end{figure}

En el sistema actual, la validez e inmutabilidad de los registros de infracciones se fundamenta en los procedimientos administrativos y en la gestión de los funcionarios responsables del sistema \parencite{C112_2018}. Los cambios en la información solo pueden ser detectados por las entidades autorizadas, lo que implica que el control sobre los registros depende directamente de la correcta aplicación de las políticas internas y del seguimiento realizado por dichas entidades \parencite{Sentencia123_2019}.

La evidencia generada se conserva bajo un modelo centralizado, en el cual la confianza en la integridad de los datos se sostiene en mecanismos administrativos y controles internos, más que en garantías técnicas accesibles públicamente \parencite{DAFP_Lineamientos_2021}. La potestad sancionatoria y el debido procedimiento administrativo aseguran la validez de los actos administrativos y la correcta motivación en la imposición de sanciones (Corte Constitucional, 2022; Gamero Casado y Fernández Ramos, s.f.).

De acuerdo con la Auditoría de Cumplimiento de la Contraloría de Bogotá (2024), en el proceso de desarrollo del sistema FÉNIX se identificaron dificultades relacionadas con la supervisión contractual, lo que derivó en retrasos, duplicidad de sistemas y un presunto detrimento patrimonial estimado en más de \$8.000 millones de pesos. Estos hallazgos reflejan que, desde su implementación, la plataforma ha enfrentado retos significativos en materia de gobernanza y gestión, los cuales han tenido impacto en la eficiencia administrativa y en la sostenibilidad financiera del proyecto.

Estas debilidades se manifiestan en la operación técnica actual. A nivel operativo, el riesgo de integridad se materializa en una fricción a gran escala con la ciudadanía. Un análisis correlacional de fuentes oficiales para el primer semestre de 2025 revela la magnitud de esta fricción: frente a 457.000 comparendos impuestos [Observatorio de Movilidad, 2025], se gestionaron 155.854 PQRSD [Informe de Gestión PQRSD, 2025].

De estos datos se deduce una Tasa de Impugnación general del 34.1\%, un indicador cuantitativo que sugiere que al menos uno de cada tres actos administrativos del sistema genera una disputa formal, reflejando una carga administrativa insostenible y un déficit de confianza.

La desconfianza generada por estas opacidades y dificultades procesales crea un vacío que es explotado por terceros, afectando directamente al ciudadano. Reportajes de prensa documentan cómo la ausencia de canales oficiales percibidos como confiables ha fomentado la aparición de redes de fraude, como el caso de Juzto.co, donde miles de ciudadanos fueron estafados con promesas de impugnaciones garantizadas, resultando en trámites inconclusos y mayores deudas \parencite{Semana_Juzto_2023}.

La identificación de estas limitaciones permite estructurar el problema en torno a variables que reflejan tanto el modelo de confianza actual como sus impactos técnicos, operativos y financieros. La Tabla~\ref{tab:comparacion_bd_blockchain} sintetiza estas variables y los indicadores asociados, mostrando cómo el paradigma centralizado de gestión condiciona la integridad de los datos, la eficiencia administrativa, la confianza ciudadana y la sostenibilidad del sistema.

\begin{table}[htbp]
    \centering
    \caption{Comparación entre bases de datos tradicionales y blockchain para gestión de registros gubernamentales}
    \begin{tabular}{p{4.5cm} p{5.2cm} p{5.2cm}}
        \toprule
        \textbf{Característica} & \textbf{Base de Datos Convencional} & \textbf{Blockchain} \\
        \midrule
        Modelo de confianza & Se basa en un administrador central (entidad de TI) & Confianza distribuida entre múltiples nodos \\
        Inmutabilidad & Registros pueden ser modificados o eliminados por administradores & Los registros son inmutables por diseño \\
        Trazabilidad / Auditoría & Depende de la implementación y control interno & Historial completo e inalterable disponible \\
        Riesgo de corrupción interna & Alto, si hay privilegios indebidos o colusión & Bajo, no se puede alterar sin consenso de la red \\
        Seguridad criptográfica & Opcional, no siempre integrada nativamente & Integrada (firmas digitales, hashes, cifrado) \\
        Disponibilidad / tolerancia a fallos & Riesgo de puntos únicos de falla & Alta disponibilidad por replicación descentralizada \\
        Velocidad de operación & Alta velocidad en lectura/escritura & Menor velocidad, prioriza integridad y consenso \\
        \bottomrule
    \end{tabular}
    \vspace{1em}
    \begin{flushleft}
        \textit{Nota.} Elaboración propia.
    \end{flushleft}
    \refstepcounter{table}\label{tab:comparacion_bd_blockchain}
\end{table}

En síntesis, el problema se formula como un Riesgo de Integridad de Datos inherente al paradigma de confianza centralizada del sistema de fotocomparendos. Este riesgo se encuentra documentado por debilidades fundacionales en la gobernanza del proyecto y se manifiesta en consecuencias medibles: (i) una Tasa de Impugnación del 34.1\%; (ii) una carga operativa superior a 155 mil PQRSD semestrales; (iii) un presunto detrimento patrimonial por más de \$8.000 millones; y (iv) la vulnerabilidad de la ciudadanía a esquemas fraudulentos derivados de la falta de transparencia institucional.

Ante este panorama, surge la necesidad de explorar arquitecturas que permitan sustituir la confianza administrativa por garantías criptográficas. La pregunta central que guía este trabajo es:

\textbf{¿Cómo mitigar el Riesgo de Integridad de Datos en el proceso de fotocomparendos en Bogotá?} 

\subsection{Objetivos}
\paragraph{Objetivo General}
Desarrollar un prototipo para apoyar el registro y trazabilidad de estados en el proceso de fotocomparendos en Bogotá, aplicando tecnologías de redes distribuidas, con el fin de fortalecer la integridad, la autenticidad de la información, y reducir los riesgos asociados a su confidencialidad.

\paragraph{Objetivos específicos}
\begin{itemize}
    \item Analizar el proceso actual de registro de fotocomparendos en Bogotá, a partir del marco jurídico y regulatorio que lo rige y de los informes de auditoría emitidos por la secretaria distrital de movilidad sobre la gestión de comparendos, para identificar requisitos funcionales, no funcionales y vulnerabilidades que el prototipo debe proporcionar.
    \item Desarrollar un prototipo con arquitectura híbrida fundamentada en la técnica de descomposición por confianza, integrando tecnologías de almacenamiento distribuido y contenido cifrado, asegurando que cada transacción incorpore los metadatos del comparendo y disponiendo de una interfaz básica para demostrar que es posible un aplicativo transparente y confiable.
    \item Evaluar la viabilidad del prototipo desarrollado, mediante la ejecución de un plan de pruebas funcionales y evaluación de métricas de desempeño en un entorno de pruebas, para validar las condiciones de inmutabilidad, trazabilidad y seguridad.
\end{itemize}

\subsection{Impacto esperado}

El desarrollo de este prototipo tiene como propósito demostrar la viabilidad técnica de una arquitectura descentralizada para la gestión de fotocomparendos, con el potencial de:

\begin{itemize}
    \item \textbf{Fortalecer la confianza ciudadana:} Mediante la verificación independiente de infracciones y el acceso transparente a la información, sin intermediarios.
    \item \textbf{Reducir la fricción operativa:} Disminuir los recursos destinados a la gestión de PQRSD y disputas administrativas, actualmente estimados en más de 155.000 casos semestrales.
    \item \textbf{Prevenir fraudes:} Mitigar la vulnerabilidad de los ciudadanos ante esquemas de estafa derivados de la falta de canales oficiales confiables.
    \item \textbf{Establecer un precedente técnico:} Servir como referencia para la implementación de soluciones similares en otros procesos gubernamentales que requieran alta integridad de datos.
\end{itemize}

\textbf{Nota:} Para la especificación detallada del alcance del proyecto, los componentes del prototipo, criterios de éxito y limitaciones metodológicas, consultar el Capítulo 6: Alcance y Limitaciones. 

% Sección 2: Justificación y Objetivos
\section{\large Justificación}
La gestión de registros públicos, como los fotocomparendos, en arquitecturas centralizadas presenta debilidades en materia de seguridad, transparencia y auditabilidad. En Bogotá, el sistema FÉNIX ilustra estos desafíos, según auditorías de la Contraloría \parencite{Informe170100005424} \parencite{InformeCumplimiento90}, que destacan limitaciones en la integridad de los datos y una fricción operativa reflejada en más de 153.000 PQRSD en un semestre. Esta situación subraya la necesidad de modelos arquitectónicos alternativos que fortalezcan la confianza pública, independizándola de la dependencia exclusiva en procedimientos y administradores internos. Se transita así de un sistema donde la integridad se presume y se audita retrospectivamente, a uno donde es intrínseca y verificable criptográficamente desde el origen.

El propósito de este proyecto no es modificar el sistema actual, sino diseñar y evaluar un prototipo autocontenido que demuestre un modelo de confianza diferente. Para ilustrar las diferencias estructurales entre el modelo convencional y el propuesto, la Tabla \ref{tab:comparacion_modelos} compara sus características clave:

\footnotesize
\renewcommand{\arraystretch}{1.15}
\setlength{\LTpre}{10pt}
\setlength{\LTpost}{10pt}
\begin{longtable}{p{1.8cm} p{2.8cm} p{2.8cm} p{3.5cm}}

\caption{Comparación entre un modelo centralizado y un modelo descentralizado}
\label{tab:comparacion_modelos} \\
\toprule
\textbf{Característica} & \textbf{Modelo Centralizado} & \textbf{Modelo Descentralizado} & \textbf{Relevancia Contextual} \\
\midrule
\endfirsthead

\caption[]{(Continuación)} \\
\toprule
\textbf{Característica} & \textbf{Modelo Centralizado} & \textbf{Modelo Descentralizado} & \textbf{Relevancia Contextual} \\
\midrule
\endhead

\midrule
\multicolumn{4}{r}{\textit{Continúa en la siguiente página}} \\
\endfoot

\bottomrule
\multicolumn{4}{p{13cm}}{\textbf{Nota.} Elaboración propia, con hallazgos basados en la Auditoría de Cumplimiento No. 90 de la Contraloría de Bogotá D.C. (octubre de 2023) y la Auditoría de Cumplimiento 170100-0054-24.} \\
\endlastfoot

Modelo de Confianza & Basado en la confianza en los administradores del sistema y en la robustez de los controles internos definidos. & Basado en un consenso criptográfico distribuido, donde la confianza reside en el protocolo y no en un intermediario. & La correcta asignación de roles es fundamental. La auditoría observó ``ausencia de un profesional responsable de Seguridad de la Información'' (págs. 20--25), subrayando la criticidad de los factores de gobernanza. \\
\midrule
Integridad de Datos & La integridad se asegura mediante controles de acceso y logs de auditoría internos gestionados por la entidad. & La integridad es una propiedad intrínseca de la estructura de datos; los registros son inmutables por diseño. & La efectividad de los controles internos es fundamental. La auditoría documentó ``Falta de control sobre la integridad y calidad de los datos migrados'' (págs. 38--40) como punto de atención. \\
\midrule
Gestión de Seguridad & Dependiente de políticas y procedimientos de seguridad definidos y ejecutados por la institución. & La seguridad es una propiedad inherente a la capa de protocolo, auditada de forma continua y global por la comunidad. & La formalización de procedimientos es clave. La auditoría identificó ``falta de gestión formal de riesgos y controles'' y ``ausencia de un plan de seguridad para la infraestructura en la nube'' (págs. 25--30). \\
\midrule
Auditabilidad y Trazabilidad & La auditoría se realiza a través de logs internos, con acceso gestionado por la entidad y sujeto a sus políticas de retención y seguridad. & La traza de auditoría es transparente, inalterable por diseño y públicamente verificable por cualquier actor autorizado. & La consistencia de los registros internos es un factor de éxito. La auditoría observó ``retrasos y baja velocidad de desarrollo'' (págs. 15--20), subrayando la importancia de una gobernanza rigurosa. \\
\bottomrule
\end{longtable}



\subsection{Pertinencia social, tecnológica y legal}

La pertinencia de este proyecto se enmarca en tres dimensiones complementarias que justifican la necesidad de explorar arquitecturas descentralizadas para la gestión de registros públicos críticos:

\begin{itemize}
    \item \textbf{Social y ciudadana:} En un contexto donde la desconfianza en los procesos administrativos genera una tasa de impugnación del 34.1\% (más de 155.000 PQRSD semestrales según se detalla en el Estado del Arte, subsección sobre el contexto de Bogotá), este proyecto ofrece un modelo alternativo que responde a la necesidad de transparencia, permitiendo la verificación independiente y empoderando al ciudadano con herramientas de auditoría directa sobre la autenticidad de las sanciones.

    \item \textbf{Tecnológica:} Demuestra cómo la integración de blockchain (para registros inmutables) e IPFS (para evidencias con contenido direccionable) puede abordar los desafíos de seguridad y trazabilidad documentados en sistemas centralizados, respondiendo específicamente a las limitaciones estructurales del sistema FÉNIX identificadas por la Contraloría de Bogotá (ver análisis detallado en el Estado del Arte).

    \item \textbf{Legal e institucional:} El prototipo se alinea con los principios de eficiencia, transparencia y rendición de cuentas exigidos por los organismos de control. Frente a los incumplimientos normativos y brechas de protección de datos identificados en el sistema actual, esta propuesta sirve como caso de estudio sobre cómo garantías técnicas intrínsecas pueden fortalecer el cumplimiento normativo y reducir los riesgos de detrimento patrimonial asociados a modelos centralizados.
\end{itemize}

Para un análisis detallado del contexto operativo y legal del sistema actual de fotocomparendos en Bogotá, así como de las limitaciones críticas identificadas en FÉNIX que motivan esta propuesta, consultar la subsección correspondiente en el Estado del Arte.

\subsection{Originalidad e innovación}
La innovación de esta monografía radica en la concepción del prototipo como un laboratorio para un nuevo modelo de confianza. Mientras los sistemas tradicionales se centran en controles administrativos, esta propuesta explora un modelo distribuido y resistente a la manipulación por diseño. La DApp funciona como una prueba de concepto que integra inmutabilidad, gobernanza automatizada y almacenamiento descentralizado para demostrar una solución a una clase de problemas que las bases de datos centralizadas, por su naturaleza, no pueden resolver de manera nativa.

\subsection{Relación del impacto con los objetivos del proyecto}

Este prototipo responde a una problemática documentada en Bogotá y se inserta en la tendencia global de GovTech. El impacto esperado del proyecto (detallado en la sección de Impacto esperado del Capítulo 1) se manifiesta en dimensiones técnicas, sociales e institucionales que se alinean directamente con los objetivos específicos planteados:

\begin{itemize}
    \item \textbf{Confianza por Diseño:} La verificación independiente fortalece la legitimidad de los procesos públicos, cumpliendo con el objetivo de garantizar integridad y autenticidad mediante arquitecturas descentralizadas.

    \item \textbf{Gobernanza Automatizada:} Los contratos inteligentes ejecutan reglas de negocio de forma predecible, reduciendo la dependencia de supervisión humana y minimizando riesgos de corrupción, alineándose con el objetivo de desarrollar un sistema transparente y confiable.

    \item \textbf{Escalabilidad en GovTech:} Este caso de uso es transferible a otros procesos que demandan alta integridad, sirviendo como precedente para futuras innovaciones en la administración pública y cumpliendo con el objetivo de establecer un modelo replicable.
\end{itemize}

La adopción de blockchain en esta propuesta no es una preferencia tecnológica, sino una respuesta técnica deliberada a los desafíos de integridad y confianza inherentes a los modelos centralizados, proponiendo una arquitectura donde la veracidad es una propiedad intrínseca y verificable del sistema. 

% Sección 3: Metodología y Diseño
% ==================== SECCIÓN 3: METODOLOGÍA Y DISEÑO ====================
\sectionframe{Metodología y Diseño del Prototipo}

% ==================== SLIDE 10: METODOLOGÍA ====================
\begin{frame}[shrink=5]{Enfoque Metodológico: Desarrollo por Prototipos}
    \textbf{Justificación del Modelo de Prototipos:}

    \begin{itemize}
        \item \textbf{Naturaleza innovadora:} Combinación de tecnologías emergentes sin precedentes locales
        \item \textbf{Requisitos evolutivos:} Marco normativo y tecnología en constante cambio
        \item \textbf{Validación temprana:} Probar hipótesis central antes de desarrollo completo
    \end{itemize}

    \vspace{1em}

    \textbf{Fases del Desarrollo del Prototipo:}

    \begin{enumerate}
        \item \textbf{Análisis de requisitos} → Marco legal + auditorías
        \item \textbf{Diseño arquitectónico} → Patrones de descomposición por confianza
        \item \textbf{Implementación iterativa} → Backend + Frontend + Smart Contracts
        \item \textbf{Pruebas y validación} → 80 casos automatizados
    \end{enumerate}

    \begin{block}{Mitigación de Riesgos}
        Decisión metodológica deliberada que mitiga riesgos técnicos y permite pivotes ágiles
    \end{block}
\end{frame}

% ==================== SLIDE 11: SELECCIÓN TECNOLÓGICA ====================
\begin{frame}[shrink=10]{Arquitectura Híbrida: Decisión Crítica}
    \textbf{Problema:} Ninguna blockchain cumple TODOS los requisitos

    \begin{itemize}
        \item Privacidad de datos personales (Ley 1581/2012)
        \item Transparencia pública ciudadana
        \item Rendimiento (457,000 comparendos semestrales)
        \item Costos operativos predecibles
    \end{itemize}

    \vspace{0.5em}

    \scriptsize
    \begin{table}
        \centering
        \begin{tabular}{p{2.5cm} p{3.5cm} p{3.5cm} p{2cm}}
            \toprule
            \textbf{Componente} & \textbf{Tecnología} & \textbf{Justificación} & \textbf{TPS} \\
            \midrule
            Capa privada & Hyperledger Fabric v2.5 & Control acceso PKI, sin gas fees & 2K-20K \\
            \midrule
            Capa pública & Ethereum (Sepolia) & Verificación ciudadana, ecosistema maduro & 15-30 \\
            \midrule
            Storage privado & IPFS privado & Evidencias sensibles, acceso controlado & - \\
            \midrule
            Storage público & IPFS público & Hashes verificación, content-addressed & - \\
            \bottomrule
        \end{tabular}
    \end{table}

    \begin{alertblock}{Arquitectura Híbrida}
        Balancea trade-offs irreconciliables mediante descomposición por niveles de confianza
    \end{alertblock}
\end{frame}

% ==================== SLIDE 12: DIAGRAMA DE DESPLIEGUE ====================
\begin{frame}[shrink=15]{Arquitectura Híbrida del Sistema}
    \begin{center}
        \includegraphics[height=0.6\textheight]{images/Despliegue.png}
    \end{center}

    \vspace{0.3em}

    \footnotesize
    \textbf{Capas del Sistema:}
    \begin{itemize}
        \item \textbf{Capa 1:} Frontend React (ciudadano + agente)
        \item \textbf{Capa 2:} API REST Node.js/Express
        \item \textbf{Capa 3:} Hyperledger Fabric (red privada permisionada)
        \item \textbf{Capa 4:} Ethereum + IPFS público (verificación transparente)
        \item \textbf{Capa 5:} IPFS privado (evidencias completas)
    \end{itemize}
\end{frame}

% ==================== SLIDE 13: DIAGRAMA DE CLASES ====================
\begin{frame}[shrink=10]{Diseño Orientado a Objetos}
    \begin{columns}[T]
        \begin{column}{0.6\textwidth}
            \includegraphics[width=\textwidth]{images/uml.png}
        \end{column}
        \begin{column}{0.4\textwidth}
            \textbf{Patrón Controller-Service-Repository}

            \vspace{1em}

            \textbf{Capas Arquitectónicas:}

            \begin{enumerate}
                \item \textbf{Servicios blockchain:}
                \begin{itemize}
                    \footnotesize
                    \item HyperledgerService
                    \item EthereumService
                    \item SyncService
                \end{itemize}

                \item \textbf{Almacenamiento:}
                \begin{itemize}
                    \footnotesize
                    \item IPFSPrivateService
                    \item IPFSPublicService
                \end{itemize}

                \item \textbf{Orquestación:}
                \begin{itemize}
                    \footnotesize
                    \item FineService
                    \item FineController (REST)
                \end{itemize}
            \end{enumerate}

            \vspace{0.5em}

            \begin{block}{\footnotesize Beneficios}
                \footnotesize
                Separación de responsabilidades, testabilidad, mantenibilidad
            \end{block}
        \end{column}
    \end{columns}
\end{frame}

% ==================== SLIDE 14: CASOS DE USO ====================
\begin{frame}[shrink=5]{Actores y Funcionalidades Principales}
    \begin{columns}[T]
        \begin{column}{0.55\textwidth}
            \includegraphics[width=\textwidth]{images/CasosUso.png}
        \end{column}
        \begin{column}{0.45\textwidth}
            \textbf{Actores Identificados:}

            \vspace{0.5em}

            \begin{enumerate}
                \item \textbf{Agente de Tránsito}
                \begin{itemize}
                    \footnotesize
                    \item Registrar comparendo
                    \item Actualizar estado
                \end{itemize}

                \vspace{0.5em}

                \item \textbf{Ciudadano}
                \begin{itemize}
                    \footnotesize
                    \item Consultar multa
                    \item Verificar autenticidad
                    \item Apelar
                \end{itemize}

                \vspace{0.5em}

                \item \textbf{Administrador}
                \begin{itemize}
                    \footnotesize
                    \item Gestionar sistema
                    \item Auditar operaciones
                \end{itemize}
            \end{enumerate}

            \vspace{0.5em}

            \begin{block}{\footnotesize Cobertura Integral}
                \footnotesize
                Sistema cubre el ciclo de vida completo del fotocomparendo
            \end{block}
        \end{column}
    \end{columns}
\end{frame}

% ==================== SLIDE 15: FLUJOS DE PROCESO ====================
\begin{frame}[shrink=15]{Diagramas de Actividad: Procesos Críticos}
    \begin{columns}[T]
        \begin{column}{0.5\textwidth}
            \centering
            \textbf{Creación de Multa}

            \includegraphics[width=0.9\textwidth]{images/ActMulta.png}

            \vspace{0.3em}

            \footnotesize
            \textbf{Flujo:}
            \begin{itemize}
                \scriptsize
                \item Captura → IPFS privado
                \item → Hyperledger
                \item → Sync → Ethereum público
            \end{itemize}
        \end{column}
        \begin{column}{0.5\textwidth}
            \centering
            \textbf{Proceso de Apelación}

            \includegraphics[width=0.9\textwidth]{images/ActApelacion.png}

            \vspace{0.3em}

            \footnotesize
            \textbf{Flujo:}
            \begin{itemize}
                \scriptsize
                \item Solicitud → Evaluación
                \item → Smart contract
                \item → Actualización estado
                \item → Notificación
            \end{itemize}
        \end{column}
    \end{columns}

    \vspace{0.5em}

    \begin{exampleblock}{Automatización y Transparencia}
        Contratos inteligentes ejecutan lógica de negocio de forma predecible y auditable
    \end{exampleblock}
\end{frame}


% Sección 4: Resultados y Validación
% ==================== SECCIÓN 4: RESULTADOS Y VALIDACIÓN ====================
\sectionframe{Resultados y Validación Experimental}

% ==================== SLIDE 16: PLAN DE PRUEBAS ====================
\begin{frame}[shrink=5]{Plan de Pruebas: Cobertura Integral}
    \textbf{Estrategia de Validación Experimental:}

    \begin{itemize}
        \item \highlight{80 casos de prueba} automatizados (Vitest v3.2.4)
        \item \highlight{Tasa de éxito: 100\%} en todos los módulos
        \item \highlight{Tiempo total: 28.98 segundos}
    \end{itemize}

    \vspace{0.5em}

    \scriptsize
    \begin{table}
        \centering
        \begin{tabular}{p{3.5cm} c c p{5cm}}
            \toprule
            \textbf{Módulo} & \textbf{Pruebas} & \textbf{Éxito} & \textbf{Cobertura} \\
            \midrule
            Utilidades (Error Handler) & 7 & 7/7 & Manejo global errores, validaciones \\
            Servicios IPFS & 8 & 8/8 & Subida, recuperación, CIDs \\
            Integración IPFS & 13 & 13/13 & Inmutabilidad, content-addressed \\
            Seguridad: Validación & 16 & 16/16 & XSS, SQL injection, path traversal \\
            Seguridad: Archivos & 10 & 10/10 & Límites 10MB, tipos válidos \\
            API REST & 26 & 26/26 & CRUD, blockchain/IPFS, integridad \\
            \midrule
            \textbf{TOTAL} & \textbf{80} & \textbf{80/80} & \textbf{100\% cobertura} \\
            \bottomrule
        \end{tabular}
    \end{table}

    \begin{exampleblock}{Ingeniería de Software Moderna}
        No es solo un prototipo conceptual - es código de producción validado
    \end{exampleblock}
\end{frame}

% ==================== SLIDE 17: INMUTABILIDAD ====================
\begin{frame}[shrink=5]{Pruebas de Inmutabilidad: Núcleo del Sistema}
    \textbf{Casos de Prueba Críticos:}

    \vspace{0.5em}

    \scriptsize
    \begin{table}
        \centering
        \begin{tabular}{p{1cm} p{4.5cm} p{5cm}}
            \toprule
            \textbf{ID} & \textbf{Caso de Prueba} & \textbf{Resultado} \\
            \midrule
            IM-001 & Intento modificación directa en ledger & Transacción RECHAZADA por consenso \\
            \midrule
            IM-002 & Alteración de imagen en IPFS & CID diferente generado → Detección automática \\
            \midrule
            IM-003 & Verificación de trazabilidad & Historial completo inmutable preservado \\
            \midrule
            IM-004 & Validación de consenso & Consenso validado correctamente \\
            \bottomrule
        \end{tabular}
    \end{table}

    \vspace{0.5em}

    \textbf{Evidencia Técnica:}
    \begin{itemize}
        \footnotesize
        \item TX Hash registro: \texttt{0xbc03e11f...42c3c069}
        \item TX Hash actualización: \texttt{0x611b696e...d315f3e48}
        \item CID IPFS evidencia: \texttt{QmadhsypxKm7b2P2w...sp8eKMF}
    \end{itemize}

    \begin{alertblock}{Validación Experimental}
        El sistema REALMENTE previene manipulación - no es teórico, está comprobado
    \end{alertblock}
\end{frame}

% ==================== SLIDE 18: MÉTRICAS DE DESEMPEÑO ====================
\begin{frame}[shrink=5]{Métricas de Desempeño}
    \begin{columns}[T]
        \begin{column}{0.5\textwidth}
            \textbf{Tiempos de Respuesta Medidos:}

            \vspace{0.5em}

            \begin{itemize}
                \item Registro completo: \highlight{< 3 segundos}
                \item Consulta de multa: \highlight{< 1 segundo}
                \item Verificación integridad: \highlight{< 2 segundos}
            \end{itemize}

            \vspace{1em}

            \begin{exampleblock}{Criterio de Éxito}
                $\checkmark$ Tiempo publicación $\leq$ 3s

                $\checkmark$ Coincidencia 100\% hash

                $\checkmark$ Trazabilidad completa
            \end{exampleblock}
        \end{column}
        \begin{column}{0.5\textwidth}
            \scriptsize
            \begin{table}
                \centering
                \begin{tabular}{p{2.5cm} p{1.3cm} p{1.3cm}}
                    \toprule
                    \textbf{Métrica} & \textbf{FÉNIX} & \textbf{Prototipo} \\
                    \midrule
                    Integridad & Admin. & Cripto. \\
                    \midrule
                    Transparencia & Opaca & Pública \\
                    \midrule
                    Auditabilidad & Logs mod. & Inmutable \\
                    \midrule
                    SPOF & Sí & No \\
                    \midrule
                    Costos disputa & 155K PQRSD & $>$50\% ↓ \\
                    \midrule
                    Confianza & Instit. & Cripto. \\
                    \bottomrule
                \end{tabular}
                \caption*{Comparación FÉNIX vs Prototipo}
            \end{table}
        \end{column}
    \end{columns}

    \vspace{0.5em}

    \begin{block}{Viabilidad Técnica Demostrada}
        El sistema es RÁPIDO y SUPERIOR al actual en dimensiones críticas
    \end{block}
\end{frame}

% ==================== SLIDE 19: CUMPLIMIENTO DE OBJETIVOS ====================
\begin{frame}[shrink=5]{Cumplimiento de Objetivos}
    \scriptsize
    \begin{table}
        \centering
        \begin{tabular}{p{3.5cm} p{2.5cm} p{4.5cm}}
            \toprule
            \textbf{Objetivo Específico} & \textbf{Técnica Validación} & \textbf{Resultado} \\
            \midrule
            Inmutabilidad blockchain & Pruebas IM-002, IM-003 & 100\% coincidencia hash blockchain-IPFS \\
            \midrule
            Almacenamiento descentralizado & 13 pruebas integración & CIDs consistentes, <500ms subida \\
            \midrule
            API REST funcional & 80 casos prueba & 80/80 pruebas superadas \\
            \midrule
            Interfaz intuitiva & 95\% cobertura comp. & Flujo registro-verificación operativo \\
            \midrule
            Transparencia & Endpoint \texttt{/integrity} & Verificación sin intervención humana \\
            \midrule
            Viabilidad técnica & Pruebas rendimiento & <2s transacciones, arq. hexagonal \\
            \bottomrule
        \end{tabular}
    \end{table}

    \vspace{1em}

    \begin{exampleblock}{Validación de Hipótesis Central}
        \centering
        $\checkmark$ \textbf{TODOS los objetivos planteados fueron alcanzados y validados cuantitativamente}
    \end{exampleblock}
\end{frame}

% ==================== SLIDE 20: INTERFACES ====================
\begin{frame}[shrink=15]{Prototipo Funcional: Interfaces Desarrolladas}
    \begin{columns}[T]
        \begin{column}{0.5\textwidth}
            \centering
            \textbf{Dashboard Agente de Tránsito}

            \includegraphics[width=\textwidth]{images/UI3.png}

            \vspace{0.3em}

            \footnotesize
            \begin{itemize}
                \item Registro de comparendo
                \item Carga de evidencia
                \item Metadatos estructurados
            \end{itemize}
        \end{column}
        \begin{column}{0.5\textwidth}
            \centering
            \textbf{Consulta Ciudadana}

            \includegraphics[width=\textwidth]{images/UI5.png}

            \vspace{0.3em}

            \footnotesize
            \begin{itemize}
                \item Búsqueda por placa
                \item Verificación integridad blockchain
                \item Visualización evidencia IPFS
            \end{itemize}
        \end{column}
    \end{columns}

    \vspace{0.5em}

    \begin{block}{Aplicación Web Real}
        No es solo arquitectura abstracta - es una aplicación FUNCIONAL que demuestra viabilidad práctica
    \end{block}
\end{frame}


% Sección 5: Conclusiones y Cierre
% ==================== SECCIÓN 5: CONCLUSIONES Y CIERRE ====================
\sectionframe{Conclusiones y Trabajo Futuro}

% ==================== SLIDE 21: CONCLUSIONES PRINCIPALES ====================
\begin{frame}[shrink=5]{Conclusiones Principales}
    \textbf{1. Viabilidad Técnica Demostrada:}
    \begin{itemize}
        \item Hyperledger Fabric + Ethereum + IPFS dual es una combinación \textbf{viable} para gestión de fotocomparendos
    \end{itemize}

    \vspace{0.5em}

    \textbf{2. Garantías Criptográficas Validadas:}
    \begin{itemize}
        \item \highlight{100\%} de intentos de modificación rechazados
        \item Detección automática de alteraciones en evidencia
        \item Tiempos de respuesta \highlight{< 3s} (apto para producción)
    \end{itemize}

    \vspace{0.5em}

    \textbf{3. Arquitectura Escalable:}
    \begin{itemize}
        \item Backend con interfaces REST estándar
        \item Frontend React facilita adopción institucional
    \end{itemize}

    \vspace{0.5em}

    \textbf{4. Modelo de Confianza Alternativo:}
    \begin{itemize}
        \item Transición de confianza administrativa a \textbf{confianza criptográfica verificable}
    \end{itemize}

    \vspace{0.5em}

    \begin{exampleblock}{Contribución Principal}
        El proyecto NO solo propone - \textbf{DEMUESTRA} que blockchain híbrida puede transformar la gestión de registros públicos críticos
    \end{exampleblock}
\end{frame}

% ==================== SLIDE 22: TRABAJO FUTURO ====================
\begin{frame}[shrink=10]{Trabajo Futuro: Líneas de Evolución}
    \begin{columns}[T]
        \begin{column}{0.5\textwidth}
            \textbf{1. Escalamiento a Producción:}
            \begin{itemize}
                \footnotesize
                \item Red Fabric multi-organizacional (SDM, Policía, Auditoría)
                \item Private Data Collections para datos ultra-sensibles
                \item Infraestructura IPFS distribuida con políticas de replicación
            \end{itemize}

            \vspace{0.5em}

            \textbf{2. Piloto Controlado:}
            \begin{itemize}
                \footnotesize
                \item Convenio con Secretaría Distrital de Movilidad
                \item Dataset real: 5,000-10,000 multas
                \item Integración con SIMIT/RUNT nacional
            \end{itemize}
        \end{column}
        \begin{column}{0.5\textwidth}
            \textbf{3. Funcionalidades Avanzadas:}
            \begin{itemize}
                \footnotesize
                \item Módulo de pagos (PSE, billeteras digitales)
                \item Sistema de apelaciones en línea automatizado
                \item Dashboard analítico para toma de decisiones
            \end{itemize}

            \vspace{0.5em}

            \textbf{4. Replicabilidad Nacional:}
            \begin{itemize}
                \footnotesize
                \item Adaptación para otras ciudades colombianas
                \item Estandarización de Smart Contracts a nivel nacional
                \item Federación de redes Fabric intercity
            \end{itemize}
        \end{column}
    \end{columns}

    \vspace{0.5em}

    \begin{alertblock}{Proyección}
        Este proyecto es punto de partida, no punto final - abre múltiples líneas de investigación aplicada en GovTech
    \end{alertblock}
\end{frame}

% ==================== SLIDE 23: AGRADECIMIENTOS ====================
\begin{frame}[plain]
    \vfill
    \begin{center}
        {\Large \textbf{Agradecimientos}}
    \end{center}

    \vspace{1em}

    \begin{columns}[c]
        \begin{column}{0.45\textwidth}
            \centering
            \includegraphics[width=0.35\textwidth]{images/Escudo_UD.png}

            \vspace{0.5em}

            \small
            Universidad Distrital \\
            Francisco José de Caldas

            \vspace{0.3em}

            \footnotesize
            Facultad de Ingeniería \\
            Ingeniería de Sistemas

            \vspace{0.5em}

            \scriptsize
            \textbf{Director:} \\
            Julio Barón Velandia
        \end{column}

        \begin{column}{0.45\textwidth}
            \centering

            \vspace{1em}

            {\Huge \textbf{¿PREGUNTAS?}}

            \vspace{1.5em}

            \footnotesize
            \textbf{Autores:}

            \vspace{0.3em}

            \scriptsize
            Laura Catalina Preciado Ballén \\[0.2em]
            Cristian Stiven Guzmán Tovar

            \vspace{0.5em}

            \tiny
            Julio 2025
        \end{column}
    \end{columns}

    \vfill
\end{frame}


% ==================== FIN DEL DOCUMENTO ====================
\end{document}
