\documentclass[10pt, aspectratio=169]{beamer}

% ==================== TEMA Y CONFIGURACIÓN ====================
\usetheme{metropolis}
\metroset{block=fill}

% ==================== PACKAGES ====================
\usepackage[utf8]{inputenc}
\usepackage[spanish]{babel}
\usepackage{csquotes}
\usepackage{graphicx}
\usepackage{booktabs}
\usepackage{longtable}
\usepackage{adjustbox}
\usepackage{tikz}
\usepackage{xcolor}
\usepackage{hyperref}
\usepackage{tabularx}
\usepackage{multirow}
\usepackage{colortbl}
\usepackage{caption}

% ==================== BIBLIOGRAFÍA APA 7 ====================
\usepackage[style=apa, backend=biber, sortcites=true, sorting=nyt]{biblatex}
\DeclareLanguageMapping{spanish}{spanish-apa}
\addbibresource{../bibliography.bib}

% ==================== CONFIGURACIÓN DE COLORES ====================
% Colores institucionales Universidad Distrital
\definecolor{udblue}{RGB}{0,51,102}
\definecolor{udlightblue}{RGB}{51,102,153}
\definecolor{udgray}{RGB}{89,89,89}
\definecolor{udorange}{RGB}{230,126,34}

\setbeamercolor{palette primary}{bg=udlightblue,fg=white}
\setbeamercolor{palette secondary}{bg=udblue,fg=white}
\setbeamercolor{frametitle}{bg=udlightblue,fg=white}
\setbeamercolor{title}{fg=udblue}
\setbeamercolor{title separator}{fg=udlightblue}
\setbeamercolor{block title}{bg=udblue,fg=white}
\setbeamercolor{alerted text}{fg=udlightblue}

% ==================== CONFIGURACIÓN DE CAPTIONS APA 7 ====================
% En beamer las tablas usan caption arriba, en cursiva
\captionsetup[table]{labelfont=bf, textfont=it, singlelinecheck=false, justification=raggedright, skip=4pt}
\captionsetup[figure]{labelfont=bf, textfont=it, singlelinecheck=false, justification=raggedright, skip=4pt}

% ==================== CONFIGURACIÓN DE HYPERREF ====================
\hypersetup{
    colorlinks=true,
    linkcolor=udblue,
    urlcolor=udlightblue,
    citecolor=udblue
}

% ==================== INFORMACIÓN DE LA PRESENTACIÓN ====================
\title{Prototipo para apoyar el registro y trazabilidad de estados en el proceso de fotocomparendos aplicando tecnologías de redes distribuidas}

\author{%
    Laura Catalina Preciado Ballén \\
    Cristian Stiven Guzmán Tovar
}

\institute{%
    Universidad Distrital Francisco José de Caldas \\
    Facultad de Ingeniería \\
    Programa de Ingeniería de Sistemas \\[1em]
    \textbf{Director:} Julio Barón Velandia
}

\date{}

% ==================== CONFIGURACIONES ADICIONALES ====================
\setbeamertemplate{frame numbering}[fraction]
\setbeamertemplate{caption}[numbered]

\setlength{\leftmargini}{1em}
\setlength{\leftmarginii}{1.5em}

% ==================== COMANDOS PERSONALIZADOS ====================
\newcommand{\sectionframe}[1]{
    \section{#1}
    \begin{frame}[plain,noframenumbering]
        \centering
        \vfill
        {\Huge \textbf{#1}}
        \vfill
    \end{frame}
}

\newcommand{\highlight}[1]{\textcolor{udlightblue}{\textbf{#1}}}

% ==================== INICIO DEL DOCUMENTO ====================
\begin{document}

% ==================== PORTADA ====================
% PRESENTADOR: CRISTIAN — Saluda, presenta el título y los autores (~30s)
\begin{frame}[plain,noframenumbering]
    \vfill
    \begin{center}
        \includegraphics[width=0.15\textwidth]{images/Escudo_UD.png}

        \vspace{0.8em}

        \normalsize
        \inserttitle

        \vspace{1.5em}

        \insertauthor

        \vspace{1em}

        \small
        Director: Julio Barón Velandia

        \vspace{1.5em}

        Universidad Distrital Francisco José de Caldas \\
        Facultad de Ingeniería \\
        Programa de Ingeniería de Sistemas
    \end{center}
    \vfill
\end{frame}

% ==================== AGENDA ====================
% PRESENTADOR: CRISTIAN — Presenta la estructura de la sustentación (~30s)
\begin{frame}{Agenda}
    \tableofcontents[hideallsubsections]
\end{frame}

% ==================== CONTENIDO ====================
% Contexto (sin sección formal - no aparece en agenda)
% ==================== CONTEXTO (sin \section - no aparece en agenda) ====================
% PRESENTADOR: CATALINA — Contexto completo (~2 min)

\section{Contexto y formulación del problema}

% ==================== SLIDE: ESCALA DEL PROBLEMA ====================
% CATALINA: Presenta cifras del sistema FÉNIX (~1 min)
\begin{frame}{Contexto: el sistema de fotocomparendos en Bogotá}
    \begin{columns}[T]
        \begin{column}{0.45\textwidth}
            \textbf{Escala operativa (Sistema FÉNIX):}
            \begin{itemize}
                \item \highlight{1.9 millones} de comparendos emitidos entre 2018--2024 \parencite{sdm2024estadisticas}
                \item \highlight{457,000} comparendos semestrales en promedio
                \item Arquitectura centralizada (BD relacional)
            \end{itemize}
        \end{column}
        \begin{column}{0.55\textwidth}
            \centering
            \textbf{Figura 1}

            \textit{Comparendos emitidos por semestre}

            \includegraphics[width=\textwidth,height=0.55\textheight,keepaspectratio]{images/numComparendos.png}

            \footnotesize
            Fuente: Secretaría Distrital de Movilidad (2024).
        \end{column}
    \end{columns}
\end{frame}

% ==================== SLIDE: INDICADORES DE LA PROBLEMÁTICA ====================
% CATALINA: Presenta indicadores clave del problema (~1 min)
\begin{frame}{Indicadores de la problemática}
    \begin{columns}[T]
        \begin{column}{0.5\textwidth}
            \begin{block}{Gestión ciudadana}
                \begin{itemize}
                    \item Tasa de impugnación: \highlight{34.1\%}
                    \item Carga operativa: \highlight{155,854 PQRSD} semestrales
                \end{itemize}
            \end{block}

            \vspace{0.5em}

            \begin{block}{Impacto fiscal}
                \begin{itemize}
                    \item Presunto detrimento patrimonial: \highlight{\$8,000 millones} \parencite{contraloria2024detrimento}
                \end{itemize}
            \end{block}
        \end{column}
        \begin{column}{0.5\textwidth}
            \textbf{Vulnerabilidades identificadas:}
            \begin{itemize}
                \item Fraude a ciudadanos mediante intermediarios ilegales \parencite{semana2023juzto}
                \item Confianza en administradores centrales sin garantías criptográficas
                \item Inmutabilidad no verificable por la ciudadanía
                \item Auditoría opaca para el control institucional
            \end{itemize}
        \end{column}
    \end{columns}
\end{frame}

% ==================== SLIDE: PREGUNTA DE INVESTIGACIÓN ====================
% CATALINA: Lee la pregunta de investigación, limitaciones e hipótesis (~1 min)
\begin{frame}{Formulación del problema}
    \begin{exampleblock}{Pregunta de investigación}
        \centering
        \small
        \textit{¿Cómo mitigar el riesgo de pérdida o alteración de la integridad de los datos asociados a todos los estados en el proceso de fotocomparendos en Bogotá mediante el uso de tecnologías de redes distribuidas que garanticen el registro, la trazabilidad, la autenticidad y la confidencialidad de la información?}
    \end{exampleblock}

    \vspace{0.3em}

    \footnotesize
    \textbf{Limitaciones del modelo actual (FÉNIX):} confianza en administradores centrales, inmutabilidad no garantizada criptográficamente, trazabilidad dependiente de controles internos, auditoría opaca para la ciudadanía.

    \vspace{0.3em}

    \normalsize
    \begin{alertblock}{Hipótesis}
        \small
        Las tecnologías de redes distribuidas (blockchain + IPFS) pueden proporcionar garantías criptográficas de integridad y transparencia verificable sin intermediarios.
    \end{alertblock}
\end{frame}



% Sección 3: Objetivos (después de mostrar cómo se resolvió)
\input{slides/01_objetivos}

% Sección 1: Metodología
% ==================== SECCIÓN 2: METODOLOGÍA ====================
% PRESENTADORES: CATALINA (enfoque + híbrida) → CRISTIAN (actores) → CATALINA (actividades) → CRISTIAN (sistema) → CATALINA (alcance)
\section{Metodología e implementación}

% ==================== SLIDE: ENFOQUE METODOLÓGICO ====================
% CATALINA: Justifica el modelo de prototipos y las fases (~1.5 min)
\begin{frame}{Enfoque metodológico: desarrollo por prototipos}
    \textbf{Justificación del modelo:}
    \begin{itemize}
        \item \textbf{Naturaleza exploratoria:} integración de tecnologías emergentes sin antecedentes en el contexto local
        \item \textbf{Requisitos evolutivos:} marco normativo y tecnológico en constante actualización
        \item \textbf{Verificación temprana:} contrastar la hipótesis central antes de un desarrollo a escala
    \end{itemize}

    \textbf{Fases del desarrollo:}
    \begin{enumerate}
        \item \textbf{Análisis de requisitos} $\rightarrow$ Marco legal + auditorías
        \item \textbf{Diseño arquitectónico} $\rightarrow$ Descomposición por niveles de confianza
        \item \textbf{Implementación iterativa} $\rightarrow$ Backend + Frontend + Smart Contracts
        \item \textbf{Pruebas y verificación} $\rightarrow$ 80 casos automatizados
    \end{enumerate}

    \begin{block}{Decisión metodológica}
        \footnotesize
        El modelo de prototipos permite mitigar riesgos técnicos y facilitar ajustes iterativos ante cambios normativos o tecnológicos.
    \end{block}
\end{frame}

% ==================== SLIDE: ARQUITECTURA HÍBRIDA ====================
% CATALINA: Explica el problema y la tabla de componentes (~1.5 min)
\begin{frame}{Arquitectura híbrida: decisión de diseño}
    \textbf{Problema:} ninguna plataforma blockchain individual satisface todos los requisitos.

    \begin{columns}[T]
        \begin{column}{0.5\textwidth}
            \begin{itemize}
                \item Privacidad de datos personales (Ley 1581/2012)
                \item Transparencia pública ciudadana (Ley 1712/2014)
            \end{itemize}
        \end{column}
        \begin{column}{0.5\textwidth}
            \begin{itemize}
                \item Rendimiento (457,000 comparendos semestrales)
                \item Costos operativos predecibles
            \end{itemize}
        \end{column}
    \end{columns}

    \vspace{0.5em}

    \centering
    \textbf{Tabla 1}

    \textit{Componentes de la arquitectura híbrida}

    \footnotesize
    \begin{table}
        \centering
        \begin{tabular}{p{2.2cm} p{3cm} p{3.8cm} p{1.5cm}}
            \toprule
            \textbf{Componente} & \textbf{Tecnología} & \textbf{Justificación} & \textbf{TPS} \\
            \midrule
            Capa privada & Hyperledger Fabric v2.5 & Control de acceso PKI, sin gas fees & 2K--20K \\
            Capa pública & Ethereum (Sepolia) & Verificación ciudadana & 15--30 \\
            Storage privado & IPFS privado & Evidencias sensibles & -- \\
            Storage público & IPFS público & Hashes de verificación & -- \\
            \bottomrule
        \end{tabular}
    \end{table}

    \footnotesize
    Fuente: Elaboración propia.
\end{frame}

% ==================== SLIDE: ACTORES Y FUNCIONALIDADES ====================
% CRISTIAN: Describe los actores y sus funcionalidades con el diagrama (~1 min)
\begin{frame}{Actores y funcionalidades principales}
    \begin{columns}[T]
        \begin{column}{0.55\textwidth}
            \centering
            \textbf{Figura 1}

            \textit{Diagrama de casos de uso}

            \includegraphics[width=\textwidth]{images/CasosUso.png}

            \footnotesize
            Fuente: Elaboración propia.
        \end{column}
        \begin{column}{0.45\textwidth}
            \textbf{Actores identificados:}

            \vspace{0.5em}

            \begin{enumerate}
                \item \textbf{Agente de tránsito}
                \begin{itemize}
                    \footnotesize
                    \item Registrar comparendo
                    \item Actualizar estado
                \end{itemize}

                \vspace{0.3em}

                \item \textbf{Ciudadano}
                \begin{itemize}
                    \footnotesize
                    \item Consultar multa
                    \item Verificar autenticidad
                    \item Apelar
                \end{itemize}

                \vspace{0.3em}

                \item \textbf{Administrador}
                \begin{itemize}
                    \footnotesize
                    \item Gestionar sistema
                    \item Auditar operaciones
                \end{itemize}
            \end{enumerate}
        \end{column}
    \end{columns}
\end{frame}

% ==================== SLIDE: DIAGRAMAS DE ACTIVIDADES ====================
% CATALINA: Explica los flujos de registro de multa y apelación (~1.5 min)
\begin{frame}{Flujos de proceso: diagramas de actividades}
    \begin{columns}[T]
        \begin{column}{0.5\textwidth}
            \centering
            \textbf{Figura 2}

            \textit{Registro de multa}

            \includegraphics[height=0.65\textheight]{images/ActMulta.png}

            \footnotesize
            Fuente: Elaboración propia.
        \end{column}
        \begin{column}{0.5\textwidth}
            \centering
            \textbf{Figura 3}

            \textit{Proceso de apelación}

            \includegraphics[height=0.65\textheight]{images/ActApelacion.png}

            \footnotesize
            Fuente: Elaboración propia.
        \end{column}
    \end{columns}
\end{frame}

% ==================== SLIDE: ARQUITECTURA DEL SISTEMA ====================
% CRISTIAN: Explica el diagrama de despliegue y las 5 capas (~1 min)
\begin{frame}{Arquitectura del sistema}
    \begin{center}
        \textbf{Figura 4}

        \textit{Diagrama de despliegue del sistema}

        \includegraphics[height=0.5\textheight]{images/Despliegue.png}

        \footnotesize
        Fuente: Elaboración propia.
    \end{center}

    \vspace{0.2em}

    \footnotesize
    \textbf{Capas:} \textbf{1.} Frontend React ~|~ \textbf{2.} API REST Node.js/Express ~|~ \textbf{3.} Hyperledger Fabric ~|~ \textbf{4.} Ethereum + IPFS público ~|~ \textbf{5.} IPFS privado
\end{frame}

% ==================== SLIDE: ALCANCE Y LIMITACIONES ====================
% CATALINA: Presenta las delimitaciones y la nota metodológica (~1.5 min)
\begin{frame}{Alcance y delimitaciones del estudio}
    \textbf{Delimitaciones metodológicas del prototipo:}

    \vspace{0.3em}

    \begin{itemize}
        \item \textbf{Datos sintéticos:} la verificación se realizó con datos generados mediante scripts de prueba, dado que no se disposede acceso a datos reales del FÉNIX, RUNT ni SIMIT.

        \vspace{0.2em}

        \item \textbf{Cobertura parcial de estados:} se implementaron 5 de los 8 estados del ciclo de vida (PENDING, PAID, APPEALED, RESOLVED\_APPEAL, CANCELLED).

        \vspace{0.2em}

        \item \textbf{Volumen controlado:} se emplearon entre 50 y 100 comparendos de prueba, frente a los 457,000 semestrales registrados en producción.

        \vspace{0.2em}

        \item \textbf{Verificación técnica:} los resultados corresponden a una \textit{verificación} en entorno controlado, no a una \textit{validación} operativa institucional.
    \end{itemize}

    \vspace{0.2em}

    \begin{block}{Nota metodológica}
        \footnotesize
        Se distingue entre \textit{verificación} (el sistema cumple las especificaciones de diseño) y \textit{validación} (el sistema opera adecuadamente en condiciones reales). Este trabajo se enmarca en la primera categoría.
    \end{block}
\end{frame}


% Sección 2: Resultados
% ==================== SECCIÓN 3: RESULTADOS ====================
% PRESENTADORES: CRISTIAN (plan + inmutabilidad) → CATALINA (métricas)
\section{Validación y pruebas}

% ==================== SLIDE: PLAN DE PRUEBAS ====================
% CRISTIAN: Presenta la estrategia y la tabla de pruebas por módulo (~1.5 min)
\begin{frame}{Plan de pruebas: cobertura del prototipo}
    \textbf{Estrategia:} \highlight{80 casos de prueba} automatizados (Vitest v3.2.4) --- \highlight{Tasa de éxito: 100\%} --- \highlight{Tiempo total: 28.98s}

    \vspace{0.3em}

    \footnotesize
    \begin{table}
        \caption{Resultados del plan de pruebas por módulo}
        \centering
        \begin{tabular}{p{3.5cm} c c p{4.5cm}}
            \toprule
            \textbf{Módulo} & \textbf{Pruebas} & \textbf{Éxito} & \textbf{Cobertura} \\
            \midrule
            Utilidades (Error Handler) & 7 & 7/7 & Manejo global de errores \\
            Servicios IPFS & 8 & 8/8 & Subida, recuperación, CIDs \\
            Integración IPFS & 13 & 13/13 & Inmutabilidad, content-addressed \\
            Seguridad: Validación & 16 & 16/16 & XSS, SQL injection, path traversal \\
            Seguridad: Archivos & 10 & 10/10 & Límites 10MB, tipos válidos \\
            API REST & 26 & 26/26 & CRUD, blockchain/IPFS \\
            \midrule
            \textbf{Total} & \textbf{80} & \textbf{80/80} & \textbf{100\% cobertura funcional} \\
            \bottomrule
        \end{tabular}
    \end{table}
\end{frame}

% ==================== SLIDE: PRUEBAS DE INMUTABILIDAD ====================
% CRISTIAN: Explica los 4 casos, evidencia técnica y resultado (~1.5 min)
\begin{frame}{Pruebas de inmutabilidad}
    \footnotesize
    \begin{table}
        \caption{Resultados de pruebas de inmutabilidad}
        \centering
        \begin{tabular}{p{1cm} p{4.5cm} p{5cm}}
            \toprule
            \textbf{ID} & \textbf{Caso de prueba} & \textbf{Resultado} \\
            \midrule
            IM-001 & Modificación directa en ledger & Transacción rechazada por consenso \\
            \midrule
            IM-002 & Alteración de imagen en IPFS & CID diferente $\rightarrow$ Detección automática \\
            \midrule
            IM-003 & Verificación de trazabilidad & Historial inmutable preservado \\
            \midrule
            IM-004 & Validación de consenso & Consenso validado correctamente \\
            \bottomrule
        \end{tabular}
    \end{table}

    \vspace{0.2em}

    \footnotesize
    \textbf{Evidencia técnica:} TX Hash registro: \texttt{0xbc03e11f...42c3c069} | TX Hash actualización: \texttt{0x611b696e...d315f3e48} | CID IPFS: \texttt{QmadhsypxKm7b2P2w...sp8eKMF}

    \vspace{0.2em}

    \begin{exampleblock}{Resultado}
        En el entorno experimental, el prototipo rechazó satisfactoriamente el 100\% de los intentos de modificación no autorizada.
    \end{exampleblock}
\end{frame}

% ==================== SLIDE: MÉTRICAS DE DESEMPEÑO ====================
% CATALINA: Presenta tiempos, criterios de éxito e interfaces (~1.5 min)
\begin{frame}{Métricas de desempeño}
    \begin{columns}[T]
        \begin{column}{0.5\textwidth}
            \textbf{Tiempos de respuesta medidos:}

            \vspace{0.5em}

            \begin{itemize}
                \item Registro completo: \highlight{$<$ 3 segundos}
                \item Consulta de multa: \highlight{$<$ 1 segundo}
                \item Verificación de integridad: \highlight{$<$ 2 segundos}
            \end{itemize}

            \vspace{0.8em}

            \begin{exampleblock}{Criterios de éxito}
                $\checkmark$ Tiempo de publicación $\leq$ 3s

                $\checkmark$ Coincidencia 100\% hash

                $\checkmark$ Trazabilidad completa en entorno de prueba
            \end{exampleblock}
        \end{column}
        \begin{column}{0.5\textwidth}
            \centering
            \textbf{Interfaces desarrolladas:}

            \vspace{0.3em}

            \includegraphics[width=0.95\textwidth,height=0.25\textheight,keepaspectratio]{images/UI3.png}
            \captionof{figure}{\textit{Dashboard del agente de tránsito}}

            \vspace{0.2em}

            \includegraphics[height=0.2\textheight]{images/UI5.png}
            \captionof{figure}{\textit{Consulta y verificación ciudadana}}
        \end{column}
    \end{columns}
\end{frame}


% Sección 4: Conclusiones
% ==================== SECCIÓN 4: CONCLUSIONES ====================
% PRESENTADORES: CATALINA (conclusiones) → CRISTIAN (respuesta a la pregunta)
\section{Conclusiones}

% ==================== SLIDE: CONCLUSIONES TÉCNICAS ====================
% CATALINA: Presenta las 3 conclusiones principales (~1.5 min)
\begin{frame}{Conclusiones principales}
    \textbf{1. Viabilidad técnica demostrada:}
    \begin{itemize}
        \item La arquitectura híbrida (Hyperledger Fabric + Ethereum + IPFS dual) demostró ser viable para la gestión de fotocomparendos en el entorno experimental.
    \end{itemize}

    \vspace{0.3em}

    \textbf{2. Garantías criptográficas verificadas:}
    \begin{itemize}
        \item 100\% de intentos de modificación no autorizada rechazados satisfactoriamente.
        \item Detección automática de alteraciones mediante \textit{content-addressing} (CIDs).
        \item Tiempos de respuesta dentro de los criterios de aceptación ($\leq$ 3s).
    \end{itemize}

    \vspace{0.3em}

    \textbf{3. Modelo de confianza alternativo:}
    \begin{itemize}
        \item Transición hacia confianza criptográfica verificable, conciliando privacidad (Ley 1581/2012) y transparencia (Ley 1712/2014).
    \end{itemize}
\end{frame}

% ==================== SLIDE: RESPUESTA A LA PREGUNTA ====================
% CRISTIAN: Responde la pregunta con evidencia y oportunidades (~1.5 min)
\begin{frame}{Respuesta a la pregunta de investigación}
    \begin{exampleblock}{Respuesta}
        La respuesta a la pregunta de investigación es \textbf{afirmativa} dentro del alcance experimental definido: las tecnologías de redes distribuidas permiten mitigar el riesgo de alteración de la integridad de los datos en el proceso de fotocomparendos.
    \end{exampleblock}

    \vspace{0.8em}

    \begin{columns}[T]
        \begin{column}{0.5\textwidth}
            \textbf{Evidencia obtenida:}
            \begin{itemize}
                \item Registro y trazabilidad de 5 estados del ciclo de vida con inmutabilidad criptográfica
                \item Detección automática de alteraciones en documentos y evidencias
                \item Modelo de confianza verificable sin intermediarios
            \end{itemize}
        \end{column}
        \begin{column}{0.5\textwidth}
            \textbf{Oportunidades de extensión:}
            \begin{itemize}
                \item Escalamiento a volúmenes operativos reales (457,000 comparendos semestrales)
                \item Integración con sistemas institucionales (SIMIT, RUNT)
                \item Incorporación de los estados restantes del proceso
                \item Estudios de aceptación tecnológica
            \end{itemize}
        \end{column}
    \end{columns}
\end{frame}

% ==================== SLIDE: EVIDENCIA DE CUMPLIMIENTO ====================
% CATALINA: Recorre la tabla y cierra con la síntesis (~1 min)
\begin{frame}{Evidencia de cumplimiento de objetivos}
    \begin{table}
        \caption{Cumplimiento de objetivos específicos}
        \centering
        \footnotesize
        \begin{tabular}{p{3cm} p{2.5cm} p{4.8cm}}
            \toprule
            \textbf{Objetivo} & \textbf{Validación} & \textbf{Resultado} \\
            \midrule
            Análisis de vulnerabilidades & Auditoría documental y normativa & Brechas en FÉNIX identificadas; requisitos definidos \\
            \midrule
            Desarrollo prototipo híbrido & Implementación iterativa & Arq. hexagonal: Fabric, Ethereum, IPFS dual, API REST \\
            \midrule
            Evaluación viabilidad técnica & 80 pruebas (Vitest v3.2.4) & 100\% superadas; $\leq$ 3s; 100\% hash match \\
            \bottomrule
        \end{tabular}
    \end{table}

    \begin{exampleblock}{Síntesis}
        Los tres objetivos específicos se cumplieron dentro del alcance experimental definido.
    \end{exampleblock}
\end{frame}



% Sección 5: Trabajo futuro
% ==================== SECCIÓN 5: TRABAJO FUTURO ====================
% PRESENTADORES: CRISTIAN (líneas de evolución) → CATALINA (agradecimientos y cierre)
\section{Trabajo futuro}

% ==================== SLIDE: LÍNEAS DE EVOLUCIÓN ====================
% CRISTIAN: Presenta las 4 líneas de evolución y la perspectiva (~1.5 min)
\begin{frame}[shrink=10]{Líneas de evolución}
    \begin{columns}[T]
        \begin{column}{0.5\textwidth}
            \textbf{1. Validación operativa:}
            \begin{itemize}
                \footnotesize
                \item Piloto controlado con 5,000--10,000 multas reales
                \item Integración con SIMIT/RUNT mediante APIs reales
                \item Estudios de aceptación tecnológica (TAM/UTAUT) con agentes de tránsito y ciudadanos
            \end{itemize}

            \vspace{0.5em}

            \textbf{2. Escalamiento a producción:}
            \begin{itemize}
                \footnotesize
                \item Red Fabric multi-organizacional (SDM, Policía, Contraloría)
                \item Migración a soluciones Layer 2 (Polygon, Arbitrum)
                \item Auditoría formal de seguridad (Slither, MythX)
            \end{itemize}
        \end{column}
        \begin{column}{0.5\textwidth}
            \textbf{3. Extensión funcional:}
            \begin{itemize}
                \footnotesize
                \item Oráculos certificadores para el estado NOTIFICADA
                \item Módulo de pagos (PSE, billeteras digitales)
                \item Sistema de apelaciones en línea automatizado
            \end{itemize}

            \vspace{0.5em}

            \textbf{4. Replicabilidad:}
            \begin{itemize}
                \footnotesize
                \item Adaptación para otras ciudades colombianas
                \item Estandarización de contratos inteligentes a nivel nacional
                \item Federación de redes Fabric intercity
            \end{itemize}
        \end{column}
    \end{columns}

    \vspace{0.5em}

    \begin{exampleblock}{Perspectiva}
        Los resultados obtenidos constituyen una base técnica para futuras investigaciones orientadas a la validación operativa e institucional del sistema propuesto.
    \end{exampleblock}
\end{frame}

% ==================== SLIDE: REFERENCIAS ====================
\begin{frame}{Referencias principales}
    \footnotesize
    % Solo mostrar referencias clave seleccionadas
    \begin{itemize}
        \item \fullcite{antonopoulos2023mastering}
        \vspace{0.3em}
        \item \fullcite{Informe170100005424}
        \vspace{0.3em}
        \item \fullcite{vanSteen2017}
        \vspace{0.3em}
        \item \fullcite{nakamoto2008bitcoin}
    \end{itemize}
\end{frame}

% ==================== SLIDE: AGRADECIMIENTOS ====================
% CATALINA: Agradece al director, la universidad (~30s)
\begin{frame}[plain]
    \vfill

    \begin{center}
        \includegraphics[width=0.12\textwidth]{images/Escudo_UD.png}

        \vspace{0.7em}

        {\Large \textbf{Agradecimientos}}

        \vspace{1.2em}

        \small
        \textbf{Universidad Distrital Francisco José de Caldas} \\
        Facultad de Ingeniería \\
        Programa de Ingeniería de Sistemas

        \vspace{1.5em}

        \footnotesize
        \textbf{Director} \\[0.4em]
        \normalsize
        Julio Barón Velandia
    \end{center}

    \vfill
\end{frame}

% ==================== SECCIÓN: PREGUNTAS ====================
% La sección aparece en la agenda y genera la slide de índice automáticamente
\section{¿Preguntas?}


% ==================== REFERENCIAS ====================
% Solo se muestra si hay preguntas sobre fuentes — no requiere presentador
\begin{frame}[allowframebreaks]{Referencias}
    \footnotesize
    \printbibliography[heading=none]
\end{frame}

% ==================== FIN DEL DOCUMENTO ====================
\end{document}
