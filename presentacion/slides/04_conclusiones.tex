% ==================== SECCIÓN 4: CONCLUSIONES ====================
% PRESENTADORES: CATALINA (conclusiones) → CRISTIAN (respuesta a la pregunta)
\section{Conclusiones}

% ==================== SLIDE: CONCLUSIONES TÉCNICAS ====================
% CATALINA: Presenta las 3 conclusiones principales (~1.5 min)
\begin{frame}{Conclusiones principales}
    \textbf{1. Viabilidad técnica demostrada:}
    \begin{itemize}
        \item La arquitectura híbrida (Hyperledger Fabric + Ethereum + IPFS dual) demostró ser viable para la gestión de fotocomparendos en el entorno experimental.
    \end{itemize}

    \vspace{0.3em}

    \textbf{2. Garantías criptográficas verificadas:}
    \begin{itemize}
        \item 100\% de intentos de modificación no autorizada rechazados satisfactoriamente.
        \item Detección automática de alteraciones mediante \textit{content-addressing} (CIDs).
        \item Tiempos de respuesta dentro de los criterios de aceptación ($\leq$ 3s).
    \end{itemize}

    \vspace{0.3em}

    \textbf{3. Modelo de confianza alternativo:}
    \begin{itemize}
        \item Transición hacia confianza criptográfica verificable, conciliando privacidad (Ley 1581/2012) y transparencia (Ley 1712/2014).
    \end{itemize}
\end{frame}

% ==================== SLIDE: RESPUESTA A LA PREGUNTA ====================
% CRISTIAN: Responde la pregunta con evidencia y oportunidades (~1.5 min)
\begin{frame}{Respuesta a la pregunta de investigación}
    \begin{exampleblock}{Respuesta}
        La respuesta a la pregunta de investigación es \textbf{afirmativa} dentro del alcance experimental definido: las tecnologías de redes distribuidas permiten mitigar el riesgo de alteración de la integridad de los datos en el proceso de fotocomparendos.
    \end{exampleblock}

    \vspace{0.8em}

    \begin{columns}[T]
        \begin{column}{0.5\textwidth}
            \textbf{Evidencia obtenida:}
            \begin{itemize}
                \item Registro y trazabilidad de 5 estados del ciclo de vida con inmutabilidad criptográfica
                \item Detección automática de alteraciones en documentos y evidencias
                \item Modelo de confianza verificable sin intermediarios
            \end{itemize}
        \end{column}
        \begin{column}{0.5\textwidth}
            \textbf{Oportunidades de extensión:}
            \begin{itemize}
                \item Escalamiento a volúmenes operativos reales (457,000 comparendos semestrales)
                \item Integración con sistemas institucionales (SIMIT, RUNT)
                \item Incorporación de los estados restantes del proceso
                \item Estudios de aceptación tecnológica
            \end{itemize}
        \end{column}
    \end{columns}
\end{frame}
