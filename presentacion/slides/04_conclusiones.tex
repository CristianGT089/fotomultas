% ==================== SECCIÓN 4: CONCLUSIONES ====================
% PRESENTADORES: CATALINA (conclusiones) → CRISTIAN (respuesta a la pregunta)
\section{Conclusiones y aportes}

% ==================== SLIDE: CONCLUSIONES TÉCNICAS ====================
% CATALINA: Presenta las 3 conclusiones principales (~1.5 min)
\begin{frame}{Conclusiones principales}
    \textbf{1. Viabilidad técnica demostrada:}
    \begin{itemize}
        \item La arquitectura híbrida (Hyperledger Fabric + Ethereum + IPFS dual) demostró ser viable para la gestión de fotocomparendos en el entorno experimental.
    \end{itemize}

    \vspace{0.3em}

    \textbf{2. Garantías criptográficas verificadas:}
    \begin{itemize}
        \item 100\% de intentos de modificación no autorizada rechazados satisfactoriamente.
        \item Detección automática de alteraciones mediante \textit{content-addressing} (CIDs).
        \item Tiempos de respuesta dentro de los criterios de aceptación ($\leq$ 3s).
    \end{itemize}

    \vspace{0.3em}

    \textbf{3. Modelo de confianza alternativo:}
    \begin{itemize}
        \item Transición hacia confianza criptográfica verificable, conciliando privacidad (Ley 1581/2012) y transparencia (Ley 1712/2014).
    \end{itemize}
\end{frame}

