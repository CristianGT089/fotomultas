% ==================== SECCIÓN 2: JUSTIFICACIÓN Y OBJETIVOS ====================
\section{Justificación y Objetivos}

% ==================== SLIDE 6: OBJETIVOS ====================
\begin{frame}{Objetivos del Proyecto}
    \begin{exampleblock}{Objetivo General}
        Desarrollar un \textbf{prototipo con arquitectura híbrida blockchain} para apoyar el registro y trazabilidad de estados en el proceso de fotocomparendos, aplicando tecnologías de redes distribuidas para fortalecer la integridad, autenticidad y confidencialidad de la información.
    \end{exampleblock}

    \vspace{1em}

    \textbf{Objetivos Específicos:}

    \begin{enumerate}
        \item \textbf{Analizar} el proceso actual y marco normativo para identificar requisitos funcionales, no funcionales y vulnerabilidades
        \vspace{0.5em}

        \item \textbf{Desarrollar} prototipo con arquitectura híbrida (blockchain + IPFS dual) con interfaz demostrable
        \vspace{0.5em}

        \item \textbf{Evaluar} viabilidad mediante plan de pruebas funcionales, de inmutabilidad y métricas de desempeño
    \end{enumerate}

    \begin{block}{Enfoque Metodológico}
        Análisis → Diseño → Validación
    \end{block}
\end{frame}

% ==================== SLIDE 7: ESTADO DEL ARTE ====================
\begin{frame}[shrink=10]{Estado del Arte: Posicionamiento Científico}
    \scriptsize
    \begin{table}
        \centering
        \begin{tabular}{p{2cm} p{2.2cm} p{3.5cm} p{3.3cm}}
            \toprule
            \textbf{Trabajo} & \textbf{Tecnologías} & \textbf{Limitaciones} & \textbf{Aporte} \\
            \midrule
            Yousfi et al. (2022) & Blockchain pública & Alto costo gas, privacidad limitada & Modelo blockchain-tráfico \\
            \midrule
            Chen et al. (2024) & BD + Blockchain & Dependencia servidor central & Hash de actas en blockchain \\
            \midrule
            Joseph (2023) & Hyperledger + IPFS & Complejidad escalamiento & Arquitectura permisionada \\
            \midrule
            Omar et al. (2024) & Blockchain híbrida & Integración parcial & Gestión descentralizada \\
            \midrule
            Anand \& Singh (2024) & IPFS + Blockchain & Persistencia IPFS & Almacenamiento distribuido \\
            \bottomrule
        \end{tabular}
    \end{table}

    \vspace{0.5em}

    \begin{alertblock}{Brecha Identificada}
        \textbf{Ningún trabajo previo integra:}
        \begin{itemize}
            \item Blockchain híbrida (privada + pública)
            \item IPFS dual (privado + público)
            \item Flujo completo de fotocomparendos
            \item Validación experimental con 80 pruebas automatizadas
        \end{itemize}
    \end{alertblock}
\end{frame}

% ==================== SLIDE 8: EVIDENCIA BIBLIOMÉTRICA ====================
\begin{frame}[shrink=5]{Relevancia Científica del Tema}
    \textbf{Análisis Bibliométrico:}
    \begin{itemize}
        \item \highlight{121 referencias bibliográficas} revisadas
        \item Tendencia creciente: blockchain + gobierno electrónico
        \item Áreas emergentes: e-governance, transparency, smart contracts
    \end{itemize}

    \vspace{0.5em}

    \begin{columns}[T]
        \begin{column}{0.33\textwidth}
            \centering
            \includegraphics[width=\textwidth]{images/MapaBibliometrix.png}
            \tiny Producción científica mundial
        \end{column}
        \begin{column}{0.33\textwidth}
            \centering
            \includegraphics[width=\textwidth]{images/GraficoLineas.png}
            \tiny Evolución temporal
        \end{column}
        \begin{column}{0.33\textwidth}
            \centering
            \includegraphics[width=\textwidth]{images/MapaTematico.jpeg}
            \tiny Mapa temático
        \end{column}
    \end{columns}

    \vspace{0.5em}

    \begin{block}{Alineación Internacional}
        El proyecto se fundamenta en investigación rigurosa y está alineado con tendencias científicas globales en GovTech.
    \end{block}
\end{frame}

% ==================== SLIDE 9: JUSTIFICACIÓN TECNOLÓGICA ====================
\begin{frame}{¿Por Qué Blockchain?}
    \textbf{Requisitos No Negociables del Dominio:}

    \begin{columns}[T]
        \begin{column}{0.5\textwidth}
            \begin{itemize}
                \item \highlight{Inmutabilidad criptográfica} verificable
                \item \highlight{Verificación sin confianza} (trustless)
                \item \highlight{Precedente legal} reconocido (eIDAS)
                \item \highlight{Auditabilidad completa} con timestamps
            \end{itemize}
        \end{column}
        \begin{column}{0.5\textwidth}
            \textbf{Por qué NO bases de datos tradicionales:}
            \begin{itemize}
                \item Admins con privilegios pueden alterar logs
                \item Verificación depende de APIs de la misma entidad
                \item NO hay resistencia computacional a manipulación
            \end{itemize}
        \end{column}
    \end{columns}

    \vspace{1em}

    \begin{exampleblock}{Conclusión Tecnológica}
        Blockchain no es una moda tecnológica - es la \textbf{única solución técnica} que cumple requisitos legales y de confianza del dominio de fotocomparendos.
    \end{exampleblock}
\end{frame}
