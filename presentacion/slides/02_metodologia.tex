% ==================== SECCIÓN 2: METODOLOGÍA ====================
% PRESENTADORES: CATALINA (enfoque + híbrida) → CRISTIAN (actores) → CATALINA (actividades) → CRISTIAN (sistema) → CATALINA (alcance)
\section{Metodología e implementación}

% ==================== SLIDE: ENFOQUE METODOLÓGICO ====================
% CATALINA: Justifica el modelo de prototipos y las fases (~1.5 min)
\begin{frame}{Enfoque metodológico: desarrollo por prototipos}
    \textbf{Justificación del modelo:}
    \begin{itemize}
        \item \textbf{Naturaleza exploratoria:} integración de tecnologías emergentes sin antecedentes en el contexto local
        \item \textbf{Requisitos evolutivos:} marco normativo y tecnológico en constante actualización
        \item \textbf{Verificación temprana:} contrastar la hipótesis central antes de un desarrollo a escala
    \end{itemize}

    \textbf{Fases del desarrollo:}
    \begin{enumerate}
        \item \textbf{Análisis de requisitos} $\rightarrow$ Marco legal + auditorías
        \item \textbf{Diseño arquitectónico} $\rightarrow$ Descomposición por niveles de confianza
        \item \textbf{Implementación iterativa} $\rightarrow$ Backend + Frontend + Smart Contracts
        \item \textbf{Pruebas y verificación} $\rightarrow$ 80 casos automatizados
    \end{enumerate}

    \begin{block}{Decisión metodológica}
        \footnotesize
        El modelo de prototipos permite mitigar riesgos técnicos y facilitar ajustes iterativos ante cambios normativos o tecnológicos.
    \end{block}
\end{frame}

% ==================== SLIDE: ARQUITECTURA HÍBRIDA ====================
% CATALINA: Explica el problema y la tabla de componentes (~1.5 min)
\begin{frame}{Arquitectura híbrida: decisión de diseño}
    \textbf{Problema:} ninguna plataforma blockchain individual satisface todos los requisitos.

    \begin{columns}[T]
        \begin{column}{0.5\textwidth}
            \begin{itemize}
                \item Privacidad de datos personales (Ley 1581/2012)
                \item Transparencia pública ciudadana (Ley 1712/2014)
            \end{itemize}
        \end{column}
        \begin{column}{0.5\textwidth}
            \begin{itemize}
                \item Rendimiento (457,000 comparendos semestrales)
                \item Costos operativos predecibles
            \end{itemize}
        \end{column}
    \end{columns}

    \vspace{0.5em}

    \centering
    \textbf{Tabla 1}

    \textit{Componentes de la arquitectura híbrida}

    \footnotesize
    \begin{table}
        \centering
        \begin{tabular}{p{2.2cm} p{3cm} p{3.8cm} p{1.5cm}}
            \toprule
            \textbf{Componente} & \textbf{Tecnología} & \textbf{Justificación} & \textbf{TPS} \\
            \midrule
            Capa privada & Hyperledger Fabric v2.5 & Control de acceso PKI, sin gas fees & 2K--20K \\
            Capa pública & Ethereum (Sepolia) & Verificación ciudadana & 15--30 \\
            Storage privado & IPFS privado & Evidencias sensibles & -- \\
            Storage público & IPFS público & Hashes de verificación & -- \\
            \bottomrule
        \end{tabular}
    \end{table}

    \footnotesize
    Fuente: Elaboración propia.
\end{frame}

% ==================== SLIDE: ACTORES Y FUNCIONALIDADES ====================
% CRISTIAN: Describe los actores y sus funcionalidades con el diagrama (~1 min)
\begin{frame}{Actores y funcionalidades principales}
    \begin{columns}[T]
        \begin{column}{0.55\textwidth}
            \centering
            \textbf{Figura 1}

            \textit{Diagrama de casos de uso}

            \includegraphics[width=\textwidth]{images/CasosUso.png}

            \footnotesize
            Fuente: Elaboración propia.
        \end{column}
        \begin{column}{0.45\textwidth}
            \textbf{Actores identificados:}

            \vspace{0.5em}

            \begin{enumerate}
                \item \textbf{Agente de tránsito}
                \begin{itemize}
                    \footnotesize
                    \item Registrar comparendo
                    \item Actualizar estado
                \end{itemize}

                \vspace{0.3em}

                \item \textbf{Ciudadano}
                \begin{itemize}
                    \footnotesize
                    \item Consultar multa
                    \item Verificar autenticidad
                    \item Apelar
                \end{itemize}

                \vspace{0.3em}

                \item \textbf{Administrador}
                \begin{itemize}
                    \footnotesize
                    \item Gestionar sistema
                    \item Auditar operaciones
                \end{itemize}
            \end{enumerate}
        \end{column}
    \end{columns}
\end{frame}

% ==================== SLIDE: DIAGRAMAS DE ACTIVIDADES ====================
% CATALINA: Explica los flujos de registro de multa y apelación (~1.5 min)
\begin{frame}{Flujos de proceso: diagramas de actividades}
    \begin{columns}[T]
        \begin{column}{0.5\textwidth}
            \centering
            \textbf{Figura 2}

            \textit{Registro de multa}

            \includegraphics[height=0.65\textheight]{images/ActMulta.png}

            \footnotesize
            Fuente: Elaboración propia.
        \end{column}
        \begin{column}{0.5\textwidth}
            \centering
            \textbf{Figura 3}

            \textit{Proceso de apelación}

            \includegraphics[height=0.65\textheight]{images/ActApelacion.png}

            \footnotesize
            Fuente: Elaboración propia.
        \end{column}
    \end{columns}
\end{frame}

% ==================== SLIDE: ARQUITECTURA DEL SISTEMA ====================
% CRISTIAN: Explica el diagrama de despliegue y las 5 capas (~1 min)
\begin{frame}{Arquitectura del sistema}
    \begin{center}
        \textbf{Figura 4}

        \textit{Diagrama de despliegue del sistema}

        \includegraphics[height=0.5\textheight]{images/Despliegue.png}

        \footnotesize
        Fuente: Elaboración propia.
    \end{center}

    \vspace{0.2em}

    \footnotesize
    \textbf{Capas:} \textbf{1.} Frontend React ~|~ \textbf{2.} API REST Node.js/Express ~|~ \textbf{3.} Hyperledger Fabric ~|~ \textbf{4.} Ethereum + IPFS público ~|~ \textbf{5.} IPFS privado
\end{frame}

% ==================== SLIDE: ALCANCE Y LIMITACIONES ====================
% CATALINA: Presenta las delimitaciones y la nota metodológica (~1.5 min)
\begin{frame}{Alcance y delimitaciones del estudio}
    \textbf{Delimitaciones metodológicas del prototipo:}

    \vspace{0.3em}

    \begin{itemize}
        \item \textbf{Datos sintéticos:} la verificación se realizó con datos generados mediante scripts de prueba, dado que no se disposede acceso a datos reales del FÉNIX, RUNT ni SIMIT.

        \vspace{0.2em}

        \item \textbf{Cobertura parcial de estados:} se implementaron 5 de los 8 estados del ciclo de vida (PENDING, PAID, APPEALED, RESOLVED\_APPEAL, CANCELLED).

        \vspace{0.2em}

        \item \textbf{Volumen controlado:} se emplearon entre 50 y 100 comparendos de prueba, frente a los 457,000 semestrales registrados en producción.

        \vspace{0.2em}

        \item \textbf{Verificación técnica:} los resultados corresponden a una \textit{verificación} en entorno controlado, no a una \textit{validación} operativa institucional.
    \end{itemize}

    \vspace{0.2em}

    \begin{block}{Nota metodológica}
        \footnotesize
        Se distingue entre \textit{verificación} (el sistema cumple las especificaciones de diseño) y \textit{validación} (el sistema opera adecuadamente en condiciones reales). Este trabajo se enmarca en la primera categoría.
    \end{block}
\end{frame}
