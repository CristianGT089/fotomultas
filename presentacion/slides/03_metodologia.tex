% ==================== SECCIÓN 3: METODOLOGÍA Y DISEÑO ====================
\sectionframe{Metodología y Diseño del Prototipo}

% ==================== SLIDE 10: METODOLOGÍA ====================
\begin{frame}[shrink=5]{Enfoque Metodológico: Desarrollo por Prototipos}
    \textbf{Justificación del Modelo de Prototipos:}

    \begin{itemize}
        \item \textbf{Naturaleza innovadora:} Combinación de tecnologías emergentes sin precedentes locales
        \item \textbf{Requisitos evolutivos:} Marco normativo y tecnología en constante cambio
        \item \textbf{Validación temprana:} Probar hipótesis central antes de desarrollo completo
    \end{itemize}

    \vspace{1em}

    \textbf{Fases del Desarrollo del Prototipo:}

    \begin{enumerate}
        \item \textbf{Análisis de requisitos} → Marco legal + auditorías
        \item \textbf{Diseño arquitectónico} → Patrones de descomposición por confianza
        \item \textbf{Implementación iterativa} → Backend + Frontend + Smart Contracts
        \item \textbf{Pruebas y validación} → 80 casos automatizados
    \end{enumerate}

    \begin{block}{Mitigación de Riesgos}
        Decisión metodológica deliberada que mitiga riesgos técnicos y permite pivotes ágiles
    \end{block}
\end{frame}

% ==================== SLIDE 11: SELECCIÓN TECNOLÓGICA ====================
\begin{frame}[shrink=10]{Arquitectura Híbrida: Decisión Crítica}
    \textbf{Problema:} Ninguna blockchain cumple TODOS los requisitos

    \begin{itemize}
        \item Privacidad de datos personales (Ley 1581/2012)
        \item Transparencia pública ciudadana
        \item Rendimiento (457,000 comparendos semestrales)
        \item Costos operativos predecibles
    \end{itemize}

    \vspace{0.5em}

    \scriptsize
    \begin{table}
        \centering
        \begin{tabular}{p{2.5cm} p{3.5cm} p{3.5cm} p{2cm}}
            \toprule
            \textbf{Componente} & \textbf{Tecnología} & \textbf{Justificación} & \textbf{TPS} \\
            \midrule
            Capa privada & Hyperledger Fabric v2.5 & Control acceso PKI, sin gas fees & 2K-20K \\
            \midrule
            Capa pública & Ethereum (Sepolia) & Verificación ciudadana, ecosistema maduro & 15-30 \\
            \midrule
            Storage privado & IPFS privado & Evidencias sensibles, acceso controlado & - \\
            \midrule
            Storage público & IPFS público & Hashes verificación, content-addressed & - \\
            \bottomrule
        \end{tabular}
    \end{table}

    \begin{alertblock}{Arquitectura Híbrida}
        Balancea trade-offs irreconciliables mediante descomposición por niveles de confianza
    \end{alertblock}
\end{frame}

% ==================== SLIDE 12: DIAGRAMA DE DESPLIEGUE ====================
\begin{frame}[shrink=15]{Arquitectura Híbrida del Sistema}
    \begin{center}
        \includegraphics[height=0.6\textheight]{images/Despliegue.png}
    \end{center}

    \vspace{0.3em}

    \footnotesize
    \textbf{Capas del Sistema:}
    \begin{itemize}
        \item \textbf{Capa 1:} Frontend React (ciudadano + agente)
        \item \textbf{Capa 2:} API REST Node.js/Express
        \item \textbf{Capa 3:} Hyperledger Fabric (red privada permisionada)
        \item \textbf{Capa 4:} Ethereum + IPFS público (verificación transparente)
        \item \textbf{Capa 5:} IPFS privado (evidencias completas)
    \end{itemize}
\end{frame}

% ==================== SLIDE 13: DIAGRAMA DE CLASES ====================
\begin{frame}[shrink=10]{Diseño Orientado a Objetos}
    \begin{columns}[T]
        \begin{column}{0.6\textwidth}
            \includegraphics[width=\textwidth]{images/uml.png}
        \end{column}
        \begin{column}{0.4\textwidth}
            \textbf{Patrón Controller-Service-Repository}

            \vspace{1em}

            \textbf{Capas Arquitectónicas:}

            \begin{enumerate}
                \item \textbf{Servicios blockchain:}
                \begin{itemize}
                    \footnotesize
                    \item HyperledgerService
                    \item EthereumService
                    \item SyncService
                \end{itemize}

                \item \textbf{Almacenamiento:}
                \begin{itemize}
                    \footnotesize
                    \item IPFSPrivateService
                    \item IPFSPublicService
                \end{itemize}

                \item \textbf{Orquestación:}
                \begin{itemize}
                    \footnotesize
                    \item FineService
                    \item FineController (REST)
                \end{itemize}
            \end{enumerate}

            \vspace{0.5em}

            \begin{block}{\footnotesize Beneficios}
                \footnotesize
                Separación de responsabilidades, testabilidad, mantenibilidad
            \end{block}
        \end{column}
    \end{columns}
\end{frame}

% ==================== SLIDE 14: CASOS DE USO ====================
\begin{frame}[shrink=5]{Actores y Funcionalidades Principales}
    \begin{columns}[T]
        \begin{column}{0.55\textwidth}
            \includegraphics[width=\textwidth]{images/CasosUso.png}
        \end{column}
        \begin{column}{0.45\textwidth}
            \textbf{Actores Identificados:}

            \vspace{0.5em}

            \begin{enumerate}
                \item \textbf{Agente de Tránsito}
                \begin{itemize}
                    \footnotesize
                    \item Registrar comparendo
                    \item Actualizar estado
                \end{itemize}

                \vspace{0.5em}

                \item \textbf{Ciudadano}
                \begin{itemize}
                    \footnotesize
                    \item Consultar multa
                    \item Verificar autenticidad
                    \item Apelar
                \end{itemize}

                \vspace{0.5em}

                \item \textbf{Administrador}
                \begin{itemize}
                    \footnotesize
                    \item Gestionar sistema
                    \item Auditar operaciones
                \end{itemize}
            \end{enumerate}

            \vspace{0.5em}

            \begin{block}{\footnotesize Cobertura Integral}
                \footnotesize
                Sistema cubre el ciclo de vida completo del fotocomparendo
            \end{block}
        \end{column}
    \end{columns}
\end{frame}

% ==================== SLIDE 15: FLUJOS DE PROCESO ====================
\begin{frame}[shrink=15]{Diagramas de Actividad: Procesos Críticos}
    \begin{columns}[T]
        \begin{column}{0.5\textwidth}
            \centering
            \textbf{Creación de Multa}

            \includegraphics[width=0.9\textwidth]{images/ActMulta.png}

            \vspace{0.3em}

            \footnotesize
            \textbf{Flujo:}
            \begin{itemize}
                \scriptsize
                \item Captura → IPFS privado
                \item → Hyperledger
                \item → Sync → Ethereum público
            \end{itemize}
        \end{column}
        \begin{column}{0.5\textwidth}
            \centering
            \textbf{Proceso de Apelación}

            \includegraphics[width=0.9\textwidth]{images/ActApelacion.png}

            \vspace{0.3em}

            \footnotesize
            \textbf{Flujo:}
            \begin{itemize}
                \scriptsize
                \item Solicitud → Evaluación
                \item → Smart contract
                \item → Actualización estado
                \item → Notificación
            \end{itemize}
        \end{column}
    \end{columns}

    \vspace{0.5em}

    \begin{exampleblock}{Automatización y Transparencia}
        Contratos inteligentes ejecutan lógica de negocio de forma predecible y auditable
    \end{exampleblock}
\end{frame}
