% ==================== SECCIÓN 1: INTRODUCCIÓN Y CONTEXTO ====================
\section{Contexto y Problemática}

% ==================== SLIDE 3: MAGNITUD DEL PROBLEMA ====================
\begin{frame}[shrink=5]{Escala Operativa del Sistema de Fotocomparendos en Bogotá}
    \begin{columns}[T]
        \begin{column}{0.5\textwidth}
            \textbf{Datos del Sistema FÉNIX:}
            \vspace{0.5em}
            \begin{itemize}
                \item \highlight{1.9 millones} de comparendos emitidos (2018-2024) \parencite{sdm2024estadisticas}
                \item \highlight{457,000} comparendos semestrales en promedio
                \item Sistema centralizado en infraestructura de nube
                \item Gestión basada en base de datos relacional tradicional
            \end{itemize}
        \end{column}
        \begin{column}{0.5\textwidth}
            \centering
            \textbf{Figura 1}

            \textit{Comparendos detectados por año}

            \includegraphics[width=\textwidth]{images/numComparendos.png}

            \footnotesize
            Fuente: Secretaría Distrital de Movilidad (2024).
        \end{column}
    \end{columns}

    \vspace{1em}
    \begin{alertblock}{Impacto}
        El sistema gestiona un volumen significativo de registros críticos que afectan directamente a ciudadanos y requiere garantías de integridad y transparencia.
    \end{alertblock}
\end{frame}

% ==================== SLIDE 4: CRISIS DE CONFIANZA ====================
\begin{frame}[shrink=5]{Crisis de Confianza: Indicadores Críticos}
    \textbf{Manifestaciones Cuantificables de la Problemática:}
    \vspace{0.5em}

    \begin{columns}[T]
        \begin{column}{0.55\textwidth}
            \begin{itemize}
                \item \textbf{Tasa de impugnación:} \highlight{34.1\%} \parencite{contraloria2024detrimento}
                    \begin{itemize}
                        \item 1 de cada 3 comparendos genera disputa formal
                    \end{itemize}
                \item \textbf{Carga operativa:} \highlight{155,854 PQRSD} semestrales \parencite{contraloria2024detrimento}
                \item \textbf{Detrimento patrimonial:} \highlight{\$8,000 millones} \parencite{contraloria2024detrimento}
                \item \textbf{Vulnerabilidad ciudadana:} Casos de fraude como Juzto.co \parencite{semana2023juzto}
            \end{itemize}
        \end{column}
        \begin{column}{0.45\textwidth}
            \centering
            \textbf{Tabla 1}

            \textit{Comparación entre base de datos tradicional y blockchain}

            \footnotesize
            \begin{table}
                \centering
                \begin{tabular}{lcc}
                    \toprule
                    \textbf{Aspecto} & \textbf{BD} & \textbf{Blockchain} \\
                    \midrule
                    Confianza & Central & Distribuida \\
                    Inmutabilidad & NO & SÍ \\
                    Trazabilidad & Limitada & Completa \\
                    Corrupción & Alto riesgo & Bajo riesgo \\
                    \bottomrule
                \end{tabular}
            \end{table}

            \footnotesize
            Fuente: Elaboración propia.
        \end{column}
    \end{columns}

    \vspace{0.5em}
    \begin{block}{Transición Necesaria}
        De problema teórico a crisis medible que requiere intervención técnica urgente
    \end{block}
\end{frame}

% ==================== SLIDE 5: FORMULACIÓN DEL PROBLEMA ====================
\begin{frame}{Formulación del Problema}
    \begin{exampleblock}{Pregunta de Investigación}
        \centering
        \textit{¿Cómo mitigar el riesgo de pérdida o alteración de la integridad de los datos en el proceso de fotocomparendos mediante tecnologías de redes distribuidas?}
    \end{exampleblock}

    \vspace{1em}

    \textbf{Limitaciones del Modelo Actual (Sistema FÉNIX):}

    \begin{columns}[T]
        \begin{column}{0.5\textwidth}
            \begin{itemize}
                \item Confianza basada en administradores centrales
                \item Inmutabilidad NO garantizada criptográficamente
            \end{itemize}
        \end{column}
        \begin{column}{0.5\textwidth}
            \begin{itemize}
                \item Trazabilidad dependiente de controles internos
                \item Auditoría opaca para ciudadanos
            \end{itemize}
        \end{column}
    \end{columns}

    \vspace{1em}

    \begin{alertblock}{Hipótesis Central}
        Las tecnologías de redes distribuidas (blockchain + IPFS) pueden proporcionar garantías criptográficas de integridad y transparencia verificable sin intermediarios.
    \end{alertblock}
\end{frame}
