% ==================== SECCIÓN: DEMOSTRACIÓN DEL PROTOTIPO ====================
% PRESENTADORES: CRISTIAN y CATALINA — Demostración en vivo del prototipo
\section{Demostración del prototipo}

% ==================== SLIDE: DEMOSTRACIÓN ====================
% CRISTIAN/CATALINA: Demostración en vivo del prototipo (~3-5 min)
\begin{frame}{Demostración en vivo}
    \centering

    \vspace{1em}

    {\Large \textbf{Demostración del prototipo}}

    \vspace{1.5em}

    \begin{columns}[T]
        \begin{column}{0.48\textwidth}
            \textbf{Entorno de despliegue:}
            \begin{itemize}
                \footnotesize
                \item \textbf{Servidor:} Grupo GNU Linux
                \item \textbf{Institución:} Universidad Distrital
                \item \textbf{URL:} \texttt{fotomultas.glud.org}
                \item \textbf{Recursos:} 8 vCPU, 16GB RAM
                \item \textbf{SO:} Ubuntu Server 22.04 LTS
            \end{itemize}
        \end{column}
        \begin{column}{0.48\textwidth}
            \textbf{Componentes desplegados:}
            \begin{itemize}
                \footnotesize
                \item Backend API (Node.js - Puerto 3000)
                \item Frontend Web (React - Puerto 80)
                \item Red Hyperledger Fabric
                \item Nodo IPFS local
                \item Conexión Ethereum Sepolia
            \end{itemize}
        \end{column}
    \end{columns}

    \vspace{1.5em}

    \begin{exampleblock}{Acceso al sistema}
        \footnotesize
        El prototipo está disponible públicamente para validación. Se demostrará el registro en Hyperledger Fabric, la publicación de hashes en Ethereum y el almacenamiento de evidencias en IPFS.
    \end{exampleblock}
\end{frame}
