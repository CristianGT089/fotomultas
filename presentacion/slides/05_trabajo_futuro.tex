% ==================== SECCIÓN 5: TRABAJO FUTURO ====================
% PRESENTADORES: CRISTIAN (líneas de evolución) → CATALINA (agradecimientos y cierre)
\section{Trabajo futuro}

% ==================== SLIDE: LÍNEAS DE EVOLUCIÓN ====================
% CRISTIAN: Presenta las 4 líneas de evolución y la perspectiva (~1.5 min)
\begin{frame}[shrink=10]{Líneas de evolución}
    \begin{columns}[T]
        \begin{column}{0.5\textwidth}
            \textbf{1. Validación operativa:}
            \begin{itemize}
                \footnotesize
                \item Piloto controlado con 5,000--10,000 multas reales
                \item Integración con SIMIT/RUNT mediante APIs reales
                \item Estudios de aceptación tecnológica (TAM/UTAUT) con agentes de tránsito y ciudadanos
            \end{itemize}

            \vspace{0.5em}

            \textbf{2. Escalamiento a producción:}
            \begin{itemize}
                \footnotesize
                \item Red Fabric multi-organizacional (SDM, Policía, Contraloría)
                \item Migración a soluciones Layer 2 (Polygon, Arbitrum)
                \item Auditoría formal de seguridad (Slither, MythX)
            \end{itemize}
        \end{column}
        \begin{column}{0.5\textwidth}
            \textbf{3. Extensión funcional:}
            \begin{itemize}
                \footnotesize
                \item Oráculos certificadores para el estado NOTIFICADA
                \item Módulo de pagos (PSE, billeteras digitales)
                \item Sistema de apelaciones en línea automatizado
            \end{itemize}

            \vspace{0.5em}

            \textbf{4. Replicabilidad:}
            \begin{itemize}
                \footnotesize
                \item Adaptación para otras ciudades colombianas
                \item Estandarización de contratos inteligentes a nivel nacional
                \item Federación de redes Fabric intercity
            \end{itemize}
        \end{column}
    \end{columns}

    \vspace{0.5em}

    \begin{exampleblock}{Perspectiva}
        Los resultados obtenidos constituyen una base técnica para futuras investigaciones orientadas a la validación operativa e institucional del sistema propuesto.
    \end{exampleblock}
\end{frame}

% ==================== SLIDE: REFERENCIAS ====================
\begin{frame}{Referencias principales}
    \footnotesize
    % Solo mostrar referencias clave seleccionadas
    \begin{itemize}
        \item \fullcite{antonopoulos2023mastering}
        \vspace{0.3em}
        \item \fullcite{Informe170100005424}
        \vspace{0.3em}
        \item \fullcite{vanSteen2017}
        \vspace{0.3em}
        \item \fullcite{nakamoto2008bitcoin}
    \end{itemize}
\end{frame}

% ==================== SLIDE: AGRADECIMIENTOS ====================
% CATALINA: Agradece al director, la universidad (~30s)
\begin{frame}[plain]
    \vfill

    \begin{center}
        \includegraphics[width=0.12\textwidth]{images/Escudo_UD.png}

        \vspace{0.7em}

        {\Large \textbf{Agradecimientos}}

        \vspace{1.2em}

        \small
        \textbf{Universidad Distrital Francisco José de Caldas} \\
        Facultad de Ingeniería \\
        Programa de Ingeniería de Sistemas

        \vspace{1.5em}

        \footnotesize
        \textbf{Director} \\[0.4em]
        \normalsize
        Julio Barón Velandia
    \end{center}

    \vfill
\end{frame}

% ==================== SECCIÓN: PREGUNTAS ====================
% La sección aparece en la agenda y genera la slide de índice automáticamente
\section{¿Preguntas?}
