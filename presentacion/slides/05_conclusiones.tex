% ==================== SECCIÓN 5: CONCLUSIONES Y CIERRE ====================
\section{Conclusiones y Trabajo Futuro}

% ==================== SLIDE 21: CONCLUSIONES PRINCIPALES ====================
\begin{frame}[shrink=5]{Conclusiones Principales}
    \textbf{1. Viabilidad Técnica Demostrada:}
    \begin{itemize}
        \item Hyperledger Fabric + Ethereum + IPFS dual es una combinación \textbf{viable} para gestión de fotocomparendos
    \end{itemize}

    \vspace{0.5em}

    \textbf{2. Garantías Criptográficas Validadas:}
    \begin{itemize}
        \item \highlight{100\%} de intentos de modificación rechazados
        \item Detección automática de alteraciones en evidencia
        \item Tiempos de respuesta \highlight{< 3s} (apto para producción)
    \end{itemize}

    \vspace{0.5em}

    \textbf{3. Arquitectura Escalable:}
    \begin{itemize}
        \item Backend con interfaces REST estándar
        \item Frontend React facilita adopción institucional
    \end{itemize}

    \vspace{0.5em}

    \textbf{4. Modelo de Confianza Alternativo:}
    \begin{itemize}
        \item Transición de confianza administrativa a \textbf{confianza criptográfica verificable}
    \end{itemize}

    \vspace{0.5em}

    \begin{exampleblock}{Contribución Principal}
        El proyecto propone y valida que la arquitectura híbrida blockchain puede mejorar la gestión de registros públicos críticos
    \end{exampleblock}
\end{frame}

% ==================== SLIDE 22: TRABAJO FUTURO ====================
\begin{frame}[shrink=10]{Trabajo Futuro: Líneas de Evolución}
    \begin{columns}[T]
        \begin{column}{0.5\textwidth}
            \textbf{1. Escalamiento a Producción:}
            \begin{itemize}
                \footnotesize
                \item Red Fabric multi-organizacional (SDM, Policía, Auditoría)
                \item Private Data Collections para datos ultra-sensibles
                \item Infraestructura IPFS distribuida con políticas de replicación
            \end{itemize}

            \vspace{0.5em}

            \textbf{2. Piloto Controlado:}
            \begin{itemize}
                \footnotesize
                \item Convenio con Secretaría Distrital de Movilidad
                \item Dataset real: 5,000-10,000 multas
                \item Integración con SIMIT/RUNT nacional
            \end{itemize}
        \end{column}
        \begin{column}{0.5\textwidth}
            \textbf{3. Funcionalidades Avanzadas:}
            \begin{itemize}
                \footnotesize
                \item Módulo de pagos (PSE, billeteras digitales)
                \item Sistema de apelaciones en línea automatizado
                \item Dashboard analítico para toma de decisiones
            \end{itemize}

            \vspace{0.5em}

            \textbf{4. Replicabilidad Nacional:}
            \begin{itemize}
                \footnotesize
                \item Adaptación para otras ciudades colombianas
                \item Estandarización de Smart Contracts a nivel nacional
                \item Federación de redes Fabric intercity
            \end{itemize}
        \end{column}
    \end{columns}

    \vspace{0.5em}

    \begin{alertblock}{Proyección}
        Este proyecto es punto de partida, no punto final - abre múltiples líneas de investigación aplicada en GovTech
    \end{alertblock}
\end{frame}

% ==================== SLIDE 22: REFERENCIAS ====================
\begin{frame}{Referencias}
    \footnotesize
    \begin{itemize}
        \item Anand, T., \& Singh, V. (2024). \textit{Traffic violation detection using blockchain} [Major project report, Jaypee University of Information Technology].
        \item Chen, C.-L., Tu, C.-Y., Deng, Y.-Y., Huang, D.-C., Liu, L.-C., \& Chen, H.-C. (2024). Blockchain-enabled transparent traffic enforcement for sustainable road safety in cities. \textit{Sustainable Cities: Smart Technologies and Cities}, \textit{6}, 1426036. \url{https://doi.org/10.3389/frsc.2024.1426036}
        \item Congreso de la República de Colombia. (2012). Ley Estatutaria 1581 de 2012: Por la cual se dictan disposiciones generales para la protección de datos personales. Diario Oficial.
        \item Contraloría General de la República de Colombia. (2024). \textit{Informe de Auditoría 170100-0054-24: Auditoría de Cumplimiento a la Secretaría Distrital de Movilidad}.
        \item Mani Joseph, P. (2023). Smart and secure blockchain structure to track vehicle record-keeping in the Sultanate of Oman. \textit{International Journal on Recent and Innovation Trends in Computing and Communication}.
        \item Omar, M. H., Taj-Eddin, I., Omar, N., \& Ibrahim, H. (2024). SECURE ROAD TRAFFIC MANAGEMENT (SRTM) SYSTEM FOR TRAFFIC VIOLATION DETECTION AND RECORDING USING BLOCKCHAIN TECHNOLOGY. \textit{Journal of Southwest Jiaotong University}, \textit{59}(2). \url{https://doi.org/10.35741/issn.0258-2724.59.2.1}
        \item Secretaría Distrital de Movilidad. (2024). \textit{Estadísticas de Comparendos Bogotá 2024}. \url{https://www.movilidadbogota.gov.co/web/observatorio}
        % \item Yousfi, N., Kmimech, M., Abbassi, I., Hamdi, H., \& Graiet, M. (2022). ITS traffic violation regulation based on blockchain smart contracts. In \textit{International Conference on Computational Collective Intelligence} (pp. 459-471). Springer. \url{https://doi.org/10.1007/978-3-031-16210-7_38}
    \end{itemize}
\end{frame}

% ==================== SLIDE 23: AGRADECIMIENTOS ====================
\begin{frame}[plain]
    \vfill

    \begin{center}
        \includegraphics[width=0.12\textwidth]{images/Escudo_UD.png}

        \vspace{0.7em}

        {\Large \textbf{Agradecimientos}}

        \vspace{1.2em}

        \small
        \textbf{Universidad Distrital Francisco José de Caldas} \\
        Facultad de Ingeniería \\
        Programa de Ingeniería de Sistemas

        \vspace{1.5em}

        \begin{columns}[t]
            \begin{column}{0.48\textwidth}
                \centering
                \footnotesize
                \textbf{Director} \\[0.4em]
                \normalsize
                Julio Barón Velandia
            \end{column}

            \begin{column}{0.48\textwidth}
                \centering
                \footnotesize
                \textbf{Autores} \\[0.4em]
                \normalsize
                Laura Catalina Preciado Ballén \\[0.25em]
                Cristian Stiven Guzmán Tovar
            \end{column}
        \end{columns}

        \vspace{2em}

        {\Huge \textcolor{udlightblue}{\textbf{ESPACIO DE PREGUNTAS}}}

        \vspace{0.8em}

        \footnotesize
    \end{center}

    \vfill
\end{frame}
