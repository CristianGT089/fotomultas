% ==================== SECCIÓN 5: CONCLUSIONES Y CIERRE ====================
\sectionframe{Conclusiones y Trabajo Futuro}

% ==================== SLIDE 21: CONCLUSIONES PRINCIPALES ====================
\begin{frame}[shrink=5]{Conclusiones Principales}
    \textbf{1. Viabilidad Técnica Demostrada:}
    \begin{itemize}
        \item Hyperledger Fabric + Ethereum + IPFS dual es una combinación \textbf{viable} para gestión de fotocomparendos
    \end{itemize}

    \vspace{0.5em}

    \textbf{2. Garantías Criptográficas Validadas:}
    \begin{itemize}
        \item \highlight{100\%} de intentos de modificación rechazados
        \item Detección automática de alteraciones en evidencia
        \item Tiempos de respuesta \highlight{< 3s} (apto para producción)
    \end{itemize}

    \vspace{0.5em}

    \textbf{3. Arquitectura Escalable:}
    \begin{itemize}
        \item Backend con interfaces REST estándar
        \item Frontend React facilita adopción institucional
    \end{itemize}

    \vspace{0.5em}

    \textbf{4. Modelo de Confianza Alternativo:}
    \begin{itemize}
        \item Transición de confianza administrativa a \textbf{confianza criptográfica verificable}
    \end{itemize}

    \vspace{0.5em}

    \begin{exampleblock}{Contribución Principal}
        El proyecto NO solo propone - \textbf{DEMUESTRA} que blockchain híbrida puede transformar la gestión de registros públicos críticos
    \end{exampleblock}
\end{frame}

% ==================== SLIDE 22: TRABAJO FUTURO ====================
\begin{frame}[shrink=10]{Trabajo Futuro: Líneas de Evolución}
    \begin{columns}[T]
        \begin{column}{0.5\textwidth}
            \textbf{1. Escalamiento a Producción:}
            \begin{itemize}
                \footnotesize
                \item Red Fabric multi-organizacional (SDM, Policía, Auditoría)
                \item Private Data Collections para datos ultra-sensibles
                \item Infraestructura IPFS distribuida con políticas de replicación
            \end{itemize}

            \vspace{0.5em}

            \textbf{2. Piloto Controlado:}
            \begin{itemize}
                \footnotesize
                \item Convenio con Secretaría Distrital de Movilidad
                \item Dataset real: 5,000-10,000 multas
                \item Integración con SIMIT/RUNT nacional
            \end{itemize}
        \end{column}
        \begin{column}{0.5\textwidth}
            \textbf{3. Funcionalidades Avanzadas:}
            \begin{itemize}
                \footnotesize
                \item Módulo de pagos (PSE, billeteras digitales)
                \item Sistema de apelaciones en línea automatizado
                \item Dashboard analítico para toma de decisiones
            \end{itemize}

            \vspace{0.5em}

            \textbf{4. Replicabilidad Nacional:}
            \begin{itemize}
                \footnotesize
                \item Adaptación para otras ciudades colombianas
                \item Estandarización de Smart Contracts a nivel nacional
                \item Federación de redes Fabric intercity
            \end{itemize}
        \end{column}
    \end{columns}

    \vspace{0.5em}

    \begin{alertblock}{Proyección}
        Este proyecto es punto de partida, no punto final - abre múltiples líneas de investigación aplicada en GovTech
    \end{alertblock}
\end{frame}

% ==================== SLIDE 23: AGRADECIMIENTOS ====================
\begin{frame}[plain]
    \vfill
    \begin{center}
        {\Large \textbf{Agradecimientos}}
    \end{center}

    \vspace{1em}

    \begin{columns}[c]
        \begin{column}{0.45\textwidth}
            \centering
            \includegraphics[width=0.35\textwidth]{images/Escudo_UD.png}

            \vspace{0.5em}

            \small
            Universidad Distrital \\
            Francisco José de Caldas

            \vspace{0.3em}

            \footnotesize
            Facultad de Ingeniería \\
            Ingeniería de Sistemas

            \vspace{0.5em}

            \scriptsize
            \textbf{Director:} \\
            Julio Barón Velandia
        \end{column}

        \begin{column}{0.45\textwidth}
            \centering

            \vspace{1em}

            {\Huge \textbf{¿PREGUNTAS?}}

            \vspace{1.5em}

            \footnotesize
            \textbf{Autores:}

            \vspace{0.3em}

            \scriptsize
            Laura Catalina Preciado Ballén \\[0.2em]
            Cristian Stiven Guzmán Tovar

            \vspace{0.5em}

            \tiny
            Julio 2025
        \end{column}
    \end{columns}

    \vfill
\end{frame}
