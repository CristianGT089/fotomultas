% ==================== CONTEXTO (sin \section - no aparece en agenda) ====================
% PRESENTADOR: CATALINA — Contexto completo (~2 min)

\section{Contexto y formulación del problema}

% ==================== SLIDE: ESCALA DEL PROBLEMA ====================
% CATALINA: Presenta cifras del sistema FÉNIX (~1 min)
\begin{frame}{Contexto: el sistema de fotocomparendos en Bogotá}
    \begin{columns}[T]
        \begin{column}{0.45\textwidth}
            \textbf{Escala operativa (Sistema FÉNIX):}
            \begin{itemize}
                \item \highlight{1.9 millones} de comparendos emitidos entre 2018--2024 \parencite{sdm2024estadisticas}
                \item \highlight{457,000} comparendos semestrales en promedio
                \item Arquitectura centralizada (BD relacional)
            \end{itemize}
        \end{column}
        \begin{column}{0.55\textwidth}
            \centering
            \textbf{Figura 1}

            \textit{Comparendos emitidos por semestre}

            \includegraphics[width=\textwidth,height=0.55\textheight,keepaspectratio]{images/numComparendos.png}

            \footnotesize
            Fuente: Secretaría Distrital de Movilidad (2024).
        \end{column}
    \end{columns}
\end{frame}

% ==================== SLIDE: INDICADORES DE LA PROBLEMÁTICA ====================
% CATALINA: Presenta indicadores clave del problema (~1 min)
\begin{frame}{Indicadores de la problemática}
    \begin{columns}[T]
        \begin{column}{0.5\textwidth}
            \begin{block}{Gestión ciudadana}
                \begin{itemize}
                    \item Tasa de impugnación: \highlight{34.1\%}
                    \item Carga operativa: \highlight{155,854 PQRSD} semestrales
                \end{itemize}
            \end{block}

            \vspace{0.5em}

            \begin{block}{Impacto fiscal}
                \begin{itemize}
                    \item Presunto detrimento patrimonial: \highlight{\$8,000 millones} \parencite{contraloria2024detrimento}
                \end{itemize}
            \end{block}
        \end{column}
        \begin{column}{0.5\textwidth}
            \textbf{Vulnerabilidades identificadas:}
            \begin{itemize}
                \item Fraude a ciudadanos mediante intermediarios ilegales \parencite{semana2023juzto}
                \item Confianza en administradores centrales sin garantías criptográficas
                \item Inmutabilidad no verificable por la ciudadanía
                \item Auditoría opaca para el control institucional
            \end{itemize}
        \end{column}
    \end{columns}
\end{frame}

% ==================== SLIDE: PREGUNTA DE INVESTIGACIÓN ====================
% CATALINA: Lee la pregunta de investigación, limitaciones e hipótesis (~1 min)
\begin{frame}{Formulación del problema}
    \begin{exampleblock}{Pregunta de investigación}
        \centering
        \small
        \textit{¿Cómo mitigar el riesgo de pérdida o alteración de la integridad de los datos asociados a todos los estados en el proceso de fotocomparendos en Bogotá mediante el uso de tecnologías de redes distribuidas que garanticen el registro, la trazabilidad, la autenticidad y la confidencialidad de la información?}
    \end{exampleblock}

    \vspace{0.3em}

    \footnotesize
    \textbf{Limitaciones del modelo actual (FÉNIX):} confianza en administradores centrales, inmutabilidad no garantizada criptográficamente, trazabilidad dependiente de controles internos, auditoría opaca para la ciudadanía.

    \vspace{0.3em}

    \normalsize
    \begin{alertblock}{Hipótesis}
        \small
        Las tecnologías de redes distribuidas (blockchain + IPFS) pueden proporcionar garantías criptográficas de integridad y transparencia verificable sin intermediarios.
    \end{alertblock}
\end{frame}

