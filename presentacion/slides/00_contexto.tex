% ==================== CONTEXTO (sin \section - no aparece en agenda) ====================
% PRESENTADOR: CATALINA — Contexto completo (~2 min)

% ==================== SLIDE: ESCALA DEL PROBLEMA ====================
% CATALINA: Presenta cifras del sistema FÉNIX y los indicadores (~1 min)
\begin{frame}{Contexto: el sistema de fotocomparendos en Bogotá}
    \begin{columns}[T]
        \begin{column}{0.5\textwidth}
            \textbf{Escala operativa (Sistema FÉNIX):}
            \begin{itemize}
                \item \highlight{1.9 millones} de comparendos emitidos entre 2018--2024 \parencite{sdm2024estadisticas}
                \item \highlight{457,000} comparendos semestrales en promedio
                \item Arquitectura centralizada (BD relacional)
            \end{itemize}
        \end{column}
        \begin{column}{0.5\textwidth}
            \centering
            \includegraphics[width=\textwidth,height=0.32\textheight,keepaspectratio]{images/numComparendos.png}
            \captionof{figure}{\textit{Comparendos emitidos por semestre}}
        \end{column}
    \end{columns}

    \vspace{0.3em}

    \textbf{Indicadores de la problemática:}
    \begin{columns}[T]
        \begin{column}{0.5\textwidth}
            \begin{itemize}
                \item Tasa de impugnación: \highlight{34.1\%}
                \item Carga operativa: \highlight{155,854 PQRSD} semestrales
            \end{itemize}
        \end{column}
        \begin{column}{0.5\textwidth}
            \begin{itemize}
                \item Presunto detrimento patrimonial: \highlight{\$8,000 millones} \parencite{contraloria2024detrimento}
                \item Vulnerabilidad ciudadana ante fraudes \parencite{semana2023juzto}
            \end{itemize}
        \end{column}
    \end{columns}
\end{frame}

% ==================== SLIDE: PREGUNTA DE INVESTIGACIÓN ====================
% CATALINA: Lee la pregunta de investigación, limitaciones e hipótesis (~1 min)
\begin{frame}{Formulación del problema}
    \begin{exampleblock}{Pregunta de investigación}
        \centering
        \small
        \textit{¿Cómo mitigar el riesgo de pérdida o alteración de la integridad de los datos asociados a todos los estados en el proceso de fotocomparendos en Bogotá mediante el uso de tecnologías de redes distribuidas que garanticen el registro, la trazabilidad, la autenticidad y la confidencialidad de la información?}
    \end{exampleblock}

    \vspace{0.3em}

    \footnotesize
    \textbf{Limitaciones del modelo actual (FÉNIX):} confianza en administradores centrales, inmutabilidad no garantizada criptográficamente, trazabilidad dependiente de controles internos, auditoría opaca para la ciudadanía.

    \vspace{0.3em}

    \normalsize
    \begin{alertblock}{Hipótesis}
        \small
        Las tecnologías de redes distribuidas (blockchain + IPFS) pueden proporcionar garantías criptográficas de integridad y transparencia verificable sin intermediarios.
    \end{alertblock}
\end{frame}
