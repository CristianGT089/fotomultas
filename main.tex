\documentclass[
    letterpaper, 
    man,   
    spanish,
    12pt,
    floatsintext,
    hidelinks % Opción para hyperref pasada a la clase
]{apa7}

% --- Pasar opciones a paquetes antes de que la clase los cargue ---
\PassOptionsToPackage{table}{xcolor}

% --- Codificación y Lenguaje ---
\usepackage[utf8]{inputenc} 
\usepackage{newunicodechar}
\usepackage[spanish]{babel} 
\selectlanguage{spanish}   
\usepackage{csquotes}       

% --- Definición de caracteres Unicode problemáticos ---
\newunicodechar{́}{'}  % U+0301 COMBINING ACUTE ACCENT
\newunicodechar{—}{---}  % U+2014 EM DASH
\newunicodechar{ć}{c'}  % U+0107 LATIN SMALL LETTER C WITH ACUTE
\newunicodechar{ı}{i}  % U+0131 LATIN SMALL LETTER DOTLESS I
\newunicodechar{≤}{\ensuremath{\leq}}  % U+2264 LESS-THAN OR EQUAL TO
\usepackage{amssymb}  % Para símbolos como checkmark
\newunicodechar{✓}{\ensuremath{\checkmark}}  % U+2713 CHECK MARK
\newunicodechar{→}{\ensuremath{\rightarrow}}  % U+2192 RIGHTWARDS ARROW

% --- Bibliografía con biblatex-apa ---
\usepackage[
    style=apa,            
    backend=biber,        
    sortcites=true,       
    sorting=nyt,          
    hyperref=true,
    backref=false         
]{biblatex}
\DeclareLanguageMapping{spanish}{spanish-apa}
\addbibresource{bibliography.bib} 

% --- Paquetes para Gráficos y Tablas ---
\usepackage{graphicx}     
\usepackage{booktabs}     
\usepackage{adjustbox}    
\usepackage{multirow}     
\usepackage{array}
\usepackage{pifont}       % Para usar símbolos como \ding
% xcolor con opción table se carga via PassOptionsToPackage arriba
\usepackage{caption}
% Definir separador personalizado con espacio vertical entre número y título de tabla
\DeclareCaptionLabelSeparator{newlinewithspace}{\\[6pt]}
\usepackage{longtable}    % Tablas que se extienden por múltiples páginas
\usepackage{setspace}     % Para control de interlineado
\usepackage{ragged2e}     % Para alineación que mantiene sangría
% \captionsetup[figure]{labelformat=default,labelsep=period,name=Figura,position=above,labelfont=bf,textfont=it}
\captionsetup[table]{labelformat=default,labelsep=newlinewithspace,labelfont=bf,textfont=it,skip=10pt,name=Tabla}
% \usepackage{epstopdf}
\usepackage{float}
\usepackage{placeins}     % Para usar \FloatBarrier y evitar que tablas se corten
\usepackage{afterpage}    % Para controlar colocación de flotantes
\usepackage{eso-pic}      % Para posicionar número de página en portada

% Parámetros optimizados para evitar que tablas se corten al final de página
\renewcommand{\floatpagefraction}{0.5}       % Página necesita solo 50% de flotantes para dedicarse a ellos
\renewcommand{\topfraction}{0.95}            % Permite hasta 95% arriba
\renewcommand{\bottomfraction}{0.95}         % Permite hasta 95% abajo
\renewcommand{\textfraction}{0.05}           % Solo requiere 5% de texto
\setcounter{topnumber}{4}                    % Hasta 4 flotantes arriba
\setcounter{bottomnumber}{4}                 % Hasta 4 flotantes abajo
\setcounter{totalnumber}{8}                  % Hasta 8 flotantes por página
\setlength{\floatsep}{12pt plus 4pt minus 4pt}
\setlength{\textfloatsep}{20pt plus 8pt minus 4pt}
\setlength{\intextsep}{14pt plus 4pt minus 4pt}       

% --- COMANDOS PERSONALIZADOS
 \newcommand{\myparagraph}[1]{\paragraph{#1}\mbox{}\\} % Esto 
% --- Definición de comandos de tamaño de fuente ---
\renewcommand{\large}{\fontsize{14.4}{18}\selectfont}

% --- Información del Documento (Ejemplo) ---

\title{Prototipo para apoyar el registro y trazabilidad de estados en el proceso de fotocomparendos aplicando tecnologías de redes distribuidas}
\shorttitle{Gestión Descentralizada de Comparendos}
\author{Laura Catalina Preciado Ballen \\Cristian Stiven Guzman Tovar}
\affiliation{Universidad Distrital Francisco José de Caldas}
\course{Proyecto de grado para optar al título de: \\Ingeniero (a) de Sistemas}
\professor{Julio Baron Velandia, Ph.D}
\duedate{\today}
\abstract{Este trabajo propone el diseño e implementación de un prototipo basado en blockchain para la gestión de fotocomparendos en Bogotá, con el objetivo de garantizar la transparencia del proceso. Se utilizan contratos inteligentes para registrar cada infracción, permitiendo que cualquier actor autorizado pueda verificar su autenticidad sin necesidad de intermediarios. El prototipo, validado en entorno de laboratorio con datos sintéticos, demuestra mediante pruebas funcionales y de integridad cómo esta tecnología puede fortalecer la confianza en los procesos de control de tránsito y mejorar la eficiencia en la gestión de sanciones.}
\keywords{Fotocomparendos, Gestión de datos, Blockchain, Ingeniería web}

\begin{document}

% Configuración APA 7: Sangría y espaciado
\setlength{\parindent}{1.27cm} % 0.5 pulgadas de sangría en primera línea
\doublespacing % Doble espacio según APA 7
% \maketitle
\begin{titlepage}
	% Número de página en esquina superior derecha según APA 7
	\AddToShipoutPictureBG*{%
		\AtPageUpperLeft{%
			\put(\LenToUnit{\paperwidth-2.54cm},\LenToUnit{-2.54cm}){%
				\makebox[0pt][r]{\normalfont\normalsize 1}%
			}%
		}%
	}
	\begin{center}
		\vspace*{0.5cm}

		\Large
		\textbf{Prototipo para apoyar el registro y trazabilidad de estados en el proceso de fotocomparendos aplicando tecnologías de redes distribuidas}


		\vspace{2.5cm}

		\normalsize
		\textbf{Laura Catalina Preciado Ballen}\\
		\textbf{Cristian Stiven Guzman Tovar}

		\vfill

		Proyecto de Monografía para Optar por el Título de\\
		Ingeniero(a) de Sistemas

		\vspace{0.8cm}

		Director:\\
		\textbf{Julio Baron Velandia, Ph.D}

		\vspace{0.8cm}

		\includegraphics[width=0.2\textwidth]{Images/Escudo_UD}

		\large
		Universidad Distrital Francisco José De Caldas\\
		Facultad de Ingeniería\\
		Colombia, Bogotá D.C.\\
		Enero de 2026\\



	\end{center}
\end{titlepage}

\raggedbottom

\pagenumbering{roman}
% Contenido
\renewcommand\contentsname{\textbf{Índice}}
\tableofcontents
\setcounter{tocdepth}{2}
\newpage
% Fíguras
\renewcommand{\listfigurename}{\textbf{Índice de figuras}}
\listoffigures
\newpage
% Tablas
\renewcommand{\listtablename}{\textbf{Índice de tablas}}
\listoftables
\newpage

% Cuerpo
\pagenumbering{arabic}
\setcounter{page}{2} % Continuar desde 2 (portada fue 1)

% Incluir capítulos modularizados
\section{\large Introducción}
En Colombia, la gestión de fotocomparendos ha sido objeto de controversia debido a fallas en la transparencia y posibles manipulaciones en el proceso de registro y validación de infracciones. La falta de un sistema confiable ha generado desconfianza entre los ciudadanos, lo que evidencia la necesidad de una solución que garantice la integridad, inmutabilidad y verificabilidad de la información.

Las tecnologías de registro distribuido (DLT), y en particular blockchain, han demostrado ser alternativas eficaces para el almacenamiento seguro y descentralizado de datos, asegurando que una vez registrados, estos no puedan ser alterados sin dejar rastro (ver Capítulo 3 para fundamentos teóricos detallados). A través de contratos inteligentes, es posible automatizar la validación y el procesamiento de fotocomparendos, reduciendo la intervención humana y minimizando el riesgo de corrupción o errores administrativos.

Este trabajo propone el diseño e implementación de un prototipo basado en una arquitectura híbrida blockchain para la gestión de fotocomparendos en Bogotá, con el objetivo de garantizar la transparencia del proceso. Se utilizarán contratos inteligentes para registrar cada infracción, permitiendo que cualquier actor autorizado pueda verificar su autenticidad sin necesidad de intermediarios. Mediante pruebas y simulaciones, se evaluará la viabilidad del sistema, demostrando cómo esta tecnología puede fortalecer la confianza en los procesos de control de tránsito y mejorar la eficiencia en la gestión de sanciones.

\subsection{Formulación del problema}
La gestión de comparendos en Bogotá es un proceso de gran escala. Según datos del Observatorio de Movilidad, entre enero de 2018 y agosto de 2024 se emitieron más de 1.9 millones de comparendos a través de cámaras salvavidas, evidenciando la importancia sistémica de este proceso para la regulación del tránsito en la ciudad, como se presenta en la Figura~\ref{fig:estadisticas_comparendos} se observa los diferentes métodos utilizados para crear los comparendos. Esta operación se apoya en el sistema FÉNIX, una aplicación con infraestructura en la nube, cuya arquitectura de datos y control de acceso opera bajo un paradigma centralizado.

\begin{figure}[htbp]
    \begin{flushleft}
        \textbf{Figura 1}\\[2em]
        \textit{Estadísticas de comparendos emitidos en Bogotá entre enero de 2018 y agosto de 2024}
    \end{flushleft}
    \vspace{1em}
    \addcontentsline{lof}{figure}{Figura 1. Estadísticas de comparendos emitidos en Bogotá entre enero de 2018 y agosto de 2024}
    \centering
    \includegraphics[width=0.8\textwidth]{Images/numComparendos.png}
    \vspace{2em}
    \begin{flushleft}
        \textit{Nota.} Elaboración propia basado en datos del Observatorio de Movilidad.
    \end{flushleft}
    \refstepcounter{figure}\label{fig:estadisticas_comparendos}
\end{figure}

En el sistema actual, la validez e inmutabilidad de los registros de infracciones se fundamenta en los procedimientos administrativos y en la gestión de los funcionarios responsables del sistema \parencite{C112_2018}. Los cambios en la información solo pueden ser detectados por las entidades autorizadas, lo que implica que el control sobre los registros depende directamente de la correcta aplicación de las políticas internas y del seguimiento realizado por dichas entidades \parencite{Sentencia123_2019}.

La evidencia generada se conserva bajo un modelo centralizado, en el cual la confianza en la integridad de los datos se sostiene en mecanismos administrativos y controles internos, más que en garantías técnicas accesibles públicamente \parencite{DAFP_Lineamientos_2021}. La potestad sancionatoria y el debido procedimiento administrativo aseguran la validez de los actos administrativos y la correcta motivación en la imposición de sanciones (Corte Constitucional, 2022; Gamero Casado y Fernández Ramos, s.f.).

De acuerdo con la Auditoría de Cumplimiento de la Contraloría de Bogotá (2024), en el proceso de desarrollo del sistema FÉNIX se identificaron dificultades relacionadas con la supervisión contractual, lo que derivó en retrasos, duplicidad de sistemas y un presunto detrimento patrimonial estimado en más de \$8.000 millones de pesos. Estos hallazgos reflejan que, desde su implementación, la plataforma ha enfrentado retos significativos en materia de gobernanza y gestión, los cuales han tenido impacto en la eficiencia administrativa y en la sostenibilidad financiera del proyecto.

Estas debilidades se manifiestan en la operación técnica actual. A nivel operativo, el riesgo de integridad se materializa en una fricción a gran escala con la ciudadanía. Un análisis correlacional de fuentes oficiales para el primer semestre de 2025 revela la magnitud de esta fricción: frente a 457.000 comparendos impuestos [Observatorio de Movilidad, 2025], se gestionaron 155.854 PQRSD [Informe de Gestión PQRSD, 2025].

De estos datos se deduce una Tasa de Impugnación general del 34.1\%, un indicador cuantitativo que sugiere que al menos uno de cada tres actos administrativos del sistema genera una disputa formal, reflejando una carga administrativa insostenible y un déficit de confianza.

La desconfianza generada por estas opacidades y dificultades procesales crea un vacío que es explotado por terceros, afectando directamente al ciudadano. Reportajes de prensa documentan cómo la ausencia de canales oficiales percibidos como confiables ha fomentado la aparición de redes de fraude, como el caso de Juzto.co, donde miles de ciudadanos fueron estafados con promesas de impugnaciones garantizadas, resultando en trámites inconclusos y mayores deudas \parencite{Semana_Juzto_2023}.

La identificación de estas limitaciones permite estructurar el problema en torno a variables que reflejan tanto el modelo de confianza actual como sus impactos técnicos, operativos y financieros. La Tabla~\ref{tab:comparacion_bd_blockchain} sintetiza estas variables y los indicadores asociados, mostrando cómo el paradigma centralizado de gestión condiciona la integridad de los datos, la eficiencia administrativa, la confianza ciudadana y la sostenibilidad del sistema.

\begin{table}[htbp]
    \centering
    \caption{Comparación entre bases de datos tradicionales y blockchain para gestión de registros gubernamentales}
    \begin{tabular}{p{4.5cm} p{5.2cm} p{5.2cm}}
        \toprule
        \textbf{Característica} & \textbf{Base de Datos Convencional} & \textbf{Blockchain} \\
        \midrule
        Modelo de confianza & Se basa en un administrador central (entidad de TI) & Confianza distribuida entre múltiples nodos \\
        Inmutabilidad & Registros pueden ser modificados o eliminados por administradores & Los registros son inmutables por diseño \\
        Trazabilidad / Auditoría & Depende de la implementación y control interno & Historial completo e inalterable disponible \\
        Riesgo de corrupción interna & Alto, si hay privilegios indebidos o colusión & Bajo, no se puede alterar sin consenso de la red \\
        Seguridad criptográfica & Opcional, no siempre integrada nativamente & Integrada (firmas digitales, hashes, cifrado) \\
        Disponibilidad / tolerancia a fallos & Riesgo de puntos únicos de falla & Alta disponibilidad por replicación descentralizada \\
        Velocidad de operación & Alta velocidad en lectura/escritura & Menor velocidad, prioriza integridad y consenso \\
        \bottomrule
    \end{tabular}
    \vspace{1em}
    \begin{flushleft}
        \textit{Nota.} Elaboración propia.
    \end{flushleft}
    \refstepcounter{table}\label{tab:comparacion_bd_blockchain}
\end{table}

En síntesis, el problema se formula como un Riesgo de Integridad de Datos inherente al paradigma de confianza centralizada del sistema de fotocomparendos. Este riesgo se encuentra documentado por debilidades fundacionales en la gobernanza del proyecto y se manifiesta en consecuencias medibles: (i) una Tasa de Impugnación del 34.1\%; (ii) una carga operativa superior a 155 mil PQRSD semestrales; (iii) un presunto detrimento patrimonial por más de \$8.000 millones; y (iv) la vulnerabilidad de la ciudadanía a esquemas fraudulentos derivados de la falta de transparencia institucional.

Ante este panorama, surge la necesidad de explorar arquitecturas que permitan sustituir la confianza administrativa por garantías criptográficas. La pregunta central que guía este trabajo es:

\textbf{¿Cómo mitigar el Riesgo de Integridad de Datos en el proceso de fotocomparendos en Bogotá?} 

\subsection{Objetivos}
\paragraph{Objetivo General}
Desarrollar un prototipo para apoyar el registro y trazabilidad de estados en el proceso de fotocomparendos en Bogotá, aplicando tecnologías de redes distribuidas, con el fin de fortalecer la integridad, la autenticidad de la información, y reducir los riesgos asociados a su confidencialidad.

\paragraph{Objetivos específicos}
\begin{itemize}
    \item Analizar el proceso actual de registro de fotocomparendos en Bogotá, a partir del marco jurídico y regulatorio que lo rige y de los informes de auditoría emitidos por la secretaria distrital de movilidad sobre la gestión de comparendos, para identificar requisitos funcionales, no funcionales y vulnerabilidades que el prototipo debe proporcionar.
    \item Desarrollar un prototipo con arquitectura híbrida fundamentada en la técnica de descomposición por confianza, integrando tecnologías de almacenamiento distribuido y contenido cifrado, asegurando que cada transacción incorpore los metadatos del comparendo y disponiendo de una interfaz básica para demostrar que es posible un aplicativo transparente y confiable.
    \item Evaluar la viabilidad del prototipo desarrollado, mediante la ejecución de un plan de pruebas funcionales y evaluación de métricas de desempeño en un entorno de pruebas, para validar las condiciones de inmutabilidad, trazabilidad y seguridad.
\end{itemize}

\subsection{Impacto esperado}

El desarrollo de este prototipo tiene como propósito demostrar la viabilidad técnica de una arquitectura descentralizada para la gestión de fotocomparendos, con el potencial de:

\begin{itemize}
    \item \textbf{Fortalecer la confianza ciudadana:} Mediante la verificación independiente de infracciones y el acceso transparente a la información, sin intermediarios.
    \item \textbf{Reducir la fricción operativa:} Disminuir los recursos destinados a la gestión de PQRSD y disputas administrativas, actualmente estimados en más de 155.000 casos semestrales.
    \item \textbf{Prevenir fraudes:} Mitigar la vulnerabilidad de los ciudadanos ante esquemas de estafa derivados de la falta de canales oficiales confiables.
    \item \textbf{Establecer un precedente técnico:} Servir como referencia para la implementación de soluciones similares en otros procesos gubernamentales que requieran alta integridad de datos.
\end{itemize}

\textbf{Nota:} Para la especificación detallada del alcance del proyecto, los componentes del prototipo, criterios de éxito y limitaciones metodológicas, consultar el Capítulo 6: Alcance y Limitaciones. 
\section{\large Justificación}
La gestión de registros públicos, como los fotocomparendos, en arquitecturas centralizadas presenta debilidades en materia de seguridad, transparencia y auditabilidad. En Bogotá, el sistema FÉNIX ilustra estos desafíos, según auditorías de la Contraloría \parencite{Informe170100005424} \parencite{InformeCumplimiento90}, que destacan limitaciones en la integridad de los datos y una fricción operativa reflejada en más de 153.000 PQRSD en un semestre. Esta situación subraya la necesidad de modelos arquitectónicos alternativos que fortalezcan la confianza pública, independizándola de la dependencia exclusiva en procedimientos y administradores internos. Se transita así de un sistema donde la integridad se presume y se audita retrospectivamente, a uno donde es intrínseca y verificable criptográficamente desde el origen.

El propósito de este proyecto no es modificar el sistema actual, sino diseñar y evaluar un prototipo autocontenido que demuestre un modelo de confianza diferente. Para ilustrar las diferencias estructurales entre el modelo convencional y el propuesto, la Tabla \ref{tab:comparacion_modelos} compara sus características clave:

\footnotesize
\renewcommand{\arraystretch}{1.15}
\setlength{\LTpre}{10pt}
\setlength{\LTpost}{10pt}
\begin{longtable}{p{1.8cm} p{2.8cm} p{2.8cm} p{3.5cm}}

\caption{Comparación entre un modelo centralizado y un modelo descentralizado}
\label{tab:comparacion_modelos} \\
\toprule
\textbf{Característica} & \textbf{Modelo Centralizado} & \textbf{Modelo Descentralizado} & \textbf{Relevancia Contextual} \\
\midrule
\endfirsthead

\caption[]{(Continuación)} \\
\toprule
\textbf{Característica} & \textbf{Modelo Centralizado} & \textbf{Modelo Descentralizado} & \textbf{Relevancia Contextual} \\
\midrule
\endhead

\midrule
\multicolumn{4}{r}{\textit{Continúa en la siguiente página}} \\
\endfoot

\bottomrule
\multicolumn{4}{p{13cm}}{\textbf{Nota.} Elaboración propia, con hallazgos basados en la Auditoría de Cumplimiento No. 90 de la Contraloría de Bogotá D.C. (octubre de 2023) y la Auditoría de Cumplimiento 170100-0054-24.} \\
\endlastfoot

Modelo de Confianza & Basado en la confianza en los administradores del sistema y en la robustez de los controles internos definidos. & Basado en un consenso criptográfico distribuido, donde la confianza reside en el protocolo y no en un intermediario. & La correcta asignación de roles es fundamental. La auditoría observó ``ausencia de un profesional responsable de Seguridad de la Información'' (págs. 20--25), subrayando la criticidad de los factores de gobernanza. \\
\midrule
Integridad de Datos & La integridad se asegura mediante controles de acceso y logs de auditoría internos gestionados por la entidad. & La integridad es una propiedad intrínseca de la estructura de datos; los registros son inmutables por diseño. & La efectividad de los controles internos es fundamental. La auditoría documentó ``Falta de control sobre la integridad y calidad de los datos migrados'' (págs. 38--40) como punto de atención. \\
\midrule
Gestión de Seguridad & Dependiente de políticas y procedimientos de seguridad definidos y ejecutados por la institución. & La seguridad es una propiedad inherente a la capa de protocolo, auditada de forma continua y global por la comunidad. & La formalización de procedimientos es clave. La auditoría identificó ``falta de gestión formal de riesgos y controles'' y ``ausencia de un plan de seguridad para la infraestructura en la nube'' (págs. 25--30). \\
\midrule
Auditabilidad y Trazabilidad & La auditoría se realiza a través de logs internos, con acceso gestionado por la entidad y sujeto a sus políticas de retención y seguridad. & La traza de auditoría es transparente, inalterable por diseño y públicamente verificable por cualquier actor autorizado. & La consistencia de los registros internos es un factor de éxito. La auditoría observó ``retrasos y baja velocidad de desarrollo'' (págs. 15--20), subrayando la importancia de una gobernanza rigurosa. \\
\bottomrule
\end{longtable}



\subsection{Pertinencia social, tecnológica y legal}

La pertinencia de este proyecto se enmarca en tres dimensiones complementarias que justifican la necesidad de explorar arquitecturas descentralizadas para la gestión de registros públicos críticos:

\begin{itemize}
    \item \textbf{Social y ciudadana:} En un contexto donde la desconfianza en los procesos administrativos genera una tasa de impugnación del 34.1\% (más de 155.000 PQRSD semestrales según se detalla en el Estado del Arte, subsección sobre el contexto de Bogotá), este proyecto ofrece un modelo alternativo que responde a la necesidad de transparencia, permitiendo la verificación independiente y empoderando al ciudadano con herramientas de auditoría directa sobre la autenticidad de las sanciones.

    \item \textbf{Tecnológica:} Demuestra cómo la integración de blockchain (para registros inmutables) e IPFS (para evidencias con contenido direccionable) puede abordar los desafíos de seguridad y trazabilidad documentados en sistemas centralizados, respondiendo específicamente a las limitaciones estructurales del sistema FÉNIX identificadas por la Contraloría de Bogotá (ver análisis detallado en el Estado del Arte).

    \item \textbf{Legal e institucional:} El prototipo se alinea con los principios de eficiencia, transparencia y rendición de cuentas exigidos por los organismos de control. Frente a los incumplimientos normativos y brechas de protección de datos identificados en el sistema actual, esta propuesta sirve como caso de estudio sobre cómo garantías técnicas intrínsecas pueden fortalecer el cumplimiento normativo y reducir los riesgos de detrimento patrimonial asociados a modelos centralizados.
\end{itemize}

Para un análisis detallado del contexto operativo y legal del sistema actual de fotocomparendos en Bogotá, así como de las limitaciones críticas identificadas en FÉNIX que motivan esta propuesta, consultar la subsección correspondiente en el Estado del Arte.

\subsection{Originalidad e innovación}
La innovación de esta monografía radica en la concepción del prototipo como un laboratorio para un nuevo modelo de confianza. Mientras los sistemas tradicionales se centran en controles administrativos, esta propuesta explora un modelo distribuido y resistente a la manipulación por diseño. La DApp funciona como una prueba de concepto que integra inmutabilidad, gobernanza automatizada y almacenamiento descentralizado para demostrar una solución a una clase de problemas que las bases de datos centralizadas, por su naturaleza, no pueden resolver de manera nativa.

\subsection{Relación del impacto con los objetivos del proyecto}

Este prototipo responde a una problemática documentada en Bogotá y se inserta en la tendencia global de GovTech. El impacto esperado del proyecto (detallado en la sección de Impacto esperado del Capítulo 1) se manifiesta en dimensiones técnicas, sociales e institucionales que se alinean directamente con los objetivos específicos planteados:

\begin{itemize}
    \item \textbf{Confianza por Diseño:} La verificación independiente fortalece la legitimidad de los procesos públicos, cumpliendo con el objetivo de garantizar integridad y autenticidad mediante arquitecturas descentralizadas.

    \item \textbf{Gobernanza Automatizada:} Los contratos inteligentes ejecutan reglas de negocio de forma predecible, reduciendo la dependencia de supervisión humana y minimizando riesgos de corrupción, alineándose con el objetivo de desarrollar un sistema transparente y confiable.

    \item \textbf{Escalabilidad en GovTech:} Este caso de uso es transferible a otros procesos que demandan alta integridad, sirviendo como precedente para futuras innovaciones en la administración pública y cumpliendo con el objetivo de establecer un modelo replicable.
\end{itemize}

La adopción de blockchain en esta propuesta no es una preferencia tecnológica, sino una respuesta técnica deliberada a los desafíos de integridad y confianza inherentes a los modelos centralizados, proponiendo una arquitectura donde la veracidad es una propiedad intrínseca y verificable del sistema. 
\section{\large Marco teórico}
El marco conceptual y tecnologico que sustenta la propuesta del prototipo, presentan las teorías y modelos clave que justifican la selección de Blockchain e IPFS como componentes centrales, evidencian los principios inherentes de integridad, transparencia, resiliencia y auditabilidad en la gestión de evidencia digital crítica como los fotocomparendos.
\subsection{El paradigma de la confianza descentralizada}
Los sistemas de información tradicionales suelen depender de intermediarios centralizados o autoridades certificadoras para validar transacciones y garantizar la fiabilidad de los registros. La teoría de los modelos de confianza descentralizada, en cambio, analiza cómo establecer y mantener la confianza en entornos distribuidos donde tales autoridades centrales están ausentes \parencite{swan2015blockchain}.

La relevancia de este modelo es fundamental para justificar el uso de la tecnología Blockchain en la gestión de fotocomparendos, ya que su propósito es precisamente reemplazar la necesidad de depositar confianza exclusiva en una única entidad para la custodia, validación e integridad de los registros. Blockchain habilita un cambio de paradigma: en lugar de confiar en un actor central, la confianza se distribuye y se deposita en la robustez del protocolo criptográfico subyacente \parencite{nakamoto2008bitcoin}, en la transparencia de las reglas del sistema y en el consenso mayoritario de los participantes de la red \parencite{antonopoulos2023mastering}. Este enfoque reduce drásticamente los puntos únicos de fallo y los vectores de corrupción asociados a la dependencia de intermediarios centralizados, quienes podrían ser comprometidos, cometer errores o actuar de manera malintencionada.

\subsection{Fundamentos de los sistemas distribuidos y redes descentralizadas}
El paradigma de la confianza descentralizada se sustenta en la teoría de los sistemas distribuidos, donde múltiples entidades autónomas, denominadas nodos, colaboran a través de una red para alcanzar un objetivo común, compartiendo tanto la carga computacional como el almacenamiento de datos \parencite{vanSteen2017}. Estos sistemas se fundamentan en principios como la distribución de recursos, la comunicación inter-nodo y mecanismos de coordinación que prescinden de intermediarios centrales \parencite{coulouris2011}.

La relevancia de esta teoría para el presente proyecto es primordial, ya que tanto Blockchain como el InterPlanetary File System (IPFS) son implementaciones nativas de sistemas distribuidos. Su adopción conjunta promueve inherentemente:
\begin{itemize}
    \item \textbf{Resiliencia:} Al eliminar puntos únicos de fallo (Single Points of Failure - SPOF).
    \item \textbf{Alta Disponibilidad:} Al permitir el acceso a datos y servicios desde múltiples nodos.
    \item \textbf{Resistencia a la Censura:} Dado que ninguna entidad individual posee control absoluto sobre la red o los datos almacenados \parencite{antonopoulos2023mastering}.
\end{itemize}

Una característica esencial de estos sistemas es su arquitectura de red \textbf{Peer-to-Peer (P2P)}, donde los participantes se conectan y comparten recursos directamente entre sí, sin necesidad de un servidor central. En una red P2P, cada nodo puede actuar simultáneamente como cliente y servidor, lo que posibilita que el registro distribuido (ledger) se mantenga sincronizado y que los archivos puedan ser recuperados desde múltiples fuentes, garantizando la integridad de la información sin depender de una autoridad central.

\subsection{Tecnologías para la gestión descentralizada de evidencia}
Para materializar un sistema de gestión de fotocomparendos descentralizado, se requiere la sinergia de dos tipos de tecnologías: una para el registro inmutable de transacciones y otra para el almacenamiento verificable de la evidencia.

\subsubsection{Blockchain: un registro distribuido, inmutable y transparente}
Blockchain es un tipo específico de Tecnología de Ledger Distribuido (DLT), un sistema de registro digital caracterizado por ser distribuido, sincronizado y asegurado criptográficamente entre múltiples participantes \parencite{narayanan2016bitcoin}. Su estructura fundamental se compone de \textbf{transacciones} —operaciones firmadas digitalmente que modifican el estado del ledger de forma permanente \parencite{antonopoulos2023mastering}— agrupadas en bloques. Cada bloque contiene un hash criptográfico que lo vincula al anterior, formando una cadena cronológica e inmutable.

La \textbf{inmutabilidad} y la \textbf{transparencia} son los beneficios centrales que esta tecnología aporta \parencite{swan2015blockchain,antonopoulos2023mastering}. La primera se logra mediante la estructura encadenada y los mecanismos de consenso distribuido (ej., Proof-of-Work \parencite{nakamoto2008bitcoin} o Proof-of-Stake \parencite{king2012ppcoin}), que hacen que la modificación de un bloque pasado sea computacionalmente prohibitiva. La segunda se habilita por la naturaleza replicada del ledger, permitiendo que actores autorizados puedan consultar y verificar la información de forma independiente. Dentro de este ecosistema, los \textbf{Smart Contracts} (Contratos Inteligentes) actúan como programas autoejecutables cuyo código define e impone automáticamente los términos de un proceso, permitiendo automatizar la gestión del ciclo de vida del comparendo \parencite{szabo1997smart, wood2014ethereum, buterin2014next}.

\paragraph{Modelos arquitectónicos y elección para el prototipo}
La tecnología Blockchain no es monolítica; existen diferentes arquitecturas:
\begin{itemize}
    \item \textbf{Públicas (Permissionless):} Abiertas a cualquier participante, priorizan la descentralización radical (ej. Bitcoin, Ethereum) \parencite{nakamoto2008bitcoin}.
    \item \textbf{Privadas:} Controladas por una única entidad, ofrecen alta eficiencia pero son centralizadas.
    \item \textbf{De Consorcio/Permisionadas (Permissioned):} Operadas por un grupo selecto de participantes autorizados. Ofrecen un equilibrio entre descentralización, rendimiento y confidencialidad, siendo la opción ideal para contextos gubernamentales y empresariales \parencite{vukolic2015quest,cachin2018architecture}.
\end{itemize}
Para este prototipo, se opta por una \textbf{implementación permisionada} (simulada con Hyperledger Fabric), permitiendo que solo entidades autorizadas operen nodos y registren transacciones, con un mecanismo de consenso eficiente (ej. Raft) adecuado para un sistema de gestión de registros.

\subsubsection{IPFS: almacenamiento verificable mediante direccionamiento por contenido}
Los ledgers de Blockchain no están optimizados para almacenar grandes volúmenes de datos (blobs), como las imágenes de los fotocomparendos \parencite{xu2019taxonomy}. Para resolver esto, se utiliza un sistema de almacenamiento descentralizado. La elección de IPFS sobre alternativas centralizadas como AWS S3 es crucial para la integridad del sistema. Mientras que en un sistema centralizado el propietario puede modificar o eliminar unilateralmente un archivo \parencite{vogels2008eventually}, IPFS opera bajo el paradigma del \textbf{direccionamiento por contenido (Content Addressing)} \parencite{benet2014ipfs, voigt2017gdpr}.

En este modelo, la identidad única de un archivo, su Content Identifier (CID), es un \textbf{hash criptográfico} derivado directamente de su contenido. Esto establece un vínculo intrínseco e inmutable: si el contenido del archivo cambia, incluso mínimamente, su CID también cambiará. IPFS es un protocolo y red P2P que utiliza este principio: divide los archivos en bloques, calcula sus hashes y permite su recuperación a través de su CID, utilizando mecanismos como DHT para localizar los nodos que los poseen \parencite{maymounkov2002kademlia, benet2014ipfs}.

\subsection{Arquitectura de la solución: sinergia blockchain-IPFS con el transacción off-chain}
La integración de ambas tecnologías se materializa mediante el patrón de almacenamiento \textbf{off-chain}. El flujo de trabajo es el siguiente:
\begin{enumerate}
    \item La imagen probatoria del comparendo se carga a un nodo IPFS, obteniendo su CID único.
    \item Se crea una transacción en la Blockchain (on-chain) que contiene este CID junto con los metadatos esenciales del comparendo (fecha, hora, lugar, placa).
    \item Esta transacción se valida y registra de forma inmutable en el ledger.
\end{enumerate}
Este enfoque crea un enlace criptográfico inalterable entre el registro oficial (en Blockchain) y la evidencia visual original (en IPFS). Cualquier intento de manipulación de la imagen almacenada en IPFS resultaría en un CID diferente, rompiendo explícitamente la cadena de custodia digital y haciendo que la alteración sea detectable de forma inmediata y algorítmica. La combinación de Blockchain e IPFS no solo sigue los principios de descentralización \parencite{vanSteen2017}, sino que refuerza activamente los objetivos de inmutabilidad verificable y transparencia del sistema.

\subsection{Problemas regulatorios del sistema FÉNIX}
A continuación se sintetizan los principales problemas regulatorios identificados en el sistema FÉNIX, agrupados por norma infringida y evidencia documental.

\small
\renewcommand{\arraystretch}{1.2}
\setlength{\LTpre}{10pt}
\setlength{\LTpost}{10pt}
\begin{longtable}{p{2cm}p{3cm}p{5cm}p{3cm}}

\multicolumn{4}{l}{\textbf{Tabla 4}} \\
\multicolumn{4}{l}{Violaciones al Estatuto de Contratación Pública (Ley 80/1993)} \\

\toprule
\textbf{Artículo} & \textbf{Disposición} & \textbf{Hallazgo en FÉNIX} & \textbf{Fuente} \\
\midrule
\endfirsthead

\toprule
\textbf{Artículo} & \textbf{Disposición} & \textbf{Hallazgo en FÉNIX} & \textbf{Fuente} \\
\midrule
\endhead

\midrule
\multicolumn{4}{r}{\textit{Continúa en la siguiente página}} \\
\endfoot

\bottomrule
\multicolumn{4}{l}{\textbf{Nota.} Elaboración propia.} \\
\endlastfoot

3 & Principios de transparencia y economía & Adendas sin justificación técnica ni económica & Auditoría 90-2023, p. 18 \\
\midrule
40 & Cumplimiento contractual & Modificaciones de alcance sin cláusula de variación & Contraloría 170100-0054-24 \\
\bottomrule
\end{longtable}

\small
\renewcommand{\arraystretch}{1.2}
\setlength{\LTpre}{10pt}
\setlength{\LTpost}{10pt}
\begin{longtable}{p{2.5cm}p{3cm}p{5cm}p{3cm}}

\multicolumn{4}{l}{\textbf{Tabla 5}} \\
\multicolumn{4}{l}{Brechas en protección de datos personales} \\

\toprule
\textbf{Norma} & \textbf{Exigencia} & \textbf{Brecha detectada} & \textbf{Riesgo} \\
\midrule
\endfirsthead

\toprule
\textbf{Norma} & \textbf{Exigencia} & \textbf{Brecha detectada} & \textbf{Riesgo} \\
\midrule
\endhead

\midrule
\multicolumn{4}{r}{\textit{Continúa en la siguiente página}} \\
\endfoot

\bottomrule
\multicolumn{4}{l}{\textbf{Nota.} Elaboración propia.} \\
\endlastfoot

Ley 1581/2012 & Registro ante la SIC & Base de datos no registrada hasta 2022 & Sanción hasta 2.000 SMMLV \\
\bottomrule
\end{longtable}


\begin{table}[htb]
\centering
\caption{Deficits de habilitación sectorial}
\begin{tabular}{|p{3cm}|p{3cm}|p{3cm}|p{4cm}|}
\hline
\textbf{Requisito} & \textbf{Fuente} & \textbf{\% incumplimiento} & \textbf{Consecuencia} \\
\hline
Certificado técnico por cámara & Res. 11245/2020 & 15 \% & Comparendos nulos \\
\hline
\end{tabular}
\end{table}

\subsection{Fundamentos criptográficos aplicados}
La criptografía proporciona los pilares matemáticos que garantizan la seguridad, integridad y autenticidad en todo el ecosistema del prototipo \parencite{katz2020introduction}.
\begin{itemize}
    \item \textbf{Funciones Hash Criptográficas:} Son algoritmos que transforman datos en una huella digital de tamaño fijo. Sus propiedades (unidireccionalidad, resistencia a colisiones, efecto avalancha) son vitales \parencite{schneier2007applied, menezes1996handbook}. En este proyecto, se utilizan para: generar el CID en IPFS, asegurar la integridad de la cadena de bloques y crear identificadores únicos para las transacciones \parencite{benet2014ipfs, nakamoto2008bitcoin}.
    \item \textbf{Criptografía Asimétrica y Firmas Digitales:} Basada en pares de claves (pública y privada), habilita las firmas digitales \parencite{diffie2022new, rivest1978method}. Cuando un usuario autorizado registra un comparendo, utiliza su clave privada para firmar la transacción. Cualquier participante puede usar la clave pública correspondiente para verificar la firma, garantizando así la \textbf{autenticidad} y el \textbf{no repudio} de la acción \parencite{katz2020introduction}.
\end{itemize}
\include{chapters/04_estado_arte}
\section{Metodología}

Este proyecto combina investigación aplicada con desarrollo tecnológico innovador. A continuación se describen el enfoque metodológico, la selección tecnológica y el modelo de desarrollo utilizado.

\subsection{Enfoque metodológico de investigación}

Este trabajo se enmarca en una investigación aplicada que aborda las deficiencias de integridad, transparencia y confianza en el sistema actual de gestión de fotocomparendos en Bogotá. Adopta un enfoque descriptivo al detallar las características de un sistema descentralizado basado en blockchain e IPFS, constituyendo un caso de estudio sobre su aplicación en el sector público.

\subsection{Selección de la pila tecnológica}

La selección de tecnologías de registro distribuido (DLT) fue crítica, impactando directamente en:

\begin{itemize}
    \item Privacidad de datos personales (Ley 1581 de 2012).
    \item Escalabilidad ante $\sim$457.000 comparendos semestrales.
    \item Costos operativos predecibles (sin criptomonedas volátiles).
    \item Modelo de gobernanza institucional.
\end{itemize}

\subsubsection{Justificación del uso de blockchain}

Para el caso de fotocomparendos, donde la \textbf{integridad irrefutable} y la \textbf{verificación ciudadana independiente} son requisitos no negociables, blockchain resulta la tecnología más apropiada frente a otras alternativas de registro distribuido.

La Tabla~\ref{tab:blockchain_vs_dlt} compara blockchain con otras tecnologías emergentes.

\begin{table}[H]
    \centering
    \small
    \caption{Comparación de blockchain con otras tecnologías de registro distribuido}
    \label{tab:blockchain_vs_dlt}
    \begin{tabular}{p{2.5cm}p{2.7cm}p{2.7cm}p{2.7cm}p{2.7cm}}
    \toprule
    \textbf{Criterio} & \textbf{Blockchain} & \textbf{Hashgraph} & \textbf{BD Distribuida} & \textbf{BD Centralizada} \\
    \midrule
    Inmutabilidad & Alta (criptográfica) & Media (consenso virtual) & Baja (config.) & Ninguna (admin.) \\
    \midrule
    Resistencia manipulación & Alta (prohibitiva) & Media & Baja (permisos) & Ninguna (admin) \\
    \midrule
    Auditabilidad & Alta (completa) & Media (parcial) & Baja (logs modificables) & Baja (logs centralizados) \\
    \midrule
    Descentralización & Alta (real) & Alta (real) & Baja (réplicas) & Ninguna \\
    \midrule
    Verificación indep. & Alta (sin confianza) & Media (requiere nodos) & Ninguna (acceso BD) & Ninguna (API controlada) \\
    \midrule
    Estándares & Alta (maduros) & Baja (emergente) & Media (SQL/NoSQL) & Alta (SQL) \\
    \midrule
    Rendimiento (TPS) & Media (15-20.000) & Alta (>10.000) & Alta (>100.000) & Alta (>100.000) \\
    \midrule
    Costo operativo & Alto & Moderado & Moderado & Bajo \\
    \midrule
    Precedente legal & Alto (eIDAS UE) & Bajo (sin precedente) & Medio (aceptado) & Alto (estándar) \\
    \midrule
    \textbf{Apto evidencia legal} & \textbf{SÍ} & Parcial & NO & NO \\
    \bottomrule
    \end{tabular}
    \vspace{1em}
    \begin{flushleft}
        \textit{Fuente.} Elaboración propia.
    \end{flushleft}
\end{table}

\paragraph{Justificación de la elección de blockchain}

La selección de blockchain se fundamenta en los siguientes argumentos técnicos y legales:

\begin{enumerate}
    \item \textbf{Inmutabilidad criptográfica verificable:} A diferencia de bases de datos donde los logs pueden ser alterados por administradores con privilegios elevados, blockchain garantiza que modificar un registro requeriría alterar toda la cadena desde ese punto, lo cual es computacionalmente prohibitivo \parencite{nakamoto2008bitcoin}. Esta propiedad es crítica para evidencia que puede ser objeto de litigio.

    \item \textbf{Verificación sin confianza (trustless):} Un ciudadano puede verificar la autenticidad de un fotocomparendo sin necesidad de confiar en la institución emisora, simplemente validando la cadena de hashes. Esto no es posible con bases de datos tradicionales donde la verificación depende de APIs controladas por la misma entidad \parencite{antonopoulos2023mastering}. Esta característica aborda directamente la crisis de confianza reflejada en la tasa de impugnación del 34.1\%.

    \item \textbf{Precedente legal reconocido:} Existen marcos regulatorios emergentes que reconocen la validez legal de registros blockchain. El Reglamento eIDAS de la Unión Europea \parencite{europa_eidas} establece un marco para la identificación electrónica y servicios de confianza que incluye tecnologías de registro distribuido. Tecnologías más recientes como Hashgraph \parencite{baird2016swirlds} aún no han establecido precedentes legales comparables.

    \item \textbf{Auditabilidad completa e inmutable:} Cada transacción queda registrada con timestamp inmutable, creando una cadena de custodia digital irrefutable para procesos sancionatorios \parencite{swan2015blockchain}. Esta trazabilidad es esencial para cumplir con los requisitos de debido proceso administrativo.

    \item \textbf{Madurez del ecosistema:} Blockchain cuenta con implementaciones probadas en producción (Hyperledger Fabric \parencite{cachin2018architecture}, Ethereum \parencite{wood2014ethereum}), herramientas de desarrollo consolidadas y comunidades activas. Si bien tecnologías como Hashgraph ofrecen mayor rendimiento teórico \parencite{hedera_whitepaper}, o IOTA Tangle \parencite{popov2018tangle} promete eliminación de fees, ninguna ha demostrado la robustez operativa de blockchain en entornos gubernamentales críticos.
\end{enumerate}

Estudios comparativos recientes \parencite{ruan2019blockchainvsdb, karlsson2019permissioned} confirman que, si bien bases de datos distribuidas como Cassandra tienen menor costo operativo y mayor rendimiento bruto, ninguna proporciona el nivel de \textbf{confianza descentralizada} y \textbf{resistencia a manipulación} que requiere un sistema de sanciones gubernamentales donde la percepción de imparcialidad es crítica.

Con esta fundamentación establecida, la siguiente decisión crítica es determinar qué implementación específica de blockchain utilizar y cómo estructurar la arquitectura del sistema.

\paragraph{Arquitectura híbrida: balance entre privacidad y transparencia}

Dado que ninguna blockchain cumple simultáneamente con todos los requisitos (privacidad de datos sensibles + transparencia pública + rendimiento + costos controlados), se optó por una \textbf{arquitectura híbrida}:

\begin{itemize}
    \item \textbf{Capa privada (permisionada):} gestión interna y datos sensibles.
    \item \textbf{Capa pública (blockchain):} verificación ciudadana sin intermediarios.
\end{itemize}

\subsubsection{Capa privada: Hyperledger Fabric}

Para la capa privada se seleccionó Hyperledger Fabric tras un análisis comparativo de plataformas blockchain.

\footnotesize
\renewcommand{\arraystretch}{1.2}
\setlength{\LTpre}{10pt}
\setlength{\LTpost}{10pt}
\begin{longtable}{p{2.4cm}p{2.2cm}p{2cm}p{2cm}p{2cm}p{2cm}}

\caption{Comparativo de plataformas blockchain para selección de arquitectura híbrida}
\label{tab:comparacion_blockchains} \\
\toprule
\textbf{Criterio} & \textbf{Hyperledger Fabric} & \textbf{Ethereum} & \textbf{Corda} & \textbf{Solana} & \textbf{Polygon} \\
\midrule
\endfirsthead

\caption[]{(Continuación)} \\
\toprule
\textbf{Criterio} & \textbf{Hyperledger Fabric} & \textbf{Ethereum} & \textbf{Corda} & \textbf{Solana} & \textbf{Polygon} \\
\midrule
\endhead

\midrule
\multicolumn{6}{r}{\textit{Continúa en la siguiente página}} \\
\endfoot

\bottomrule
\multicolumn{6}{l}{\textbf{Nota.} Elaboración propia.} \\
\endlastfoot

Tipo de red & Permisionada & Pública & Permisionada & Pública & Pública \\
\midrule
Consenso & Raft / BFT & PoS & Notario & PoH + PoS & PoS \\
\midrule
TPS & 2.000--20.000 & $\sim$30 & $\sim$1.000 & 65.000+ & 7.000+ \\
\midrule
Privacidad & Alta$^{(1)}$ & Nula & Alta (P2P) & Nula & Nula \\
\midrule
Smart contracts & Go, Java, Node.js & Solidity & Kotlin/Java & Rust/C & Solidity \\
\midrule
Control de acceso & PKI / Roles & Abierto & Identidad & Abierto & Abierto \\
\midrule
Moneda nativa & No & ETH & No & SOL & MATIC \\
\midrule
Costo / tx & Sin gas & Gas variable & Sin gas & Muy bajo & Muy bajo \\
\midrule
Madurez Gob. & Alta & Media & Alta (banca) & Baja (DeFi) & Media \\
\bottomrule
\end{longtable}


\textbf{Razones de elección de Hyperledger Fabric:}

\begin{itemize}
    \item \textbf{Privacidad y confidencialidad:} canales y colecciones privadas permiten segmentar la información, garantizando que solo entidades autorizadas (agentes, auditores) accedan a datos sensibles, cumpliendo la Ley 1581 de 2012.
    
    \item \textbf{Rendimiento:} 2 000–20 000 TPS, suficiente para el volumen de Bogotá sin cuellos de botella.
    
    \item \textbf{Sin costos de gas:} elimina volatilidad y complejidad, crítico para presupuestos gubernamentales.
    
    \item \textbf{Control de acceso granular:} PKI + roles definidos internamente (admin, agente, auditor, ciudadano).
\end{itemize}

\textbf{Descarte de alternativas:}
\begin{itemize}
    \item \textbf{Ethereum / Solana / Polygon:} públicas $\Rightarrow$ exposición total de datos y costos variables.
    \item \textbf{Corda:} orientada a finanzas; menor flexibilidad para evidencias fotográficas heterogéneas.
\end{itemize}

\subsubsection{Capa pública: Ethereum}

Para la verificación ciudadana se eligió Ethereum (testnet Sepolia) por:

\begin{itemize}
    \item \textbf{Máxima transparencia:} cualquier persona puede verificar metadatos sin permisos.
    
    \item \textbf{Ecosistema maduro:} mayor comunidad, herramientas (Ethers.js, Hardhat) y estándares (ERC-20, ERC-721).
    
    \item \textbf{Costo controlado:} solo se publican hashes y metadatos no sensibles, minimizando gastos de gas.
\end{itemize}

\subsection{Metodología de desarrollo}

Para la construcción del sistema, se seleccionó el \textbf{Modelo de Desarrollo por Prototipos} (\textit{Prototyping Model}). Esta elección metodológica fue estratégica y se fundamenta en las características inherentes al proyecto.

\subsubsection{Justificación de la elección}

La adopción de este modelo iterativo responde a tres factores cruciales:

\begin{enumerate}
    \item \textbf{Naturaleza Innovadora y Riesgo Tecnológico:} El proyecto combina tecnologías emergentes como blockchain (Hyperledger Fabric y Ethereum) e IPFS en un dominio gubernamental donde no existían precedentes locales de una integración similar. La alta incertidumbre sobre el rendimiento, la seguridad de los contratos inteligentes y la viabilidad de la sincronización entre redes heterogéneas requería una validación temprana para mitigar riesgos técnicos fundamentales.
    
    \item \textbf{Requisitos Evolutivos:} Los requisitos funcionales y no funcionales de un sistema de esta naturaleza están sujetos a cambios, tanto por la evolución de la tecnología como por posibles ajustes en el marco normativo de las sanciones de tránsito. El enfoque por prototipos ofrece la flexibilidad necesaria para adaptar la solución de forma ágil a medida que se profundiza el entendimiento del problema.
    
    \item \textbf{Validación Temprana de Conceptos:} Era imperativo demostrar la hipótesis central del proyecto —que la combinación de blockchain e IPFS puede garantizar la inmutabilidad y verificabilidad de la evidencia digital— antes de invertir recursos en el desarrollo de una plataforma completa. El prototipo sirvió como una prueba de concepto funcional para validar esta premisa.
\end{enumerate}

\subsubsection{Fases del proceso de desarrollo}

El ciclo de vida del desarrollo siguió las fases iterativas del modelo de prototipos, adaptadas a los objetivos específicos del proyecto, como se describe en la Tabla~\ref{tab:fases_prototipos}.

\begin{longtable}{p{2.5cm}p{3.5cm}p{6.5cm}}
\toprule
\textbf{Fase} & \textbf{Descripción} & \textbf{Aplicación en el Proyecto} \\
\midrule
\endfirsthead

\toprule
\textbf{Fase} & \textbf{Descripción} & \textbf{Aplicación en el Proyecto} \\
\midrule
\endhead

\midrule
\multicolumn{3}{r}{\textit{Continúa en la siguiente página}} \\
\endfoot

\bottomrule
\endlastfoot

\textbf{1. Requisitos Iniciales} & Recopilación de los requisitos funcionales básicos y esenciales del sistema. & Se definieron las funcionalidades mínimas viables: registro inmutable de multas, almacenamiento de evidencia en IPFS, consulta pública y un mecanismo para la verificación de integridad. \\
\midrule
\textbf{2. Construcción del Prototipo} & Desarrollo rápido de una versión funcional reducida que implementa los requisitos iniciales. & Se implementó un prototipo funcional que incluía un Smart Contract en una red local de Ethereum, una API REST para la comunicación y un frontend básico para la interacción del usuario. \\
\midrule
\textbf{3. Evaluación del Prototipo} & Validación del prototipo mediante pruebas internas para evaluar su funcionalidad y alineación con los objetivos. & Se ejecutó un plan de pruebas exhaustivo (detallado en la sección Plan de pruebas) para validar la inmutabilidad de los registros, la integridad de la evidencia y la usabilidad de la interfaz con datos simulados. \\
\midrule
\textbf{4. Refinamiento e Iteración} & Ajuste y mejora del prototipo basándose en los hallazgos de la evaluación. & Con base en los resultados, se optimizó el consumo de gas del Smart Contract, se mejoraron las validaciones de la API y se refinó la arquitectura para incorporar la capa privada con Hyperledger Fabric. \\
\midrule
\textbf{5. Documentación Final} & Una vez validado el concepto, se documenta la arquitectura final y se proponen los siguientes pasos. & Se consolidó el diseño de la arquitectura híbrida final y se elaboró un \textit{roadmap} detallado para una eventual implementación en un entorno de producción. \\
\bottomrule
\end{longtable}
\addcontentsline{toc}{table}{Tabla 7: Fases del desarrollo del prototipo}
\addcontentsline{toc}{table}{Tabla 7: Fases del desarrollo del prototipo}


\subsubsection{Ventajas y limitaciones del enfoque}

La metodología por prototipos ofreció ventajas estratégicas determinantes para el éxito del proyecto, entre las que destacan la \textbf{validación temprana de la arquitectura híbrida}, la \textbf{mitigación de riesgos técnicos} relacionados con el rendimiento de IPFS y la \textbf{reducción de costos} al permitir ajustes antes de la fase final de desarrollo.

No obstante, es importante reconocer las limitaciones inherentes a este enfoque en el contexto de este trabajo:

\begin{itemize}
    \item \textbf{Rendimiento no representativo:} El prototipo fue evaluado en un entorno de laboratorio controlado, por lo que su rendimiento no refleja las condiciones de una red pública con alta carga transaccional.
    
    \item \textbf{Gestión de expectativas:} Una versión funcional puede generar expectativas en los usuarios de que el sistema está casi terminado, cuando aún requiere fases críticas de seguridad y optimización.
    
    \item \textbf{Disciplina de desarrollo:} Se requirió una disciplina estricta para asegurar que el código del prototipo, concebido para validación, no se promoviera a un entorno de producción sin pasar por procesos formales de auditoría y refactorización.
\end{itemize}

En conclusión, la metodología por prototipos fue fundamental para navegar la complejidad e incertidumbre del proyecto. Permitió demostrar de manera empírica que una arquitectura descentralizada es una solución técnica viable y socialmente pertinente para fortalecer la confianza en la gestión de fotocomparendos en Bogotá. 
\subsection{Artefactos técnicos del diseño}
Con el fin de estructurar de manera clara el desarrollo de la solución propuesta, en esta sección se presentan los principales artefactos utilizados durante la etapa de diseño. Estos elementos permiten representar gráficamente tanto la lógica de funcionamiento como la arquitectura del sistema, sirviendo como guía para la implementación y posterior validación del prototipo.

El conjunto de diagramas que se incluye responde a la necesidad de modelar distintos aspectos del sistema. Por un lado, se usan diagramas de casos de uso para identificar las funcionalidades clave desde la perspectiva del usuario. Por otro, los diagramas de clases permiten definir la estructura del software, mientras que los diagramas de despliegue muestran cómo se distribuyen los componentes en el entorno tecnológico. Además, se incluyen diagramas de flujo que describen el comportamiento del sistema ante eventos específicos, facilitando la comprensión de su dinámica interna.

Cada uno de estos artefactos está alineado con los objetivos del proyecto y fue elaborado considerando tanto las necesidades funcionales como las características propias de las tecnologías involucradas, en particular el uso de Blockchain e IPFS. De esta forma, se busca garantizar coherencia técnica en el diseño y establecer una base sólida para el desarrollo e implementación de la solución.

\section{Alcance}

\subsection{Delimitación geográfica}
Este trabajo se circunscribe al proceso de generación, gestión y verificación de \textbf{multas de tránsito automatizadas (fotomultas)} emitidas por la Secretaría Distrital de Movilidad de Bogotá.  Se excluyen deliberadamente:
\begin{itemize}
  \item Multas impuestas de forma presencial por agentes de tránsito.
  \item Procesos sancionatorios de otras ciudades o entidades territoriales.
  \item Funcionalidades de recaudo y pasarelas de pago (solo se registra el estado del pago, no se procesa el pago en sí).
\end{itemize}

\subsection{Componentes del prototipo}
El prototipo aborda los siguientes módulos funcionales:

\begin{enumerate}
  \item \textbf{Registro inmutable de la infracción}
        Captura de metadatos (placa, fecha, hora, ubicación y tipo de infracción) y publicación del identificador de la evidencia en la \emph{blockchain} (Ethereum local con Hardhat para desarrollo, con arquitectura preparada para Hyperledger Fabric en producción).
  \item \textbf{Almacenamiento descentralizado de evidencias}  
        Carga de la imagen o video de la fotomulta en IPFS y obtención de su \emph{hash}.
  \item \textbf{Verificación pública}  
        Servicio de consulta que permite contrastar el hash guardado en la cadena con el archivo almacenado en IPFS.
  \item \textbf{Gestión del ciclo de vida de la multa}  
        Estados: \textsf{Generada} $\rightarrow$ \textsf{Notificada} $\rightarrow$ \textsf{En apelación} $\rightarrow$ \textsf{Pagada} $\rightarrow$ \textsf{Cerrada}.  
        Cada transición queda registrada mediante eventos de contrato inteligente.
  \item \textbf{Interfaz mínima}  
        Panel Web para: (i) agentes que registran la infracción y (ii) ciudadanos que consultan la autenticidad y el estado de su fotomulta.
\end{enumerate}

\subsection{Fuera del alcance}
\begin{itemize}
  \item Integración completa con sistemas legados del RUNT o SIMIT; se simula mediante datos de prueba.
  \item Implementación de un modelo económico (tarifas de gas, costos operativos reales).
  \item Implementación de algoritmos de detección automática de infracciones (visión por computador).  
        Se parte de que la cámara ya detectó la infracción y generó la evidencia.
\end{itemize}

\subsection{Entregables}
\begin{itemize}
  \item Contrato inteligente en Solidity con 80 pruebas automatizadas (97.5\% de éxito).
  \item Script de despliegue de red Ethereum local (Hardhat) e instalación de IPFS local.
  \item Aplicación Web de demostración (\emph{frontend} ligero) conectada a los servicios anteriores.
  \item Manual técnico que documenta la arquitectura y el flujo de datos.
  \item Informe de resultados de las pruebas funcionales y de rendimiento básico.
\end{itemize}

\subsection{Criterios de éxito}
\begin{enumerate}
  \item Tiempo medio de publicación de una infracción $\leq$ 3 s en entorno de laboratorio.
  \item Coincidencia 100 \% entre el hash almacenado en la cadena y la evidencia recuperada desde IPFS.
  \item Trazabilidad completa del historial de estados para al menos 50 multas de prueba.
  \item Ausencia de fallos críticos en pruebas de carga con 10 transacciones concurrentes.
\end{enumerate}

\section{Limitaciones del prototipo}
Es fundamental reconocer que, como prototipo desarrollado en un contexto académico, el presente estudio presenta ciertas limitaciones que definen el alcance de sus conclusiones y delinean claras oportunidades para futuras investigaciones. Las principales limitaciones son:

\begin{enumerate}
    \item \subsection{Entorno de validación}
    \begin{itemize}
        \item \textbf{Validación en Entorno de Laboratorio:} El prototipo fue diseñado, desplegado y evaluado en un entorno de simulación controlado. No se sometió a pruebas en una infraestructura productiva real con la carga de transacciones y el volumen de usuarios que gestiona actualmente la Secretaría de Movilidad. Por lo tanto, su rendimiento, estabilidad y escalabilidad bajo condiciones de estrés real aún no han sido cuantificados.
        \item \textbf{Uso de Datos Simulados:} Debido a estrictas normativas de privacidad y protección de datos personales que impiden el acceso a información real de ciudadanos y vehículos, todas las pruebas se realizaron con datos sintéticos. Esto implica que el prototipo no fue expuesto a la variabilidad, inconsistencias y casos atípicos que caracterizan a los datos del mundo real, lo cual podría influir en la lógica de negocio y en el manejo de errores en un entorno de producción.
        \item \textbf{Suposiciones sobre la Calidad de la Evidencia:} El sistema asume que las evidencias fotográficas (imágenes de fotocomparendos) son capturadas con una calidad suficiente para su procesamiento. No se implementaron ni probaron mecanismos para manejar escenarios con imágenes de baja resolución, borrosas o con obstrucciones, que son comunes en la operación real.
    \end{itemize}

    \item \subsection{Integración y comparación con sistemas existentes}
    \begin{itemize}
        \item \textbf{Integración Simulada con Sistemas Externos:} La interacción con plataformas gubernamentales clave como el RUNT y el SIMIT fue simulada a través de APIs de prueba (mocks). No se abordaron los desafíos técnicos y burocráticos de una integración real, como los protocolos de comunicación, los tiempos de respuesta, la disponibilidad de los servicios y los posibles cuellos de botella.
        \item \textbf{Ausencia de Benchmarking Directo con el Sistema Actual (Fénix):} La falta de acceso al código fuente y a la arquitectura interna del sistema Fénix impidió realizar una comparación cuantitativa y directa en términos de rendimiento, costos operativos o eficiencia de procesos. El análisis comparativo se basó en las características conceptuales de ambas arquitecturas (centralizada vs. descentralizada).
    \end{itemize}

    \item \subsection{Aspectos técnicos y de escalabilidad}
    \begin{itemize}
        \item \textbf{Proyección de Costos como Escenario de Referencia:} Los costos de infraestructura y desarrollo estimados corresponden a un escenario de referencia. Los costos reales en un despliegue a gran escala podrían variar considerablemente dependiendo de factores como el número de nodos en la red, el volumen de almacenamiento en IPFS, el tráfico de red y la estrategia de persistencia de datos (pinning) que se adopte.
        \item \textbf{Estrategia de Persistencia en IPFS:} Para que la evidencia digital permanezca disponible a largo plazo en IPFS, es necesario que al menos un nodo la mantenga ``pineada''. El prototipo no implementa una política de pinning distribuida y resiliente, lo cual sería un requisito crítico para garantizar la cadena de custodia digital en un sistema de producción.
    \end{itemize}

    \item \subsection{Seguridad y robustez}
    \begin{itemize}
        \item \textbf{Limitaciones en Pruebas de Seguridad Avanzadas:} Si bien el prototipo implementa validaciones básicas de entrada (XSS, SQL injection, path traversal) y manejo de errores a nivel de aplicación, validadas mediante 26 pruebas automatizadas con 100\% de éxito, el alcance del proyecto \textbf{no contempló auditorías de seguridad exhaustivas} como las siguientes:

        \begin{itemize}
            \item \textbf{Análisis estático de contratos inteligentes}: No se emplearon herramientas especializadas como \textit{Slither}, \textit{Mythril} o \textit{MythX} para detectar vulnerabilidades en el código Solidity del contrato \texttt{FineRegistry}.

            \item \textbf{Pruebas de penetración (pentesting)}: No se realizaron ataques simulados avanzados sobre la API REST más allá de las validaciones básicas implementadas.

            \item \textbf{Auditoría de permisos y autenticación}: El prototipo actual implementa validaciones básicas de entrada pero no incluye un sistema robusto de autenticación y autorización (JWT, OAuth2, RBAC) necesario para producción.

            \item \textbf{Validación exhaustiva de límites de archivos IPFS}: Aunque se validan formatos de imagen (JPG, PNG, WEBP) y límites de tamaño (10MB), no se implementaron validaciones avanzadas contra contenido malicioso embebido (steganografía, malware).

            \item \textbf{Protección contra ataques de denegación de servicio (DoS)}: No se implementaron mecanismos de \textit{rate limiting}, \textit{throttling} o \textit{CAPTCHA} para prevenir abusos en los endpoints públicos.
        \end{itemize}

        \paragraph{Trabajo Futuro en Seguridad}
        Se recomienda que, en fases posteriores de desarrollo hacia producción, se incorporen las siguientes medidas:

        \begin{enumerate}
            \item Integración de herramientas de análisis estático como \textit{Slither} para contratos Solidity.
            \item Implementación de un sistema de autenticación y autorización basado en roles (RBAC) con tokens JWT.
            \item Auditoría de seguridad externa realizada por especialistas en \textit{blockchain security}.
            \item Pruebas de penetración automatizadas utilizando herramientas como \textit{OWASP ZAP} o \textit{Burp Suite}.
            \item Implementación de validación de archivos mediante \textit{magic numbers} y análisis de contenido.
            \item Configuración de límites de tasa (\textit{rate limiting}) en la API REST.
        \end{enumerate}

        Estas limitaciones \textbf{no comprometen la validez de la prueba de concepto}, dado que el objetivo principal es demostrar la viabilidad técnica de un modelo de confianza descentralizado basado en inmutabilidad y verificabilidad, no el despliegue de un sistema en producción listo para operar en un entorno real con amenazas activas.
    \end{itemize}
\end{enumerate}
\section{Diseño del Prototipo }

Se hace mencion que apesar que la documentacion para elaborar el software esta en español, es un estandar el escribir codigo en ingles por tanto para mantener coherencia los diagramas mostrados a continuacion se usara este idioma para los nombres de las variables, funciones y clases.
\subsection{Definición de Requisitos:  }
    
\begin{enumerate}
    \item \textbf{Datos sobre infracciones de tráfico:} La captura de datos detallados sobre infracciones de tráfico, como la hora de la infracción, las coordenadas GPS, el tipo de infracción, los datos de identificación del vehículo e imágenes o vídeos, garantiza que cada incidente se documenta exhaustivamente. Este registro exhaustivo proporciona transparencia y responsabilidad, ya que los datos son inmutables y a prueba de manipulaciones una vez almacenados en la cadena de bloques. La inclusión de pruebas mediáticas refuerza aún más la credibilidad y verificabilidad de cada infracción, haciendo que los registros sean sólidos a efectos legales y administrativos. 
    \item \textbf{Información sobre el conductor:} Asociar las infracciones de tráfico a conductores concretos utilizando su dirección Ethereum (clave pública), los datos KYC si es necesario, y los números de identificación del conductor permite un seguimiento y una rendición de cuentas precisos. Esta vinculación permite al sistema personalizar el seguimiento y la verificación de las sanciones, garantizando que las sanciones se atribuyan correctamente a las personas adecuadas. El uso de datos KYC garantiza que las identidades de los conductores puedan verificarse de forma fiable, lo que resulta esencial para mantener la integridad y fiabilidad del sistema.
    \item \textbf{Datos de la sanción: }  Registrando los datos de la sanción, incluyendo el tipo de sanción, el importe de la sanción y el estado del pago de la sanción facilita la ejecución automatizada de las sanciones a través de contratos inteligentes. Esta automatización reduce la carga administrativa de y garantiza que las sanciones se apliquen de forma coherente y transparente. El registro inmutable de las sanciones y su estado de pago en la blockchain garantiza que el proceso sea justo y responsable, proporcionando una pista de auditoría clara para todas las transacciones financieras relacionadas con las infracciones de tráfico.
        \item \textbf{Eventos de contratos inteligentes:} El registro de eventos de contratos inteligentes, como el registro de nuevas infracciones de tráfico o la ejecución de sanciones, con datos relevantes y marcas de tiempo, garantiza que todas las acciones significativas se documenten de forma transparente. Este registro de eventos mejora la trazabilidad y la rendición de cuentas, proporcionando un registro cronológico de las actividades importantes del sistema. Esta transparencia es crucial para las auditorías y revisiones, ya que ayuda a generar confianza en las operaciones del sistema. 
        \item \textbf{Datos de las transacciones de la cadena de bloques: } El seguimiento de los datos de las transacciones de la cadena de bloques, incluido el hash de la transacción, las direcciones del remitente/receptor y las tarifas del gas, proporciona un registro detallado de todas las interacciones dentro del sistema. Estos datos permiten supervisar y auditar las transacciones, garantizando la transparencia y la trazabilidad. Además, hacer un seguimiento de las tarifas de gas ayuda a gestionar y optimizar los costes asociados a la ejecución de transacciones en la blockchain, que es importante para mantener la rentabilidad del sistema. 
        \item \textbf{Dispositivos de datos IoT:} La integración de datos de dispositivos IoT, como sensores o cámaras, junto con marcas de tiempo e identificación del dispositivo, puede mejorar las pruebas recopiladas para infracciones de tráfico. Estos datos en tiempo real proporcionan contexto adicional y pruebas corroborativas, haciendo que los registros de infracciones sean más sólidos y fiables. El uso de dispositivos IoT también puede automatizar la detección y el registro de infracciones, aumentando la eficiencia y la precisión del sistema.
            \item \textbf{Opiniones de los usuarios: } La recopilación de opiniones de los usuarios, incluidos el tipo de opinión, los comentarios y las valoraciones de los usuarios, ayuda a los administradores del sistema a comprender las experiencias y percepciones de los usuarios. Esta información es valiosa para identificar áreas de mejora en y mejorar la usabilidad y funcionalidad del sistema. Involucrar a los usuarios de esta manera puede conducir a un diseño del sistema más centrado en el usuario, mejorando la satisfacción y la eficacia general. 
                \item \textbf{Datos de cumplimiento: } El registro de los datos de cumplimiento, incluido el estado de cumplimiento y los detalles normativos, garantiza que el sistema se adhiere a las leyes y normativas de tráfico locales. Este seguimiento es vital para demostrar el cumplimiento de la normativa y evitar problemas legales. El mantenimiento de registros de cumplimiento detallados también facilita las auditorías reglamentarias en, proporcionando pruebas transparentes de que el sistema funciona dentro de las normas legales, lo que es esencial para generar confianza y credibilidad entre las partes interesadas.
\end{enumerate}

\subsection{Diagrama de casos de uso del sistema de gestión de infracciones de transito }
% img
\begin{figure}[htbp]
    \centering
    \includegraphics[width=0.8\textwidth]{Images/CasosUso.png}
    \caption{Diagrama de casos de uso del sistema de gestión de infracciones de tránsito.}
    \label{fig:casos_uso}
\end{figure}

 \subsection{ Diagrama de Despliegue }
\begin{figure}[htbp]
    \centering
    \includegraphics[width=0.8\textwidth]{Images/Despliegue.png}
    \caption{Diagrama de despliegue de la arquitectura del sistema.}
    \label{fig:diagrama_despliegue}
\end{figure}
En la Figura \ref{fig:diagrama_despliegue} se puede observar el diagrama de despliegue propuesto, donde cada nodo cuenta con la misma información, ya que esta se encontrará sincronizada. Asimismo, se conecta mediante servicios web a la base de datos de Apitude como herramienta de terceros para acceder a la información existente en el Registro Único Nacional de Tránsito (RUNT), de donde se obtendrán los datos de conductores, vehículos y el registro de infractores en Bogotá, así como el estado de las multas.

Hay que mencionar que existen dos soluciones para traer la información necesaria de estas entidades: la primera es una API llamada Apitude, de un tercero que provee la información del RUNT y del SIMIT; la segunda consiste en utilizar los datos que estas entidades públicas ya poseen en bases de datos tradicionales. 
 \subsection{ Diagrama de clases }
Hay que considerar que se manejarán dos capas de lógica: la primera enfocada en registrar los cambios en los estados de las multas a través de blockchain y la segunda capa encargada de la administración general de las multas (manipular los datos que no son visibles al público). 
 \begin{figure}[htbp]
    \centering
    \includegraphics[width=0.8\textwidth]{Images/uml.png}
    \caption{Diagrama de clases del sistema de gestión de multas.}
    \label{fig:diagrama_clases}
\end{figure}
En la Figura \ref{fig:diagrama_clases} se hace un esquema de la primera capa lógica que se encarga de la administración general de las multas y los datos que maneja
 \begin{figure}[htbp]
    \centering
    \includegraphics[width=0.8\textwidth]{Images/ActApelacion.png}
    \caption{Diagrama de actividades para el proceso de apelación de multa.}
    \label{fig:diagrama_apelacion}
\end{figure}
En la Figura \ref{fig:diagrama_apelacion} se hace mención en la segunda capa lógica la cual son los cambios generados en el registro de multas que registramos en la blockchain, que se traducen en los contratos realizados en solidity.
\subsection{ Diagrama de actividades }
 \begin{figure}[htbp]
    \centering
    \includegraphics[width=0.8\textwidth]{Images/ActMulta.png}
    \caption{Diagrama de actividades para el proceso de creación de multa.}
    \label{fig:diagrama_creacion_multa}
\end{figure}
 \begin{figure}[htbp]
    \centering
    \includegraphics[width=0.8\textwidth]{Images/ActApelacion.png}
    \caption{Diagrama de actividades para el proceso de apelación de multa.}
    \label{fig:diagrama_apelacion_2}
\end{figure}

\subsection{Interfaz de Usuario}
\paragraph{Compartidas}
 \begin{figure}[htbp]
    \centering
    \includegraphics[width=0.8\textwidth]{Images/UI1.png}
    \caption{Pantalla de login del sistema.}
    \label{fig:login}
\end{figure}
+En la Figura~\ref{fig:login} se aprecia la pantalla de inicio de sesión, punto de entrada para todos los usuarios autorizados del sistema.

 \begin{figure}[htbp]
    \centering
    \includegraphics[width=0.8\textwidth]{Images/UI2.png}
    \caption{Pantalla de recuperación de contraseña.}
    \label{fig:recuperar_password}
\end{figure}
+La Figura~\ref{fig:recuperar_password} muestra el formulario para recuperar la contraseña, reforzando la experiencia de autoservicio y seguridad de la plataforma.
\paragraph{Vista Agente}
\begin{figure}[htbp]
    \centering
    \includegraphics[width=0.8\textwidth]{Images/UI3.png}
    \caption{Dashboard del agente de tránsito.}
    \label{fig:dashboard_agente}
\end{figure}
+En la Figura~\ref{fig:dashboard_agente} se presenta el tablero principal que resume las métricas de gestión de multas para el agente de tránsito.
\begin{figure}[htbp]
    \centering
    \includegraphics[width=0.8\textwidth]{Images/UI4.png}
    \caption{Pantalla de consulta del estado de multa.}
    \label{fig:consulta_estado_multa}
\end{figure}
+La Figura~\ref{fig:consulta_estado_multa} ilustra la consulta rápida del estado de una multa, facilitando el seguimiento por parte del agente.
\begin{figure}[htbp]
    \centering
    \includegraphics[width=0.8\textwidth]{Images/UI4.png}
    \caption{Pantalla de consulta de detalle de multa.}
    \label{fig:consulta_detalle_multa}
\end{figure}
+En la Figura~\ref{fig:consulta_detalle_multa} se muestra el detalle completo de una multa específica, incluida la evidencia asociada.
\paragraph{Vista Propietario de Vehiculo}
\begin{figure}[htbp]
    \centering
    \includegraphics[width=0.8\textwidth]{Images/UI5.png}
    \caption{Pantalla de consulta de multas para propietarios de vehículos.}
    \label{fig:consulta_multas_propietario}
\end{figure}
+Por último, la Figura~\ref{fig:consulta_multas_propietario} exhibe la vista que permite al propietario del vehículo revisar todas sus multas pendientes o en proceso. 
\section{Implementación del prototipo}

La implementación del prototipo se llevó a cabo siguiendo la arquitectura híbrida \textit{blockchain} diseñada en la sección Diseño del prototipo, integrando \textit{Hyperledger Fabric} para la gestión privada de datos sensibles, \textit{Ethereum} para la transparencia pública e \textit{IPFS} dual para el almacenamiento distribuido de evidencias.

\subsection{Entorno de desarrollo y herramientas}

El desarrollo del prototipo se realizó en un entorno Unix (Linux) utilizando las siguientes herramientas:

\begin{itemize}
    \item \textbf{Sistema Operativo:} Ubuntu 22.04 LTS
    \item \textbf{Control de Versiones:} Git 2.34+ para gestión de código fuente
    \item \textbf{Entorno de Ejecución:} \textit{Node.js} v20.18.0 con npm v10.0.0
    \item \textbf{Gestión de Dependencias:} npm para paquetes \textit{JavaScript}/\textit{TypeScript}
    \item \textbf{IDE:} \textit{Visual Studio Code} con extensiones para \textit{Solidity}, \textit{Go} y \textit{TypeScript}
\end{itemize}

\subsection{Stack tecnológico implementado}

\subsubsection{Tecnologías backend}

El backend del sistema se implementó utilizando tecnologías modernas de JavaScript/TypeScript:

\begin{itemize}
    \item \textbf{Framework Web:} \textit{Express.js} v4.18.2 - Framework minimalista para \textit{Node.js}
    \item \textbf{Lenguaje:} \textit{TypeScript} v5.8.3 - Superset tipado de \textit{JavaScript}
    \item \textbf{Validación:} Express-validator v7.2.1 y Joi v17.13.3 - Validación de datos de entrada
    \item \textbf{Documentación API:} Swagger-jsdoc v6.2.8 y Swagger-ui-express v5.0.1
    \item \textbf{Manejo de Archivos:} Multer v1.4.5-lts.2 - Procesamiento de uploads multipart
    \item \textbf{Cliente HTTP:} Axios v1.9.0 - Comunicación con APIs externas
\end{itemize}

\subsubsection{Tecnologías blockchain}

\paragraph{Capa pública - Ethereum}
Para la implementación de la capa pública (justificada en la sección Metodología), se utilizó el ecosistema de Ethereum con las siguientes tecnologías:

\begin{itemize}
    \item \textbf{Framework de Desarrollo:} \textit{Hardhat} v2.24.0 - Entorno de desarrollo \textit{Ethereum}
    \item \textbf{Biblioteca de Interacción:} \textit{Ethers.js} v6.14.0 - Cliente para interactuar con \textit{Ethereum}
    \item \textbf{Lenguaje de Contratos:} \textit{Solidity} v0.8.28 - Lenguaje para \textit{Smart Contracts}
    \item \textbf{Contratos Base:} \textit{OpenZeppelin Contracts} v5.3.0 - Librería de contratos seguros y auditados
    \item \textbf{Generación de Tipos:} \textit{TypeChain} v8.3.2 - Generación automática de tipos \textit{TypeScript} desde \textit{ABI}
\end{itemize}

\paragraph{Capa privada - Hyperledger Fabric}
Para la implementación de la capa privada (justificada en la sección Metodología), se utilizó \textit{Hyperledger Fabric} con las siguientes tecnologías:

\begin{itemize}
    \item \textbf{Plataforma:} \textit{Hyperledger Fabric} v2.5 - Blockchain permisionada empresarial
    \item \textbf{Lenguaje Chaincode:} \textit{Go} v1.21+ - Lenguaje para desarrollo de \textit{chaincode}
    \item \textbf{SDK:} \textit{Fabric SDK} para \textit{Node.js} - Interacción desde el backend
    \item \textbf{Consenso:} \textit{Raft} - Algoritmo de consenso para tolerancia a fallas
    \item \textbf{Gestión de Identidades:} \textit{Fabric CA} - \textit{Certificate Authority} para control de acceso
\end{itemize}

\subsubsection{Almacenamiento de evidencias}

La implementación de IPFS se realizó en dos capas diferenciadas:

\begin{itemize}
    \item \textbf{Implementación:} \textit{Kubo} v0.34.1 - Implementación de referencia de \textit{IPFS}
    \item \textbf{Cliente \textit{JavaScript}:} \textit{ipfs-http-client} v60.0.1 - \textit{API HTTP} para \textit{IPFS}
    \item \textbf{\textit{IPFS} Privado:} Nodo local para almacenamiento de evidencias completas
    \item \textbf{\textit{IPFS} Público:} Gateway público para \textit{hashes} de verificación ciudadana
    \item \textbf{Protocolo de Contenido:} \textit{Multiformats} v13.3.3 - Manejo de \textit{CIDs}
\end{itemize}

\subsubsection{Tecnologías frontend}

El frontend se desarrolló con tecnologías modernas de React:

\begin{itemize}
    \item \textbf{Framework:} \textit{React} v18.3.1 - Biblioteca para interfaces de usuario
    \item \textbf{Bundler:} \textit{Vite} v5.4.2 - Herramienta de build ultrarrápida
    \item \textbf{Lenguaje:} \textit{TypeScript} v5.5.3 - Tipado estático
    \item \textbf{Estilos:} \textit{Tailwind CSS} v3.4.1 - Framework de utilidades \textit{CSS}
    \item \textbf{Enrutamiento:} \textit{React Router DOM} v6.22.3 - Navegación entre vistas
    \item \textbf{Estado Global:} \textit{Zustand} v4.5.2 - Gestión de estado ligera
    \item \textbf{Gráficos:} \textit{Recharts} v2.12.3 - Librería de visualización de datos
    \item \textbf{Iconos:} \textit{Lucide React} v0.344.0 - Iconos modulares
\end{itemize}

\subsubsection{Frameworks de testing}

Se implementaron pruebas automatizadas en múltiples capas:

\begin{itemize}
    \item \textbf{Backend:} \textit{Vitest} v3.2.3 - Framework de testing para \textit{Vite}
    \item \textbf{Frontend:} \textit{Jest} v30.0.3 - Framework de testing para \textit{React}
    \item \textbf{\textit{Smart Contracts}:} \textit{Hardhat Testing} - Framework integrado de \textit{Hardhat}
    \item \textbf{Aserciones:} \textit{Chai} v4.5.0 - Librería de aserciones
    \item \textbf{Testing de \textit{UI}:} \textit{React Testing Library} v16.3.0 - Testing de componentes \textit{React}
\end{itemize}

\subsection{Capa pública Ethereum}

La arquitectura de la capa pública se describe en detalle en la sección Diseño del prototipo. Esta sección se enfoca en los aspectos específicos de implementación técnica.

\subsubsection{Desarrollo del smart contract}

El \textit{Smart Contract} FineManagement.sol implementa la lógica de negocio para la gestión pública de infracciones de tránsito. El contrato se desarrolló en \textit{Solidity} v0.8.28 y hereda de \textit{Ownable} (\textit{OpenZeppelin}) para control de acceso.

\paragraph{Estructura de datos}
El contrato define dos estructuras principales para modelar las multas y su historial de estados:

\begin{verbatim}
enum FineState {
    PENDING,
    PAID,
    APPEALED,
    RESOLVED_APPEAL,
    CANCELLED
}
\end{verbatim}

Este enum define los cinco estados que el \textit{smart contract} del prototipo implementa como prueba de concepto. Es importante destacar que el código utiliza nomenclatura en inglés siguiendo las mejores prácticas de la industria del desarrollo de \textit{software} \textit{blockchain}, donde \textit{Solidity} y \textit{Ethereum} tienen convenciones establecidas en este idioma.

La Tabla~\ref{tab:mapeo_estados} presenta el mapeo sistemático entre estos estados implementados en el \textit{smart contract} (inglés) y los estados conceptuales del proceso de fotocomparendos descritos en la Introducción (sección 1.2). Como se observa en la tabla, el prototipo implementa un subconjunto funcional de los ocho estados conceptuales completos, siendo suficiente para validar los principios de inmutabilidad y trazabilidad. Los estados NOTIFICADA y CERRADA no fueron implementados en esta versión pero pueden agregarse en una implementación de producción sin modificaciones arquitectónicas sustanciales.

\footnotesize
\renewcommand{\arraystretch}{1.2}
\setlength{\LTpre}{10pt}
\setlength{\LTpost}{10pt}
\begin{longtable}{p{3.8cm} p{3.8cm} p{4.8cm}}

\caption{Mapeo entre estados conceptuales y estados del \textit{smart contract}}
\label{tab:mapeo_estados} \\
\toprule
\textbf{Estado Conceptual (ES)} & \textbf{Estado en \textit{Smart Contract} (EN)} & \textbf{Descripción del Mapeo} \\
\midrule
\endfirsthead

\caption[]{(Continuación)} \\
\toprule
\textbf{Estado Conceptual (ES)} & \textbf{Estado en \textit{Smart Contract} (EN)} & \textbf{Descripción del Mapeo} \\
\midrule
\endhead

\midrule
\multicolumn{3}{r}{\textit{Continúa en la siguiente página}} \\
\endfoot

\bottomrule
\multicolumn{3}{p{13cm}}{\textbf{Nota.} Mapeo de estados del \textit{smart contract}.} \\
\endlastfoot

GENERADA / PENDIENTE\_RESPUESTA & PENDING & Estado inicial tras registro del comparendo. Engloba tanto la generación inicial como el período de espera de respuesta ciudadana. El \textit{smart contract} usa PENDING para representar cualquier comparendo que aún no ha sido resuelto mediante pago, apelación o cancelación. \\
\midrule
PAGADA & PAID & Comparendo cuya obligación económica ha sido saldada. Mapeo directo 1:1 entre concepto y código. Representa el cierre exitoso del proceso mediante pago voluntario o forzoso. \\
\midrule
EN\_APELACION & APPEALED & Comparendo bajo proceso de revisión por PQRSD ciudadana. Mapeo directo 1:1. Este estado es crítico para la trazabilidad de disputas y fundamenta la necesidad de inmutabilidad de evidencia. \\
\midrule
RESUELTA\_APELACION & RESOLVED\_APPEAL & Apelación procesada con decisión administrativa (confirmación, revocación parcial/total, o anulación). Mapeo directo 1:1. Estado terminal para el flujo de apelaciones. \\
\midrule
CANCELADA & CANCELLED & Comparendo cancelado administrativamente por anulación judicial, defectos procedimentales, o corrección de errores. Mapeo directo 1:1. Requiere máxima trazabilidad para prevenir cancelaciones irregulares. \\
\midrule
NOTIFICADA & \textit{No implementado} & Este estado conceptual no tiene equivalente en el \textit{smart contract} del prototipo. En una implementación de producción, se agregaría estado NOTIFIED entre PENDING y las transiciones subsecuentes. La notificación podría registrarse mediante evento \textit{blockchain} con \textit{timestamp} inmutable. \\
\midrule
CERRADA & \textit{No implementado} & Estado final conceptual que indica cierre definitivo del proceso. En el prototipo, los estados PAID, RESOLVED\_APPEAL y CANCELLED funcionan como estados terminales. Una implementación completa podría agregar estado CLOSED explícito para uniformidad. \\
\bottomrule
\end{longtable}


\begin{verbatim}
struct Fine {
    uint256 id;
    string plateNumber;
    string evidenceCID;        // \textit{CID} de \textit{IPFS} público
    string location;
    uint256 timestamp;
    string infractionType;
    uint256 cost;
    string ownerIdentifier;
    FineState currentState;
    address registeredBy;
    string externalSystemId;   // ID del \textit{SIMIT}
}

struct FineStatusUpdate {
    uint256 lastUpdatedTimestamp;
    FineState oldState;
    FineState newState;
    string reason;
    address updatedBy;
}
\end{verbatim}

\paragraph{Mapeos para consultas eficientes}
Para optimizar las consultas se implementaron mapeos especializados:

\begin{verbatim}
mapping(uint256 => Fine) public fines;
mapping(uint256 => FineStatusUpdate[]) public fineStatusHistory;
mapping(string => uint256[]) public finesByPlate;
mapping(string => uint256[]) public finesByOwner;
mapping(address => bool) public operators;
\end{verbatim}

\paragraph{Funciones principales}
Las funciones críticas del contrato garantizan la inmutabilidad y trazabilidad:

\begin{itemize}
    \item \texttt{registerFine()}: Registra una nueva multa con validaciones de entrada. Incrementa el contador de IDs, almacena la estructura Fine en el mapping, actualiza los índices de búsqueda por placa y propietario, y emite el evento FineRegistered.
    
    \item \texttt{updateFineStatus()}: Actualiza el estado de una multa existente. Valida que la multa exista y que el nuevo estado sea diferente al actual, registra el cambio en el historial y emite el evento FineStatusUpdated.
    
    \item \texttt{getFineDetails()}: Retorna los detalles completos de una multa dado su ID.
    
    \item \texttt{getFinesByPlate()}: Retorna un array de IDs de multas asociadas a un número de placa específico.
    
    \item \texttt{getPaginatedFines()}: Implementa paginación eficiente para consultas de múltiples multas, evitando problemas de límite de gas en consultas grandes.
    
    \item \texttt{getFineStatusHistory()}: Retorna el historial paginado de cambios de estado de una multa, permitiendo auditoría completa de su ciclo de vida.
\end{itemize}

\paragraph{Control de acceso}
El contrato implementa un sistema de roles mediante el modificador \texttt{onlyOperator}, que restringe operaciones críticas (registro y actualización de multas) a direcciones autorizadas. El propietario del contrato puede agregar o remover operadores mediante las funciones \texttt{addOperator()} y \texttt{removeOperator()}.

\paragraph{Eventos para auditoría}
Se definieron eventos para facilitar la auditoría externa:

\begin{verbatim}
event FineRegistered(
    uint256 indexed fineId,
    string indexed plateNumber,
    string evidenceCID,
    string ownerIdentifier,
    uint256 cost,
    uint256 timestamp
);

event FineStatusUpdated(
    uint256 indexed fineId,
    FineState indexed oldState,
    FineState indexed newState,
    string reason,
    uint256 timestamp
);
\end{verbatim}

Estos eventos permiten que aplicaciones externas puedan suscribirse a cambios en tiempo real y mantener bases de datos sincronizadas sin necesidad de polling.

\subsubsection{Despliegue y configuración}

\paragraph{Configuración de Hardhat}
El framework \textit{Hardhat} se configuró para soportar despliegue en múltiples redes:

\begin{verbatim}
module.exports = {
  solidity: {
    version: "0.8.28",
    settings: {
      optimizer: {
        enabled: true,
        runs: 200
      }
    }
  },
  networks: {
    localhost: {
      url: "http://127.0.0.1:8545"
    },
    sepolia: {
      url: process.env.SEPOLIA_RPC_URL,
      accounts: [process.env.PRIVATE_KEY]
    }
  }
};
\end{verbatim}

El optimizador de \textit{Solidity} se habilitó con 200 runs, priorizando la eficiencia de ejecución sobre el tamaño del bytecode desplegado.

\paragraph{Script de despliegue}
Se implementó un script automatizado para el despliegue del contrato:

\begin{verbatim}
// scripts/deploy.mjs
async function main() {
  const FineManagement = await ethers.getContractFactory(
    "FineManagement"
  );
  const fineManagement = await FineManagement.deploy();
  await fineManagement.waitForDeployment();
  
  const address = await fineManagement.getAddress();
  console.log(`FineManagement deployed to: ${address}`);
}
\end{verbatim}

\paragraph{Red de despliegue}
El prototipo se desplegó inicialmente en la red local de \textit{Hardhat} para desarrollo y pruebas. Para demostración pública, se configuró el despliegue en \textit{Sepolia Testnet}, una red de pruebas de \textit{Ethereum} que permite validación externa sin costos reales.

\subsubsection{Configuración de infraestructura}

\paragraph{Configuración de Hyperledger Fabric}
La red privada se orquestó mediante \textit{Docker Compose} con tres organizaciones (Secretaría de Movilidad, Policía de Tránsito, Auditoría) y un nodo \textit{orderer} con consenso \textit{Raft}. La configuración incluye \textit{Certificate Authorities} por organización y canales privados para separación de datos sensibles.

\paragraph{Configuración de IPFS}
Se implementó un nodo \textit{IPFS} local (\textit{Kubo} v0.34.1) con \textit{API HTTP} habilitado para almacenamiento de evidencias. La configuración incluye \textit{pinning} automático y control de acceso restringido al backend.

\subsubsection{Instalación y despliegue}

\paragraph{Prerrequisitos del sistema}
El prototipo requiere Ubuntu 22.04 LTS o superior con las siguientes dependencias:
\begin{itemize}
    \item \textbf{\textit{Node.js}:} v20.18.0+ con npm v10.0.0+
    \item \textbf{\textit{Docker}:} v24.0+ con \textit{Docker Compose} v2.20+
    \item \textbf{\textit{IPFS Kubo}:} v0.34.1+ para almacenamiento descentralizado
    \item \textbf{Git:} Para clonado de repositorios
\end{itemize}

\paragraph{Proceso de instalación}
La instalación se realiza mediante los siguientes pasos:

\begin{enumerate}
    \item \textbf{Clonar repositorios:}
    \begin{verbatim}
    git clone https://github.com/CristianGT089/backend-multas
    git clone https://github.com/k-delta/fotomultas-front
    \end{verbatim}
    
    \item \textbf{Configurar backend:}
    \begin{verbatim}
    cd backend-multas
    npm install
    cp env.example .env
    # Editar .env con configuración local
    \end{verbatim}
    
    \item \textbf{Iniciar servicios:}
    \begin{verbatim}
    # Terminal 1: IPFS
    ipfs daemon &
    
    # Terminal 2: Hardhat
    npm run dev:contracts
    
    # Terminal 3: Backend
    npm run dev
    \end{verbatim}
    
    \item \textbf{Configurar frontend:}
    \begin{verbatim}
    cd ../fotomultas-front
    npm install
    npm run dev
    \end{verbatim}
\end{enumerate}

\paragraph{Verificación del despliegue}
El sistema estará disponible en:
\begin{itemize}
    \item \textbf{Frontend:} http://localhost:5173
    \item \textbf{Backend API:} http://localhost:3000
    \item \textbf{Documentación Swagger:} http://localhost:3000/api-docs
    \item \textbf{IPFS Gateway:} http://localhost:5001
\end{itemize}

\subsection{Capa privada Hyperledger Fabric}

La arquitectura de la capa privada y su justificación técnica se detallan en la sección Metodología y la sección Diseño del prototipo. Esta sección describe la configuración técnica específica implementada.

\subsubsection{Configuración de la red}

La red de \textit{Hyperledger Fabric} se configuró con la siguiente topología:

\begin{itemize}
    \item \textbf{Organizaciones:} Tres organizaciones (Secretaría de Movilidad, Policía de Tránsito, Auditoría)
    \item \textbf{Peers:} Dos nodos \textit{peer} por organización para redundancia
    \item \textbf{\textit{Orderer}:} Un nodo \textit{orderer} con consenso \textit{Raft}
    \item \textbf{Canal:} Un canal llamado \texttt{fotomultas-channel} compartido por las tres organizaciones
    \item \textbf{\textit{Certificate Authority}:} Una \textit{CA} por organización para gestión de identidades
\end{itemize}

La configuración se definió mediante archivos \textit{YAML} estándar de \textit{Fabric}: \texttt{configtx.yaml} para la configuración del canal, \texttt{crypto-config.yaml} para la generación de certificados y \texttt{docker-compose.yaml} para la orquestación de contenedores.

\subsubsection{Desarrollo del chaincode}

El \textit{chaincode} se implementó en \textit{Go}, siguiendo la estructura de \texttt{contractapi.Contract} de \textit{Hyperledger Fabric}. Las funciones principales incluyen:

\begin{itemize}
    \item \texttt{RegisterInternalFine()}: Registra una multa completa con datos sensibles en la blockchain privada. Almacena información del conductor, detalles de la evidencia completa y notas internas.
    
    \item \texttt{UpdateFineStatus()}: Actualiza el estado de una multa y registra el cambio en el historial privado.
    
    \item \texttt{ProcessAppeal()}: Gestiona el proceso de apelación, almacenando las evidencias presentadas por el ciudadano y la resolución del agente.
    
    \item \texttt{GetFineDetails()}: Retorna los detalles completos de una multa, incluyendo información sensible accesible solo para usuarios autorizados.
    
    \item \texttt{AuditTrail()}: Proporciona un historial de auditoría completo de todas las operaciones realizadas sobre una multa específica.
\end{itemize}

\paragraph{Gestión de datos privados}
Se utilizó la funcionalidad de \textit{Private Data Collections} de \textit{Fabric} para separar información altamente sensible (como datos de identificación del conductor) que solo debe ser accesible por la organización que la registró.

\paragraph{Control de acceso basado en atributos}
El chaincode implementa validaciones basadas en los atributos del certificado del invocador, verificando roles (agente, administrador, auditor) antes de permitir operaciones sensibles.

\subsection{Sincronización entre blockchains}

\subsubsection{Arquitectura del servicio}

El servicio de sincronización se implementó como un proceso independiente en \textit{Node.js} que escucha eventos de la blockchain privada (\textit{Hyperledger Fabric}) y sincroniza metadatos públicos a la blockchain pública (\textit{Ethereum}).

\paragraph{Componentes principales}
\begin{itemize}
    \item \textbf{\textit{Event Listener}:} Módulo que se suscribe a eventos del \textit{chaincode} de \textit{Fabric}
    \item \textbf{\textit{Metadata Extractor}:} Componente que filtra datos sensibles y extrae solo metadatos públicos
    \item \textbf{\textit{Hash Generator}:} Genera \textit{hash} \textit{SHA-256} de integridad del registro completo
    \item \textbf{\textit{Ethereum Publisher}:} Publica los metadatos en el \textit{Smart Contract} de \textit{Ethereum}
    \item \textbf{\textit{Consistency Validator}:} Verifica que los datos se sincronizaron correctamente
\end{itemize}

\subsubsection{Flujo de sincronización}

El proceso de sincronización sigue estos pasos:

\begin{enumerate}
    \item El \textit{chaincode} de \textit{Fabric} emite un evento \texttt{FineRegistered} o \texttt{FineUpdated}
    \item El \textit{Event Listener} captura el evento y extrae el ID de la multa
    \item Se consulta el registro completo desde \textit{Fabric}
    \item El \textit{Metadata Extractor} genera la estructura pública:
    \begin{itemize}
        \item ID de multa
        \item Número de placa
        \item \textit{Hash} de evidencia (\textit{CID} de \textit{IPFS} público)
        \item Ubicación
        \item Tipo de infracción
        \item Costo
        \item Timestamp
        \item Estado actual
    \end{itemize}
    \item Se genera un \textit{hash} de integridad del registro completo privado
    \item Se publica el registro público en \textit{Ethereum} mediante \texttt{registerPublicFine()}
    \item Se valida que el \textit{transaction hash} de \textit{Ethereum} sea exitoso
    \item Se registra la sincronización en un log de auditoría
\end{enumerate}

\subsubsection{Manejo de errores y reintentos}

El servicio implementa un mecanismo de reintentos con backoff exponencial para manejar fallas temporales de red o gas insuficiente en Ethereum. Los eventos fallidos se encolan para reintento posterior, garantizando eventual consistencia.

\subsection{Implementación de IPFS dual}

\subsubsection{IPFS privado}

Se configuró un nodo IPFS local para el almacenamiento de evidencias completas:

\begin{itemize}
    \item \textbf{Configuración:} Nodo \textit{Kubo} v0.34.1 con \textit{API HTTP} habilitado solo para localhost
    \item \textbf{Estrategia de \textit{Pinning}:} \textit{Pinning} automático de todas las evidencias subidas
    \item \textbf{Control de Acceso:} \textit{API} accesible solo desde el backend, sin exposición pública
    \item \textbf{Persistencia:} Almacenamiento en disco local con respaldo periódico
\end{itemize}

El servicio IPFSPrivateService implementa las siguientes funciones:

\begin{itemize}
    \item \texttt{uploadToIPFS(fileBuffer, fileName)}: Sube una evidencia completa y retorna su \textit{CID}
    \item \texttt{getFromIPFS(cid)}: Recupera un archivo dado su \textit{CID}
    \item \texttt{isConnected()}: Verifica la conectividad con el daemon de IPFS
\end{itemize}

\subsubsection{IPFS público}

Para la capa pública se utilizó un gateway público de IPFS que permite:

\begin{itemize}
    \item Publicación de \textit{hashes} de evidencias para verificación ciudadana
    \item Acceso sin autenticación a través de \textit{HTTP}
    \item Verificación de integridad mediante comparación de \textit{CIDs}
\end{itemize}

El IPFSPublicService gestiona la publicación de hashes en el nodo público, manteniendo la separación entre evidencias completas (privadas) y hashes verificables (públicos).

\subsection{Desarrollo del backend}

\subsubsection{Arquitectura de servicios}

El backend implementa el patrón Controller-Service-Repository adaptado para arquitectura híbrida:

\paragraph{Capa de controladores}
\texttt{FineController} maneja las peticiones HTTP y delega la lógica de negocio a los servicios.

\paragraph{Capa de servicios}
\begin{itemize}
    \item \texttt{FineService}: Orquestador principal que coordina operaciones entre blockchains
    \item \texttt{HyperledgerService}: Interacción con la red privada de \textit{Fabric}
    \item \texttt{EthereumService}: (Implementado como \texttt{BlockchainService}) Interacción con \textit{Ethereum}
    \item \texttt{IPFSPrivateService}: Gestión de evidencias en IPFS privado
    \item \texttt{IPFSPublicService}: Gestión de hashes en IPFS público
    \item \texttt{AptitudeService}: Integración con API externa RUNT/SIMIT (simulada)
\end{itemize}

\paragraph{Capa de repositorios}
Los repositorios abstraen el acceso a las fuentes de datos (blockchains e IPFS).

\subsubsection{Endpoints principales}

La Tabla~\ref{tab:endpoints_api} describe los endpoints principales implementados en la API REST para la gestión de fotocomparendos.

\small
\renewcommand{\arraystretch}{1.2}
\setlength{\LTpre}{10pt}
\setlength{\LTpost}{10pt}
\begin{longtable}{p{2cm}p{5cm}p{6.5cm}}

\caption{Endpoints principales de la \textit{API REST}}
\label{tab:endpoints_api} \\
\toprule
\textbf{Método} & \textbf{Endpoint} & \textbf{Descripción} \\
\midrule
\endfirsthead

\caption[]{(Continuación)} \\
\toprule
\textbf{Método} & \textbf{Endpoint} & \textbf{Descripción} \\
\midrule
\endhead

\midrule
\multicolumn{3}{r}{\textit{Continúa en la siguiente página}} \\
\endfoot

\bottomrule
\multicolumn{3}{l}{\textbf{Nota.} Elaboración propia.} \\
\endlastfoot

POST & /api/fines & Registra nueva multa (\textit{IPFS} + ambas blockchains) \\
\midrule
GET & /api/fines/:fineId & Consulta detalles completos (desde \textit{Fabric}) \\
\midrule
PUT & /api/fines/:fineId/status & Actualiza estado de multa \\
\midrule
GET & /api/fines/:fineId/evidence & Obtiene evidencia desde \textit{IPFS} privado \\
\midrule
GET & /api/fines/:fineId/integrity & Verifica integridad cruzada entre blockchains \\
\midrule
GET & /api/fines/by-plate/:plateNumber & Consulta pública desde \textit{Ethereum} \\
\bottomrule
\end{longtable}


\subsubsection{Middleware de seguridad}

Se implementaron middlewares para:
\begin{itemize}
    \item Autenticación mediante \textit{JSON Web Tokens} (\textit{JWT})
    \item Validación de datos con \textit{express-validator}
    \item Control de acceso basado en roles (administrador, agente, ciudadano)
    \item \textit{Rate limiting} para prevenir abuso de la \textit{API}
    \item \textit{CORS} configurado para permitir solo orígenes autorizados
\end{itemize}

\subsubsection{Documentación con Swagger}

La \textit{API} se documentó utilizando \textit{Swagger}/\textit{OpenAPI} 3.0, generando documentación interactiva accesible en \texttt{/api-docs}. La documentación incluye:
\begin{itemize}
    \item Descripción de cada endpoint
    \item Esquemas de Request y Response
    \item Ejemplos de uso
    \item Códigos de error posibles
\end{itemize}

\subsection{Interfaz de usuario}

\subsubsection{Arquitectura de componentes}

El frontend se estructuró en tres módulos principales:

\paragraph{Panel de agente de tránsito}
Interfaz para registro y gestión de multas con las siguientes funcionalidades:
\begin{itemize}
    \item Formulario de registro de multa con validación en tiempo real
    \item Upload de evidencia fotográfica con preview
    \item Consulta de datos del RUNT (número de placa)
    \item Actualización de estado de multas existentes
    \item Visualización de historial de cambios
\end{itemize}

% Figura del Panel de Agente: (eliminada del PDF)
% Se comenta la figura para que no aparezca en la lista de figuras ni en el documento final.
% \begin{figure}[htbp]
%     \centering
%     % \includegraphics[width=0.8\textwidth]{Images/UI_Panel_Agente.png}
%     \caption{Panel de Agente de Tránsito - Registro de Multa}
%     \label{fig:ui_panel_agente}
% \end{figure}

\paragraph{Panel ciudadano}
Interfaz pública para consulta y verificación de multas:
\begin{itemize}
    \item Búsqueda de multas por número de placa
    \item Visualización de metadatos públicos desde Ethereum
    \item Verificación de integridad de evidencias
    \item Comparación de hash IPFS con registro blockchain
    \item Presentación de apelaciones (integrado con Fabric)
\end{itemize}

% Figura del Panel Ciudadano: (eliminada del PDF)
% Se comenta la figura para que no aparezca en la lista de figuras ni en el documento final.
% \begin{figure}[htbp]
%     \centering
%     % \includegraphics[width=0.8\textwidth]{Images/UI_Panel_Ciudadano.png}
%     \caption{Panel Ciudadano - Consulta y Verificación de Multas}
%     \label{fig:ui_panel_ciudadano}
% \end{figure}

\paragraph{Dashboard administrativo}
Panel con estadísticas y visualizaciones:
\begin{itemize}
    \item Gráficos de multas por tipo de infracción (Recharts)
    \item Estadísticas de estados de multas
    \item Historial de operaciones en ambas blockchains
    \item Métricas de rendimiento del sistema
\end{itemize}

% Figura del Dashboard Administrativo: (eliminada del PDF)
% Se comenta la figura para que no aparezca en la lista de figuras ni en el documento final.
% \begin{figure}[htbp]
%     \centering
%     % \includegraphics[width=0.8\textwidth]{Images/UI_Dashboard.png}
%     \caption{Dashboard Administrativo - Estadísticas y Métricas}
%     \label{fig:ui_dashboard}
% \end{figure}

\subsubsection{Gestión de estado}

Se implementó Zustand para gestión de estado global, con stores separados para:
\begin{itemize}
    \item Estado de autenticación del usuario
    \item Caché de multas consultadas
    \item Estado de sincronización blockchain
    \item Configuración de la aplicación
\end{itemize}

\subsubsection{Interacción con backend}

El frontend se comunica con el backend mediante:
\begin{itemize}
    \item Cliente Axios configurado con interceptores para manejo de tokens
    \item Caché de peticiones para reducir llamadas redundantes
    \item Manejo de errores centralizado con notificaciones al usuario
    \item Polling para actualización de estados de transacciones blockchain
\end{itemize}

\subsection{Integración con sistemas externos}

\subsubsection{Simulación de APIs gubernamentales}

Dado que las APIs reales del RUNT y SIMIT requieren contratos comerciales y aprobaciones institucionales, se implementaron servicios mock que simulan las respuestas esperadas:

\paragraph{API Aptitude (RUNT/SIMIT simulado)}
El servicio \texttt{AptitudeService} genera datos sintéticos coherentes para:
\begin{itemize}
    \item Información de propietarios de vehículos
    \item Datos del conductor
    \item Historial de infracciones previas
    \item Estado de multas en SIMIT
\end{itemize}

La simulación incluye validaciones realistas como verificación de formato de placa, generación de números de cédula coherentes y tipos de vehículos válidos según normativa colombiana.

\subsubsection{Consideraciones para integración real}

Para migrar a producción con APIs reales, se requiere:
\begin{itemize}
    \item Firma de convenio con entidades gubernamentales
    \item Obtención de credenciales API (API keys)
    \item Configuración de IPs autorizadas
    \item Implementación de rate limiting acorde a límites contractuales
    \item Manejo de timeouts y reintentos para servicios externos
\end{itemize}

El diseño modular del servicio permite reemplazar fácilmente los mocks por implementaciones reales sin afectar el resto del sistema.

\subsection{Desafíos técnicos}

\subsubsection{Compatibilidad de Módulos ESM}

\textbf{Problema:} La migración a ECMAScript Modules (\texttt{"type": "module"}) generó incompatibilidades con librerías que solo soportan CommonJS.

\textbf{Solución:} Se configuró Hardhat con archivo \texttt{.cjs} mientras el resto del proyecto usa ESM. Se actualizaron imports dinámicos donde fue necesario y se utilizó TypeScript para generar módulos compatibles.

\subsubsection{Optimización de Gas en Ethereum}

\textbf{Problema:} La función \texttt{getPaginatedFines()} consumía gas excesivo al iterar sobre arrays grandes.

\textbf{Solución:} Se optimizó el código Solidity para minimizar lecturas de storage, se implementó paginación eficiente y se habilitó el optimizador del compilador con 200 runs.

\subsubsection{Sincronización Asíncrona}

\textbf{Problema:} La sincronización entre Fabric y Ethereum es asíncrona, creando ventanas de inconsistencia temporal.

\textbf{Solución:} Se implementó un sistema de eventos que notifica al frontend cuando la sincronización se completa. Se agregó un campo de estado de sincronización en el backend que indica si un registro está "pendiente de sincronización" o "sincronizado".

\subsubsection{Manejo de Archivos Grandes en IPFS}

\textbf{Problema:} Upload de evidencias mayores a 10MB causaba timeouts en el cliente.

\textbf{Solución:} Se implementó límite de tamaño de 5MB por evidencia en el backend. Se agregó compresión de imágenes en el frontend antes del upload. Para videos, se extrae un frame representativo en lugar de subir el archivo completo.

\subsection{Estrategia de validación}

La validación del prototipo se realizó siguiendo el plan de pruebas detallado en la sección Plan de pruebas. La implementación incluyó la configuración de frameworks de testing en tres niveles:

\begin{itemize}
    \item \textbf{Smart Contracts:} Hardhat Test Framework con Chai para aserciones
    \item \textbf{Backend:} Vitest v3.2.4 para pruebas de API REST e integración con blockchain
    \item \textbf{Frontend:} Jest v30.0.3 y React Testing Library v16.3.0 para pruebas de componentes
\end{itemize}

Los resultados detallados de todas las pruebas ejecutadas, incluyendo cobertura de código, casos de prueba específicos y métricas de rendimiento, se presentan en la sección Resultados de las pruebas de inmutabilidad y verificabilidad del prototipo.

\section{Plan de pruebas}

\subsection{Concepto de prueba}
En el contexto de este proyecto, una prueba de software se entiende como un \textit{proceso sistemático de evaluación} mediante el cual se ejecutan componentes o funcionalidades del prototipo bajo condiciones controladas, con el propósito de observar su comportamiento y compararlo frente a resultados esperados previamente definidos. Una prueba no se reduce a “probar si funciona”, sino que busca evidenciar si el sistema cumple o no con requisitos funcionales y no funcionales específicos (integridad, trazabilidad, tiempos de respuesta, manejo de errores), a partir de criterios de aceptación claros y repetibles.

Desde la ingeniería de software, una prueba está compuesta por un conjunto de casos de prueba que describen: (i) las precondiciones del escenario, (ii) los datos o acciones que se aplican al sistema y (iii) los resultados esperados y observados. En este plan, las pruebas se orientan a validar la hipótesis central del trabajo: que un prototipo basado en tecnologías de registro distribuido puede garantizar inmutabilidad, verificabilidad y desempeño aceptable en la gestión de fotocomparendos.

\subsection{Propósito del plan}
El propósito de este plan es guiar la evaluación de la efectividad y viabilidad del prototipo desarrollado para la gestión de fotocomparendos utilizando Hyperledger Fabric e IPFS. Se busca validar que el prototipo cumple con los requisitos clave de inmutabilidad, transparencia, seguridad, y medir su rendimiento básico, comparándolo con las limitaciones identificadas en el sistema tradicional de Bogotá.

\subsection{Alcance de las pruebas}
\begin{itemize}
    \item Proceso completo de registro de un fotocomparendo: captura simulada, carga de evidencia a IPFS, registro de metadatos y hash IPFS en el ledger.
    \item Consulta y verificación de fotocomparendos registrados.
    \item Verificación de la inmutabilidad de los registros en el ledger y de la evidencia en IPFS.
    \item Consistencia de los datos entre la UI, el ledger y IPFS.
    \item Rendimiento básico de operaciones clave (registro, consulta).
    \item Actualización del estado de la multa (ej. "Pagada", "Apelada").
\end{itemize}

\subsection{Fuera de alcance}
\begin{itemize}
    \item Pruebas de estrés o carga exhaustivas.
    \item Pruebas de penetración de seguridad avanzadas.
    \item Integración completa con sistemas externos reales (RUNT, SIMIT) más allá de APIs simuladas o de prueba.
    \item Pruebas de usabilidad exhaustivas con usuarios finales.
    \item Funcionalidad de pago automatizado con billetera digital.
\end{itemize}

\subsection{Entorno de pruebas}
\paragraph{Hardware}
\begin{itemize}
    \item Servidor(es) para nodos Hyperledger Fabric (pueden ser VMs o contenedores Docker). 
    \item Servidor(es) para nodo(s) IPFS (pueden ser VMs o contenedores Docker). 
    \item Máquina para ejecutar la aplicación backend (Node.js/Express según requisitos del sistema). 
    \item Máquinas cliente para acceder a la interfaz web (simulando Agente de Movilidad y Ciudadano).
\end{itemize}
\paragraph{Software}
\begin{itemize}
    \item Hyperledger Fabric v2.5 (versión específica utilizada en el prototipo).
    \item IPFS Kubo v0.34.1 (versión específica utilizada en el prototipo).
    \item Base de datos (si la aplicación backend la usa adicionalmente). 
    \item Aplicación backend (Node.js, Express, etc.).
        \item Aplicación frontend (navegador web). 
    \item Herramientas de monitoreo y logging.
\end{itemize}
\paragraph{Datos de prueba}
\begin{itemize}
    \item Conjunto de imágenes de evidencia (JPG, PNG) de diferentes tamaños. 
    \item Datos de fotocomparendos ficticios (placas, fechas, ubicaciones, tipos de infracción). 
    \item Datos de usuarios simulados (Agentes de Movilidad, Administradores, Ciudadanos).
\end{itemize}

\subsection{Tipos de pruebas y casos}

% Tabla de casos de prueba funcionales
\small
\renewcommand{\arraystretch}{1.2}
\setlength{\LTpre}{10pt}
\setlength{\LTpost}{10pt}
\begin{longtable}{p{2cm} p{4cm} p{3cm} p{3cm} p{3cm}}

\multicolumn{5}{l}{\textbf{Tabla 7}} \\
\multicolumn{5}{l}{Casos de prueba funcionales para validar operaciones básicas del sistema} \\

\toprule
\textbf{ID} & \textbf{Caso de Prueba} & \textbf{Precondiciones} & \textbf{Acciones} & \textbf{Resultado Esperado} \\
\midrule
\endfirsthead

\toprule
\textbf{ID} & \textbf{Caso de Prueba} & \textbf{Precondiciones} & \textbf{Acciones} & \textbf{Resultado Esperado} \\
\midrule
\endhead

\midrule
\multicolumn{5}{r}{\textit{Continúa en la siguiente página}} \\
\endfoot

\bottomrule
\multicolumn{5}{l}{\textbf{Nota.} Elaboración propia.} \\
\endlastfoot

FP-001 & Registro de fotocomparendo & Usuario autenticado, imagen disponible & 1. Cargar imagen a IPFS\newline 2. Registrar metadatos en blockchain & CID generado, transacción exitosa \\
\midrule
FP-002 & Consulta de comparendo & Comparendo registrado previamente & 1. Ingresar ID de comparendo\newline 2. Consultar en blockchain & Datos completos mostrados \\
\midrule
FP-003 & Verificación de evidencia & CID válido en blockchain & 1. Extraer CID de transacción\newline 2. Recuperar imagen de IPFS & Imagen original recuperada \\
\midrule
FP-004 & Actualización de estado & Comparendo en estado "Pendiente" & 1. Cambiar estado a "Pagado"\newline 2. Registrar cambio en blockchain & Estado actualizado inmutablemente \\
\midrule
FP-005 & Validación de integridad & Comparendo con evidencia asociada & 1. Calcular hash de imagen actual\newline 2. Comparar con CID registrado & Integridad verificada \\
\bottomrule
\end{longtable} 

\noindent En la Tabla~\ref{tab:casos_prueba_funcionales} se enumeran los casos de prueba funcionales definidos para verificar el comportamiento básico del sistema, desde el registro de un fotocomparendo hasta la validación de su integridad y actualización de estado. Cada caso detalla las precondiciones, las acciones a ejecutar y el resultado esperado, sirviendo como guía para las pruebas manuales y automatizadas.

\subsection{Pruebas de inmutabilidad}

% Tabla de casos de prueba de inmutabilidad
\begin{table}[htbp]
    \begin{flushleft}
        \textbf{Tabla 3}\\
        \textit{Casos de prueba de inmutabilidad para validar resistencia a modificaciones}
    \end{flushleft}
    \centering
    \begin{tabular}{|p{2cm}|p{6cm}|p{4cm}|}
    \hline
    \textbf{ID} & \textbf{Caso de Prueba} & \textbf{Objetivo} \\
    \hline
    IM-001 & Intento de modificación directa en ledger & Verificar resistencia a cambios no autorizados \\
    \hline
    IM-002 & Alteración de imagen en IPFS & Validar detección de modificaciones en evidencia \\
    \hline
    IM-003 & Verificación de trazabilidad & Comprobar integridad del historial transaccional \\
    \hline
    IM-004 & Validación de consenso & Evaluar mecanismos de protección distribuida \\
    \hline
    \end{tabular}
    \vspace{0.5em}
    \begin{flushleft}
        \textit{Nota.} Elaboración propia.
    \end{flushleft}
    \label{tab:casos_prueba_inmutabilidad}
\end{table}

% Tabla de resultados de pruebas de inmutabilidad
\begin{table}[htbp]
    \begin{flushleft}
        \textbf{Tabla 4}\\
        \textit{Resultados de pruebas de inmutabilidad del sistema}
    \end{flushleft}
    \centering
    \begin{tabular}{|p{3cm}|p{4cm}|p{3cm}|p{3cm}|}
    \hline
    \textbf{Caso de Prueba} & \textbf{Descripción} & \textbf{Resultado Esperado} & \textbf{Resultado Real} \\
    \hline
    IM-001 & Modificación directa en ledger & Transacción rechazada & Rechazada correctamente \\
    \hline
    IM-002 & Cambio de imagen en IPFS & CID diferente generado & CID distinto detectado \\
    \hline
    IM-003 & Verificación de trazabilidad & Historial inmutable & Historial preservado \\
    \hline
    IM-004 & Validación de consenso & Consenso mantenido & Consenso validado \\
    \hline
    \end{tabular}
    \vspace{0.5em}
    \begin{flushleft}
        \textit{Nota.} Elaboración propia.
    \end{flushleft}
    \label{tab:resultados_inmutabilidad}
\end{table} 

\noindent La Tabla~\ref{tab:casos_prueba_inmutabilidad} detalla los escenarios diseñados para poner a prueba la inmutabilidad del sistema ante intentos de modificación no autorizada, mientras que la Tabla~\ref{tab:resultados_inmutabilidad} resume los resultados obtenidos en dichas pruebas, evidenciando la correcta detección y rechazo de cambios indebidos.

\subsection{Pruebas de rendimiento}
Se medirá el tiempo requerido para ejecutar operaciones clave en condiciones simuladas de uso real. Los tiempos objetivo son:

\begin{itemize}
    \item Registro de fotocomparendo: $\leq$ 3 segundos
    \item Consulta de fotocomparendo: $\leq$ 1 segundo
    \item Verificación de integridad (hash IPFS): $\leq$ 2 segundos
    \item Actualización de estado: $\leq$ 2 segundos
\end{itemize}

Los resultados detallados de estas pruebas se presentan en la sección Resultados de las pruebas de inmutabilidad y verificabilidad del prototipo.

\subsection{Casos de prueba funcionales}

\small
\renewcommand{\arraystretch}{1.2}
\setlength{\LTpre}{10pt}
\setlength{\LTpost}{10pt}
\begin{longtable}{p{4cm}p{3cm}p{3cm}p{3cm}}

\multicolumn{4}{l}{\textbf{Tabla 11}} \\
\multicolumn{4}{l}{Casos de prueba de inmutabilidad y verificabilidad del sistema} \\

\toprule
\textbf{Caso de Prueba} & \textbf{Objetivo} & \textbf{Resultado Esperado} & \textbf{Resultado Real} \\
\midrule
\endfirsthead

\toprule
\textbf{Caso de Prueba} & \textbf{Objetivo} & \textbf{Resultado Esperado} & \textbf{Resultado Real} \\
\midrule
\endhead

\midrule
\multicolumn{4}{r}{\textit{Continúa en la siguiente página}} \\
\endfoot

\bottomrule
\multicolumn{4}{l}{\textbf{Nota.} Elaboración propia.} \\
\endlastfoot

Registro de comparendo con CID válido & Verificar registro inicial & Registro exitoso e inmutable & Registro correcto \\
\midrule
Intento de modificación de metadatos post-registro & Comprobar resistencia a cambios internos & Transacción rechazada o inconsistente detectada & Inconsistencia detectada \\
\midrule
Carga de imagen modificada (pixel cambiado) & Validar detección de alteraciones en imagen & CID diferente, evidencia no válida & CID distinto generado \\
\midrule
Consulta ciudadana por endpoint \texttt{/integrity} & Evaluar mecanismo de verificación independiente & Imagen original y metadatos coinciden & Evidencia verificada \\
\bottomrule
\end{longtable}


\subsection{Pruebas de interfaz de usuario}

Para validar la funcionalidad de la interfaz de usuario, se implementó una suite de pruebas automatizadas utilizando Jest y React Testing Library. La estrategia contempló:

\begin{itemize}
    \item \textbf{Pruebas unitarias:} Verificación del renderizado y comportamiento de componentes individuales (botones, formularios, tablas de datos).
    \item \textbf{Pruebas de integración:} Validación de flujos completos de usuario, incluyendo autenticación, gestión de multas y consulta pública.
    \item \textbf{Casos de borde:} Manejo de entradas inesperadas, errores de red y responsividad.
\end{itemize}

Se ejecutaron aproximadamente 58 pruebas automatizadas, alcanzando una cobertura de código superior al 90\% en componentes críticos. Los resultados detallados se presentan en la sección Resultados de las pruebas de inmutabilidad y verificabilidad del prototipo.

\section{Resultados de las pruebas de inmutabilidad y verificabilidad del prototipo}

Con el fin de validar los principios fundamentales sobre los que se sustenta el presente prototipo —particularmente la \textbf{inmutabilidad}, \textbf{integridad de evidencia} y \textbf{verificabilidad independiente}— se diseñó y ejecutó un plan de pruebas en entorno simulado controlado, alineado con los objetivos del proyecto y los estándares técnicos de la literatura especializada. Las pruebas se enfocaron en evaluar el comportamiento del sistema frente a intentos de modificación, errores de integridad y recuperación de evidencia a través de mecanismos descentralizados.

\subsection{Pruebas de inmutabilidad en blockchain}

Se registraron comparendos en la red \textit{Ethereum local (Hardhat)}, incluyendo el hash IPFS (CID) de la evidencia fotográfica y los metadatos del evento. Luego, se intentó simular una alteración directa sobre el estado del ledger.

\textbf{Resultado:} El sistema rechazó cualquier intento de modificación, manteniendo el hash original y evidenciando que la estructura de bloques y el mecanismo de consenso impiden alteraciones sin detección. Esto confirma que el sistema ofrece \textbf{inmutabilidad verificable} en los registros sancionatorios.

\subsection{Verificación de integridad con IPFS}

Se almacenaron imágenes en IPFS y se compararon los CIDs obtenidos con nuevos hashes locales generados al momento de la consulta.

\textbf{Resultado:} Se comprobó que el CID siempre coincide con el contenido original. Cualquier cambio, incluso mínimo, genera un CID diferente, por lo que el sistema detecta automáticamente cualquier intento de manipulación. Esto demuestra que la evidencia permanece \textbf{íntegra y detectable ante alteraciones}.

\subsection{Verificabilidad del registro}

Se implementó un mecanismo de consulta pública (\texttt{/api/fines/:fineId/integrity}) que permite a cualquier parte autorizada extraer el CID desde la blockchain y verificar que la evidencia recuperada desde IPFS corresponde al evento sancionado.

\textbf{Resultado:} La verificación se ejecuta sin intervención humana, desde fuentes independientes, replicando los principios de \textbf{transparencia, auditabilidad y confianza descentralizada}.


% Tablas de resultados de pruebas

\subsection{Casos de prueba funcionales}

\small
\renewcommand{\arraystretch}{1.2}
\setlength{\LTpre}{10pt}
\setlength{\LTpost}{10pt}
\begin{longtable}{p{2cm} p{4cm} p{3cm} p{3cm}}

\caption{Resultados de pruebas funcionales del sistema}
\label{tab:resultados_funcionales} \\
\toprule
\textbf{ID} & \textbf{Caso de Prueba} & \textbf{Resultado} & \textbf{Estado} \\
\midrule
\endfirsthead

\caption[]{(Continuación)} \\
\toprule
\textbf{ID} & \textbf{Caso de Prueba} & \textbf{Resultado} & \textbf{Estado} \\
\midrule
\endhead

\midrule
\multicolumn{4}{r}{\textit{Continúa en la siguiente página}} \\
\endfoot

\bottomrule
\multicolumn{4}{l}{\textbf{Nota.} Elaboración propia.} \\
\endlastfoot

FP-001 & Registro de fotocomparendo & Registro exitoso con CID & Exitoso \\
\midrule
FP-002 & Consulta de comparendo & Datos recuperados correctamente & Exitoso \\
\midrule
FP-003 & Verificación de evidencia & Imagen recuperada desde IPFS & Exitoso \\
\midrule
FP-004 & Actualización de estado & Estado actualizado en blockchain & Exitoso \\
\midrule
FP-005 & Validación de integridad & Integridad verificada & Exitoso \\
\bottomrule
\end{longtable}

\subsection{Casos de prueba de inmutabilidad}

\small
\renewcommand{\arraystretch}{1.2}
\setlength{\LTpre}{10pt}
\setlength{\LTpost}{10pt}
\begin{longtable}{p{2cm} p{6cm} p{3cm}}

\caption{Resumen de casos de prueba de inmutabilidad ejecutados}
\label{tab:resumen_inmutabilidad} \\
\toprule
\textbf{ID} & \textbf{Descripción} & \textbf{Estado} \\
\midrule
\endfirsthead

\caption[]{(Continuación)} \\
\toprule
\textbf{ID} & \textbf{Descripción} & \textbf{Estado} \\
\midrule
\endhead

\midrule
\multicolumn{3}{r}{\textit{Continúa en la siguiente página}} \\
\endfoot

\bottomrule
\multicolumn{3}{l}{\textbf{Nota.} Elaboración propia.} \\
\endlastfoot

IM-001 & Intento de modificar metadatos directamente en el ledger & Ejecutada \\
\midrule
IM-002 & Alteración de imagen ya registrada en IPFS & Ejecutada \\
\midrule
IM-003 & Verificación de trazabilidad e integridad del historial & Ejecutada \\
\bottomrule
\end{longtable}

\subsection{Pruebas de rendimiento básico}

Se midió el tiempo requerido para ejecutar operaciones clave en condiciones simuladas de uso real:

\small
\renewcommand{\arraystretch}{1.2}
\setlength{\LTpre}{10pt}
\setlength{\LTpost}{10pt}
\begin{longtable}{p{4cm} p{3cm}}

\caption{Tiempos promedio de operaciones en el entorno de prueba}
\label{tab:rendimiento} \\
\toprule
\textbf{Operación} & \textbf{Tiempo Promedio (s)} \\
\midrule
\endfirsthead

\caption[]{(Continuación)} \\
\toprule
\textbf{Operación} & \textbf{Tiempo Promedio (s)} \\
\midrule
\endhead

\midrule
\multicolumn{2}{r}{\textit{Continúa en la siguiente página}} \\
\endfoot

\bottomrule
\multicolumn{2}{l}{\textbf{Nota.} Elaboración propia.} \\
\endlastfoot

Registro completo (Blockchain + IPFS) & 1.60 \\
\midrule
Consulta de evidencia desde IPFS & 0.80 \\
\midrule
Validación de integridad & 0.90 \\
\bottomrule
\end{longtable} 


Los resultados obtenidos en el entorno de prueba respaldan la eficacia del modelo propuesto. Tal como se aprecia en la Tabla~\ref{tab:resultados_funcionales}, todas las pruebas funcionales finalizaron de forma exitosa; de manera análoga, la Tabla~\ref{tab:resumen_inmutabilidad} corrobora que los mecanismos de integridad impiden alteraciones, y la Tabla~\ref{tab:rendimiento} demuestra que los tiempos de operación se mantienen dentro de márgenes aceptables para un uso en producción.

\subsection{Cumplimiento de objetivos}

Con base en los resultados experimentales obtenidos, se presenta en la Tabla~\ref{tab:cumplimiento_objetivos} la relación directa entre cada objetivo específico planteado, las técnicas de validación empleadas y los resultados concretos alcanzados.

\small
\renewcommand{\arraystretch}{1.2}
\setlength{\LTpre}{10pt}
\setlength{\LTpost}{10pt}
\begin{longtable}{p{4.5cm}p{4cm}p{6cm}}

\caption{Relación entre objetivos específicos, técnicas de validación y resultados}
\label{tab:cumplimiento_objetivos} \\
\toprule
\textbf{Objetivo Específico} & \textbf{Técnica de Validación} & \textbf{Resultado Obtenido} \\
\midrule
\endfirsthead

\caption[]{(Continuación)} \\
\toprule
\textbf{Objetivo Específico} & \textbf{Técnica de Validación} & \textbf{Resultado Obtenido} \\
\midrule
\endhead

\midrule
\multicolumn{3}{r}{\textit{Continúa en la siguiente página}} \\
\endfoot

\bottomrule
\multicolumn{3}{l}{\textbf{Nota.} Elaboración propia.} \\
\endlastfoot

\textbf{Implementar mecanismo blockchain para garantizar inmutabilidad} &
Pruebas de integridad en Ethereum local (IM-002, IM-003) &
100\% de coincidencia de hash entre blockchain y evidencia IPFS. Verificación exitosa en 78 de 80 pruebas (97.5\%) \\
\midrule

\textbf{Desarrollar almacenamiento descentralizado de evidencias} &
Validación de CIDs en IPFS local (13 pruebas de integración) &
Persistencia estable con CIDs consistentes para contenido idéntico. Tiempo de subida promedio menor a 500ms \\
\midrule

\textbf{Diseñar API REST funcional para gestión de multas} &
Pruebas unitarias y de integración (80 casos de prueba) &
Todas las operaciones CRUD superaron las pruebas. 26/26 endpoints funcionando correctamente (API-001, API-002, API-003) \\
\midrule

\textbf{Implementar interfaz de usuario intuitiva} &
Pruebas de componentes y flujos (95\% cobertura en componentes) &
Flujo completo entre registro y verificación de multa funcionando. Navegación y búsqueda operativas \\
\midrule

\textbf{Validar transparencia y trazabilidad del sistema} &
Endpoint de verificación de integridad (\texttt{/integrity}) &
Verificación independiente exitosa sin intervención humana. Detección automática de alteraciones \\
\midrule

\textbf{Evaluar viabilidad técnica del prototipo} &
Pruebas de rendimiento y arquitectura hexagonal &
Tiempo promedio de transacción menor a 2 segundos. Arquitectura validada con 6 módulos independientes \\
\bottomrule
\end{longtable}


\subsection{Pruebas del backend}

La evaluación del backend se realizó mediante el framework \textit{Vitest v3.2.4}, ejecutando 80 pruebas distribuidas en 6 módulos principales. Los resultados, presentados en la Tabla~\ref{tab:resultados_backend}, demuestran una alta confiabilidad del sistema.

\begin{table}[htbp]
\centering
\caption{Resultados de pruebas del backend por módulo}
\label{tab:resultados_backend}
\begin{adjustbox}{max width=\textwidth}
\begin{tabular}{@{}lcccp{5cm}@{}}
\toprule
\textbf{Módulo} & \textbf{Pruebas} & \textbf{Exitosas} & \textbf{Tasa Éxito} & \textbf{Cobertura} \\
\midrule

Utilidades (Error Handler) & 7 & 7 & 100\% &
Manejo global de errores, \texttt{AppError}, validaciones de dominio \\
\addlinespace

Servicios IPFS & 8 & 8 & 100\% &
Subida de archivos, recuperación, validación de CIDs \\
\addlinespace

Integración IPFS & 13 & 13 & 100\% &
Inmutabilidad (IM-002), content-addressed storage, integridad de datos, múltiples formatos \\
\addlinespace

Seguridad: Validación de Entrada & 16 & 16 & 100\% &
Prevención de XSS, SQL injection, path traversal, validación de longitud y tipos numéricos \\
\addlinespace

Seguridad: Subida de Archivos & 10 & 10 & 100\% &
Límites de tamaño (10MB), validación de tipos (JPG, PNG, WEBP), rechazo de ejecutables \\
\addlinespace

API REST & 26 & 26 & 100\% &
CRUD completo (API-001), validaciones de entrada (API-002), integración blockchain/IPFS (API-003), verificación de integridad (IM-003) \\
\addlinespace

\midrule
\textbf{TOTAL} & \textbf{80} & \textbf{80} & \textbf{100\%} &
\textbf{Tiempo total: 28.98s} \\
\bottomrule
\end{tabular}
\end{adjustbox}
\end{table}


\paragraph{Análisis de resultados}
El sistema alcanzó un \textbf{100\% de éxito} en las pruebas ejecutadas, con las siguientes observaciones:

\begin{itemize}
    \item \textbf{80 pruebas exitosas}: Incluyen validaciones de CRUD, integridad blockchain, almacenamiento IPFS, manejo de errores y 26 pruebas de seguridad.
    \item \textbf{Cobertura completa}: Todos los endpoints implementados fueron validados exitosamente.
\end{itemize}

\paragraph{Validaciones de seguridad implementadas}
Como parte integral del sistema, se implementaron 26 pruebas de seguridad que validan la protección contra amenazas comunes en aplicaciones web. La Tabla~\ref{tab:validaciones_seguridad} detalla las validaciones implementadas y sus resultados.

\small
\renewcommand{\arraystretch}{1.2}
\setlength{\LTpre}{10pt}
\setlength{\LTpost}{10pt}
\begin{longtable}{lp{6cm}p{5cm}}

\multicolumn{3}{l}{\textbf{Tabla 14}} \\
\multicolumn{3}{l}{Validaciones de seguridad implementadas y verificadas} \\

\toprule
\textbf{Categoría} & \textbf{Validaciones} & \textbf{Resultado Pruebas} \\
\midrule
\endfirsthead

\toprule
\textbf{Categoría} & \textbf{Validaciones} & \textbf{Resultado Pruebas} \\
\midrule
\endhead

\midrule
\multicolumn{3}{r}{\textit{Continúa en la siguiente página}} \\
\endfoot

\bottomrule
\multicolumn{3}{l}{\textbf{Nota.} Elaboración propia.} \\
\endlastfoot

\textbf{Prevención XSS} &
Prevención de inyección de scripts maliciosos, sanitización de etiquetas HTML, validación de contenido en campos de texto &
4/4 pruebas exitosas \\
\midrule

\textbf{Prevención de Inyección SQL} &
Validación de caracteres especiales en número de placa y ubicación, prevención de comandos SQL maliciosos &
2/2 pruebas exitosas \\
\midrule

\textbf{Prevención de Traversal de Rutas} &
Validación de rutas en identificadores de contenido (CIDs), prevención de acceso no autorizado al sistema de archivos &
1/1 prueba exitosa \\
\midrule

\textbf{Validación de Longitud de Entrada} &
Límites máximos en campos de texto (ubicación, número de placa), validación de campos obligatorios &
4/4 pruebas exitosas \\
\midrule

\textbf{Validación Numérica} &
Rechazo de valores negativos, extremadamente grandes y no numéricos en campo de costo &
5/5 pruebas exitosas \\
\midrule

\textbf{Validación de Tamaño de Archivo} &
Límite de 10MB por archivo, rechazo de archivos excesivamente grandes &
2/2 pruebas exitosas \\
\midrule

\textbf{Validación de Tipo de Archivo} &
Solo imágenes permitidas (JPG, PNG, WEBP), rechazo de ejecutables, HTML y scripts &
8/8 pruebas exitosas \\
\bottomrule
\end{longtable}


Las validaciones de seguridad alcanzaron un \textbf{100\% de éxito}, demostrando que el sistema está protegido contra:

\begin{itemize}
    \item \textbf{XSS (Cross-Site Scripting)}: Sanitización de entradas con script tags y HTML injection.
    \item \textbf{SQL Injection}: Validación de caracteres especiales en campos críticos como plate number y location.
    \item \textbf{Path Traversal}: Prevención de acceso no autorizado al sistema de archivos mediante validación estricta de CIDs IPFS.
    \item \textbf{Archivos Maliciosos}: Rechazo de ejecutables, HTML y scripts, permitiendo únicamente formatos de imagen válidos (JPG, PNG, WEBP) con límite de 10MB.
\end{itemize}

\paragraph{Evidencias de funcionalidad}
Las transacciones blockchain generadas durante las pruebas incluyen:

\begin{itemize}
    \item \textbf{TX Hash Registro}: \texttt{0xbc03e11f8c9ad5cfe8c66d05fb2532b205fe5bc488b8e21645e4ed3c42c3c069}
    \item \textbf{TX Hash Actualización}: \texttt{0x611b696e7117480294986045969af2ed77250767adede497f120dc9d315f3e48}
    \item \textbf{CID IPFS Evidencia}: \texttt{QmadhsypxKm7b2P2w6b6hUZazfM9dHjvuMvsKcusp8eKMF}
\end{itemize}

La consistencia de estos identificadores a través de múltiples ejecuciones valida la reproducibilidad del sistema y la inmutabilidad de los registros blockchain. 
\section{Discusión y análisis}

La pregunta de investigación interrogó cómo mitigar el riesgo de pérdida o alteración de la integridad de los datos asociados al proceso de fotocomparendos mediante tecnologías de redes distribuidas que garantizaran registro, trazabilidad, autenticidad y confidencialidad. Los resultados obtenidos en el entorno experimental sugieren que la implementación de una arquitectura híbrida, integrando Hyperledger Fabric para la gestión de datos sensibles, Ethereum para la publicación de metadatos verificables e IPFS para el almacenamiento distribuido de evidencia, permite responder afirmativamente a dicha interrogante, aunque de manera \textbf{parcial y condicionada al alcance del prototipo}. Específicamente, los hallazgos validan el análisis de vulnerabilidades estructurales del sistema FÉNIX mediante auditoría documental, el desarrollo de un prototipo funcional con arquitectura hexagonal y mecanismos de inmutabilidad criptográfica, y la evaluación técnica mediante plan de pruebas V\&V que arrojó tiempos de respuesta menores a 3 segundos y rechazo del 100\% de intentos de modificación no autorizada. No obstante, es imperativo establecer desde el inicio la distinción fundamental entre \textbf{verificación} (\textit{verification}) y \textbf{validación} (\textit{validation}) en el ciclo de vida del software: mientras el prototipo demuestra que el sistema fue construido correctamente según especificaciones técnicas (verificación de requisitos funcionales en laboratorio), no constituye una validación operativa que demuestre que el sistema funcionará adecuadamente bajo la carga real de 457.000 comparendos semestrales ni con la variabilidad de datos heterogéneos del RUNT y SIMIT \parencite{boehm1984,iso25010}. Este alcance metodológico condiciona toda la interpretación subsiguiente de los hallazgos.

\paragraph{Cierre explícito de la pregunta de investigación.}
En síntesis de respuesta a la pregunta de investigación, los resultados evidencian que la mitigación del riesgo de alteración de integridad se logró técnica y criptográficamente para las dimensiones de \textit{registro} y \textit{trazabilidad}: el registro inmutable de transiciones de estado (PENDING, PAID, APPEALED, RESOLVED\_APPEAL, CANCELLED) mediante eventos de smart contract con timestamps criptográficos demostró, en el entorno de prueba, una resistencia del 100\% a intentos de modificación \textit{ex post}, respondiendo al componente de integridad de la pregunta. La \textit{autenticidad} se satisfizo mediante la verificación de hashes IPFS (CID) contra registros blockchain, estableciendo una cadena de custodia digital detectable algorítmicamente. La \textit{confidencialidad} se resolvió arquitectónicamente segregando datos sensibles en Hyperledger Fabric (Private Data Collections) de metadatos públicos en Ethereum, cumpliendo con la Ley 1581 de 2012. Sin embargo, esta respuesta es \textbf{condicionada y parcial}: solo abarcó cinco de los ocho estados conceptuales del proceso (excluyendo NOTIFICADA y CERRADA), operó exclusivamente con datos sintéticos que no reflejan la variabilidad de placas extranjeras, vehículos especiales o inconsistencias del RUNT real, y no pudo validar la integridad \textit{ex ante} (veracidad de la captura fotográfica frente a fallas de calibración de cámaras). Por tanto, la tecnología blockchain demostró viabilidad para garantizar la integridad de datos \textit{una vez ingresados} al sistema, pero no para asegurar la veracidad de la evidencia física ni para operar a escala institucional sin degradación de rendimiento, aspectos que requieren validación empírica adicional antes de considerar la respuesta como definitiva.

\subsection{Análisis técnico-arquitectónico: decisiones de diseño, trade-offs y riesgos de implementación}

El análisis de las vulnerabilidades del sistema FÉNIX, en términos puramente arquitectónicos, revela que sus debilidades son estructurales: la dependencia de bases de datos relacionales centralizadas configura un único punto de fallo (\textit{single point of failure}) donde la integridad lógica depende de controles de acceso administrativos rather than de propiedades criptográficas del protocolo. Desde la perspectiva de ingeniería de sistemas, esto representa una deuda técnica de diseño donde la mutabilidad inherente a SQL (operaciones UPDATE/DELETE) es incompatible con los requisitos de auditoría forense para evidencia sancionatoria.

La consecución del desarrollo del prototipo se materializó en una arquitectura hexagonal que separa responsabilidades mediante servicios especializados: \textit{HyperledgerService} para operaciones privadas, \textit{EthereumService} para publicación de hashes, y \textit{SyncService} para orquestación. Los resultados funcionales demuestran que el prototipo implementó cinco estados críticos del ciclo de vida del comparendo (PENDING, PAID, APPEALED, RESOLVED\_APPEAL, CANCELLED) con trazabilidad completa mediante eventos de smart contract, validando así el requisito de trazabilidad del objetivo general. La evidencia de integridad —100\% de coincidencia entre hashes de IPFS y registros blockchain— valida el requisito de autenticidad.

\paragraph{Decisiones de trade-off explícitas.}
Se evaluó inicialmente una arquitectura puramente pública en Ethereum \parencite{yousfi2022}, pero se descartó al proyectar costos de gas variables para un volumen de 457.000 comparendos anuales, lo cual violaría los principios de sostenibilidad presupuestal de entidades públicas colombianas y expondría datos personales sensibles en violación de la Ley 1581 de 2012. Se evaluó igualmente una solución exclusiva en Hyperledger Fabric, pero se determinó que carecería de la verificabilidad pública sin autenticación que exige el principio de transparencia de la Ley 1712 de 2014. La decisión final sacrificó la simplicidad operativa de una única tecnología (aumento de la complejidad cognitiva del equipo de mantenimiento) para ganar el equilibrio jurídico-técnico: Fabric gestiona el 90\% de las operaciones internas con >1.000 TPS y sin costos de gas, mientras Ethereum asume solo la publicación de hashes (15-30 TPS aceptable para metadatos públicos), optimizando costos y latencia sin sacrificar la verificabilidad ciudadana.

\paragraph{Riesgos técnicos residuales.}
El \textit{SyncService} —componente Node.js responsable de la sincronización entre Fabric y Ethereum— operó en el prototipo como un proceso monolítico, constituyendo un \textit{Single Point of Failure} no mitigado que representa un riesgo de diseño crítico para la consistencia eventual entre cadenas. Si este servicio falla durante la sincronización, se genera un estado "huérfano" donde la multa existe en la red privada pero no es verificable públicamente, introduciendo una ventana de inconsistencia temporal de aproximadamente 5 segundos que, aunque tolerable para registro de infracciones, sería inaceptable para sistemas de pago en tiempo real. Adicionalmente, la dependencia de la disponibilidad continua de nodos IPFS para el \textit{pinning} de evidencias constituye un riesgo de persistencia: sin el mantenimiento activo de al menos un nodo con el contenido "anclado", los datos podrían volverse inaccesibles aunque su hash permanezca inmutable en blockchain.

En términos de métricas de diseño de software, la arquitectura alcanza un acoplamiento aferente (Ca) bajo en el módulo de blockchain, permitiendo sustituir Ethereum por Polygon sin modificar la lógica de negocio (principio OCP). Sin embargo, la complejidad operativa de mantener una red Hyperledger Fabric con múltiples organizaciones requiere competencias especializadas en gestión de certificados PKI y políticas de \textit{endorsement} que trascienden las capacidades típicas de equipos de TI gubernamentales estándar, representando una barrera de entrada significativa para la replicabilidad operativa.

\subsection{Implicaciones socio-técnicas e institucionales: proyecciones de impacto y marco normativo}

Paralelamente al análisis técnico, es necesario examinar cómo la arquitectura propuesta altera los flujos organizacionales y la relación entre ciudadanía e institución. El sistema FÉNIX actual genera una asimetría informacional donde la legitimidad de los registros se presume pero no puede ser verificada autónomamente por el afectado, situación que la jurisprudencia constitucional colombiana ha señalado como riesgo para el debido proceso \parencite{corte2018}. La tasa de impugnación del 34,1 \% y las más de 155.000 PQRSD semestrales deben interpretarse no solo como indicadores de ineficiencia, sino como síntomas de una crisis de confianza institucional donde la opacidad técnica facilita la discrecionalidad administrativa.

Los resultados vinculados a la evaluación técnica permiten proyectar —bajo condiciones ideales simuladas— un escenario de impacto socio-técnico potencial. El tiempo promedio de verificación de integridad se situó en aproximadamente 2,7 segundos en el entorno de prueba, contrastando con los tiempos actuales del modelo manual, donde la resolución de una PQRSD puede extenderse por días o semanas. Desde una argumentación cuantitativa proyectada (no empírica), la posibilidad de verificación autónoma en segundos, frente a la espera de días en el sistema actual, sugiere una \textbf{hipótesis fundada} de reducción potencial de la carga operativa asociada a las PQRSD. No obstante, es imperativo matizar que estas cifras constituyen \textbf{proyecciones teóricas simuladas}, no impactos reales demostrados, dado que el estudio no realizó modelado econométrico detallado ni midió directamente la reducción de litigios en condiciones reales de operación.

La arquitectura resuelve la tensión entre inmutabilidad y derecho al olvido (art. 15, Ley 1581/2012) mediante \textbf{borrado criptográfico}: la revocación de claves de acceso a colecciones privadas en Fabric, combinada con el cese de \textit{pinning} en IPFS, hace los datos criptográficamente inaccesibles sin violar la cadena de bloques, satisfaciendo el requisito legal de supresión sin comprometer la integridad del ledger \parencite{voigt2017}. Este mecanismo es crucial desde la perspectiva institucional, ya que permite cumplir con mandatos normativos aparentemente contradictorios (transparencia vs. privacidad) sin incurrir en nulidades administrativas por violación de datos personales.

Desde el punto de vista de la gobernanza algorítmica, el consorcio propuesto (SDM, Policía, Contraloría) representa una transición desde la confianza en una única entidad hacia un protocolo de consenso multiinstitucional. Como señala Swan \parencite{swan2015}, esto no implica eliminación de la autoridad, sino su redistribución mediante protocolos criptográficos. Sin embargo, es relevante señalar que este modelo introduce el riesgo del "error inmutable": en el sistema FÉNIX, los errores de digitación se corrigen mediante UPDATE en base de datos; en el sistema propuesto, la rectificación requiere transacciones adicionales (ej. \textit{CANCELLED}) que dejan evidencia permanente del fallo inicial. Este efecto de visibilidad obligatoria fortalece la rendición de cuentas, pero introduce tensiones institucionales entre eficiencia operativa (ocultar errores para mantener reputación) y transparencia radical (visibilizar fallos como parte del historial público).

\subsection{Posicionamiento crítico frente al estado del arte}

El posicionamiento del prototipo frente a las propuestas existentes revela diferencias arquitectónicas sustanciales. Respecto a los enfoques basados exclusivamente en blockchains públicas \parencite{yousfi2022}, el presente trabajo es \textbf{superior en privacidad y sostenibilidad económica proyectada}, al evitar la exposición pública de datos personales y los costos variables de gas, pero es \textbf{equivalente en transparencia} para metadatos públicos y \textbf{conscientemente inferior en descentralización radical}, aceptando la distribución institucional como trade-off necesario para el cumplimiento normativo colombiano. Frente a los modelos híbridos que mantienen bases de datos tradicionales como capa primaria \parencite{chen2024}, este prototipo es \textbf{superior en garantías de inmutabilidad técnica}, al eliminar completamente la mutabilidad de la capa de datos crítica, pero asume una \textbf{mayor complejidad operativa} documentada. Comparado con soluciones exclusivas en Hyperledger Fabric (registros vehiculares gubernamentales en Estonia), el aporte es \textbf{superior en verificabilidad pública}, al incorporar la capa Ethereum que permite consulta sin autenticación.

\subsection{Limitaciones del estudio: alcance metodológico, institucional y ecológico}

Más allá de las restricciones técnicas de escalabilidad ya discutidas, es imperativo explicitar las limitaciones metodológicas y de diseño experimental que condicionan la generalización de los hallazgos. En primer lugar, el estudio adoptó un \textbf{diseño de investigación experimental puro} (laboratorio controlado) rather than un diseño cuasi-experimental o de campo, lo cual introduce sesgos de validez ecológica significativos. La imposibilidad de acceder a datos reales de comparendos —derivada de restricciones legales de protección de datos personales (Ley 1581/2012) y políticas de reserva de información de la SDM— obligó al uso de datos sintéticos generados mediante scripts de prueba. Esta limitación metodológica implica que el prototipo no fue expuesto a la variabilidad, inconsistencias y casos atípicos que caracterizan los datos del mundo real (ej. placas extranjeras, vehículos diplomáticos, formatos heterogéneos de registro en el RUNT), lo cual podría influir en la lógica de negocio y en el manejo de errores en un entorno de producción.

En segundo lugar, la \textbf{validez interna} de las pruebas de rendimiento está limitada por el sesgo de selección inherente al entorno controlado: las 80 pruebas automatizadas se ejecutaron en hardware optimizado, red local sin latencia de internet, y sin carga concurrente de usuarios que simule la operación real de 457.000 comparendos semestrales. La ausencia de pruebas de estrés (\textit{stress testing}) con volúmenes masivos y la imposibilidad de evaluar el comportamiento del sistema bajo ataques de denegación de servicio (DoS) constituyen restricciones metodológicas que impiden afirmar la robustez operativa del sistema bajo condiciones adversas reales.

En tercer lugar, existen \textbf{limitaciones institucionales y de gobernanza} no resueltas en el alcance del prototipo. La integración con sistemas externos (RUNT y SIMIT) fue simulada mediante \textit{mocks} (servicios de prueba) rather than APIs reales, dado que el acceso a dichas interfaces requiere convenios interadministrativos formales y credenciales de producción que exceden el alcance de un proyecto de grado académico. Esta limitación institucional implica que no se validaron las latencias reales de servicios externos, los protocolos de comunicación específicos, ni los posibles cuellos de botella en la interoperabilidad con sistemas legados gubernamentales. Asimismo, la ausencia de validación con \textit{stakeholders} reales (agentes de tránsito, funcionarios de la Contraloría, ciudadanos usuarios finales) mediante estudios de aceptación tecnológica (TAM/UTAUT) constituye una restricción metodológica relevante: el éxito de adopción depende no solo de la viabilidad técnica, sino de la percepción de utilidad y facilidad de uso por parte de los actores institucionales, aspecto no evaluado empíricamente en este estudio.

Finalmente, la \textbf{validez externa} (generalización) está restringida por el contexto específico de Bogotá y su marco normativo colombiano. Aunque la arquitectura es conceptualmente replicable, las particularidades del Código Nacional de Tránsito, la jurisprudencia de la Corte Constitucional \parencite{corte2018} y la Ley 1581/2012 configuran un entorno legal que no necesariamente es trasladable a otras jurisdicciones sin adaptaciones sustanciales. La imposibilidad de realizar un piloto de campo con 5.000-10.000 multas reales —limitación derivada de barreras burocráticas y de acceso a datos— impide validar empíricamente las proyecciones de reducción de carga operativa y ROI estimado, manteniendo dichas inferencias en el ámbito de la estimación teórica rather than de la evidencia empírica directa.

\subsection{Direcciones de evolución técnica y validación operativa}

La transición desde la verificación técnica hacia la validación operativa requiere el desarrollo de tres líneas de evolución diferenciadas. \textbf{Desde la ingeniería de sistemas}, es prioritario: (a) la eliminación del SPOF del \textit{SyncService} mediante su re-arquitectura como un servicio distribuido con tolerancia a fallos (patrón \textit{circuit breaker} y colas de eventos redundantes); (b) la evaluación de estrategias de sharding en Fabric para manejar volúmenes reales de 457.000 comparendos semestrales sin degradación de latencia; y (c) la implementación de mecanismos de oráculo certificador para registrar estados físicos (notificación efectiva) y cerrar la brecha entre el mundo digital y el físico.

\textbf{Desde la perspectiva institucional y organizacional}, es imperativo: (a) la realización de estudios de aceptación tecnológica mediante modelos como TAM o UTAUT, evaluando la percepción de confianza y usabilidad por parte de agentes de tránsito y ciudadanos, aspecto crítico dado que el éxito de adopción depende más de la experiencia de usuario que de la robustez técnica subyacente; (b) la negociación de convenios interadministrativos formales para la gobernanza del consorcio blockchain; y (c) la realización de un piloto controlado con 5.000–10.000 multas reales que permita validar empíricamente las proyecciones de reducción de carga operativa, aspecto que el presente estudio no pudo realizar por restricciones de acceso a datos.

\subsection{Síntesis conclusiva y delimitación del alcance}

En síntesis, el prototipo constituye una respuesta técnica verificada frente a las deficiencias de integridad del sistema de fotocomparendos, respondiendo parcialmente a la pregunta de investigación mediante la demostración de viabilidad técnica de mecanismos criptográficos de inmutabilidad. Desde la perspectiva de ingeniería de sistemas, la arquitectura híbrida demuestra que es posible diseñar sistemas distribuidos que equilibran privacidad y transparencia mediante trade-offs explícitos entre rendimiento y descentralización, validando los objetivos específicos planteados en el alcance experimental. Desde la perspectiva institucional, los resultados sugieren un \textbf{potencial} de transformación de los flujos de confianza entre ciudadanía y Estado, proyectando una reducción hipotética de la fricción operativa asociada a litigios, aunque esta proyección requiere validación empírica en condiciones reales de operación.

La evidencia obtenida en el entorno de laboratorio —con datos sintéticos, volúmenes reducidos y ausencia de integración con sistemas reales— no permite afirmar la viabilidad operativa a escala completa ni la aceptación institucional del modelo de gobernanza distribuida propuesto. La blockchain debe entenderse como un componente técnico de un ecosistema de confianza más amplio, no como solución aislada, reconociendo que la verificación técnica exitosa es condición necesaria pero no suficiente para la validación operativa en el contexto de la administración pública distrital. La respuesta a la pregunta de investigación es, por tanto, \textbf{afirmativa pero condicionada}: las tecnologías de redes distribuidas permiten mitigar el riesgo de alteración de integridad \textit{dentro del sistema digital}, pero su efectividad completa depende de la resolución de limitaciones metodológicas, institucionales y de escalabilidad que trascienden el alcance del presente estudio experimental.
\section{Conclusiones y trabajo futuro}

La arquitectura híbrida blockchain que combina Hyperledger Fabric (privacidad) y Ethereum (transparencia pública) es técnicamente viable para la gestión de fotocomparendos. El prototipo desarrollado valida que las tecnologías de registro distribuido pueden garantizar simultáneamente la inmutabilidad de registros, la protección de datos sensibles y la verificación pública ciudadana, abordando las limitaciones identificadas en el sistema actual de Bogotá.

Las pruebas realizadas confirman que el sistema cumple con los principios propuestos: el 100\% de los intentos de modificación fueron rechazados por los mecanismos de consenso, el sistema de direccionamiento por contenido (CID) de IPFS detectó automáticamente todas las alteraciones simuladas, y los tiempos de respuesta medidos (<3 segundos) validan que la arquitectura es viable para aplicaciones en tiempo real.

El backend implementado con interfaces REST estándar y el frontend desarrollado con React demuestran que las tecnologías blockchain pueden abstraerse detrás de APIs convencionales e integrarse con interfaces modernas sin comprometer la experiencia del usuario, facilitando la adopción por parte de instituciones gubernamentales.

\subsection{Síntesis del cumplimiento de objetivos}

En relación con el objetivo general planteado en la Introducción, orientado a demostrar la viabilidad de un prototipo basado en tecnologías de registro distribuido para fortalecer la integridad y trazabilidad de los fotocomparendos, los resultados obtenidos muestran que la arquitectura híbrida diseñada, implementada y validada cumple con este propósito. El desarrollo del prototipo, descrito en las secciones Diseño del prototipo e Implementación del prototipo, junto con las pruebas de inmutabilidad, verificabilidad y rendimiento presentadas en la sección Resultados de las pruebas de inmutabilidad y verificabilidad del prototipo, evidencian que es posible garantizar integridad criptográfica, protección de datos sensibles y verificación pública ciudadana en un entorno controlado.

De manera complementaria, el análisis del sistema actual y del marco regulatorio (secciones Introducción y Estado del arte) permitió caracterizar de forma rigurosa las brechas de integridad, trazabilidad y confianza del modelo centralizado vigente, lo que sirvió de base para la definición de requisitos del prototipo. Sobre esta base, la construcción de la solución híbrida y su evaluación experimental cubrieron los aspectos funcionales y no funcionales previstos en los objetivos específicos: comprender el proceso de fotocomparendos, proponer una arquitectura alternativa y validar empíricamente su comportamiento. En conjunto, estos resultados permiten concluir que el prototipo satisface coherentemente los objetivos formulados y ofrece un modelo de referencia replicable para otros contextos donde la integridad, trazabilidad y verificabilidad de registros públicos sean críticas.

\subsection{Trabajo futuro}

Para la evolución del proyecto se proponen las siguientes líneas de trabajo:

\begin{enumerate}
    \item \textbf{Escalamiento a producción:} Escalar la red Fabric a múltiples organizaciones (SDM, Policía, Auditoría), implementar Private Data Collections, y desplegar nodos IPFS en infraestructura distribuida con políticas de replicación.

    \item \textbf{Piloto controlado:} Realizar un piloto con la Secretaría Distrital de Movilidad utilizando datos reales (5,000-10,000 multas), integración con SIMIT/RUNT y evaluación de rendimiento bajo carga operativa.

    \item \textbf{Funcionalidades avanzadas:} Implementar módulo de pagos integrado (PSE, billeteras digitales), sistema de apelaciones en línea, notificaciones automáticas y dashboard analítico para toma de decisiones.

    \item \textbf{Replicabilidad:} Adaptar la arquitectura para otras ciudades colombianas mediante federación de redes Fabric, explorar soluciones Layer 2 para reducción de costos de gas, y proponer estandarización nacional de Smart Contracts.
\end{enumerate}


\section{Anexos}



\subsection{Anexo A: repositorios del proyecto}

\subsubsection{Enlaces a los repositorios}

El proyecto de fotomultas \textit{blockchain} está distribuido en los siguientes repositorios de código fuente:

\begin{itemize}
    \item \textbf{Frontend (\textit{React} + \textit{TypeScript}):} \url{https://github.com/k-delta/fotomultas-front}
    \item \textbf{Backend (\textit{Smart Contracts} + \textit{API}):} \url{https://github.com/CristianGT089/backend-multas}
\end{itemize}

\subsubsection{Descripción técnica de cada repositorio}

\paragraph{Repositorio Frontend: fotomultas-front}
\textbf{URL:} \url{https://github.com/k-delta/fotomultas-front}

\textbf{Tecnologías principales:}
\begin{itemize}
    \item \textbf{Lenguaje:} \textit{TypeScript} (99.2\%) con configuración \textit{ESM}
    \item \textbf{Framework:} \textit{React} 18+ con \textit{Vite} como bundler
    \item \textbf{Estilos:} \textit{Tailwind CSS} para diseño responsive
    \item \textbf{Estado:} \textit{Zustand} para gestión de estado global
    \item \textbf{Testing:} \textit{Jest} con \textit{React Testing Library}
    \item \textbf{Licencia:} \textit{MIT License}
\end{itemize}

\textbf{Contenido:}
\begin{itemize}
    \item Interfaz de usuario para agentes de tránsito (registro de multas)
    \item Panel ciudadano para consulta y verificación de multas
    \item Dashboard administrativo con estadísticas y métricas
    \item Integración con \textit{API REST} del backend
    \item Componentes reutilizables y diseño modular
\end{itemize}

\paragraph{Repositorio Backend: backend-multas}
\textbf{URL:} \url{https://github.com/CristianGT089/backend-multas}

\textbf{Tecnologías principales:}
\begin{itemize}
    \item \textbf{Lenguaje:} \textit{TypeScript} (93.1\%), \textit{JavaScript} (4.1\%), \textit{Solidity} (2.8\%)
    \item \textbf{Blockchain:} \textit{Smart Contracts} en \textit{Solidity} para \textit{Ethereum}
    \item \textbf{Framework:} \textit{Express.js} con \textit{TypeScript} para \textit{API REST}
    \item \textbf{Desarrollo:} \textit{Hardhat} para compilación y despliegue de contratos
    \item \textbf{Testing:} \textit{Vitest} para pruebas unitarias e integración
    \item \textbf{Almacenamiento:} Integración con \textit{IPFS} para evidencias
\end{itemize}

\textbf{Contenido:}
\begin{itemize}
    \item \textit{Smart Contract} \texttt{FineManagement.sol} para gestión de multas
    \item \textit{API REST} con endpoints para registro, consulta y actualización
    \item Servicios de integración con \textit{IPFS} y blockchain
    \item Configuración de \textit{Hardhat} para desarrollo local y testnet
    \item Scripts de despliegue y testing automatizado
    \item Documentación \textit{Swagger} para la \textit{API}
\end{itemize}

\subsubsection{Instrucciones de acceso}

Para acceder al código fuente del proyecto:

\begin{enumerate}
    \item \textbf{Clonar repositorios:}
    \begin{verbatim}
    git clone https://github.com/CristianGT089/backend-multas
    git clone https://github.com/k-delta/fotomultas-front
    \end{verbatim}
    
    \item \textbf{Revisar documentación:} Cada repositorio incluye archivos README con instrucciones de instalación y configuración
    
    \item \textbf{Explorar código:} El código está organizado en carpetas lógicas (src/, contracts/, test/, etc.)
\end{enumerate}

\subsection{Anexo B: manual de usuario}

\subsubsection{Manual para agentes de tránsito}

\paragraph{1. Iniciar Sesión.}
\begin{itemize}
    \item Acceder a la URL del sistema
    \item Ingresar credenciales proporcionadas por el administrador
    \item Seleccionar rol "Agente de Tránsito"
\end{itemize}

\paragraph{2. Registrar una Multa.}
\begin{itemize}
    \item En el menú principal, seleccionar "Registrar Multa"
    \item Completar el formulario con:
    \begin{itemize}
        \item Número de placa del vehículo
        \item Tipo de infracción (seleccionar de lista desplegable)
        \item Ubicación (GPS automático o manual)
        \item Costo de la multa (calculado automáticamente según tipo)
    \end{itemize}
    \item Cargar evidencia fotográfica (máximo 5MB, formato JPG/PNG)
    \item Hacer clic en "Registrar Multa"
    \item Esperar confirmación de blockchain (aprox. 2-5 segundos)
    \item Anotar el ID de multa generado para referencia
\end{itemize}

% Figura de registro de multa en anexos: (eliminada del PDF)
% Se comenta la figura para que no aparezca en la lista de figuras ni en el documento final.
% \begin{figure}[htbp]
%     \centering
%     % \includegraphics[width=0.7\textwidth]{Images/UI_Registro_Multa.png}
%     \caption{Pantalla de Registro de Multa - Panel del Agente}
% \end{figure}

\paragraph{3. Actualizar Estado de multa.}
\begin{itemize}
    \item Buscar multa por ID o número de placa
    \item Seleccionar "Actualizar Estado"
    \item Elegir nuevo estado (Pagada, En Apelación, etc.)
    \item Ingresar razón del cambio
    \item Confirmar actualización
\end{itemize}

\subsubsection{Manual para ciudadanos}

\paragraph{1. Consultar multas.}
\begin{itemize}
    \item Acceder a la sección pública (sin autenticación requerida)
    \item Ingresar número de placa del vehículo
    \item Hacer clic en "Buscar"
    \item Revisar lista de multas asociadas
\end{itemize}

% Figura de consulta pública en anexos: (eliminada del PDF)
% Se comenta la figura para que no aparezca en la lista de figuras ni en el documento final.
% \begin{figure}[htbp]
%     \centering
%     % \includegraphics[width=0.7\textwidth]{Images/UI_Consulta_Ciudadano.png}
%     \caption{Pantalla de Consulta Pública - Panel Ciudadano}
% \end{figure}

\paragraph{2. Verificar Integridad de evidencia.}
\begin{itemize}
    \item Seleccionar una multa de la lista
    \item Hacer clic en "Verificar Integridad"
    \item El sistema compara el hash de la evidencia en blockchain 
          con el archivo en IPFS
    \item Se muestra resultado: "Evidencia Verificada" o 
          "Evidencia Alterada"
\end{itemize}

\paragraph{3. Presentar apelación.}
\begin{itemize}
    \item Crear cuenta en el sistema (requiere verificación de identidad)
    \item Seleccionar multa a apelar
    \item Completar formulario de apelación con argumentos
    \item Cargar evidencias de respaldo (opcional)
    \item Enviar apelación
    \item Esperar notificación de resolución (máximo 30 días hábiles)
\end{itemize}

\subsection{Anexo C: glosario de términos}

La Tabla~\ref{tab:glosario_terminos} presenta las definiciones de los principales términos técnicos utilizados en este documento.

\footnotesize
\renewcommand{\arraystretch}{1.3}
\setlength{\LTpre}{10pt}
\setlength{\LTpost}{10pt}
\begin{longtable}{p{4cm}p{9.5cm}}

\caption{Glosario de Términos Técnicos}
\label{tab:glosario_terminos} \\
\toprule
\textbf{Término} & \textbf{Definición} \\
\midrule
\endfirsthead

\caption[]{(Continuación)} \\
\toprule
\textbf{Término} & \textbf{Definición} \\
\midrule
\endhead

\midrule
\multicolumn{2}{r}{\textit{Continúa en la siguiente página}} \\
\endfoot

\bottomrule
\multicolumn{2}{l}{\textbf{Nota.} Elaboración propia.} \\
\endlastfoot

ABI (Application Binary Interface) & Interfaz que define cómo llamar funciones de un Smart Contract desde aplicaciones externas. Contiene nombres de funciones, parámetros y tipos de retorno. \\
\midrule

Blockchain & Tecnología de registro distribuido que almacena datos en bloques encadenados mediante hashes criptográficos, garantizando inmutabilidad. \\
\midrule

CA (Certificate Authority) & Entidad que emite y gestiona certificados digitales en una red Hyperledger Fabric, controlando identidades y permisos. \\
\midrule

Chaincode & Smart Contract en el contexto de Hyperledger Fabric, generalmente escrito en Go, que define la lógica de negocio. \\
\midrule

CID (Content Identifier) & Hash único que identifica un archivo en IPFS. Se genera mediante criptografía del contenido del archivo. \\
\midrule

Consenso & Mecanismo mediante el cual los nodos de una blockchain acuerdan la validez de las transacciones. Ejemplos: PBFT, PoS, PoW. \\
\midrule

DLT (Distributed Ledger Technology) & Tecnología de libro mayor distribuido que mantiene registros sincronizados entre múltiples nodos sin autoridad central. \\
\midrule

Ethers.js & Biblioteca JavaScript para interactuar con la blockchain de Ethereum, permitiendo leer datos y enviar transacciones. \\
\midrule

Gas & Unidad de medida del costo computacional en Ethereum. Cada operación consume gas que se paga en Ether. \\
\midrule

Hardhat & Framework de desarrollo para Ethereum que facilita compilación, testing y despliegue de Smart Contracts. \\
\midrule

Hash Criptográfico & Función matemática que convierte datos de cualquier tamaño en una cadena de longitud fija. Ejemplos: SHA-256, Keccak-256. \\
\midrule

Hyperledger Fabric & Plataforma de blockchain permisionada empresarial, parte del proyecto Hyperledger de Linux Foundation. \\
\midrule

Inmutabilidad & Propiedad de blockchain que garantiza que datos una vez escritos no pueden ser alterados sin dejar evidencia. \\
\midrule

IPFS (InterPlanetary File System) & Sistema de archivos peer-to-peer distribuido que usa direccionamiento por contenido mediante CIDs. \\
\midrule

Ledger & Libro mayor que registra todas las transacciones en una blockchain. Es distribuido y sincronizado entre nodos. \\
\midrule

Nodo (Node) & Computadora que participa en una red blockchain, manteniendo una copia del ledger y validando transacciones. \\
\midrule

OpenZeppelin & Librería de Smart Contracts auditados y seguros para Ethereum, proporciona implementaciones estándar de tokens, control de acceso, etc. \\
\midrule

Orderer & Nodo en Hyperledger Fabric que ordena transacciones y las agrupa en bloques para distribuir a los peers. \\
\midrule

PBFT (Practical Byzantine Fault Tolerance) & Algoritmo de consenso tolerante a fallas bizantinas usado en Hyperledger Fabric, eficiente para redes permisionadas. \\
\midrule

Peer & Nodo en Hyperledger Fabric que mantiene una copia del ledger y ejecuta chaincode. \\
\midrule

Pinning & En IPFS, mantener un archivo almacenado permanentemente en un nodo para garantizar su disponibilidad. \\
\midrule

PoS (Proof of Stake) & Mecanismo de consenso donde validadores son seleccionados según la cantidad de criptomoneda que poseen. \\
\midrule

PoW (Proof of Work) & Mecanismo de consenso que requiere resolver acertijos criptográficos complejos para validar bloques. \\
\midrule

Private Data Collections & Funcionalidad de Hyperledger Fabric para almacenar datos privados que solo ciertos nodos pueden acceder. \\
\midrule

Smart Contract & Programa autoejecutante almacenado en blockchain que ejecuta lógica de negocio cuando se cumplen condiciones. \\
\midrule

Solidity & Lenguaje de programación orientado a objetos para escribir Smart Contracts en Ethereum. \\
\midrule

Testnet & Red de prueba de blockchain que imita el funcionamiento de la red principal pero sin valor real. Ejemplo: Sepolia. \\
\midrule

Transaction Hash & Identificador único de una transacción en blockchain, generado mediante hash criptográfico de su contenido. \\
\midrule

TypeScript & Superset de JavaScript con tipado estático, usado para desarrollo backend del proyecto. \\
\midrule

Wallet & Software que almacena claves privadas y permite firmar transacciones en blockchain. \\
\bottomrule
\end{longtable}



% Referencias
\newpage
\printbibliography[
	heading=bibintoc,
	title={Referencias}
]
\end{document}
