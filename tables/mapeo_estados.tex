\footnotesize
\renewcommand{\arraystretch}{1.2}
\setlength{\LTpre}{10pt}
\setlength{\LTpost}{10pt}
\begin{longtable}{p{3.8cm} p{3.8cm} p{4.8cm}}

\caption{Mapeo entre estados conceptuales y estados del \textit{smart contract}}
\label{tab:mapeo_estados} \\
\toprule
\textbf{Estado Conceptual (ES)} & \textbf{Estado en \textit{Smart Contract} (EN)} & \textbf{Descripción del Mapeo} \\
\midrule
\endfirsthead

\caption[]{(Continuación)} \\
\toprule
\textbf{Estado Conceptual (ES)} & \textbf{Estado en \textit{Smart Contract} (EN)} & \textbf{Descripción del Mapeo} \\
\midrule
\endhead

\midrule
\multicolumn{3}{r}{\textit{Continúa en la siguiente página}} \\
\endfoot

\bottomrule
\multicolumn{3}{p{13cm}}{\textbf{Nota.} Mapeo de estados del \textit{smart contract}.} \\
\endlastfoot

GENERADA / PENDIENTE\_RESPUESTA & PENDING & Estado inicial tras registro del comparendo. Engloba tanto la generación inicial como el período de espera de respuesta ciudadana. El \textit{smart contract} usa PENDING para representar cualquier comparendo que aún no ha sido resuelto mediante pago, apelación o cancelación. \\
\midrule
PAGADA & PAID & Comparendo cuya obligación económica ha sido saldada. Mapeo directo 1:1 entre concepto y código. Representa el cierre exitoso del proceso mediante pago voluntario o forzoso. \\
\midrule
EN\_APELACION & APPEALED & Comparendo bajo proceso de revisión por PQRSD ciudadana. Mapeo directo 1:1. Este estado es crítico para la trazabilidad de disputas y fundamenta la necesidad de inmutabilidad de evidencia. \\
\midrule
RESUELTA\_APELACION & RESOLVED\_APPEAL & Apelación procesada con decisión administrativa (confirmación, revocación parcial/total, o anulación). Mapeo directo 1:1. Estado terminal para el flujo de apelaciones. \\
\midrule
CANCELADA & CANCELLED & Comparendo cancelado administrativamente por anulación judicial, defectos procedimentales, o corrección de errores. Mapeo directo 1:1. Requiere máxima trazabilidad para prevenir cancelaciones irregulares. \\
\midrule
NOTIFICADA & \textit{No implementado} & Este estado conceptual no tiene equivalente en el \textit{smart contract} del prototipo. En una implementación de producción, se agregaría estado NOTIFIED entre PENDING y las transiciones subsecuentes. La notificación podría registrarse mediante evento \textit{blockchain} con \textit{timestamp} inmutable. \\
\midrule
CERRADA & \textit{No implementado} & Estado final conceptual que indica cierre definitivo del proceso. En el prototipo, los estados PAID, RESOLVED\_APPEAL y CANCELLED funcionan como estados terminales. Una implementación completa podría agregar estado CLOSED explícito para uniformidad. \\
\bottomrule
\end{longtable}
