\small
\renewcommand{\arraystretch}{1.2}
\setlength{\LTpre}{10pt}
\setlength{\LTpost}{10pt}
\begin{longtable}{p{4.5cm} p{4.5cm} p{3.5cm} p{2.5cm}}

\caption{Variables del problema de investigación}
\label{tab:variables_problema} \\
\toprule
\textbf{Variable} & \textbf{Definición} & \textbf{Medición Actual} & \textbf{Meta con Prototipo} \\
\midrule
\endfirsthead

\caption[]{(Continuación)} \\
\toprule
\textbf{Variable} & \textbf{Definición} & \textbf{Medición Actual} & \textbf{Meta con Prototipo} \\
\midrule
\endhead

\midrule
\multicolumn{4}{r}{\textit{Continúa en la siguiente página}} \\
\endfoot

\bottomrule
\multicolumn{4}{l}{\textbf{Nota.} Elaboración propia basada en datos de \textcite{sdm2024estadisticas} y \textcite{contraloria2024detrimento}.} \\
\endlastfoot

Tasa de Impugnación & Porcentaje de comparendos que generan PQRSD por parte de ciudadanos que cuestionan su validez o evidencia & 34.1\% (155,854 PQRSD de 457,000 comparendos semestrales) & Reducción esperada por mayor confianza en integridad de evidencia \\
\midrule
Detrimento Patrimonial & Pérdida económica estimada para el Distrito por comparendos impugnados exitosamente o declarados nulos & \$8,000+ millones de pesos semestrales & Cuantificación de reducción mediante trazabilidad verificable \\
\midrule
Carga Operativa PQRSD & Cantidad de solicitudes de petición, queja, reclamo y denuncia que deben procesarse administrativamente & 155,854 solicitudes por semestre (2024-I) & Reducción por transparencia y verificabilidad autónoma \\
\midrule
Vulnerabilidad Ciudadana & Exposición del ciudadano a fraudes o manipulación de registros de comparendos (ej. casos Juzto.co) & Casos documentados de suplantación y modificación irregular & Mitigación por inmutabilidad criptográfica y registro distribuido \\
\bottomrule
\end{longtable}
