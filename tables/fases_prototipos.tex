\begin{longtable}{p{2.5cm}p{3.5cm}p{6.5cm}}
\toprule
\textbf{Fase} & \textbf{Descripción} & \textbf{Aplicación en el Proyecto} \\
\midrule
\endfirsthead

\toprule
\textbf{Fase} & \textbf{Descripción} & \textbf{Aplicación en el Proyecto} \\
\midrule
\endhead

\midrule
\multicolumn{3}{r}{\textit{Continúa en la siguiente página}} \\
\endfoot

\bottomrule
\endlastfoot

\textbf{1. Requisitos Iniciales} & Recopilación de los requisitos funcionales básicos y esenciales del sistema. & Se definieron las funcionalidades mínimas viables: registro inmutable de multas, almacenamiento de evidencia en IPFS, consulta pública y un mecanismo para la verificación de integridad. \\
\midrule
\textbf{2. Construcción del Prototipo} & Desarrollo rápido de una versión funcional reducida que implementa los requisitos iniciales. & Se implementó un prototipo funcional que incluía un Smart Contract en una red local de Ethereum, una API REST para la comunicación y un frontend básico para la interacción del usuario. \\
\midrule
\textbf{3. Evaluación del Prototipo} & Validación del prototipo mediante pruebas internas para evaluar su funcionalidad y alineación con los objetivos. & Se ejecutó un plan de pruebas exhaustivo (detallado en la sección Plan de pruebas) para validar la inmutabilidad de los registros, la integridad de la evidencia y la usabilidad de la interfaz con datos simulados. \\
\midrule
\textbf{4. Refinamiento e Iteración} & Ajuste y mejora del prototipo basándose en los hallazgos de la evaluación. & Con base en los resultados, se optimizó el consumo de gas del Smart Contract, se mejoraron las validaciones de la API y se refinó la arquitectura para incorporar la capa privada con Hyperledger Fabric. \\
\midrule
\textbf{5. Documentación Final} & Una vez validado el concepto, se documenta la arquitectura final y se proponen los siguientes pasos. & Se consolidó el diseño de la arquitectura híbrida final y se elaboró un \textit{roadmap} detallado para una eventual implementación en un entorno de producción. \\
\bottomrule
\end{longtable}
\addcontentsline{toc}{table}{Tabla 7: Fases del desarrollo del prototipo}
\addcontentsline{toc}{table}{Tabla 7: Fases del desarrollo del prototipo}
