\small
\renewcommand{\arraystretch}{1.2}
\setlength{\LTpre}{10pt}
\setlength{\LTpost}{10pt}
\begin{longtable}{p{4.5cm} p{5.2cm} p{5.2cm}}

\caption{\textbf{Tabla 2} \textit{Comparación entre bases de datos tradicionales y blockchain para gestión de registros gubernamentales}}
\label{tab:comparacion_bd_blockchain} \\
\toprule
\textbf{Característica} & \textbf{Base de Datos Convencional} & \textbf{Blockchain} \\
\midrule
\endfirsthead

\caption[]{(Continuación)} \\
\toprule
\textbf{Característica} & \textbf{Base de Datos Convencional} & \textbf{Blockchain} \\
\midrule
\endhead

\midrule
\multicolumn{3}{r}{\textit{Continúa en la siguiente página}} \\
\endfoot

\bottomrule
\multicolumn{3}{l}{\textbf{Nota.} Comparativa técnica BD vs Blockchain.} \\
\endlastfoot

Modelo de confianza & Se basa en un administrador central (entidad de TI) & Confianza distribuida entre múltiples nodos \\
\midrule
Inmutabilidad & Registros pueden ser modificados o eliminados por administradores & Los registros son inmutables por diseño \\
\midrule
Trazabilidad / Auditoría & Depende de la implementación y control interno & Historial completo e inalterable disponible \\
\midrule
Riesgo de corrupción interna & Alto, si hay privilegios indebidos o colusión & Bajo, no se puede alterar sin consenso de la red \\
\midrule
Seguridad criptográfica & Opcional, no siempre integrada nativamente & Integrada (firmas digitales, hashes, cifrado) \\
\midrule
Disponibilidad / tolerancia a fallos & Riesgo de puntos únicos de falla & Alta disponibilidad por replicación descentralizada \\
\midrule
Velocidad de operación & Alta velocidad en lectura/escritura & Menor velocidad, prioriza integridad y consenso \\
\bottomrule
\end{longtable}
