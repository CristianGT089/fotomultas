% Tabla de comparación entre bases de datos tradicionales y blockchain
\begin{table}[htbp]
\centering
\begin{tabular}{|p{4.5cm}|p{5.2cm}|p{5.2cm}|}
\hline
\textbf{Característica} & \textbf{Base de Datos Convencional} & \textbf{Blockchain} \\
\hline
Modelo de confianza & Se basa en un administrador central (entidad de TI) & Confianza distribuida entre múltiples nodos \\
\hline
Inmutabilidad & Registros pueden ser modificados o eliminados por administradores & Los registros son inmutables por diseño \\
\hline
Trazabilidad / Auditoría & Depende de la implementación y control interno & Historial completo e inalterable disponible \\
\hline
Riesgo de corrupción interna & Alto, si hay privilegios indebidos o colusión & Bajo, no se puede alterar sin consenso de la red \\
\hline
Seguridad criptográfica & Opcional, no siempre integrada nativamente & Integrada (firmas digitales, hashes, cifrado) \\
\hline
Disponibilidad / tolerancia a fallos & Riesgo de puntos únicos de falla & Alta disponibilidad por replicación descentralizada \\
\hline
Velocidad de operación & Alta velocidad en lectura/escritura & Menor velocidad, prioriza integridad y consenso \\
\hline
\end{tabular}
\caption{Comparación entre bases de datos tradicionales y blockchain}
\end{table} 