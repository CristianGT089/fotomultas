% Tabla de casos de prueba funcionales
\paragraph{Pruebas Funcionales}
\begin{table}[htbp]
    \centering
    \footnotesize
    \caption{Casos de Prueba Funcionales}
    \label{tab:casos_funcionales}

    \begin{tabular}{|
        >{\raggedright\arraybackslash}p{0.07\textwidth}|
        >{\raggedright\arraybackslash}p{0.20\textwidth}|
        >{\raggedright\arraybackslash}p{0.40\textwidth}|
        >{\raggedright\arraybackslash}p{0.20\textwidth}|}
        \hline
        \textbf{ID} & \textbf{Descripción} & \textbf{Pasos de Ejecución} & \textbf{Datos de Entrada} \\
        \hline
        % Fila 1
        \textbf{FT-001} & 
        Registro exitoso de fotocomparendo & 
        1. Login en SisFotocomp. \newline 
        2. Ir a "Registrar nueva multa". \newline 
        3. Ingresar datos (placa, fecha, tipo). \newline 
        4. Adjuntar imagen. \newline 
        5. Enviar. & 
        Placa: XYZ789, Fecha: [Hoy], Tipo: Exceso Velocidad, Imagen: evidencia01.jpg \\
        \hline
        % Fila 2
        \textbf{FT-002} & 
        Consulta y verificación (Agente/Admin) & 
        1. Login como Agente/Admin. \newline 
        2. Ir a "Gestión de multas". \newline 
        3. Buscar multa FT-001 por ID o placa. \newline 
        4. Ver detalles. \newline 
        5. Verificar información e imagen IPFS. & 
        ID/Placa de la multa FT-001. \\
        \hline
        % Fila 3
        \textbf{FT-003} & 
        Consulta ciudadana & 
        1. Acceder a "Consulta de Multas". \newline 
        2. Ingresar documento, número y placa. \newline 
        3. Ingresar CAPTCHA. \newline 
        4. Consultar. & 
        Datos del propietario/vehículo de FT-001. \\
        \hline
        % Fila 4
        \textbf{FT-004} & 
        Registro con datos incompletos & 
        1. Intentar registrar multa sin placa o sin imagen. & 
        Placa: Vacía, Imagen: No adjuntada. \\
        \hline
        % Fila 5
        \textbf{FT-005} & 
        Actualización de estado & 
        1. Seleccionar multa FT-001. \newline 
        2. Cambiar estado (ej. "Apelada", "Pagada"). \newline 
        3. Guardar. & 
        Multa FT-001, Nuevo estado: "Apelada". \\
        \hline
        % Fila 6
        \textbf{FT-006} & 
        Consistencia Ledger-IPFS & 
        1. Registrar multa (similar a FT-001). \newline 
        2. Anotar CID de IPFS y metadatos. \newline 
        3. Recuperar transacción del ledger. \newline 
        4. Recuperar imagen de IPFS. & 
        Nueva multa, nueva imagen. \\
        \hline
    \end{tabular}
\end{table} 