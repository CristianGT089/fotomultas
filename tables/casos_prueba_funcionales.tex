% Tabla de casos de prueba funcionales
\begin{table}[htbp]
    \begin{flushleft}
        \textbf{Tabla 2}\\[2em]
        \textit{Casos de prueba funcionales para validar operaciones básicas del sistema}
    \end{flushleft}
    \vspace{1em}
    \addcontentsline{lot}{table}{Tabla 2. Casos de prueba funcionales para validar operaciones básicas del sistema}
    \centering
    \begin{tabular}{p{2cm} p{4cm} p{3cm} p{3cm} p{3cm}}
        \toprule
        \textbf{ID} & \textbf{Caso de Prueba} & \textbf{Precondiciones} & \textbf{Acciones} & \textbf{Resultado Esperado} \\
        \midrule
        FP-001 & Registro de fotocomparendo & Usuario autenticado, imagen disponible & 1. Cargar imagen a IPFS\newline 2. Registrar metadatos en Blockchain & CID generado, transacción exitosa \\
        FP-002 & Consulta de comparendo & Comparendo registrado previamente & 1. Ingresar ID de comparendo\newline 2. Consultar en Blockchain & Datos completos mostrados \\
        FP-003 & Verificación de evidencia & CID válido en Blockchain & 1. Extraer CID de transacción\newline 2. Recuperar imagen de IPFS & Imagen original recuperada \\
        FP-004 & Actualización de estado & Comparendo en estado "Pendiente" & 1. Cambiar estado a "Pagado"\newline 2. Registrar cambio en Blockchain & Estado actualizado inmutablemente \\
        FP-005 & Validación de integridad & Comparendo con evidencia asociada & 1. Calcular hash de imagen actual\newline 2. Comparar con CID registrado & Integridad verificada \\
        \bottomrule
    \end{tabular}
    \vspace{2em}
    \begin{flushleft}
        \textit{Nota.} Elaboración propia.
    \end{flushleft}
    \refstepcounter{table}\label{tab:casos_prueba_funcionales}
\end{table} 