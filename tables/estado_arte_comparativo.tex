\begin{longtable}{p{2.5cm} p{2.2cm} p{2.2cm} p{2.5cm} p{3.2cm}}
    \caption{Análisis Comparativo del Estado del Arte en Gestión de Infracciones con Blockchain} \\
        \toprule
        \textbf{Trabajo} & \textbf{Ámbito} & \textbf{Tecnologías} & \shortstack{Limitaciones\\Identificadas} & \shortstack{Aporte Relevante\\para el Prototipo} \\
        \midrule
        \endfirsthead

        \toprule
        \textbf{Trabajo} & \textbf{Ámbito} & \textbf{Tecnologías} & \shortstack{Limitaciones\\Identificadas} & \shortstack{Aporte Relevante\\para el Prototipo} \\
        \midrule
        \endhead
        \endfoot
        
        \bottomrule
        \endlastfoot
        Yousfi et al. (2022) & Gestión de tráfico urbano & Blockchain pública, Smart Contracts & Alto costo de transacciones (gas), privacidad limitada para datos personales & Modelo conceptual de integración blockchain-tráfico, solución a la transparencia y trazabilidad \\
        \midrule
        Chen et al. (2024) & Sistema de multas electrónicas & Base de datos centralizada + Blockchain & Falta de inmutabilidad completa, dependencia del servidor central & Propuesta de registrar hash de actas en blockchain para mayor integridad y transparencia \\
        \midrule
        Joseph (2023) & Registros vehiculares gubernamentales & Hyperledger Fabric, IPFS & Complejidad para escalar, gestión de identidades & Arquitectura permisionada para manejo seguro de datos sensibles \\
        \midrule
        Dutta et al. (2023) & Seguros automotrices & Ethereum, Smart Contracts & Latencia en transacciones, costos operativos & Automatización de procesos mediante contratos inteligentes \\
        \midrule
        Omar et al. (2024) & Gestión de infracciones de tránsito & Blockchain híbrida, base de datos & Integración parcial, falta de flujo completo & Aproximación hacia una gestión descentralizada con uso mixto de tecnologías \\
        \midrule
        Choquevilca Quispe \& Morales Valencia (2024) & Fotocomparendos en Latinoamérica & Análisis conceptual & Falta de implementaciones prácticas en la región & Identificación de brechas y oportunidades para implementación blockchain \\
        \midrule
        Proyectos e-gov (Suecia, Estonia) & Registros gubernamentales & Blockchain permisionada, KSI & Limitado a registros específicos, no multas & Validación técnica y mejora en eficiencia para registros oficiales \\
        \midrule
        Anand \& Singh (2024) & Gestión de documentos oficiales & IPFS + Blockchain & Persistencia en IPFS, costos de almacenamiento & Almacenamiento distribuido para evidencias con verificación en blockchain \\
        \midrule
        JUIT Research Group (2024) & Sistema de pagos gubernamentales & Stablecoins, Smart Contracts & Adopción de criptomonedas, entorno regulatorio restringido & Automatización de pagos en ecosistemas blockchain \\
    \end{longtable}
    \vspace{1em}
    \begin{flushleft}
        \textit{Nota.} Elaboración propia.
    \end{flushleft}
    \label{tab:estado_arte_comparativo}

