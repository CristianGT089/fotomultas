\footnotesize
\renewcommand{\arraystretch}{1.3}
\setlength{\LTpre}{10pt}
\setlength{\LTpost}{10pt}
\begin{longtable}{p{4cm}p{9.5cm}}

\caption{Glosario de Términos Técnicos}
\label{tab:glosario_terminos} \\
\toprule
\textbf{Término} & \textbf{Definición} \\
\midrule
\endfirsthead

\caption[]{(Continuación)} \\
\toprule
\textbf{Término} & \textbf{Definición} \\
\midrule
\endhead

\midrule
\multicolumn{2}{r}{\textit{Continúa en la siguiente página}} \\
\endfoot

\bottomrule
\multicolumn{2}{l}{\textbf{Nota.} Elaboración propia.} \\
\endlastfoot

ABI (\textit{Application Binary Interface}) & Interfaz que define cómo llamar funciones de un \textit{Smart Contract} desde aplicaciones externas. Contiene nombres de funciones, parámetros y tipos de retorno. \\
\midrule

\textit{Blockchain} & Tecnología de registro distribuido que almacena datos en bloques encadenados mediante \textit{hashes} criptográficos, garantizando inmutabilidad. \\
\midrule

CA (\textit{Certificate Authority}) & Entidad que emite y gestiona certificados digitales en una red \textit{Hyperledger Fabric}, controlando identidades y permisos. \\
\midrule

\textit{Chaincode} & \textit{Smart Contract} en el contexto de \textit{Hyperledger Fabric}, generalmente escrito en \textit{Go}, que define la lógica de negocio. \\
\midrule

\textit{CID} (\textit{Content Identifier}) & \textit{Hash} único que identifica un archivo en \textit{IPFS}. Se genera mediante criptografía del contenido del archivo. \\
\midrule

Consenso & Mecanismo mediante el cual los nodos de una \textit{blockchain} acuerdan la validez de las transacciones. Ejemplos: \textit{PBFT}, \textit{PoS}, \textit{PoW}. \\
\midrule

\textit{DLT} (\textit{Distributed Ledger Technology}) & Tecnología de libro mayor distribuido que mantiene registros sincronizados entre múltiples nodos sin autoridad central. \\
\midrule

\textit{Ethers.js} & Biblioteca \textit{JavaScript} para interactuar con la \textit{blockchain} de \textit{Ethereum}, permitiendo leer datos y enviar transacciones. \\
\midrule

\textit{Gas} & Unidad de medida del costo computacional en \textit{Ethereum}. Cada operación consume \textit{gas} que se paga en \textit{Ether}. \\
\midrule

\textit{Hardhat} & Framework de desarrollo para \textit{Ethereum} que facilita compilación, testing y despliegue de \textit{Smart Contracts}. \\
\midrule

\textit{Hash} Criptográfico & Función matemática que convierte datos de cualquier tamaño en una cadena de longitud fija. Ejemplos: \textit{SHA-256}, \textit{Keccak-256}. \\
\midrule

\textit{Hyperledger Fabric} & Plataforma de \textit{blockchain} permisionada empresarial, parte del proyecto \textit{Hyperledger} de \textit{Linux Foundation}. \\
\midrule

Inmutabilidad & Propiedad de \textit{blockchain} que garantiza que datos una vez escritos no pueden ser alterados sin dejar evidencia. \\
\midrule

\textit{IPFS} (\textit{InterPlanetary File System}) & Sistema de archivos \textit{peer-to-peer} distribuido que usa direccionamiento por contenido mediante \textit{CIDs}. \\
\midrule

\textit{Ledger} & Libro mayor que registra todas las transacciones en una \textit{blockchain}. Es distribuido y sincronizado entre nodos. \\
\midrule

Nodo (\textit{Node}) & Computadora que participa en una red \textit{blockchain}, manteniendo una copia del \textit{ledger} y validando transacciones. \\
\midrule

\textit{OpenZeppelin} & Librería de \textit{Smart Contracts} auditados y seguros para \textit{Ethereum}, proporciona implementaciones estándar de tokens, control de acceso, etc. \\
\midrule

\textit{Orderer} & Nodo en \textit{Hyperledger Fabric} que ordena transacciones y las agrupa en bloques para distribuir a los \textit{peers}. \\
\midrule

\textit{PBFT} (\textit{Practical Byzantine Fault Tolerance}) & Algoritmo de consenso tolerante a fallas bizantinas usado en \textit{Hyperledger Fabric}, eficiente para redes permisionadas. \\
\midrule

\textit{Peer} & Nodo en \textit{Hyperledger Fabric} que mantiene una copia del \textit{ledger} y ejecuta \textit{chaincode}. \\
\midrule

\textit{Pinning} & En \textit{IPFS}, mantener un archivo almacenado permanentemente en un nodo para garantizar su disponibilidad. \\
\midrule

\textit{PoS} (\textit{Proof of Stake}) & Mecanismo de consenso donde validadores son seleccionados según la cantidad de criptomoneda que poseen. \\
\midrule

\textit{PoW} (\textit{Proof of Work}) & Mecanismo de consenso que requiere resolver acertijos criptográficos complejos para validar bloques. \\
\midrule

\textit{Private Data Collections} & Funcionalidad de \textit{Hyperledger Fabric} para almacenar datos privados que solo ciertos nodos pueden acceder. \\
\midrule

\textit{Smart Contract} & Programa autoejecutante almacenado en \textit{blockchain} que ejecuta lógica de negocio cuando se cumplen condiciones. \\
\midrule

\textit{Solidity} & Lenguaje de programación orientado a objetos para escribir \textit{Smart Contracts} en \textit{Ethereum}. \\
\midrule

\textit{Testnet} & Red de prueba de \textit{blockchain} que imita el funcionamiento de la red principal pero sin valor real. Ejemplo: \textit{Sepolia}. \\
\midrule

\textit{Transaction Hash} & Identificador único de una transacción en \textit{blockchain}, generado mediante \textit{hash} criptográfico de su contenido. \\
\midrule

\textit{TypeScript} & \textit{Superset} de \textit{JavaScript} con tipado estático, usado para desarrollo backend del proyecto. \\
\midrule

\textit{Wallet} & Software que almacena claves privadas y permite firmar transacciones en \textit{blockchain}. \\
\bottomrule
\end{longtable}
