% Tablas de resultados de pruebas

\subsection{Casos de prueba funcionales}

\small
\renewcommand{\arraystretch}{1.2}
\setlength{\LTpre}{10pt}
\setlength{\LTpost}{10pt}
\begin{longtable}{p{2cm} p{4cm} p{3cm} p{3cm}}

\caption{Resultados de pruebas funcionales del sistema}
\label{tab:resultados_funcionales} \\
\toprule
\textbf{ID} & \textbf{Caso de Prueba} & \textbf{Resultado} & \textbf{Estado} \\
\midrule
\endfirsthead

\caption[]{(Continuación)} \\
\toprule
\textbf{ID} & \textbf{Caso de Prueba} & \textbf{Resultado} & \textbf{Estado} \\
\midrule
\endhead

\midrule
\multicolumn{4}{r}{\textit{Continúa en la siguiente página}} \\
\endfoot

\bottomrule
\multicolumn{4}{l}{\textbf{Nota.} Elaboración propia.} \\
\endlastfoot

FP-001 & Registro de fotocomparendo & Registro exitoso con CID & Exitoso \\
\midrule
FP-002 & Consulta de comparendo & Datos recuperados correctamente & Exitoso \\
\midrule
FP-003 & Verificación de evidencia & Imagen recuperada desde IPFS & Exitoso \\
\midrule
FP-004 & Actualización de estado & Estado actualizado en blockchain & Exitoso \\
\midrule
FP-005 & Validación de integridad & Integridad verificada & Exitoso \\
\bottomrule
\end{longtable}

\subsection{Casos de prueba de inmutabilidad}

\small
\renewcommand{\arraystretch}{1.2}
\setlength{\LTpre}{10pt}
\setlength{\LTpost}{10pt}
\begin{longtable}{p{2cm} p{6cm} p{3cm}}

\caption{Resumen de casos de prueba de inmutabilidad ejecutados}
\label{tab:resumen_inmutabilidad} \\
\toprule
\textbf{ID} & \textbf{Descripción} & \textbf{Estado} \\
\midrule
\endfirsthead

\caption[]{(Continuación)} \\
\toprule
\textbf{ID} & \textbf{Descripción} & \textbf{Estado} \\
\midrule
\endhead

\midrule
\multicolumn{3}{r}{\textit{Continúa en la siguiente página}} \\
\endfoot

\bottomrule
\multicolumn{3}{l}{\textbf{Nota.} Elaboración propia.} \\
\endlastfoot

IM-001 & Intento de modificar metadatos directamente en el ledger & Ejecutada \\
\midrule
IM-002 & Alteración de imagen ya registrada en IPFS & Ejecutada \\
\midrule
IM-003 & Verificación de trazabilidad e integridad del historial & Ejecutada \\
\bottomrule
\end{longtable}

\subsection{Pruebas de rendimiento básico}

Se midió el tiempo requerido para ejecutar operaciones clave en condiciones simuladas de uso real:

\small
\renewcommand{\arraystretch}{1.2}
\setlength{\LTpre}{10pt}
\setlength{\LTpost}{10pt}
\begin{longtable}{p{4cm} p{3cm}}

\caption{Tiempos promedio de operaciones en el entorno de prueba}
\label{tab:rendimiento} \\
\toprule
\textbf{Operación} & \textbf{Tiempo Promedio (s)} \\
\midrule
\endfirsthead

\caption[]{(Continuación)} \\
\toprule
\textbf{Operación} & \textbf{Tiempo Promedio (s)} \\
\midrule
\endhead

\midrule
\multicolumn{2}{r}{\textit{Continúa en la siguiente página}} \\
\endfoot

\bottomrule
\multicolumn{2}{l}{\textbf{Nota.} Elaboración propia.} \\
\endlastfoot

Registro completo (Blockchain + IPFS) & 1.60 \\
\midrule
Consulta de evidencia desde IPFS & 0.80 \\
\midrule
Validación de integridad & 0.90 \\
\bottomrule
\end{longtable} 
