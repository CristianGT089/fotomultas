\footnotesize
\renewcommand{\arraystretch}{1.15}
\setlength{\LTpre}{10pt}
\setlength{\LTpost}{10pt}
\begin{longtable}{p{1.8cm} p{2.8cm} p{2.8cm} p{3.5cm}}

\caption{Comparación entre un modelo centralizado y un modelo descentralizado}
\label{tab:comparacion_modelos} \\
\toprule
\textbf{Característica} & \textbf{Modelo Centralizado} & \textbf{Modelo Descentralizado} & \textbf{Relevancia Contextual} \\
\midrule
\endfirsthead

\caption[]{(Continuación)} \\
\toprule
\textbf{Característica} & \textbf{Modelo Centralizado} & \textbf{Modelo Descentralizado} & \textbf{Relevancia Contextual} \\
\midrule
\endhead

\midrule
\multicolumn{4}{r}{\textit{Continúa en la siguiente página}} \\
\endfoot

\bottomrule
\multicolumn{4}{p{13cm}}{\textbf{Nota.} Elaboración propia, con hallazgos basados en la Auditoría de Cumplimiento No. 90 de la Contraloría de Bogotá D.C. (octubre de 2023) y la Auditoría de Cumplimiento 170100-0054-24.} \\
\endlastfoot

Modelo de Confianza & Basado en la confianza en los administradores del sistema y en la robustez de los controles internos definidos. & Basado en un consenso criptográfico distribuido, donde la confianza reside en el protocolo y no en un intermediario. & La correcta asignación de roles es fundamental. La auditoría observó ``ausencia de un profesional responsable de Seguridad de la Información'' (págs. 20--25), subrayando la criticidad de los factores de gobernanza. \\
\midrule
Integridad de Datos & La integridad se asegura mediante controles de acceso y logs de auditoría internos gestionados por la entidad. & La integridad es una propiedad intrínseca de la estructura de datos; los registros son inmutables por diseño. & La efectividad de los controles internos es fundamental. La auditoría documentó ``Falta de control sobre la integridad y calidad de los datos migrados'' (págs. 38--40) como punto de atención. \\
\midrule
Gestión de Seguridad & Dependiente de políticas y procedimientos de seguridad definidos y ejecutados por la institución. & La seguridad es una propiedad inherente a la capa de protocolo, auditada de forma continua y global por la comunidad. & La formalización de procedimientos es clave. La auditoría identificó ``falta de gestión formal de riesgos y controles'' y ``ausencia de un plan de seguridad para la infraestructura en la nube'' (págs. 25--30). \\
\midrule
Auditabilidad y Trazabilidad & La auditoría se realiza a través de logs internos, con acceso gestionado por la entidad y sujeto a sus políticas de retención y seguridad. & La traza de auditoría es transparente, inalterable por diseño y públicamente verificable por cualquier actor autorizado. & La consistencia de los registros internos es un factor de éxito. La auditoría observó ``retrasos y baja velocidad de desarrollo'' (págs. 15--20), subrayando la importancia de una gobernanza rigurosa. \\
\bottomrule
\end{longtable}

